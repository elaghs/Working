Sequential model checkers can construct proofs of very complex models.  However, the results reported by the tool when a proof succeeds do not generally provide much insight to the user.  It is often useful for users to have traceability information: which portions of the model were necessary to the proof.  This traceability information can be used to diagnose a variety of modeling problems such as overconstrained axioms and underconstrained properties, and can also be used to measure {\em completeness} of a set of requirements over a model.  In this paper, we present a new algorithm to efficiently compute the set of support within a model necessary for inductive proofs of safety properties for sequential systems.  The algorithm is based on the UNSAT core support built into current SMT solvers and a novel encoding of the inductive problem to try to generate a minimal set of support.  We prove our algorithm correct, and describe its implementation in the jkind model checker for Lustre models.  We then present an experiment in which we benchmark the algorithm in terms of speed, robustness, and minimality, with promising results.

