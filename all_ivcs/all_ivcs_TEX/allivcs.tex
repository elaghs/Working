\section{Method}
\label{sec:allivcs}
 
\newcommand{\ucalg}{IVC\_UC\xspace}  

\newcommand{\mink}{\textsc{MinimizeK}\xspace}
\newcommand{\reduceinv}{\textsc{ReduceInvariants}\xspace}
\newcommand{\minivc}{\textsc{MinimizeIvc}\xspace}

\newcommand{\checksat}{\textsc{CheckSat}\xspace}
\newcommand{\isadeq}{\textsc{CheckAdq}\xspace}
\newcommand{\actlit}{\textsc{ActLit}}
\newcommand{\unsatcore}{\textsc{UnsatCore}\xspace}
\newcommand{\unsat}{\texttt{UNSAT}\xspace}
\newcommand{\sat}{\texttt{SAT}\xspace}

As mentioned, the contribution of this paper is to provide an efficient and complete method for calculating $AIVC$ of a property. To this end, we first introduce several additional notions and definitions. 
%most of which are inspired by the MUS enumeration technique proposed in \cite{marco2016fast}.

The technique for enumerating all IVCs is a generalization of exploring the power set of $T$ (denoted by $ \mathcal{P}(T) $).
Basically, the algorithm needs to explore the provability of a
given property by any subset of $T$, which may sound impractical.
However, this section, first, proposes some notions that make it possible to have a complete
enumeration algorithm that only needs to explore a (small) portion of $\mathcal{P}(T)$
in order to compute $AIVC$.

\begin{definition} {\emph{Adequacy:}}
\label{def:adeq}
Given property $P$ and every $S \in \mathcal{P}(T)$, we have either $(I, S) \vdash P$ or $(I, S) \nvdash P$. In the former case, we say $S$ is \textbf{adequate} for $P$; in the latter, we say that $S$ is \textbf{inadequate} for the proof of $P$.
\end{definition}

\begin{definition}{\emph{Maximal Inadequate Subset (MIS):}}
  \label{def:mis}
  $S \subset T$ for $(I, T) \vdash P$ is a Maximal Inadequate Subset (MIS) iff
  $(I, S) \nvdash P$ and $\forall T_i \in T\setminus S.~ (I, S\cup{T_i}) \vdash P$.
\end{definition}
\begin{note}
If the second condition in $MIS$ does not hold, $S$ is not maximal, but it is an inadequate set.
\end{note}

\begin{definition}{\emph{Minimal Correction Set (MCS):}}
  \label{def:mcs}
  $S \subset T$ for $(I, T) \vdash P$ is a Minimal Correction Set (MCS) iff
  $(I, T \setminus S) \nvdash P$ and $\forall T_i \in S.~ (I, (T \setminus S)\cup \{T_i\}) \vdash P$.
\end{definition}

%It should be mentioned that minimality and maximality are about minimum or maximum cardinality subsets.
Note that $MCS$ is more of syntactic sugar that specifies sets that can also be specified by $MIS$; i.e. for any $MIS$ of $T$, there is a corresponding $MCS$ such that adding any element of that $MCS$ to the $MIS$, makes the property provable by the $MIS$.
And, that's why it is called the ``minimal correction'' set. 
\begin{note}
If $S \subset T$ is inadequate, but not maximal, there will be a correction set $C \subset T$ for it such that $S \cup C$ is adequate.
\end{note}

\begin{lemma}
\label{lem:adeq}
For $(I, T) \vdash P$, if $S \subset T$ is adequate for property $P$, then all of its supersets are adequate for $P$ as well:
\allowbreak $$\forall S_1 \subset S_2 \subset T.~ (I, S_1) \vdash P \Rightarrow (I, S_2) \vdash P$$
\end{lemma}
\begin{proof}
From $S_1 \subseteq S_2$ we have $S_2 \Rightarrow S_1$. Thus the
  reachable states of $(I, S_2)$ are a subset of the reachable states
  of $(I, S_1)$.
\end{proof}

\begin{lemma}
\label{lem:inadeq}
For $(I, T) \vdash P$, if a given subset $S$ is inadequate, then all of its subsets are inadequate as well:
\allowbreak $$\forall S_1 \subset S_2 \subset T.~ (I, S_2) \nvdash P \Rightarrow (I, S_1) \nvdash P$$
\end{lemma}
\begin{proof}
Given the fact that $(I, T) \vdash P$ and $S_2$ is inadequate,
there is a correction set $C \subseteq (T\setminus S_2)$ such that
$C \cup S_2$ is adequate, which implies every $S' \subseteq (T \setminus C)$ is inadequate.
So, since $S_1 \subset S_2 \subseteq (T \setminus C)$, set $S_1$ is also inadequate.
\end{proof}

In exploring the power set,
we define two functions \checksat and \isadeq; function \isadeq tells whether or not property $P$ is provable by a subset of $T$:
$$\isadeq : \mathcal{P}(T) \rightarrow \{true~(adequate), false~(inadequate)\}$$
\noindent Function \checksat determines if an
existentially quantified formula is satisfiable or not
\footnote{We assume readers are familiar with the boolean satisfiability problem, which is the problem of determining whether there exists an assignment that satisfies a given propositional formula. For more information, refer to \cite{cook1971complexity}.}:
$$\checksat: Boolean \rightarrow \{ true~(\sat), false~(\unsat) \}$$

\noindent The main core of an algorithm for computing $AIVC(P)$ is to choose an \emph{unexplored} $S \in \mathcal{P}(T)$ and examine whether or not it is adequate for $P$; if so, then compute an $S' \subseteq S$ such that $IVC(P, S')$. 
To choose $S$ and keep track of which parts of the the power set
have been explored, we use a Boolean variable $map$, which is
in conjunctive normal form (CNF) and built gradually as the algorithm proceeds. 
As $map$ gets updated, it guides our exploration algorithm for $AIVC$. 
In other words, $map$ shows the \emph{search space} we have to explore for finding all IVCs. 

More precisely, given $T$ with $n$ top-level conjuncts, 
we define an ordered
set of activation literals $\mathcal{A} = \{a_1...a_n\}$, where every $a_i$ is of type of Boolean. Function $\actlit : T \rightarrow \mathcal{A}$ 
is a one-to-one map assigning every $T_i \in T$ to an $a_i \in \mathcal{A}$. 
Then, a $map$ for $AIVC(P)$ is a CNF formula built over the elements of $\mathcal{A}$ such that:
\begin{itemize}
  \item Initially $map = true$ which we will interpret as the fact that the whole $\mathcal{P}(T)$ is unexplored.
  \item For every explored set $S \in \mathcal{P}(T)$: 
  \begin{itemize}
    \item if $S$ is adequate for $P$, then $map$ contains a clause $\bigvee_{i: T_{i}\in S} \neg {\actlit (T_i)}$. Note that, according to Lemma \ref{lem:adeq}, such a clause makes every superset of $S$ as explored, which shrinks our search space.
    \item if $S$ is inadequate for $P$, then $map$ contains a clause $\bigvee_{i: T_{i}\in (T \setminus S)} \actlit (T_i)$.
  \end{itemize}
\end{itemize}



If we consider the power set as a lattice in a Hasse diagram,
$MAP$ tells us which parts of the lattice have been explored:
$$MAP: \mathcal{P}(T) \rightarrow \{true~(unexplored), false~(explored)\}$$




\begin{definition}{\emph{Unexplored subset problem:}}
  \label{def:usp}
  
\end{definition}





 

%\begin{algorithm}[t]
%  $k' \leftarrow 1$ \\
%  \While{$\checksat(\neg\iq_{k'}(T, P, P)) = \sat$} {
%    $k' \leftarrow k' + 1$ \\
%    }
%  \Return{$k'$} \\
%\caption{$\mink(T, P)$}
%\label{alg:minimize-k}
%\end{algorithm}
 

%\begin{theorem}
%\label{thm:minimal-hard}
%\end{theorem}
%\begin{proof}
%\end{proof}
