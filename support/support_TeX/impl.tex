\section{Implementation}
\label{sec:impl}

We have implemented the inductive validity core algorithms in the
previous section in two tools: {\em JKind}, which performs the \ucalg
algorithm, and {\em JSupport}, which can compute either the \bfalg or
the \ucbfalg algorithm (using JKind as a subprocess). Moreover, our
implementation of \ucbfalg uses an additional feature of JKind to
store and re-use discovered invariants between separate runs. This
reduces some of the cost of attempting to re-prove a property multiple
times. These tools operate over the Lustre
language~\cite{Halbwachs91:lustre}, which we briefly illustrate below.

\subsection{Lustre and IVCs}

Lustre~\cite{Halbwachs91:lustre} is a synchronous dataflow language
used as an input language for various model checkers. The textual
models in Figures~\ref{fig:ex-before} and \ref{fig:ex-after} are
written in Lustre. We will use model in Figure~\ref{fig:ex-before} as
a running example in this section. For our purposes, a Lustre program
consists of 1) input variables, {\tt x} in the example, 2) output
variables, {\tt a}, {\tt b}, and {\tt y} in the example, and 3) an
equation for each output variable. A Lustre program runs over discrete
time steps. On each step, the input variables take on some values and
are used to compute values for the output variables on the same step.
In addition, equations may refer to the previous value of a variable
using the {\tt pre} operator. This operator is underspecified in the
initial step, so the arrow operator, {\tt ->}, is used to guard the
{\tt pre} operator. In the initial step the expression {\tt e1 -> e2}
evaluates to {\tt e1}, and it evaluates to {\tt e2} in all other steps.

We interpret a Lustre program as a model specification by considering
the behavior of the program under all possible input traces. Safety
properties over Lustre can then be expressed as Boolean expressions in
Lustre. A safety property holds if the corresponding expression is
always true for all input traces. For example, the property for
Figure~\ref{fig:ex-before} is {\tt y >= 0}, which is a valid property.

It is straightforward to translate this interpretation of Lustre into
the traditional initial and transition relations. We will show this by
continuing with the example in Figure~\ref{fig:ex-before}. First we
introduce a new Boolean variable $init$ into the state space to denote
when the system is in its initial step. Then we define,
\begin{align*}
  &I((x, a, b, y, \mathit{init})) = \mathit{init} \\
  &T((x, a, b, y, \mathit{init}), (x', a', b', y', \mathit{init'})) = \\
  &\hspace{1.5cm} (a' = f(x', \ite{init}{0}{y})) \land~ \\
  &\hspace{1.5cm} (b' = \ite{a' \geq 0}{a'}{-a'}) \land~ \\
  &\hspace{1.5cm} (y' = b' + (\ite{init}{0}{y})) \land ~\\
  &\hspace{1.5cm} \neg\mathit{init'}
\end{align*}
Note that $f$ is unspecified in Figure~\ref{fig:ex-before}. In a real
system, $f$ would be defined in the Lustre model and expanded in the
translation relation. A safety property such as {\tt y >= 0} is
translated into $\mathit{init} \lor (y \geq 0)$. Nested uses of arrow
and pre operators are handled by introducing new output variables for
nested expressions, though such details are unimportant for our
purposes.

Each equation in the Lustre program is translated into a single
top-level conjunct in the transition relation. This is very convenient
as the IVC of a Lustre property can be reported in terms of the output
variables whose equations are part of the IVC. Equivalently, the
interpretation of an IVC for a Lustre property is that any output
variable that is not part of the IVC can be turned into an input
variable, its equation thrown away, while preserving the validity of
the property. Thus the granularity of the IVC analysis is determined
by the granularity of the Lustre equations and can be adjusted by
introducing auxiliary variables for subexpressions if desired.

\subsection{JKind}

JKind is an infinite-state model checker for safety properties. JKind
proves safety properties using multiple cooperative engines in
parallel including $k$-induction, property directed reachability, and
template-based lemma generation. JKind accepts Lustre programs written
over the theory of linear integer and real arithmetic. In the
back-end, JKind uses an SMT solver such as Yices, Z3, MathSAT, or
SMTInterpol.

JKind works on multiple properties simultaneously. When a property is
proven and IVC generation is enabled, an additional parallel engine
executes Algorithm~\ref{alg:ivc} to generate a near-minimal IVC.

JKind accepts an annotation on its input Lustre program indicating
which outputs variables to consider for IVC generation. Output
variables not mentioned in the annotation are implicitly included in
all IVCs. This allows the implementation focus on the variables
important to the user and ignore, for example, administrative
equations. This is even more important for tools which generate Lustre
as they often create many such administrative equations which simply
wire together more interesting expressions.

%%  LocalWords:  JSupport subprocess IVCs dataflow underspecified IVC
%%  LocalWords:  init Yices MathSAT SMTInterpol subexpressions
