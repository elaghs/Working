\subsection{Minimality of Inductive Validity Cores}
\label{subsec:minimality}
\newcommand{\ucalg}{\texttt{\small{IVC\_UC}}}
\newcommand{\ucbfalg}{\texttt{\small{IVC\_UCBF}}}

To compute all $IVC$s of a given property, the \aivcalg ~algorithm
 makes a major assumption on the outcome of the \getivc ~module;
it assumes \getivc ~returns \emph{minimal} Inductive Validity Cores.
However, as shown in \cite{Ghass16},
determining  the minimality of  an  $IVC$ is  hard. Although the $IVC$ generation algorithm proposed by Ghassabani et al. \cite{Ghass16}, called \ucalg ,
is efficient and fast, it does not guarantee minimality;
i.e. an $IVC$ obtained from \ucalg ~is either minimal or approximately minimal.
In the \ucalg ~algorithm, $IVC$s are built upon inductive proofs,
which usually rely on additional
lemmas (or invariants) derived by the model checking algorithms
for strengthening the property and proving it
inductively.
Depending on the structure of the inductive
proofs, an $IVC$ computed by \ucalg ~might still not be minimal.
They also proposed another algorithm that guarantees
the minimality of an $IVC$ \cite{Ghass16}, called \ucbfalg, which is a lot more expensive than the \ucalg. The \ucbfalg ~takes an $IVC$ generated by
the \ucalg ~as input and then, minimizes it brute-forcefully. In this section,
first we show that the \aivcalg ~algorithm is correct even if
 the \getivc ~module does not guarantee minimality.
Then, we mention how to modify Algorithm \ref{alg:aivc}
 such that the output set $A$ only contains the \textbf{minimal} $IVC$s.

%\begin{algorithm}
%  \SetKwInOut{Input}{input}
%  \SetKwInOut{Output}{output}
%  \Input{$(I, T) \vdash P$}
%  \Output{Minimal IVC for $(I, T) \vdash P$}
%  \BlankLine
%  $S \leftarrow \ucalg((I, T) \vdash P)$ \\
%  \For{$x \in S$} {
%    \If{$(I, S\setminus\{x\}) \vdash P$}{
%      $S \leftarrow S\setminus \{x\}$
%    }
%  }
%  \Return{S}
%\caption{An abstract representation of \ucbfalg \cite{Ghass16}}
%\label{alg:ucbf}
%\end{algorithm}

\begin{theorem}
\label{theorem:ivc-not-min}
  Algorithm \ref{alg:aivc} enumerates all minimal Inductive Validity Cores
  even if \getivc ~does not guarantee minimality of an Inductive Validity Core.
\end{theorem}
\begin{proof}
In an iteration of the \texttt{while} loop that $\isadeq (P, M)$ is $true$,
let \getivc ~returns a set $S'$ which is not minimal:
$\exists \mathcal{M} \subset S'.~ (I, \mathcal{M}) \vdash P$ and
$\forall T_i \in \mathcal{M} . ~ (I, \mathcal{M} \setminus \{T_i\}) \nvdash P$.
Therefore, there is an unexplored set $S \supseteq \mathcal{M}$ that all its supersets have already been explored, which means
$map$ is still satisfiable.
So, eventually there will be a model of $map$ that makes $M = S$ in line 7.
Again at that point, suppose \getivc ~returns an adequate set $S''$ which is not minimal.
We know that
  $\mathcal{M} \subset S'' \subseteq S \wedge S'' \neq S' \wedge |S''| \leq |S'|$.
  Previously, set $S'$ marked all its supersets as explored. And now, the
  same thing happens with $S''$.
  In the worst case scenario, \getivc
   ~explores all the supersets of $\mathcal{M}$
   before generating $\mathcal{M}$; hence with the same reasoning,
  eventually, all the supersets of $\mathcal{M}$ will be explored, and there will be a model of $map$ that makes $M = \mathcal{M}$ in line 7,
  which results in a minimal $IVC$ by \getivc ~(because $\mathcal{M}$ is minimal itself).
  Now,  $\mathcal{M}$  prunes itself from the $map$.
  This reasoning can be applied to all minimal Inductive Validity Cores.
  Therefore, Algorithm \ref{alg:aivc} explores all minimal adequate sets even if \getivc ~
  does not guarantee minimality.
\end{proof}

\begin{theorem}
 Algorithm \ref{alg:aivc} will terminate even if \getivc ~does not guarantee minimality.
\end{theorem}
\begin{proof}
  Immediate from Theorem \ref{theorem:ivc-not-min}, \ref{theorem:aivc}, and \ref{theorem:termination}.
\end{proof}

Although we have proved that \getivc ~does not have to return minimal sets,
with the current version of the \aivcalg ~algorithm, if \getivc ~does not
return minimal adequate sets, at the end of the process,
 set $A$ may contain some supersets of
minimal IVCs as well. So, to make sure that the algorithm only returns
the minimal adequate sets ($IVC$s), all we need is to
remove supersets before returning $A$; for example, adding the following code snippet after line 9 of Algorithm \ref{alg:aivc} serves this purpose:

\vspace{0.09in}
\indent  \For{$X \in A$}{
\indent        \If{$S' \subset X$}{ $A \leftarrow A \setminus \{X\}$}
\indent      }
\vspace{0.09in}
\noindent Obviously, the closer to minimal the results of \getivc ~are,
the more quickly the search space shrinks and Algorithm \ref{alg:aivc} terminates.

%
%\begin{algorithm}[t]
%  \SetKwInOut{Input}{input}
%  \SetKwInOut{Output}{output}
%  \Input{$(I, T) \vdash P$}
%  \Output{$AIVC (P)$}
%  \BlankLine
%  $A \leftarrow \varnothing$\\
%  Create activation literals $\{a_1, \ldots, a_n\}$ \\
% % $map \leftarrow true$ \\
%  $map \leftarrow \top$ \\
% % $L \leftarrow \varnothing$ \\
%  \BlankLine
%
%  \While{$\checksat (map)$} {
%    $model \leftarrow $ build a maximal model of $map$ \\
%    $M \leftarrow$ extract the set of variables assigned $true$ in $model$ \\
%    $M \leftarrow \bigcup_{a_i \in M} \actlit ^{-1}(a_i)$ \\
%\BlankLine
%    \If{$\isadeq (P, M)$}{
%    \BlankLine
%      $S' \leftarrow \getivc (P, M)$ \\
%
%      \For{$X \in A$}{
%        \If{$S' \subset X$}{ $A \leftarrow A \setminus X$}
%      }
%      $A \leftarrow A \cup S'$ \\
%
%      $map \leftarrow map \wedge (\bigvee_{T_{i}\in S'} \neg {\actlit (T_i)})$
%    }
%    \Else{
%      $map \leftarrow map \wedge (\bigvee_{T_{i}\in (T \setminus M)} \actlit (T_i))$ \\
%    %  $L \leftarrow L \cup (\bigcup_{T_i \in C} T_i)$
%    }
%  }
%  \Return{$A$}
%\caption{Modified version of the \aivcalg ~algorithm that
%guarantees $AIVC(P)$ contains minimal Inductive Validity Cores}
%\label{alg:aivc2}
%\end{algorithm} 