%Increasing usage of computer systems in safety-critical applications demands the utmost care in their specification, design, and implementation.  Formal verification is a useful method for mathematical/systematic examination of the requirements a system must meet. However, due to the complexity of modern systems, safety analysis is challenging. One of the most successful and powerful methods for formal verification is symbolic model checking.
Symbolic model checkers can construct proofs of properties over
very complex models. However, the results reported by the tool
when a proof succeeds do not generally provide much insight to
the user. It is often useful for users to have traceability information related to the proof: which portions of the model were necessary to construct it.  This traceability information can be used to diagnose a variety of modeling problems such as overconstrained axioms and underconstrained properties, and can also be used to measure {\em completeness} of a set of requirements over a model.

We propose the notion of {\em inductive validity cores} (IVCs), which are intended to trace a property to a \emph{minimal} set of model elements necessary for proof. Such cores are not unique, and algorithms for efficiently producing both single IVC and all IVCs are presented. IVCs can be used for several interesting analyses, including
regression analysis for testing/proof, determination of the minimum (as
opposed to minimal) number of model elements necessary for proof, the
diversity examination of model elements leading to proof, and analyzing fault
tolerance.
%We, fisrt, propose a method to efficiently compute a single IVC within a model necessary for inductive proofs of safety properties for sequential systems.  The algorithm is based on the UNSAT core support built into current SMT solvers and a novel encoding of the inductive problem to try to generate a minimal inductive validity core.
%Then, we propose an efficient method for finding \emph{all minimal} IVCs of a
%given property. Finally, we introduce several useful applications of this idea.
%We prove our algorithms are correct, and describe their implementation in the JKind model checker for Lustre models.
%We then present an experiment in which we benchmark the algorithms in terms of speed, diversity of produced cores, and minimality, with promising results.
