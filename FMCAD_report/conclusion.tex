\section{Conclusions}
\label{sec:conclusion}
This paper has introduced the notion of inductive validity core (IVC), which
appears to be a useful for inductive model checking. We have presented a novel algorithm for
computing IVCs, published in \cite{Ghass16} and
applicable to all forms of inductive SAT/SMT-based model checking
including $k$-induction, PDR, and interpolation. We have implemented our IVC algorithm as part of the open source model
checker JKind, and shown the algorithm requires only a
small overhead in practice \cite{Ghass16, expr}. Our work has recently been integrated into the AADL/AGREE tool
suite~\cite{QFCS15:backes, hilt2013}, which supports compositional
reasoning about system architectures. A pilot project at Rockwell
Collins is using the traceability information produced by the IVC
support in the AGREE tool. Finally, We are in the process of
comparing our approach against other approaches measuring completeness of
requirements \cite{expr}.
%

%% \begin{itemize}
%%     \item Write this at the end.
%%     \ela{We can add something about fault tolerance for future work.
%%     if we have all sets of support, and their intersection is empty, we have redundancy.
%%     we can talk about algorithm for minimum support set...}
%% \end{itemize}

%%  LocalWords:  IVC IVCs PDR Yices MathSAT SMTInterpol
