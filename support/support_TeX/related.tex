\section{Related work}
\label{sec:related}

%\andrew{This section seems a bit strange since it focuses on MUS more
%  than IVC. Are people doing this for SAT/SMT solving or for model
%  checking?}

Our work builds on top of a substantial foundation building Minimally Unsatisfiable Subformulas
(MUSes) from UNSAT cores~\cite{Cimatti2007:UNSAT}, including \cite{marques2010minimal, belov2012towards, ryvchin2011faster, belov2012computing, nadel2010boosting}.  Recent algorithms can handle very large problems, but computing MUSes is still a resource-intensive task.  While some work is aimed at providing a set of minimal unsatisfiable formulae, minimality is usually defined such that given a set of clauses $\mathbb{M}$, removing any member of $\mathbb{M}$ makes it satisfiable \cite{belov2012computing}.  The step of producing minimal invariants for proofs has been investigated in depth by Ivrii et al. in~\cite{Ivrii14:invariants}.

UNSAT cores and MUSes are used for many different activities within
formal verification. Gupta et al. \cite{gupta2003iterative} and
McMillan and Amla \cite{mcmillan2003automatic} introduced the use of
unsatisfiable cores in proof-based abstraction engines. Their goal is
to shrink the abstraction size by omitting the parts of the design
that are irrelevant to the proof of the property under verification.
Torlak et al. in~\cite{Torlak08:cores} finds MUSes of Alloy
specifications, and considers semantic vacuity, which we consider in
Section~\ref{sec:intro}. Alloy models are only analyzed up to certain
size bounds, however, and in general are unable to prove properties
for arbitrary models. Also, because we are extracting information from
proofs, it is possible to use IVCs for additional purposes (proof
explanation and completeness checking).

%[SLICING]
If we view Lustre as a programming language, our work can be viewed as a more accurate form of program slicing~\cite{Tip95asurvey}.  We perform {\em backwards slicing} from the formula that defines the property of interest of the model.  The slice produced is smaller and more accurate than a static slice of the formula~\cite{Weiser:1981:slicing}, but guaranteed to be a sound slice for the formula for all program executions, unlike dynamic slicing~\cite{Agrawal:1990:slicing}.  Predicate-based slicing has been used~\cite{Li04:slicing} to try to minimize the size of a dynamic slice.  Our approach may have utility for some concerns of program slicing (such as model understanding) by constructing simple ``requirements'' of a model and using the tool to find the relevant portions of the model.

Another potential use of our work is for ``semantic'' vacuity detection.  A standard definition of vacuity is syntactic and defined as follows~\cite{Kupferman:2006:SCF}: {\em A system K satisfies a formula $\phi$ vacuously iff $K \vdash \phi$ and there is some subformula $\psi$ of $\phi$ such that $\psi$ does not affect $\phi$ in K}.  Vacuity has been extensively studied~\cite{Gurfinkel:2012:RVB,Chockler2008,DBLP:Ben-DavidK13,Kupferman:2006:SCF,Chockler:2007,Beer1997} considering a range of different temporal logics and definitions of ``affect''.  On the other hand, our work can be used to consider a broader definition of vacuity.  Even if all subformulae are required (the property is not syntactically vacuous), it may not require substantial portions of the model, and so may be provable for vacuous reasons.  The problem is exacerbated when the modeling and property language are the same (as in JKind), because whether a subformula is considered part of the model or part of the property, from the perspective of checking tools, can be unclear.

Determining completeness of properties has also been extensively studied. Certification standards such as DO-178C~\cite{DO178C} require that requirements-derived tests achieve some level of structural coverage (MC/DC, decision, statement) depending on the criticality level of the software, in order to approximate completeness.  If coverage is not achieved, then additional requirements and tests are added until coverage is achieved.  Chockler~\cite{chockler_coverage_2003} defined the first completeness metrics directly on formal properties based on mutation coverage.  Later work by Kupferman, Chockler and Vardi~\cite{Kupferman:2006:SCF} defines completeness as an extension of vacuity to elements in the model.  We present an alternative approach that uses the proof directly, which we expect to be considerably less expensive to compute.  Recent work by Murugesan~\cite{murugesan2015we} and Schuller~\cite{schuler_assessing_2011} attempts to combine test coverage metrics with requirements to determine completeness.


%[DETERMINING VACUITY]
%- Vardi and Kupferman's work


\iffalse
\begin{itemize}
    \item MUS's : checked
    \item Work on Alloy: checked
    \item Work that Teme pointed us to : will be added
    \item Anything else Elaheh has found : \%60 checked
\end{itemize}
\fi

%%  LocalWords:  Subformulas MUSes UNSAT formulae et al Amla Torlak
%%  LocalWords:  IVCs iff subformula subformulae criticality Chockler
%%  LocalWords:  Kupferman Vardi Murugesan Schuller MUS's Teme Elaheh
