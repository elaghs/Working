\section{Implementation}
\label{sec:impl}

\mike{My thoughts on this section:}
\begin{itemize}
    \item introduce (and shrink) explanation of Lustre
    \item describe that $\widehat T$ is the set of equations in Lustre.
    \item describe what it means to remove an equation (to make valid Lustre, 
        turn assigned variable into input)?  I'm not sure this is necessary, and it 
        gets messy.
\end{itemize}

There are two important solver-based engines in \texttt{JKind} that can be considered as its primary proof-engines: \texttt{PDR} and \texttt{K-induction}. Current version of \texttt{JKind} has a new solver-based engine called \texttt{ReduceSupport},
which is an implementation of Algorithm~\ref{alg:set-of-support}.
One major goal of our experiments was to evaluate the minimality of the support sets computed by \texttt{ReduceSupport}.
To do so, we needed to compare the output of our algorithm with another algorithm
which computes a truly minimal support set, i.e. Algorithm~\ref{alg:naive}. Hence, we extended \texttt{JKind}
with another support computing tool, called \texttt{JSupport}, which is an implementation of Algorithm~\ref{alg:naive}.

As described in~\ref{subsec:jkind}, \texttt{JKind} accepts LUS models for verification.
Such models are made of a set of nodes each of which embodies a set of inputs, outputs, equations, assertions, and properties. For each property, each equation can be considered as an element of an initial set called \textit{\%SUPPORT}. Given initial support set $S$ for property $P$, \texttt{ReduceSupport} finds which of the items in $S$ are necessary to prove $P$. The output of this computation will be a (closely) minimal set of support for $P$. In light of this process, the extended version of \texttt{JKind} we used for the experiments includes:

\begin{itemize}
    \item an algorithm to compute a truly minimal set of support, i.e. \texttt{JSupport}.
    \item given a LUS model, a static crawler which automatically marks all equations of a node in the initial support set of a property.
    \item some trackers that measure the verification time with/ without support computation.
    \item some minor changes in the XML writers.
\end{itemize}
