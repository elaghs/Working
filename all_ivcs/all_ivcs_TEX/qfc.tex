\iffalse
\section{Case Study}

\label{sec:qfc}
We demonstrate the applicability of the algorithms toward ``requirements coverage'' (or complete traceability as described by Murugesan~\cite{Murugesan16:renext}) on two large-scale models from a NASA Quad-redundant Flight Control System (QFCS)~\cite{NFM2015:backes}: the Flight Control System (FCS) (with 3423 equations and 2 properties) and the Flight Control Computer (FCC) (with 9953 equations and 9 properties).  The goal was to determine the scalability and utility of the approach for determining model redundancy and model coverage given the requirements. 

We performed our experiment by examining the conjunction of properties for each model.  The experiment was conducted with a timeout of 6 hours per model. Table \ref{tab:qfcs} shows the result of our study.  We examine {\em model coverage} of the requirements, using two measurements: {\em MAY} coverage, which examines which elements might be necessary for proof, and {\em MUST} coverage, which includes only equations necessary for all proofs.  We also calculate the number and diversity of IVCs, which is an indication of redundancy in the model for its requirements.



\begin{table}
  \caption{\aivcalg\ on the QFCS models}
  \vspace{-0.1in}
  \centering
\begin{tabular}{|c|c|c|}
  \hline
  % after \\: \hline or \cline{col1-col2} \cline{col3-col4} ...
  models & FCS & FCC \\
   \hline
   Initial proof time (min)  & 19.5 & 172 \\
   All-IVC runtime (h) & 6 & 6 \\
   \hline
   \# of IVCs  & 17 & 1   \\
   \hline
   IVC size by \ucbfalg\  & 1659 & 4302  \\
   min IVC size by \aivcalg & 1659 & 4302 \\
   avg IVC size by \aivcalg  & 1659.9 & 4302 \\
   max IVC size by \aivcalg  & 1660 & 4302 \\
  \hline

  coverage score of MAY & 0.62 & 0.43  \\
  coverage score of MUST & 0.47 & 0.43 \\
  \hline
\end{tabular}
\label{tab:qfcs}
\vspace{-0.2in}
\end{table}

%We ran the algorithm on each property separately, which left us with 11 complex problems to study.
%The FCS 3,423  104 prop
%fcc 9953 1511? prop

\fi
