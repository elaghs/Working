\subsection{Using Approximate Inductive Validity Cores for Computing AIVC}
\label{subsec:minimality}
\newcommand{\ucalg}{\texttt{\small{IVC\_UC}}}
\newcommand{\ucbfalg}{\texttt{\small{IVC\_UCBF}}}

To compute all \ivc s of a given property, it appears that the \aivcalg ~algorithm
assumes that \getivc\ always returns \emph{minimal} Inductive Validity Cores.
However, as shown in \cite{Ghass16}, it is computationally cheap to find an
approximately minimal \ivc\ using the algorithm \ucalg; however, finding a {\em guaranteed} minimal \ivc\ using the \ucbfalg\ algorithm is computationally expensive.  In this section,
we demonstrate that the \aivcalg ~algorithm is correct and complete even if the \getivc ~module does not guarantee minimality.  %Then, we mention how to modify Algorithm \ref{alg:aivc} such that the output set $A$ only contains \textbf{minimal} \ivc s.

%\begin{algorithm}
%  \SetKwInOut{Input}{input}
%  \SetKwInOut{Output}{output}
%  \Input{$(I, T) \vdash P$}
%  \Output{Minimal IVC for $(I, T) \vdash P$}
%  \BlankLine
%  $S \leftarrow \ucalg((I, T) \vdash P)$ \\
%  \For{$x \in S$} {
%    \If{$(I, S\setminus\{x\}) \vdash P$}{
%      $S \leftarrow S\setminus \{x\}$
%    }
%  }
%  \Return{S}
%\caption{An abstract representation of \ucbfalg \cite{Ghass16}}
%\label{alg:ucbf}
%\end{algorithm}

\mike{The proof below seems a bit informal.  Moreover, as best I can tell, Lemma 3, corollary 2, Theorem 1, and Theorem 2 all work without modification.  Theorem 1 is a bit informal - the set A {\em contains} all \mivc s but can contain more.  As written, I think the proof works either way.}

\begin{theorem}
\label{theorem:ivc-not-min}
  Algorithm \ref{alg:aivc} enumerates all minimal Inductive Validity Cores
  even if \getivc returns a superset $S'$ of an IVC $S$.
\end{theorem}
\begin{proof}
In an iteration of the \texttt{while} loop that $\isadeq (P, M)$ is $true$,
let \getivc ~return a set $S'$ which is not minimal:
$\exists \mathcal{M} \subset S'.~ (I, \mathcal{M}) \vdash P$ and
$\forall T_i \in \mathcal{M} . ~ (I, \mathcal{M} \setminus \{T_i\}) \nvdash P$.
Then there is an unexplored set $S \supseteq \mathcal{M}$ that all its supersets have already been explored, so $map$ is still satisfiable.
Eventually there will be a model of $map$ that makes $M = S$ in line 7 \mike{why?}.
Again at that point, suppose \getivc ~returns an adequate set $S''$ which is not minimal.
We know that
  $\mathcal{M} \subset S'' \subseteq S \wedge S'' \neq S' \wedge |S''| \leq |S'|$.
  Previously, set $S'$ marked all its supersets as explored. And now, the
  same thing happens with $S''$.
  In the worst case scenario, \getivc
   ~explores all the supersets of $\mathcal{M}$
   before generating $\mathcal{M}$; hence with the same reasoning,
  eventually, all the supersets of $\mathcal{M}$ will be explored, and there will be a model of $map$ that makes $M = \mathcal{M}$ in line 7,
  which results in a minimal \ivc\ by \getivc ~(because $\mathcal{M}$ is minimal itself).
  Now,  $\mathcal{M}$  prunes itself from the $map$.
  This reasoning can be applied to all minimal Inductive Validity Cores.
  Therefore, Algorithm \ref{alg:aivc} explores all minimal adequate sets even if \getivc ~
  does not guarantee minimality.
\end{proof}

\begin{theorem}
 Algorithm \ref{alg:aivc} will terminate even if \getivc ~does not guarantee minimality.
\end{theorem}
\begin{proof}
  Immediate from Theorem \ref{theorem:ivc-not-min}, \ref{theorem:aivc}, and \ref{theorem:termination}.
\end{proof}

Although we have proved that \getivc ~does not have to return minimal sets,
with the current version of the \aivcalg ~algorithm, if \getivc ~does not
return minimal adequate sets, at the end of the process,
 set $A$ may contain some supersets of
minimal IVCs as well. So, to make sure that the algorithm only returns
the minimal adequate sets (\ivc s), all we need is to
remove supersets before returning $A$; for example, adding the following code snippet after line 9 of Algorithm \ref{alg:aivc} serves this purpose:

\vspace{0.09in}
\indent  \For{$X \in A$}{
\indent        \If{$S' \subset X$}{ $A \leftarrow A \setminus \{X\}$}
\indent      }
\vspace{0.09in}
\noindent Obviously, the closer to minimal the results of \getivc ~are,
the fewer iterations are required for Algorithm 1 to terminate.  Each non-minimal adequate set returned by \textsc{GetIVC} will induce an additional iteration for Algorithm 1.

%
%\begin{algorithm}[t]
%  \SetKwInOut{Input}{input}
%  \SetKwInOut{Output}{output}
%  \Input{$(I, T) \vdash P$}
%  \Output{$AIVC (P)$}
%  \BlankLine
%  $A \leftarrow \varnothing$\\
%  Create activation literals $\{a_1, \ldots, a_n\}$ \\
% % $map \leftarrow true$ \\
%  $map \leftarrow \top$ \\
% % $L \leftarrow \varnothing$ \\
%  \BlankLine
%
%  \While{$\checksat (map)$} {
%    $model \leftarrow $ build a maximal model of $map$ \\
%    $M \leftarrow$ extract the set of variables assigned $true$ in $model$ \\
%    $M \leftarrow \bigcup_{a_i \in M} \actlit ^{-1}(a_i)$ \\
%\BlankLine
%    \If{$\isadeq (P, M)$}{
%    \BlankLine
%      $S' \leftarrow \getivc (P, M)$ \\
%
%      \For{$X \in A$}{
%        \If{$S' \subset X$}{ $A \leftarrow A \setminus X$}
%      }
%      $A \leftarrow A \cup S'$ \\
%
%      $map \leftarrow map \wedge (\bigvee_{T_{i}\in S'} \neg {\actlit (T_i)})$
%    }
%    \Else{
%      $map \leftarrow map \wedge (\bigvee_{T_{i}\in (T \setminus M)} \actlit (T_i))$ \\
%    %  $L \leftarrow L \cup (\bigcup_{T_i \in C} T_i)$
%    }
%  }
%  \Return{$A$}
%\caption{Modified version of the \aivcalg ~algorithm that
%guarantees $AIVC(P)$ contains minimal Inductive Validity Cores}
%\label{alg:aivc2}
%\end{algorithm} 