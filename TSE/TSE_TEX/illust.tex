\section{Illustration}
\label{sec:illust}
To illustrate the \aivcalg ~algorithm we use the example presented in Section \ref{sec:example} with $P = ({\small \texttt{on\_p}})$ .
%As inspired by the MARCO algorithm in \cite{marco2016fast}, we visualize $\mathcal{P}(\mathcal{A})$ as a lattice in a Hasse diagram (Fig.~\ref{fig:lattice1}) to demonstrate the progress of the algorithm. As you can see, each level of the lattice contains sets with the same size linked to sets that are their immediate
%supersets (upper level regions) and subsets (lower level regions). Note that the power set is viewed as a Boolean formula, so each member in the lattice shows
% variable with $true$.
For better description, we view $T$ as an ordered set of its top-level conjuncts; i.e. $T = \{$ {\small \texttt{a1\_below, a2\_below, a1\_above, a2\_above, one\_below, both\_above, doi\_on, on\_p}} $\}$.
The algorithm starts with creating activation literals for each $T_i \in T$. Let the ordered set of Boolean variables $\{ a_1, \ldots , a_8 \}$ be the corresponding literals to the elements of $T$ (e.g. $\actlit ({ \small \texttt{a1\_below}}) = a_1$ and $\actlit ({\small \texttt{on\_p}}) = a_8$). Then, line 3 initializes $map$ with $\top$.

In the first iteration of the \texttt{while} loop, since $map$ is
empty, it is satisfiable, and a model for it can be any subset of
literals. So obviously, the first maximal model of $map$ contains all
the literals, which means, in line~\ref{alg:aivc:assignm}, $M = \{a_1,
\ldots, a_8\}$, and in line~\ref{alg:aivc:assigns}, $S = T$. Since $S$
is adequate for $P$, the \getivc ~module is called in
line~\ref{alg:aivc:getivc}. Suppose the returned \mivc\ by this function
is $S' = \{ { \small \texttt{a1\_below},~\texttt{one\_below},
  ~\texttt{doi\_on},~\texttt{on\_p}}\}$; this set is added to $A$ in
line~\ref{alg:aivc:addset}, and thus it comes to adding a new clause
to $map$ (line~\ref{alg:aivc:aadd}), which makes $map = (\neg a_1 \vee
\neg a_5 \vee \neg a_7 \vee \neg a_8)$. As discussed, this constraint
marks all the supersets of $S'$ as blocked and prunes them off the
search space.

For the second iteration, $map$ is still satisfiable,
so the algorithm gets to find a maximal model of it in line~\ref{alg:aivc:maxsat}. Suppose this time, the maximal model makes $M = \{a_1, \ldots, a_7\}$,
which leads to $S = T \setminus \{ {\small \texttt{on\_p}}\} $ in line~\ref{alg:aivc:assigns}.
Since $S$ is inadequate for $P$,
the algorithm jumps to line 12 updating
$map$ as $map \leftarrow map \wedge a_8$.
Adding this new clause removes all the subsets of
$T \setminus \{{\small \texttt{on\_p}}\}$
from the search space. Similarly, in the third iteration, if the maximal model of $map$
yields $M = \{a_1, \ldots, a_4, a_6, \ldots, a_8\}$, then $S = T \setminus \{ {\small \texttt{one\_below}}\} $ will be another inadequate set that makes $map$ become
$map \leftarrow map \wedge a_5$
in line~\ref{alg:aivc:iadd}.

Suppose, in the fourth iteration, the maximal model leads to $M = \{a_2, \ldots, a_8\}$ and
$S = T \setminus \{ {\small \texttt{a1\_below}}\} $ in lines~\ref{alg:aivc:assignm} and~\ref{alg:aivc:assigns}.
Since this $S$ is adequate for $P$, \getivc ~computes a new \mivc\ in line~\ref{alg:aivc:getivc}.
Let the new \mivc\ be $S' = \{ {\small \texttt{a2\_below},~\texttt{one\_below}, ~\texttt{doi\_on},~\texttt{on\_p}}\}$; after adding this set to $A$,
it is time to constrain $map$ by a new clause in line~\ref{alg:aivc:addset},
which results in $map \leftarrow map \wedge (\neg a_2 \vee \neg a_5 \vee \neg a_7 \vee \neg a_8)$.

After these iterations, $map$ is still satisfiable, and the maximal model is
 $S = T \setminus \{ { \small \texttt{a1\_below}}, { \small \texttt{a2\_below}}\}$ in line~\ref{alg:aivc:assigns}.
In this case, $S$ is inadequate, so we update $map$ as
$map \leftarrow map \wedge (a_1 \vee a_2)$ (line~\ref{alg:aivc:iadd}). After adding this new clause to $map$,
all the subsets of $T \setminus \{ { \small \texttt{a1\_below}}, {\small \texttt{a2\_below}}\}$
will be blocked. The algorithm continues similar to the forth iteration leading to $S$ (in line~\ref{alg:aivc:assigns}) and $map$ (in line~\ref{alg:aivc:iadd}) to be as
 $S = T \setminus \{ {\small  \texttt{doi\_on}}\}$ and $map \leftarrow map \wedge a_7$.

Finally, after the sixth iteration, $map$ becomes \unsat and the algorithm terminates.
Note that $MIS$es and $IVC$s may be discovered in different orders from what explained here. The order by which sets are explored is
quite dependent on the maximal model returned in line~\ref{alg:aivc:maxsat} as well as the \mivc s returned in line~\ref{alg:aivc:getivc} because there could be several distinct maximal models (\mis es) and \mivc s. For this example with a $|T| = 8$ and $|\mathcal{P}(T)| = 2^8$, a brute force approach of power set exploration needs to look into  256 cases. However, the \aivcalg ~algorithm only explored 6 cases to cover the entire power set. %Depending on the order by which \mivc s and \mis es are discovered, the total cases to explore by the algorithm may change.
% All in all, it is fair to say the \aivcalg ~algorithm is  linear in the size of $T$.
%This issue could affect the performance of the algorithm as well.



