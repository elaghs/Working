\section{Conclusions \& Future Work}
\label{sec:conc}
The idea of extracting a minimal IVC for a given property and its applications was recently introduced in \cite{Ghass16}.  However, a single IVC often does not provide a complete picture of the traceability from a property to a model.  In this paper,
we have addressed the problem of extracting {\em all minimal} IVCs. We have shown
the correctness and completeness of our method and algorithm.  In addition, we have a substantial evaluation that shows that the practicality and efficiency of our technique.

Our method is inspired by a recent work in the domain of satisfiability analysis \cite{marco2016fast}. One interesting future direction is to devise similar MIVC enumeration algorithms based on other studies on MUSes extraction such as \cite{nadel2014accelerated}.  We are also looking into improving our implementation by using more  efficient methods for the \isadeq ~and \getivc ~modules used by our algorithm. Another interesting direction is to parallelize the enumeration process: it is certainly possible to ask for multiple distinct maximal models to be solved in parallel.
%, though this may result in unnecessary work performed by some of the parallel solvers.

We also plan to investigate additional applications of the idea.  When performing {\em compositional verification}, the All-IVCs technique may be able to determine {\em minimal component sets} within an architecture that can satisfy a given set of requirements, which may be helpful for design-space exploration and synthesis. Finally, we are interested in adapting the notion of (all) validity cores for \emph{bounded} model checking for quantifying how much of models have been explored by bounded analysis. 