
\documentclass[10pt, conference]{IEEEtran}



\usepackage{cite}

\ifCLASSINFOpdf
\usepackage[pdftex]{graphicx}
  % declare the path(s) where your graphic files are
  % \graphicspath{{../pdf/}{../jpeg/}}
  % and their extensions so you won't have to specify these with
  % every instance of \includegraphics
  % \DeclareGraphicsExtensions{.pdf,.jpeg,.png}
\else
  % or other class option (dvipsone, dvipdf, if not using dvips). graphicx
  % will default to the driver specified in the system graphics.cfg if no
  % driver is specified.
\usepackage[dvips]{graphicx}
  % declare the path(s) where your graphic files are
  % \graphicspath{{../eps/}}
  % and their extensions so you won't have to specify these with
  % every instance of \includegraphics
  % \DeclareGraphicsExtensions{.eps}
\fi
% graphicx was written by David Carlisle and Sebastian Rahtz. It is
% required if you want graphics, photos, etc. graphicx.sty is already
% installed on most LaTeX systems. The latest version and documentation can
% be obtained at:
% http://www.ctan.org/tex-archive/macros/latex/required/graphics/
% Another good source of documentation is "Using Imported Graphics in
% LaTeX2e" by Keith Reckdahl which can be found as epslatex.ps or
% epslatex.pdf at: http://www.ctan.org/tex-archive/info/
%
% latex, and pdflatex in dvi mode, support graphics in encapsulated
% postscript (.eps) format. pdflatex in pdf mode supports graphics
% in .pdf, .jpeg, .png and .mps (metapost) formats. Users should ensure
% that all non-photo figures use a vector format (.eps, .pdf, .mps) and
% not a bitmapped formats (.jpeg, .png). IEEE frowns on bitmapped formats
% which can result in "jaggedy"/blurry rendering of lines and letters as
% well as large increases in file sizes.
%
% You can find documentation about the pdfTeX application at:
% http://www.tug.org/applications/pdftex





% *** MATH PACKAGES ***
%
\usepackage[cmex10]{amsmath}
\usepackage{amssymb}
\usepackage{stmaryrd}
\usepackage{amsthm}
\usepackage{algorithmic}
\usepackage{array}
\usepackage{mdwmath}
\usepackage{mdwtab}
\usepackage{eqparbox}
\usepackage[tight,normalsize]{subfigure}
\usepackage[font=normalsize]{caption}
\usepackage{tabularx,colortbl}
\usepackage[dvipsnames]{xcolor}
\usepackage{flushend}
\usepackage{cite}
\usepackage{amsmath}
%\usepackage[font=footnotesize]{subfig}
%\usepackage[caption=false,font=footnotesize]{subfig}
\usepackage{fixltx2e}
\usepackage[ruled, vlined, linesnumbered]{algorithm2e}
\usepackage{stfloats}
\usepackage{url}
\usepackage{xspace}

\hyphenation{op-tical net-works semi-conduc-tor}
\newcommand{\mkeyword}[1]{\mbox{\texttt{#1}}}
\DeclareMathOperator{\kuop}{uop}
\DeclareMathOperator{\kbop}{bop}
\DeclareMathOperator{\kite}{ite}
\DeclareMathOperator{\kpre}{pre}
\DeclareMathOperator{\dom}{dom}
\DeclareMathOperator{\ktrue}{true}
\DeclareMathOperator{\kfalse}{false}
\DeclareMathOperator{\kselect}{select}
\DeclareMathOperator{\ran}{range}
\newcommand{\lbb}{[\![}
\newcommand{\rbb}{]\!]}
\newcommand{\expr}{\phi}
\newcommand{\exprS}{\Phi}

\begin{document}

\definecolor{gold}{rgb}{0.90,.66,0}
\definecolor{dgreen}{rgb}{0,0.6,0}
\newcommand{\mike}[1]{\textcolor{red}{#1}}
\newcommand{\fixed}[1]{\textcolor{purple}{#1}}
\newcommand{\andrew}[1]{\textcolor{green}{#1}}
\newcommand{\ela}[1]{\textcolor{blue}{#1}}
\newcommand{\stateequiv}{\equiv_{s}}
\newcommand{\traceequiv}{\equiv_{\sigma}}
\newcommand{\ta}{\text{TA}}
\newcommand{\cta}{\text{TA$_{C}$}}
\newcommand{\tta}{\text{TA$_{T}$}}

\newcommand{\bfalg}{{IVC\_BF}\xspace}
\newcommand{\ucalg}{{IVC\_UC}\xspace}
\newcommand{\ucbfalg}{{IVC\_UCBF}\xspace}
\newcommand{\mustalg}{{IVC\_MUST}\xspace}

\newtheorem{definition}{Definition}
\newtheorem{lemma}{Lemma}
\newtheorem{theorem}{Theorem}
\newtheorem{coroll}{Corollary}
%\newdef{lemma}{Lemma}
%\newdef{definition}{Definition}
%\newdef{theorem}{Theorem}
%\newdef{note}{Note}
%
% paper title
% can use linebreaks \\ within to get better formatting as desired
\title{Inductive Validity Cores for Formal Verification}


% author names and affiliations
% use a multiple column layout for up to two different
% affiliations

\author{\IEEEauthorblockN{Elaheh Ghassabani and Michael W. Whalen}

\IEEEauthorblockA{Department of Computer Science and Engineering\\
University of Minnesota\\
Minneapolis, MN, USA\\
Email: ghassaba, whalen@cs.umn.edu}
\and
\IEEEauthorblockN{Andrew Gacek}
\IEEEauthorblockA{Rockwell Collins\\
Advanced Technology Center\\
Cedar Rapids, IA, USA\\
Email: andrew.gacek@rockwellcollins.com}
}

% conference papers do not typically use \thanks and this command
% is locked out in conference mode. If really needed, such as for
% the acknowledgment of grants, issue a \IEEEoverridecommandlockouts
% after \documentclass

% for over three affiliations, or if they all won't fit within the width
% of the page, use this alternative format:
%
%\author{\IEEEauthorblockN{Michael Shell\IEEEauthorrefmark{1},
%Homer Simpson\IEEEauthorrefmark{2},
%James Kirk\IEEEauthorrefmark{3},
%Montgomery Scott\IEEEauthorrefmark{3} and
%Eldon Tyrell\IEEEauthorrefmark{4}}
%\IEEEauthorblockA{\IEEEauthorrefmark{1}School of Electrical and Computer Engineering\\
%Georgia Institute of Technology,
%Atlanta, Georgia 30332--0250\\ Email: see http://www.michaelshell.org/contact.html}
%\IEEEauthorblockA{\IEEEauthorrefmark{2}Twentieth Century Fox, Springfield, USA\\
%Email: homer@thesimpsons.com}
%\IEEEauthorblockA{\IEEEauthorrefmark{3}Starfleet Academy, San Francisco, California 96678-2391\\
%Telephone: (800) 555--1212, Fax: (888) 555--1212}
%\IEEEauthorblockA{\IEEEauthorrefmark{4}Tyrell Inc., 123 Replicant Street, Los Angeles, California 90210--4321}}




% use for special paper notices
%\IEEEspecialpapernotice{(Invited Paper)}




% make the title area
\maketitle


\begin{abstract}
This paper introduces the concept of Inductive Validity Core (IVC), which provides an explanation of an inductive proof, such as those constructed by modern model checking algorithms using k-induction and Property-Directed Reachability (PDR). This notion has been implemented and evaluated in the JKind model checker. IVCs provide a lot of useful applications in formal verification which is discussed in this paper.
\end{abstract}

\begin{IEEEkeywords}
  formal verification; SAT-based model checking; IVC; inductive proofs;
\end{IEEEkeywords}

\IEEEpeerreviewmaketitle

\section{Introduction}
\label{sec:intro}
Most modern sequential model checking techniques for safety properties, including IC3/PDR~\cite{Een2011:PDR} and $k$-induction~\cite{SheeranSS00}, use a form of induction to establish proof.  These techniques are very powerful, and can often reason successfully over very larger or even infinite state spaces.  The proofs provided by these tools can provide rigorous evidence that a software or hardware system works as intended.

On the other hand, there many situations in which properties can be proved, but systems still will not perform as intended.  Issues such as vacuity~\cite{Kupferman03:Vacuity}, incorrect environmental assumptions~\cite{Whalen07:FMICS}, and errors either in English language requirements or formalization~\cite{Pike06:axioms} can all lead to failures of ``proved'' systems.  Thus, even if proofs are established, one must approach verification with skepticism.

Recently, Ghassabani et al.~\cite{Ghass16} introduced the idea of {\em Inductive Validity Cores} (IVCs) in order to provide additional information with proofs. IVCs offer proof explanation as to why a property is satisfied by a model in a formal and human-understandable way.  The idea lifts UNSAT cores~\cite{zhang2003extracting}
to the level of sequential model checking algorithms using induction.  Informally, if a model is viewed as a conjunction of constraints,
a minimal IVC (MIVC) is a set of constraints that is sufficient to construct a proof such that if any constraint is removed, the property is no longer valid.
IVCs and MIVCs can be used for several purposes, including performing traceability between specification and design elements, assessing model coverage, and explanation of unsatisfiable test obligations when using model checkers for test case generation. Ghassabani et al.~\cite{Ghass16} presented two algorithms: \ucalg, which computes an approximately minimal IVC that is computationally inexpensive, and \ucbfalg,
an algorithm that produces a
MIVC but is considerably more expensive to compute.
%
The IVC idea shares many similarities with approaches for computing minimal
invariant sets for inductive proofs (such as is performed for inductive proof certificates~\cite{piskac2016, ivrii2014small}), and in fact the \ucalg\ algorithm performs a minimal lemma set computation.  However, there is a substantive difference: to find a guaranteed minimal set of constraints, it is usually necessary to find new proofs involving {\em new lemmas} not used in the original proof, which accounts for the expense of the \ucbfalg\ algorithm.

It is often the case that there are multiple MIVCs for a given property.  In this case, the algorithms from~\cite{Ghass16} give, at best, an
incomplete picture of the traceability information associated with the proof.  Depending on the model and property to be analyzed, there is often substantial diversity between the IVCs used for proof, and there can be substantial difference in the size between the {\em minimal} IVC returned by the \ucbfalg\ algorithm and a {\em minimum} IVC, which is the (not necessarily unique) smallest MIVC.
 If {\em all} MIVCs can be found, then several additional analyses can be performed:
\begin{itemize}
    \item Coverage Analysis: MIVCs can be used to define coverage metrics by examining the percentage of model elements required for a proof.  However, since MIVCs are not unique, there are multiple, equally legitimate coverage scores possible.  Having \emph{all} MIVCs allows one to define additional metrics: coverage of MAY elements, coverage of MUST elements, as well as policies for the existing MIVC metric: e.g., choose the smallest MIVC. %\ela{I'm not sure if introducing MAY/MUST would make sense to the readers }
    \item Optimizing Logic Synthesis:  synthesis tools can benefit from MIVCs in the process of transforming an abstract behavior into a design implementation. A practical way of calculating MIVCs allows to find a minimum set of design elements (optimal implementation) for a certain behavior. Such optimizations can be performed at the different levels of synthesis.
    \item Impact Analysis: Given all MIVCs, it is possible to determine which requirements may be falsified by changes to the model.  This analysis allows for selective regression verification of tests and proofs: if there are alternate proof paths that do not require the modified portions of the model, then the requirement does not need to be re-verified.
    \item Robustness Analysis: As proposed by Murugesan et. al in~\cite{Murugesan16:renext}, it is possible partition the model elements into MUST and MAY sets based on whether they are in every MIVC or only some MIVCs, respectively.  This may allow insight into the relative importance of different model elements for property.  For example, if the MUST set is empty, then the requirement has been implemented in multiple ways, such as would be expected in a fault-tolerant system.  Moreover, examining the diversity of all MIVCs could lead to changes in how traceability
        ~\cite{COEST,cleland2007best}
     %~\cite{COEST,hayes2003improving,cleland2007best}
        is performed and managed in critical systems.
\end{itemize}
%\noindent In addition, the Requirements Engineering community is keenly interested in approaches to manage requirements traceability.  In most cases, it is assumed that there is a single ``golden'' set of trace links that describes how requirements are implemented in software~\cite{COEST,hayes2003improving,cleland2007best}.  However, if there are multiple MIVCs, then it is possible that there are several equally valid sets of trace links.  Examining the diversity of all MIVCs could lead to changes in how traceability is performed and managed in critical systems.

As far as commercial tools are concerned, we have found some of them that use the term \emph{proof-core} ~\cite{hanna2015formal, jasper_gold}, which sounds similar to the idea of a \emph{single} MIVC. However,
to the best of our knowledge, none of them offer the calculation of \emph{all} proof-cores.
Moreover, solutions provided by these tools are quite underspecified:
no formal description of the proof-core notion or algorithms are provided. In addition, no implementations or experimental results are provided, so it is not possible to compare their approach with IVCs.

In this paper, we propose a new method for computing \emph{all minimal} IVCs. In  recent  years,  a  number  of  efficient
algorithms  for  extracting  all MUSes  have  been proposed \cite{bacchus2015using, belov2012muser2, belov2013core, belov2012towards, nadel2014accelerated, liffiton2005max}.  In this paper, we adapt the recent work by Liffiton et al. \cite{marco2016fast} from the generation of MUSes from UNSAT-cores to all IVCs for inductive model checking.  This requires changing the underlying mechanisms that are used to construct candidate solutions and also changing the structure of the proof of correctness.  In addition, in our proof, we demonstrate that the approach terminates with all minimal IVCs even if the witness generator only generates approximately minimal IVCs (utilizing the ``fast'' \ucalg\ algorithm from~\cite{Ghass16}).  In our empirical results, this allows our algorithm to be quite efficient to the extent that in many cases, the cost of extracting all minimal IVCs is similar to the cost of finding a single guaranteed-minimal IVC, and on average is approximately 1.6x the cost of determining a single minimal IVC.
The contributions of the work are therefore as follows:
\begin{itemize}
    \item An algorithm for computing all minimal IVCs.
    \item A proof of correctness and completeness of the algorithm.
    \item An evaluation of the algorithm for performance and diversity of result sets against a benchmark suite.
    \item An industrial case study with over 10K design elements that demonstrates the practicality and usefulness of our technique.
\end{itemize}

%\ela{I think we need to make it clear that IVCs are different from MUSes, proof-certificates or minimal invariants, abstraction, slicing. Currently, the introduction doesn't say anything about these. You had an idea on having a table... Perhaps you want to include a discussion section?\\ Or, Maybe we could expand the introduction with these things and make it more motivating}

%\ela{Also, I think the contributions don't stand out. finding \emph{all} \textbf{minimal} IVCs itself is two contribution. I think minimality is important. Maybe we should stress on it a little bit more}

The rest of the paper is organized as follows.
Section \ref{sec:example} introduces a running example used to illustrate concepts and our method.
Section \ref{sec:background} covers the formal preliminaries for the approach.
In Section \ref{sec:allivcs}, we present our method for enumerating all minimal IVCs,
which is illustrated in
Section \ref{sec:illust}. In Section \ref{sec:impl}, we talk about implementation and evaluation of our method. Section \ref{sec:qfc} presents an industrial case study. Finally, Section \ref{sec:conc} mentions conclusions and future work. 
\newcommand{\minproofcov}{\text{\sc MinProof-Cov}}

%\clearpage
\section{Proof-Based Metrics}
\label{sec:method}

We propose a new approach for measuring property completeness based on proof rather than mutation.  We first define notation, then describe different possible metrics given a set of {\em minimal proofs}.%\footnote{Section~\ref{sec:impl} describes how these proofs are discovered in practice.}
%\subsection{Coverage and Minimal Proofs}
%Alternatively, we can consider using the proofs themselves as a mechanism for determining adequacy of requirements.

\begin{definition} {\emph{IVC coverage (\ivccov):}} \\
\label{def:coverage-justi}
Given $S \in AIVC(P)$, $T_i$ is covered by $P$ via $S$ \emph{iff} $T_i \in S$.
%Given $S \in AIVC(P)$, $T_i \in T$ is covered by $P$ \emph{iff} $T_i \in S$,
%denoted by $T_i \in \ivccov (P, S)$
\end{definition}

%For the sake of simplicity, we refer to the coverage function
%formalized in Definition \ref{def:coverage-justi} as \ivccov\.
%
We call Definition \ref{def:coverage-justi} a \emph{proof-preserving} metric because, with a set of the model elements marked as covered by \ivccov, $P$ is provable.  Other notions, as will be discussed in Section~\ref{subsec:method-disc}, may yield subsets of the model that are insufficient to reconstruct the proof of the property.
%\footnote{\noindent ~Throughout the paper, when a coverage metric is justifiable, like \ivccov, we say that it preserves provability of the property.}
%Thus, the coverage score for \ivccov\ is often higher than the score for \nondetcov.
The coverage score for \ivccov\ can be calculated with: $$\frac{|S|}{|T|}$$
%Note that because minimal proofs are not unique, there are several possible coverage scores.
Because $P$ may have multiple \mivc s,  \ivccov\ metric can lead to various scores that belong to the following set:
\[
\{~\frac{ |S|}{|T|}~|~S \in AIVC(P)~\}
\]

\noindent Note that if an \mivc ~contains all model elements (i.e., the model is {\em completely covered}), then there is only one possible \mivc , so in this case there is no diversity of scores.

%the model is {\em completely covered}, on the other hand, then there is only one possible minimal set: the set of all elements.

Given {\em all} proofs of a particular property, it is possible to define additional, complementary coverage notions.  To do so, we use the following categorization of the model elements based on \mivc ~and $AIVC$ relations for $P$:
\begin{itemize}
 % \item  \textbf{$MUST$} contains model elements in all the \mivc s of $P$.
%      %$$ MUST_x = \{\forall i (S_xi \in \Sigma_x) \mid \bigcap S_xi \}$$
%      \[
%      MUST (P) = \bigcap AIVC(P)
%      \]
%
%  \item  \textbf{$MAY$} includes model elements that are used in some, but not all, \mivc s.
%      \[
%      MAY(P) = (\bigcup AIVC (P)) \setminus MUST (P)
%      \]
%
%  \item  \textbf{$IRR$} specifies model elements that are not in any of the possible \mivc s of $P$.
%  $$IRR(P) = T \setminus (\bigcup AIVC (P))$$
\item $MUST (P) = \bigcap AIVC(P)$
\item $MAY(P) = (\bigcup AIVC (P)) \setminus MUST (P)$
\item $IRR(P) = T \setminus (\bigcup AIVC (P))$
\end{itemize}

\noindent This categorization helps to identify the role and relevance of each design element in satisfying a property. Function $MUST$ specifies the parts of the model absolutely necessary for the property satisfaction.  Any change to these parts will affect provability of the property. On the other hand, any single element in $MAY (P)$, may be modified without affecting satisfaction of $P$(though modifying multiple elements may require re-proof). The $IRR$ denotes model elements that are irrelevant to the validity of $P$ \cite{Murugesan16:renext}.

Using the notions of $MAY$ and $MUST$, we can introduce additional coverage metrics.
%Since the primary goal of
% this paper has been to provide a complementary coverage notion in
%  formal verification, it is worth exploring other possible notions based on the idea of provability and $AIVC$, which is beneficial, as with testing, because if a coverage notion is an over-approximation, when the coverage
% is high, it does not necessarily mean the quality of
% the specification (or test suite) is high, or when it is an under-approximation, a low coverage score does not always mean the specification is of poor quality.

\begin{definition} {\emph{(\maycov):}}
  \label{def:comp-1}
 $T_i \in T$ is covered by $P$ \emph{iff} $T_i \in \maycov (P)$, where
   $\maycov (P) = \{T_i ~|~ \exists S \in AIVC(P)~.~T_i \in S \}$.
\end{definition}

\begin{definition} {\emph{(\mustcov):}}
  \label{def:mustcov}
 $T_i \in T$ is covered by $P$ \emph{iff} $T_i \in \mustcov (P)$, where
   $\mustcov (P) = \{T_i ~|~ \forall S \in AIVC(P)~.~T_i \in S \}$.
\end{definition}

The $\maycov$ notion aims to deal with the fact that a property $P$ may have
several distinct \mivc s. In such cases, \ivccov\ only looks at an arbitrary \mivc\
that may contain a subset of $MAY(P)$, which means, depending on
which \mivc\ it considers, every time it may report a different part of $MAY(P)$
as uncovered. However, \maycov\ resolves this issue reporting the entire set of $MAY(P)$ as covered, which also leads to higher coverage scores.  \mustcov\ takes the opposite view, considering a model element as covered only if it affects all the proofs of $P$.


It is still possible to build more relaxed coverage metrics in which coverage
is captured by looking at individual properties, rather than their conjunction.
%for example, in the definition of \ivccov , it is wise to look at $P$ as
%the conjunction of all properties. However,
We can, for example, describe a metric in which any element used by an \mivc ~for any property is considered covered.
%with this view,
%elements around IVCs that do not have common \emph{must}
%elements with others will be treated as uncovered while they are at least covered by one
% IVC of an individual property in the specification.
%
The next definition, \allcov, formalizes this notion.
\begin{definition} {\emph{(\allcov):}}
  \label{def:comp-2}
     Given a set of properties $\Delta$ over $T$, $T_i \in T$ is covered
   \emph{iff} $T_i \in \allcov (T)$, where
   $\allcov (T) = \{T_i ~|~ \exists P \in \Delta ,~ S \in AIVC(P).~T_i \in S \}$.
\end{definition}

%Considering $MAY$ and $MUST$ categorization, we can formalize another
%coverage metric that takes into account the \emph{must} set;
%however, such a metric is the same as \nondetcov\ as we discuss in the next sub-section.

\subsection{Discussion}
\label{subsec:method-disc}


Based on the categorization of elements, we will state some relationships about \mivc s so to compare different proof-based metrics proposed earlier.

\begin{lemma}
  \label{lem:must-not-enough}
  If $MAY(P) \neq \varnothing$, then $P$ is not provable by $MUST(P)$.
\end{lemma}
\begin{proof}
  $MAY(P) \neq \varnothing \Rightarrow  \exists T_i \in MAY(P).$
$T_i \in \bigcup AIVC(P) \wedge T_i \notin MUST(P)$,
which implies $\exists S \in AIVC(P).~ T_i \in S$.
Considering the fact that $S$ is minimal and
$MUST(P) \subset S$ (since $T_i \in S \wedge T_i \notin MUST(P)$),
 $\nexists S' \subset S.~ (I,S') \vdash P$,  which means $(I, MUST(P)) \nvdash P$.
\end{proof}
\vspace{2mm}

%\begin{lemma}
%    \label{lem:must-mustcov}
%    $T_i \in MUST(P) \Leftrightarrow T_i \in \mustcov(P)$
%\end{lemma}
%\begin{proof}
%Immediate from the definition of $MUST$ and \mustcov.
%\end{proof}

Now we focus on the relationship between non-deterministic mutation-based coverage and proof-based metrics. In Chockler et. al. \cite{chockler2010coverage}, each mutant design changes the type of a single node to \inputnode\ (see Section \ref{sec:background}).
Given a suitable encoding of the netlist, assigning a ``fresh'' input is an isomorphic operation to simply removing a $T_i$ from $T$. The mapping is as follows: the net-list becomes a conjunction
of equations, where each vertex becomes a variable $v_i \in U$, and where each non-input vertex becomes an assignment equation $T_i \in T$.
For example, given an AND-vertex $v_i$ with three input edges from other vertexes $\{v_a, v_b, v_c\}$, we would define an equation $T_i \in T$ of the form $(v_i = (v_a \wedge v_b \wedge v_c))$.
%
%As the variable is no longer constrained by a defining equation, it is effectively an %input.

Given this encoding, we can reframe the non-deterministic coverage proposed in \cite{chockler2010coverage} as follows:

\begin{definition} {\emph{Nondeterministic coverage (alternate specification) (\nondetcovalt) ~\cite{chockler2010coverage}.} }
\label{def:non-det-2}
$T_i \in T$ is covered by property $P$ \emph{iff} $T_i \in \nondetcovalt (P)$, where
$\nondetcovalt (P) = \{T_i~|~ (I, T) \vdash P \wedge (I, T \setminus \{T_i\}) \nvdash P\}$.
\end{definition}


%\begin{definition} {\emph{Nondeterministic coverage alternate definition (\nondetcovalt) ~\cite{chockler2010coverage}.} }
%\label{def:non-det-2}
%$T_i \in T$ is covered by property $P$ \emph{iff} $T_i \in \nondetcovalt (P)$, where
%$\nondetcovalt (P) = \{T_i~|~ (I, T) \vdash P \wedge (I, T \setminus \{T_i\}) \nvdash P\}$.
%\end{definition}
%
%\begin{lemma}
%    \label{lem:nondet-nondetaltcov}
%    $\nondetcov(P) = \nondetcovalt(P)$
%\end{lemma}
%\begin{proof}
%\mike{obvious?} \ela{   not so sure if obvious}
%\end{proof}

\noindent Given this definition, it becomes straightforward to define some additional properties.

\begin{lemma}
  \label{lem:must-coverage}
$T_i \in \nondetcovalt (P) \Leftrightarrow T_i \in \mustcov(P)$.
\end{lemma}
\begin{proof}
$T_i \in \nondetcovalt (P)$ means that $(I, T \setminus \{ T_i \}) \nvdash P$ then
%$T_i$ is necessary to prove $P$,  which means
$\forall S \subset T .~ T_i \notin S \Rightarrow (I, S) \nvdash P$.
Therefore, since $(I, T) \vdash P$, $T_i \in \bigcap AIVC(P)$, which means  $T_i \in MUST(P)$.
On the other hand, let $T_i \in MUST(P)$; then $\forall S \in AIVC(P).~ T_i \in S$.
By definition, any proof of $P$ is a superset of some minimal IVC in $AIVC(P)$.
Thus, any subset $S$ of $T$ leading to proof contains $T_i$.
Therefore, $T \setminus \{ T_i \}$ does not lead to a proof.
%On the other hand, by definition, $MUST(P)$ is the intersection of all IVCs.
%From the definition of $MUST$, removing a $T_i \in MUST(P)$ from $T$
%results in $ \bigcap AIVC(P) \setminus \{ T_i \} $.
%And since all IVCs in $AIVC$ are \emph{minimal} removing an element from all possible IVCs makes
% $P$ unprovable by every single of them:
% $\forall S \in AIVC(P),~ T_i \in \bigcap AIVC(P).~ (I, S \setminus \{ T_i \}) \nvdash P$. And, we know $S \subseteq T$, so $S \setminus \{ T_i \} \subseteq T \setminus \{ T_i \}$, which means the reachable states of
% $(I, T \setminus \{ T_i \})$ are a subset of the reachable states from
%   $(I, S \setminus \{ T_i \})$. Therefore,
%   $ (I, S \setminus \{ T_i \}) \nvdash P \Rightarrow (I, T \setminus \{ T_i \}) \nvdash P$.
\end{proof}
\vspace{2mm}

In light of Lemma \ref{lem:must-coverage}, the \nondetcovalt\ coverage score of specification $P$ can be also calculated by
$$\frac{|MUST(P)|}{|T|}$$
%Therefore, for set of properties $\Delta$, the coverage score is computed by $$\frac{|MUST(\Pi)|}{|T|},\quad  \Pi= \bigwedge_{i} {P_i \in \Delta}$$


%\mike{after all metrics presented, contrast them on the example.  Introduce the properties HERE and then discuss the coverage sets}
%
%\mike{Then, you can talk about justification, etc.}
\begin{coroll}
\label{cor:must-not-provable}
\nondetcovalt\ is not proof-preserving.
\end{coroll}
\begin{proof}
Immediate from Lemma \ref{lem:must-not-enough} and Lemma \ref{lem:must-coverage}
\end{proof}
\vspace{2mm}
\begin{coroll}
\label{cor:ivc-provable}
\ivccov\ is proof-preserving.
\end{coroll}
\begin{proof}
Immediate from Definition~\ref{def:minimal-ivc} and Definition \ref{def:coverage-justi}
\end{proof}
\vspace{2mm}

%It should be pointed out that \ivccov\ is accurate meaning that it does not result in false positives. In other words, since IVCs are \emph{minimal}, \ivccov\ does not mark
%any \emph{actual} uncovered element as covered.

To conclude this section, we should mention that one can define many more proof-based coverage metrics based on the $\mivc /AIVC$ idea. Metrics that make use of the $AIVC$ relation are computationally more expensive to compute than \ivccov\ although they might be easier to satisfy (i.e., result in higher coverage scores).

%\ela{please read the following paragraph and improve it. I've been trying to justify why we only have implementation for \ivccov \\}
The proposed coverage metrics can be ranked in terms of their scores as follows:
$$\nondetcovalt\ \leq \ivccov\ \leq \maycov\ \leq \allcov$$
\ivccov\ and \nondetcovalt\ are equivalent when all elements within the model are covered: if all model elements are MUST elements, then there can only be one \mivc , and this \mivc ~uses all of the model elements.   In the implementation and experiments, we will focus on the \ivccov\ and \nondetcovalt\ metrics.  Both metrics are fairly rigorous and can be computed reasonably efficiently.  The equivalence of \mustcov\ and \nondetcovalt\ allows us to compare our algorithms against state-of-the-art mutation based coverage.


%However, we will show \ivccov\ is a lot cheaper to compute
%and in terms of rigor, it is neither too hard (like \nondetcovalt)
%nor too easy (like \allcov) to satisfy.
%Let $\Delta$ be a set of properties over $T$. When we define a new property $P = \bigwedge p_i$, where $p_i \in \Delta$, possible \mivc s of $P$ are the parts of the models that affect the proof of every property $p_i \in \Delta$. In this way, we are able to identify portions of the model that are constrained by some properties without having to calculate the $AIVC$ relation for individual properties, which will significantly reduce computational cost.
%\footnote{Next section will illustrate this idea.} For these reasons, in Section \ref{sec:impl},
%we consider computation of \ivccov\
%and in order to benchmark it against the existing methods,
%we also provide implementation of \mustcov\ which is the same as \nondetcovalt .

%Based on our preliminary evaluation, we believe that metrics based on
%$AIVC$ relation (like \maycov\ and \allcov) are approximately as computationally expensive as \nondetcov.\footnote{\noindent ~The reason is that \nondetcov\ computes the must set which is also based on $AIVC$ relation. However, in terms of preserving provability, a set of design elements marked as covered by \allcov\ and \maycov\ are
%sufficient to reconstruct the proof of the properties.}
%In the following sections, we first illustrate how the different metrics measure coverage of our ASW example with some sample requirements, and then perform a larger experiment with the \nondetcov\ and \ivccov\ metrics.

%So, in order to examine the proof-based metrics, Section \ref{sec:impl} considers the implementation of two major notions: \nondetcov\ and
% \ivccov ; because \nondetcov\ is based on a recent work in the literature,
% and among all the other proposed notions, \ivccov\ is the
% one that does not take into account $AIVC$.
 %Besides, in terms of coverage score, \ivccov\ is not too easy (or hard) to satisfy.  

\section{Implementation}
\label{sec:impl}

The algorithm for efficiently computing IVCs can be found in a forthcoming FSE paper~\cite{Ghass16} and is implemented in the JKind \cite{jkind}, which is an infinite-state model checker for safety properties using multiple cooperative engines in parallel (such as k-induction and PDR). JKind accepts
Lustre programs written over the theory of linear integer and real
arithmetic. In the back-end, JKind uses an SMT solver such as
Z3, Yices, MathSAT, or SMTInterpol.
JKind works on multiple properties simultaneously. When a
property is proven and IVC generation is enabled, an additional
parallel engine executes the IVC generation algorithm to compute a minimal
IVC. We demonstrated the efficiency and precision of the approach using a set of Lustre models developed
as a benchmark suite for~\cite{Hagen08:FMCAD}, augmented with additional models from industrial projects (~\cite{QFCS15:backes,hilt2013}). The results show that our algorithm for computing IVCs is quite efficient even for industrial models with an average overhead of ~10\%. 

\section{Experiment}
\label{sec:experiment}

%\mike{What do we want to call our efficient algorithm: IVC?}

We would like to investigate both the {\em efficiency} and {\em
  minimality} of our three algorithms: the n{\"a}ive brute-force
algorithm (\bfalg), the UNSAT core-based algorithm (\ucalg), and the
combined UNSAT core followed by brute-force minimization algorithm
(\ucbfalg). Efficiency is computed in terms of wall-clock time: how
much overhead does the IVC algorithm introduce? Minimality is
determined by the size of the IVC: cores with a smaller number of
variables are preferred to cores with a larger number of variables.
Finally, we are interested in the {\em diversity} of solutions: how
often do different tools/algorithms generate different minimal IVCs?

The use of JKind allows additional dimensions to our investigation: it supports two different inductive algorithms: $k$-induction and PDR, and a ``fastest'' mode, that runs both algorithms in parallel.  In addition, JKind supports multiple back-end SMT solvers including Z3~\cite{DeMoura08:z3}, Yices~\cite{Dutertre06:yices}, MathSAT~\cite{Cimatti2013:MathSAT}, and SMTInterpol~\cite{Christ2012:SMTInterpol}.  We would like to determine whether the choice of inductive algorithm affects the size of the IVC, whether different solvers are more or less efficient at producing IVCs, and whether running different solvers/algorithms leads to {\em diversity} of IVC solutions.

Therefore, we investigate the following research questions:
\begin{itemize}
    \item \textbf{RQ1:} How expensive is it to compute inductive validity cores using the \bfalg, \ucalg, and \ucbfalg algorithms?
    \item \textbf{RQ2:} How close to minimal are the support sets computed by \ucalg as opposed to the (guaranteed minimal) \ucbfalg?  How do the sizes of IVCs compare to static slices of the model?
    \item \textbf{RQ3:} How much {\em diversity} exists in the solutions produced by different solver/induction algorithm configurations?
\end{itemize}

\subsection{Experimental Setup}
In this study, we started from a suite of 700 Lustre models developed
as a benchmark suite for~\cite{Hagen08:FMCAD}. We augmented this suite
with 82 additional models from recent verification projects including
avionics and medical devices~\cite{QFCS15:backes,hilt2013}. Most of
the benchmark models from~\cite{Hagen08:FMCAD} are small (10k or less,
with 6-40 equations) and contain a range of hardware benchmarks and
software problems involving counters. The additional models are much
larger: around 80k with over 300 equations. We added the new
benchmarks to better check the scalability for the tools, especially
with respect to the brute force algorithm.
%
%\mike{MORE HERE...stats on size, reasons for add'l models.}
Each benchmark model has a single property to analyze.  For our purposes, we are only interested in models with a {\em valid} property (though it is perhaps worth noting that there is no additional computation---and thus no overhead---using the JKind IVC options for {\em invalid} properties).  In our benchmark set, 295 models yield counterexamples, and 10 additional models are neither provable nor yield counterexamples in our test configuration (see next paragraph for configuration information).  The benchmark suite therefore contains 476 models with valid properties, which we use as our test subjects.

For each test model, we computed \ucalg in 12+1 configurations: the
twelve configurations were the cross product of all solvers \{Z3,
Yices, MathSAT, SMTInterpol\} and inductive algorithms
\{$k$-induction, PDR, fastest\}, and the remaining (+1) configuration
was an instance of \bfalg run on Yices, which is the default solver in
JKind. In addition, for each of the 12 configurations, we ran an
instance of JKind without IVC to examine overhead. The experiments
were run on an Intel(R) i5-2430M, 2.40GHz, 4GB memory machine, with a
1 hour timeout for each analysis on any model. The data gathered for
each configuration of each model included the time required to check
the model without IVC, with IVC, and also the set of elements in the
computed IVC.\footnote{The benchmarks, all raw experimental results,
  and computed data are available on \cite{expr}.}

Note that not all analysis problems were solvable with all algorithms: for all solvers, $k$-induction (without IVC) was unable to solve 172 of the examples.  When comparing minimality of different solving algorithms, we only considered cases where both algorithms provided a solution (as will be discussed in more detail in Section~\ref{sec:minimality}).

\iffalse
\begin{itemize}
    \item an algorithm to compute a truly minimal set of support, i.e. \texttt{JSupport}.
    \item given a LUS model, a static crawler which automatically marks all equations of a node in the initial support set of a property.
    \item some trackers that measure the verification time with/ without support computation.
   % \item some minor changes in the XML writers.
\end{itemize}

\mike{My thoughts on this section: mostly, it needs more structure: more information on the properties of the models: size, provenance, etc., a broken out subsection on the description of the experimental setup, etc}

\mike{I think we want to split out the results in another top-level section}

Experiment:
\begin{itemize}
    \item (Overview) describe research questions and goals.
    \item Experimental setup: tell me about the models: how many, how big are they?  Then, tell me about the experiment: the tool configurations, the machine used for test.
    \item Data generation: Describe what you measured for each model analysis.
\end{itemize}
\fi


%%  LocalWords:  minimality ive UNSAT IVC Minimality IVCs PDR Yices
%%  LocalWords:  MathSAT SMTInterpol RQ JSupport


\section{Conclusions \& Future Work}
\label{sec:conclusion}

In this paper, we have defined a novel coverage notion for formal verification using
the IVC concept, a useful measure in relation to
a valid safety property for inductive model checking. We have shown that our method
 is computationally efficient while 
 being accurate about the covered parts of a given design. 
 We have referred to this accuracy as preserving provability, which means 
 that a set of elements considered covered by our algorithm is sufficient 
 to establish the validity proof for every requirement in the set of specifications.
 
 We have implemented
our algorithm as part of the open source model checker JKind. Using our approach, measuring coverage is quite possible as we have shown in our experiments.
 We also benchmarked our implementation and compared it with other techniques in the literature. 
 The experiments show that the computation imposes a small overhead to the verification process. We have described how the justifiable notion of coverage proposed in this paper can be used as a
means of quantifying requirements completeness.
 
 In addition, based on the idea of multiple support sets for a specification, we 
 have introduced and discussed some other complementary coverage notions in the context of formal verification. Finally, we are in the process of developing some efficient algorithms for exploring the space of IVCs, e.g., finding a
minimum, rather than minimal support set, or finding all support sets. Having such algorithms makes the utilization of other proposed coverage practical.

%\section*{Acknowledgment}
%
%
%The authors would like to thank...
%more thanks here


\bibliographystyle{IEEEtran}
\bibliography{biblio}
\end{document}


