\section{Method}
\label{sec:allivcs}
 
\newcommand{\ucalg}{IVC\_UC\xspace}  

\newcommand{\mink}{\textsc{MinimizeK}\xspace}
\newcommand{\reduceinv}{\textsc{ReduceInvariants}\xspace}
\newcommand{\minivc}{\textsc{MinimizeIvc}\xspace}

\newcommand{\checksat}{\textsc{CheckSat}\xspace}
\newcommand{\unsatcore}{\textsc{UnsatCore}\xspace}
\newcommand{\unsat}{\textsc{UNSAT}\xspace}
\newcommand{\sat}{\textsc{SAT}\xspace}

As mentioned, the contribution of this paper is to provide an efficient and complete method for calculating $AIVC$ of a property. To this end, we first introduce several additional notions and definitions, most of which are inspired by the MUS enumeration technique proposed in \cite{marco2016fast}.

\begin{definition}{\emph{Maximal Irrelevant Subset (MIS):}}
  \label{def:mis}
  $S \subset T$ for $(I, T) \vdash P$ is a Maximal Irrelevant Subset (MIS) iff 
  $(I, S) \nvdash P$ and $\forall T_i \in T\setminus S.~ (I, S\cup{T_i}) \vdash P$.  
\end{definition}

\begin{definition}{\emph{Minimal Correction Set (MCS):}}
  \label{def:mcs}
  $S \subset T$ for $(I, T) \vdash P$ is a Minimal Correction Set (MCS) iff
  $(I, T \setminus S) \nvdash P$ and $\forall T_i \in S.~ (I, (T \setminus S)\cup \{T_i\}) \vdash P$.
\end{definition}

%It should be mentioned that minimality and maximality are about minimum or maximum cardinality subsets. 
Note that $MCS$ is more of syntactic sugar that specifies sets that can also be specified by $MSS$; i.e. for any $MIS$ of $T$, there is a corresponding $MCS$ such that adding any element of that $MCS$ to the $MIS$, makes the property provable by the $MIS$. 
And, that's why it is called the ``minimal correction'' set. 

The technique for enumerating all IVCs is a generalization of exploring the power set of $T$ (denoted by $ \mathcal{P}(T) $).
Basically, the algorithm needs to explore the provability of a 
given property by any subset of $T$, which may sound impractical. 
However, using some facts about $IVC$s and $MIS$es we can have a complete
enumeration algorithm that only needs to explore a (small) subset of $T$ 
in order to calculate $AIVC$:
\begin{itemize}
  \item Given property $P$ and every subset $S$ of $\mathcal{P}(T)$, we have either $(I, S) \vdash P$ or $(I, S) \nvdash P$. In the former case, we say $S$ is \textbf{adequate} (to prove $P$); in the latter, we say that $S$ is \textbf{inadequate} (for the proof of P).
  \item If a given subset $S$ is adequate (inadequate), then all of its supersets (subsets) are (in)adequate as well. 
\end{itemize}

In exploring the power set, 
we define two functions $MAP$ and $\mathcal{A}$
from subsets to truth values. 
Function $\mathcal{A}$ tells whether or not property $P$ is provable by a subset of $T$:
$$\mathcal{A} : \mathcal{P}(T) \rightarrow \{true~(adequate), false~(inadequate)\}$$  
\noindent Function $MAP$ is defined to guide the exploration algorithm. 
If we consider the power set as a lattice in a Hasse diagram, 
$MAP$ tells us which parts of the lattice have been explored:
$$MAP: \mathcal{P}(T) \rightarrow \{true~(unexplored), false~(explored)\}$$
\noindent The core of an algorithm for computing $AIVC(P)$ is to choose an \emph{unexplored} subset $S$ of $\mathcal{P}(T)$ and examine whether or not $S$ is adequate for $P$. If so, then compute an $S' \subseteq S$ such that $IVC(P, S')$.

To choose 

\begin{definition}{\emph{Unexplored subset problem:}}
  \label{def:usp}
  
\end{definition}





 

%\begin{algorithm}[t]
%  $k' \leftarrow 1$ \\
%  \While{$\checksat(\neg\iq_{k'}(T, P, P)) = \sat$} {
%    $k' \leftarrow k' + 1$ \\
%    }
%  \Return{$k'$} \\
%\caption{$\mink(T, P)$}
%\label{alg:minimize-k}
%\end{algorithm}
 

%\begin{theorem}
%\label{thm:minimal-hard}
%\end{theorem}
%\begin{proof}
%\end{proof}
