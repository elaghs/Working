\section{Inductive Validity Cores}
\label{sec:ivc}

\newcommand{\bfalg}{IVC\_BF\xspace}
\newcommand{\ucalg}{IVC\_UC\xspace}
\newcommand{\ucbfalg}{IVC\_UCBF\xspace}
\newcommand{\bq}{\textsc{BaseQuery}\xspace}
\newcommand{\iq}{\textsc{IndQuery}\xspace}
\newcommand{\fq}{\textsc{FullQuery}\xspace}

\newcommand{\mink}{\textsc{MinimizeK}\xspace}
\newcommand{\reduceinv}{\textsc{ReduceInvariants}\xspace}
\newcommand{\minivc}{\textsc{MinimizeIvc}\xspace}

\newcommand{\checksat}{\textsc{CheckSat}\xspace}
\newcommand{\unsatcore}{\textsc{UnsatCore}\xspace}
\newcommand{\unsat}{\textsc{UNSAT}\xspace}
\newcommand{\sat}{\textsc{SAT}\xspace}

Given a transition system which satisfies a safety property $P$, we
want to know which parts of the system are necessary for satisfying
the safety property. One possible way of asking this is, ``What is the
most general version of this transition system that still satisfies
the property?'' The answer is disappointing. The most general system is
$I(s) = P(s)$ and $T(s, s') = P(s')$, i.e., you start in any state
satisfying the property and can transition to any state that still
satisfies the property. This answer gives no insight into the original
system because it has no connection to the original system. In this
section we introduce the notion of {\em inductive validity cores} (IVC)
which looks at generalizing the original transition system while
preserving a safety property.

In order to talk about generalizing a transition system, we assume the
transition relation of the system has the structure of a top-level
conjunction. This assumption gives us a structure that we can easily
manipulate as we generalize the system. Given $T(s, s') = T_1(s, s')
\land \cdots \land T_n(s, s')$ we will write $T = T_1 \land \cdots
\land T_n$ for short. By further abuse of notation we will identify
$T$ with the set of its top-level conjuncts. Thus we will write $x \in
T$ to mean that $x$ is a top-level conjunct of $T$. We will write $S
\subseteq T$ to mean that all top-level conjuncts of $S$ are top-level
conjuncts of $T$. We will write $T \setminus \{x\}$ to mean $T$
with the top-level conjunct $x$ removed. We will use the same notation
when working with sets of invariants.

\begin{definition}{\emph{Inductive Validity Core:}}
  \label{def:ivc}
  Let $(I, T)$ be a transition system and let $P$ be a
  safety property with $(I, T)\vdash P$. We say $S \subseteq
  T$ is an {\em inductive validity core} for $(I, T)\vdash P$ iff $(I,
  S) \vdash P$. When $I$, $T$, and $P$ can be inferred from
  context we will simply say $S$ is an inductive validity core.
\end{definition}

\begin{definition}{\emph{Minimal Inductive Validity Core:}}
  \label{def:minimal-ivc}
  An inductive validity core $S$ for $(I, T)\vdash P$ is minimal iff
  there does not exist $M \subset S$ such that $M$ is an inductive validity core
  for $(I, T)\vdash P$.
\end{definition}

Note that minimal inductive validity cores are not necessarily unique.
For example, take $I = a \land b$, $T = a' \land b'$, and $P = a \lor
b$. Then both $\{a'\}$ and $\{b'\}$ are minimal inductive validity
cores for $(I, T)\vdash P$. However, inductive validity cores do have
the following monotonicity property.

\begin{lemma}
  \label{lem:ivc-monotonic}
  Let $(I, T)$ be a transition system and let $P$ be a safety property
  with $(I, T)\vdash P$. Let $S_1 \subseteq S_2 \subseteq T$. If $S_1$
  is an inductive validity core for $(I, T)\vdash P$ then $S_2$ is an
  inductive validity core for $(I, T)\vdash P$.
\end{lemma}
\begin{proof}
  From $S_1 \subseteq S_2$ we have $S_2 \Rightarrow S_1$. Thus the
  reachable states of $(I, S_2)$ are a subset of the reachable states
  of $(I, S_1)$. \qed
\end{proof}

\begin{algorithm}[t]
  \SetKwInOut{Input}{input}
  \SetKwInOut{Output}{output}
  \Input{$(I, T)\vdash P$}
  \Output{Minimal inductive validity core for $(I, T)\vdash P$}
  \BlankLine
  $S \leftarrow T$ \\
  \For{$x \in S$} {
    \If{$(I, S\setminus\{x\}) \vdash P$}{
      $S \leftarrow S\setminus \{x\}$
    }
  }
  \Return{S}
\caption{\bfalg: Brute-force algorithm for computing a minimal IVC}
\label{alg:naive}
\end{algorithm}

This lemma gives us a simple, brute-force algorithm for computing
a minimal inductive validity core, Algorithm \bfalg~(\ref{alg:naive}). The
resulting set of this algorithm is obviously an inductive validity
core for $(I, T)\vdash P$. The following lemma shows that it is also
minimal.

\begin{lemma}
  The result of Algorithm~\ref{alg:naive} is a minimal inductive validity core
  for $(I, T)\vdash P$.
\end{lemma}
\begin{proof}
  Let the result be $R$. Suppose towards contradiction that $R$ is not
  minimal. Then there is an inductive validity core $M$ with $M
  \subset R$. Take $x \in R\setminus M$. Since $x \in R$ it must be
  that during the algorithm $(I, S\setminus\{x\})\vdash P$ is not true
  for some set $S$ where $R \subseteq S$. We have $M \subset R
  \subseteq S$ and $x\not\in M$, thus $M \subseteq S\setminus \{x\}$.
  Since $M$ is an inductive validity core,
  Lemma~\ref{lem:ivc-monotonic} says that $S\setminus \{x\}$ is an
  inductive validity core, and so $(I, S\setminus\{x\})\vdash P$. This
  is a contradiction, thus $R$ must be minimal.
\end{proof}

This algorithm has two problems. First, checking if a safety property
holds is undecidable in general thus the algorithm may never terminate
even when the safety property is easily provable over the original
transition system. Second, this algorithm is very inefficient since it
tries to re-prove the property multiple times.

\begin{algorithm}[t]
  \SetKwInOut{Input}{input}
  \SetKwInOut{Output}{output}
  \Input{$P$ with invariants $Q$ is $k$-inductive for $(I, T)$}
  \Output{Inductive validity core for $(I, T)\vdash P$}
  \BlankLine
  $k \leftarrow \mink(T, P \land Q)$ \\
  $R \leftarrow \reduceinv_k(T, Q, P)$ \\
  \Return{$\minivc_k(I, T, R)$}\\
\caption{\ucalg: Efficient algorithm for computing a nearly minimal inductive validity core from UNSAT cores}
\label{alg:ivc}
\end{algorithm}

The key to a more efficient algorithm is to make better use of the
information that comes out of model checking. In addition to knowing
that $P$ holds on a system $(I, T)$, suppose we also know something
stronger: $P$ with the invariant set $Q$ is $k$-inductive for $(I,
T)$. This gives us the broad structure of a proof for $P$ which allows
us to reconstruct the proof over a modified transition system.
However, we must be careful since this proof structure may be more
than is actually needed to establish $P$. In particular, $Q$ may
contain unneeded invariants which could cause the inductive validity
core for $P \land Q$ to be larger than the inductive validity core for
$P$. Thus before computing the inductive validity core we first try to
reduce the set of invariants to be as small as possible. This
operation is expensive when $k$ is large so as a first step we
minimize $k$. This is the motivation behind
Algorithm \ucalg~(\ref{alg:ivc}).

\begin{figure}
\begin{align*}
  &\bq_1(I, T, P) \equiv \forall s_0.~ I(s_0) \Rightarrow P(s_0) \\
%%%
  &\bq_{k+1}(I, T, P) \equiv \bq_k(I, T, P) \land~ \\
%
  &\hspace{10pt}\left(\forall s_0, \ldots, s_k.~ I(s_0) \land T(s_0,
  s_1) \land \cdots \land T(s_{k-1}, s_k) \Rightarrow P(s_k)\right)
  \\[5pt]
%%%
  &\iq_k(T, Q, P) \equiv (\forall s_0, \ldots, s_k.~\\
%
  &\hspace{10pt} Q(s_0) \land T(s_0,
  s_1) \land \cdots \land Q(s_{k-1}) \land T(s_{k-1}, s_k) \Rightarrow
  P(s_k)) \\[5pt]
%%%
  &\fq_k(I, T, P) \equiv \\
%
  &\hspace{10pt}\bq_k(I, T, P) \land \iq_k(T, P, P)
\end{align*}
\caption{$k$-induction queries}
\label{fig:queries}
\end{figure}

To describe the details of Algorithm~\ref{alg:ivc} we define queries
for the base and inductive steps of $k$-induction
(Figure~\ref{fig:queries}). Note, in $\iq(T, Q, P)$ we separate the
assumptions made on each step, $Q$, from the property we try to show
on the last step, $P$. We use this separation when reducing the set of
invariants.

We assume that our queries are checked by an SMT solver. That is, we
assume we have a function $\checksat(F)$ which determines if $F$, an
existentially quantified formula, is satisfiable or not. In order to
efficiently manipulate our queries, we assume the ability to create
{\em activation literals} which are simply distinguished Boolean
variables. The call $\checksat(A, F)$ holds the activation literals in
$A$ true while checking $F$. When $F$ is unsatisfiable, we assume we
have a function $\unsatcore()$ which returns a minimal subset of the
activation literals such that the formula is unsatisfiable with those
activation literals held true. In practice, SMT solvers often return a
non-minimal set, but we can minimize the set via repeated calls to
\checksat. We assume both \checksat and \unsatcore are always
terminating.

\begin{algorithm}[t]
  $k' \leftarrow 1$ \\
  \While{$\checksat(\neg\iq_{k'}(T, P, P)) = \sat$} {
    $k' \leftarrow k' + 1$ \\
    }
  \Return{$k'$} \\
\caption{$\mink(T, P)$}
\label{alg:minimize-k}
\end{algorithm}

The function $\mink(T, P)$ is defined in
Algorithm~\ref{alg:minimize-k}. This function assumes that $P$ is
$k$-inductive for $(I, T)$. It returns the smallest $k'$ such that $P$
is $k'$-inductive for $(I, T)$. We start checking at $k' = 1$ since
smaller values of $k'$ are much quicker to check than larger ones. The
checking must eventually terminate since $P$ is $k$-inductive. We also
only check the inductive query since we know the base query will be
true for all $k' \leq k$. Although we describe each query in
Algorithm~\ref{alg:minimize-k} separately, in practice they can be
done incrementally to improve efficiency.

\begin{algorithm}[t]
  $R \leftarrow \{P\}$ \\
  Create activation literals $A = \{a_1, \ldots, a_n\}$ \\
  $C \leftarrow (a_1 \Rightarrow Q_1) \land \cdots \land (a_n \Rightarrow Q_n)$ \\
  \While{$true$} {
    $\checksat(A, \neg\iq_k(T, C, R))$ \\
    \If{$\unsatcore() = \emptyset$}{
      \Return{R}
    }
    \For{$a_i \in \unsatcore()$}{
      $R \leftarrow R \cup \{Q_i\}$ \\
      $C \leftarrow C \setminus \{a_i \Rightarrow Q_i\}$ \\
    }
  }
\caption{$\reduceinv_k(T, \{Q_1, \ldots, Q_n\}, P)$}
\label{alg:reduce-invariants}
\end{algorithm}

The function $\reduceinv_k(T, \{Q_1, \ldots, Q_n\}, P)$ is defined in
Algorithm~\ref{alg:reduce-invariants}. This function assumes that $P
\land Q_1 \land \cdots \land Q_n$ is $k$-inductive for $(I, T)$. It
returns a set $R \subseteq \{P, Q_1, \ldots, Q_n\}$ such that $R$ is
$k$-inductive for $(I, T)$ and $P \in R$. Like \mink, this function
only checks the inductive query since each element of $R$ is an
invariant and therefore will always pass the base query. A significant
complication for reducing invariants is that some invariants may
mutually need each other, even though none of them are needed to prove
$P$. Thus in Algorithm~\ref{alg:reduce-invariants} we find a minimal
set of invariants needed to prove $P$, then we find a minimal set of
invariants to prove those invariants, and so on. We terminate when no
more invariants are needed to prove the properties in $R$.
Algorithm~\ref{alg:reduce-invariants} is guaranteed to terminate since
$R$ gets larger in every iteration of the outer loop and it is bounded
above by $\{P, Q_1, \ldots, Q_n\}$. As with
Algorithm~\ref{alg:minimize-k}, we describe each query in
Algorithm~\ref{alg:reduce-invariants} separately, though in practice
large parts of the queries can be re-used to improve efficiency.

This iterative lemma determination does not guarantee a minimal
result. For example, we may find $P$ requires just $Q_1$, that $Q_1$
requires just $Q_2$, and that $Q_2$ does not require any other
invariants. This gives the result $\{P, Q_1, Q_2\}$, but it may be
that $Q_2$ alone is enough to prove $P$ thus the original result is
not minimal. Also note, we do not care about the result of \checksat,
only the \unsatcore that comes out of it. Since $P \land Q_1 \land
\cdots \land Q_n$ is $k$-inductive, we know the \checksat call will
always return \unsat.

\begin{algorithm}[t]
  Create activation literals $A = \{a_1, \ldots, a_n\}$ \\
  $T \leftarrow (a_1 \Rightarrow T_1) \land \cdots \land (a_n \Rightarrow T_n)$ \\
  $\checksat(A, \neg\fq_k(I, T, P))$ \\
  $R \leftarrow \emptyset$ \\
  \For{$a_i \in \unsatcore()$}{
    $R \leftarrow R \cup \{T_i\}$
  }
  \Return{R}
\caption{$\minivc_k(I, \{T_1, \ldots, T_n\}, P)$}
\label{alg:minimize-ivc}
\end{algorithm}

The function $\minivc_k(I, \{T_1, \ldots, T_n\}, P)$ is defined in
Algorithm~\ref{alg:minimize-ivc}. This function assumes that $P$ is
$k$-inductive for $(I, T)$. It returns a minimal inductive validity
core $R \subseteq \{T_1, \ldots, T_n\}$ such that $P$ is $k$-inductive
for $(I, R)$. It is trivially terminating. Since
Algorithms~\ref{alg:minimize-k}, \ref{alg:reduce-invariants}, and
\ref{alg:minimize-ivc} are terminating, Algorithm~\ref{alg:ivc} is
always terminating.

Our full inductive validity core algorithm in Algorithm~\ref{alg:ivc}
does not guarantee a minimal inductive validity core. One reason is
that \reduceinv does not guarantee a minimal set of invariants. A
larger reason is that we only consider the invariants that the
algorithm is given at the outset. It is possible that there are other
invariants which could lead to a smaller inductive validity core, but
we do not search for them. In Sections~\ref{sec:experiment} and
\ref{sec:results}, we show that in practice our algorithm is nearly
minimal and much more efficient than the naive algorithm. The
following theorem shows that minimality checking is at least as hard
as model checking and therefore undecidable in many settings.

\begin{theorem}
\label{thm:minimal-hard}
Determining if an IVC is minimal is as hard as model checking.
\end{theorem}
\begin{proof}
Consider an arbitrary model checking problem $(I, T)\vdash^? P$ where
$P$ is not a tautology. We will construct an IVC for a related model
checking problem which will be minimal if and only if $(I, T)\nvdash
P$. Let $x$ and $y$ be fresh variables. Construct a transition system
with initial predicate $I\land \neg x$ and transition predicate $(x'
\Rightarrow y') \land ((y' \Rightarrow P') \land T)$. The constructed
system clearly satisfies the property $x \Rightarrow P$. Thus $S = \{x'
\Rightarrow y', (y' \Rightarrow P') \land T\}$ is an IVC. $S$ is
minimal if and only if neither $\{x' \Rightarrow y'\}$ nor $\{(y'
\Rightarrow P') \land T\}$ is an IVC. Since $x$ and $y$ are fresh and
$P$ is not a tautology, $\{x' \Rightarrow y'\}$ is not an IVC. Since
$x$ and $y$ are fresh, $\{(y' \Rightarrow P') \land T\}$ is an IVC for
the property $x \Rightarrow P$ if and only if $(I, T)\vdash P$.
Therefore, $S$ is minimal if and only if $(I, T)\nvdash P$.
\end{proof}

When minimality is a necessity, we can combine \bfalg and \ucalg into
a single algorithm which aims to efficiently guarantee minimality. The
hybrid algorithm, \ucbfalg, consists of running \ucalg to generate an
initial nearly minimal IVC which is then run through \bfalg to
guarantee minimality. The resulting algorithm is not guaranteed to
terminate since \bfalg is not guaranteed to terminate.

%%% Local Variables:
%%% mode: latex
%%% TeX-master: "main.tex"
%%% End

%%  LocalWords:  Lustre iff TODO invariants Minimality BaseQuery IVC
%%  LocalWords:  InductiveQuery FullQuery MinimizeK ReduceInvariants
%%  LocalWords:  MinimizeIVC CheckSat UnsatCore UNSAT UC UCBF
%%  LocalWords:  IndQuery MinimizeIvc minimality
