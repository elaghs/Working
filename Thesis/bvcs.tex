One straightforward technique for extracting BVCs is based on the conventional bounded model checking.
We described $k$-induction in Chapter \ref{ch:background}. Bounded model checking is the base-case in \label{fig:k-induction} for a given $k$. To obtain validity cores from the base-case query, we need to unroll transition relation step by step (for $i=0$ to $i \leq k$). Assuming the property of interest is valid, at least up to bound $k$, in each step the result of $\bq_k(I, T, P)$ (Figure \ref{alg:bvc} and \ref{fig:queries}) will be \unsat . In the same way with IVCs, we can map the unsat-cores to the model elements, which will result in BVCs.

\begin{algorithm}[t]
  \SetKwInOut{Input}{input}
  \SetKwInOut{Output}{output}
  \Input{$P \vdash _{k'}(I, T)$ and bound $k \leq k'$}
  \Output{\bvc ~for $(I, T)\vdash P$ at depth $k$}
  \BlankLine
   Create activation literals $A = \{a_1, \ldots, a_n\}$ \\
  $T \leftarrow (a_1 \Rightarrow T_1) \land \cdots \land (a_n \Rightarrow T_n)$ \\
  $\checksat(A, \neg\bq_k(I, T, P))$ \\
  $R \leftarrow \emptyset$ \\
  \For{$a_i \in \unsatcore()$}{
    $R \leftarrow R \cup \{T_i\}$
  }
  \Return{R}
\caption{\bvcalg: Algorithm for computing a bounded validity core}
\label{alg:bvc}
\end{algorithm} 

