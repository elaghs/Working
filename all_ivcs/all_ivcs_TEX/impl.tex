\section{Implementation}
\label{sec:impl}
%\subsection{Implementation}
We have implemented the \aivcalg ~algorithm
in an industrial model checker called \texttt{JKind} ~\cite{jkind},
which verifies safety properties of  infinite-state synchronous systems.
It accepts Lustre programs \cite{Halbwachs91:lustre} as input.  The translation of Lustre
into a symbolic transition system in \texttt{JKind} is straightforward and is similar to what is described
in~\cite{Hagen08:FMCAD}.
Verification is supported by multiple ``proof engines'' that execute in parallel, including K-induction,
property directed reachability (PDR), and lemma generation engines that attempt to prove
multiple properties in parallel.  To implement the engines,
\texttt{JKind} emits SMT problems using the theories of linear integer and real arithmetic.  \texttt{JKind} supports the
\texttt{Z3}~\cite{DeMoura08:z3},
\texttt{Yices}~\cite{Dutertre06:yices}, \texttt{MathSAT}~\cite{Cimatti2013:MathSAT},
\texttt{SMTInterpol}~\cite{Christ2012:SMTInterpol}, and \texttt{CVC4} \cite{barrett2011cvc4} SMT solvers as back-ends.  When a property is
proved and IVC generation is enabled, an additional parallel engine
executes the \ucalg ~algorithm \cite{Ghass16} to generate an (approximately) minimal IVC.
%
To implement our method, we have extended \texttt{JKind} with a new engine that
implements Algorithm \ref{alg:aivc} on top of \texttt{Z3}.
%The source code is publicly available on \cite{mygit}.
We use the \texttt{JKind} IVC generation engine to implement the \getivc\ procedure in  Algorithm \ref{alg:aivc}.
%It worth mentioning that we could have employed the \ucbfalg ~algorithm for this purpose as well.
%One might say that since \ucbfalg ~guarantees minimality, it would help to the \aivcalg ~algorithm terminate more quickly.
%However, as we will show in the experiments, on the one hand, the \ucbfalg ~algorithm is very expensive. On the other hand, the \ucalg ~algorithm is not only fast, but it also generates \ivc s that are
%quite close to the \mivc s computed by \ucbfalg.

One technical issue that needs to
be handled in any implementation of the \aivcalg ~algorithm involves the undecidability of the model checking algorithms;
in each iteration of the \texttt{while} loop, Algorithm~\ref{alg:aivc}
has to decide on the adequacy of a given set $S \subseteq T$ (line 8).
This decision should be made by the use of a proof method that
tries to prove the property $P$ over $S$. Although we know that $(I, T) \vdash P$,
there is no guarantee that we can determine whether or not any $S \subset T$ is adequate; it is known~\cite{Ghass16} that
this problem is as hard as model checking, which is undecidable for infinite state systems.
Therefore, in practice, we have to set timeouts for the model checking algorithm in each iteration.
In our implementation, we set a timeout based on the time the property takes to be proved given all model elements.
During this initial proof, we obtain its required \emph{proof-time}. 
Obtaining the initial proof guarantees the success of calculating first IVC by \ucalg. 
Therefore, in our implementation, a proof is first passed to the IVC generation engine, and then the result of this engine is sent to our All-IVCs generation engine\footnote{This information flow is useful because we are able to initialize $map$ with the disjunction of the negated literals of the initial IVC set.}.
At this point, we have obtained the runtime of single IVC extraction (\emph{\ucalg-time}) and \emph{proof-time}.
Then, the timeout we set for each iteration of the algorithm is ($30$ sec  $+\ 5\ \times$ (\emph{proof-time} $+$ \emph{\ucalg-time})). 
When the \texttt{while} loop times out for $S$ in line 8 of Algorithm~\ref{alg:aivc},
we treat $S$ as an \emph{inadequate} set;
otherwise results may be incorrect; i.e. sets that are not actually adequate may be reported as adequate.
When we mark an indeterminate set $S$ as inadequate, line 13 will prune off $S$ and all its subsets from
the search space.  This can occasionally lead to anomalies where the brute force algorithm (like \ucbfalg) discovers a {\em smaller}
MIVC than is discovered by the \aivcalg ~procedure.  Note that this can happen regardless of the timeout threshold used
due to the lemma generation procedure; this aspect will be discussed in more detail in
Section~\ref{sec:experiment-discussion}. It should be pointed out the IVCs found by the algorithm are only guaranteed to be minimal in the absence of timeouts.


%The source code is publicly available on \cite{mygit}.
%This section points out some technical issues in the implementation.
%
%We have extended \texttt{JKind} with a new engine that
% implements Algorithm \ref{alg:aivc}.
% First a property is proved and a single IVC set $S$ is computed
% by the IVC generation engine. Then, the set $S$ is sent to our engine.
% For efficiency, instead of initializing $map$ with $\top$,
% we initialize $map$ with $\bigvee_{T_{i}\in S} \neg {\actlit (T_i)}$,
% which prunes off set $S$ and all its supersets from $\mathcal{P}(T)$ at the beginning of the algorithm. If $S$ is minimal, then chances are a large
% subset of  $\mathcal{P}(T)$ is marked as explored even before
% the algorithm starts the search.

