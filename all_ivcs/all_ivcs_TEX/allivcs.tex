\section{Method}
\label{sec:allivcs}

\newcommand{\getivc}{\textsc{GetIVC}}

\newcommand{\blockup}{\textsc{BlockUp}}
\newcommand{\blockdown}{\textsc{BlockDown}}
\newcommand{\minivc}{\textsc{MinimizeIvc}\xspace}

\newcommand{\checksat}{\textsc{CheckSat}}
\newcommand{\isadeq}{\textsc{CheckAdq}}
\newcommand{\actlit}{\textsc{ActLit}}
\newcommand{\unsatcore}{\textsc{UnsatCore}\xspace}
\newcommand{\unsat}{\texttt{UNSAT}\xspace}
\newcommand{\sat}{\texttt{SAT}\xspace}

As mentioned, the contribution of this paper is to provide an efficient and complete method for calculating $AIVC$ of a property. To this end, we first introduce several additional notions and definitions, most of which are inspired by the MUS enumeration technique proposed in \cite{marco2016fast}.

The technique for enumerating all IVCs is a generalization of exploring the power set of $T$ (denoted by $ \mathcal{P}(T) $).
Basically, the algorithm needs to explore the provability of a
given property by any subset of $T$, which may sound impractical.
However, this section, first, proposes some notions that make it possible to have a complete
enumeration algorithm that only needs to explore a (small) portion of $\mathcal{P}(T)$
in order to compute $AIVC$.

\begin{definition} {\emph{Adequacy:}}
\label{def:adeq}
Given property $P$ and every $S \in \mathcal{P}(T)$, we have either $(I, S) \vdash P$ or $(I, S) \nvdash P$. In the former case, we say $S$ is \textbf{adequate} for $P$; in the latter, we say that $S$ is \textbf{inadequate} for the proof of $P$.
\end{definition}

\begin{definition}{\emph{Maximal Inadequate Subset (MIS):}}
  \label{def:mis}
  $S \subset T$ for $(I, T) \vdash P$ is a Maximal Inadequate Subset (MIS) iff
  $(I, S) \nvdash P$ and $\forall T_i \in T\setminus S.~ (I, S\cup{T_i}) \vdash P$.
\end{definition}
\begin{note}
If the second condition in $MIS$ does not hold, $S$ is not maximal, but it is an inadequate set.
\end{note}

\begin{definition}{\emph{Minimal Correction Set (MCS):}}
  \label{def:mcs}
  $S \subset T$ for $(I, T) \vdash P$ is a Minimal Correction Set (MCS) iff
  $(I, T \setminus S) \nvdash P$ and $\forall T_i \in S.~ (I, (T \setminus S)\cup \{T_i\}) \vdash P$.
\end{definition}

%It should be mentioned that minimality and maximality are about minimum or maximum cardinality subsets.
Note that $MCS$ is more of syntactic sugar that specifies sets that can also be specified by $MIS$; i.e. for any $MIS$ of $T$, there is a corresponding $MCS$ such that adding any element of that $MCS$ to the $MIS$, makes the property provable by the $MIS$.
And, that's why it is called the ``minimal correction'' set.
\begin{note}
If $S \subset T$ is inadequate, but not maximal, there will be a correction set $C \subset T$ for it such that $S \cup C$ is adequate.
\end{note}

\begin{lemma}
\label{lem:adeq}
For $(I, T) \vdash P$, if $S \subset T$ is adequate for property $P$, then all of its supersets are adequate for $P$ as well:
\allowbreak $$\forall S_1 \subset S_2 \subset T.~ (I, S_1) \vdash P \Rightarrow (I, S_2) \vdash P$$
\end{lemma}
\begin{proof}
From $S_1 \subseteq S_2$ we have $S_2 \Rightarrow S_1$. Thus the
  reachable states of $(I, S_2)$ are a subset of the reachable states
  of $(I, S_1)$.
\end{proof}

\begin{lemma}
\label{lem:inadeq}
For $(I, T) \vdash P$, if a given subset $S$ is inadequate, then all of its subsets are inadequate as well:
\allowbreak $$\forall S_1 \subset S_2 \subset T.~ (I, S_2) \nvdash P \Rightarrow (I, S_1) \nvdash P$$
\end{lemma}
\begin{proof}
Given the fact that $(I, T) \vdash P$ and $S_2$ is inadequate,
there is a correction set $C \subseteq (T\setminus S_2)$ such that
$C \cup S_2$ is adequate, which implies every $S' \subseteq (T \setminus C)$ is inadequate.
So, since $S_1 \subset S_2 \subseteq (T \setminus C)$, set $S_1$ is also inadequate.
\end{proof}

\begin{corol}
$\forall S \in AIVC(P)$, S is a minimal adequate set for $P$.
\end{corol}
\begin{proof}
  Immediate from the definitions of $IVC$ and $AIVC$, and Lemma \ref{lem:adeq}.
\end{proof}

The main core of an algorithm for computing $AIVC(P)$ is to choose an \emph{unexplored} $S \in \mathcal{P}(T)$ and examine whether or not it is adequate for $P$; if so, then compute an $S' \subseteq S$ such that $IVC(P, S')$.
To choose $S$ and keep track of which parts of the the power set
have been explored, we use a Boolean variable $map$, which is
in conjunctive normal form (CNF) and built gradually as the algorithm proceeds.
As $map$ gets updated, it guides our exploration algorithm for $AIVC$.
In other words, $map$ shows the \emph{search space} we have to explore for finding all IVCs.

More precisely, given $T$ with $n$ top-level conjuncts,
we define an ordered
set of activation literals $\mathcal{A} = \{a_1...a_n\}$, where every $a_i$ is of type of Boolean. Function $\actlit : T \rightarrow \mathcal{A}$
is a bijection function assigning every $T_i \in T$ to an $a_i \in \mathcal{A}$ and vice versa.
Then, a $map$ for $AIVC(P)$ is a CNF formula built over the elements of $\mathcal{A}$ such that:
\begin{itemize}
  \item Initially $map$ is $\top$ interpreted as the fact that the whole $\mathcal{P}(T)$ is unexplored.
  \item When $map$ is satisfiable, a model of it is a set
  $M \in \mathcal{P}(\mathcal{A})$.
  \item Every model $M$ of $map$ corresponds to a set $S \in \mathcal{P}(T)$ such that
$S = \bigcup_{a_i \in M} \actlit ^{-1} (a_i)$ and $M = \bigcup_{T_i \in S} \actlit(T_i)$.
  \item For every explored set $S \in \mathcal{P}(T)$:
  \begin{itemize}
    \item if $S$ is adequate for $P$, then $map$ contains a clause $\bigvee_{T_{i}\in S} \neg {\actlit (T_i)}$. Note that, according to Lemma \ref{lem:adeq}, such a clause automatically makes every superset of $S$ as explored, which shrinks our search space. So, it is said those supersets are \emph{dominated} by $S$.
    \item if $S$ is inadequate for $P$, then $map$ contains a clause $\bigvee_{T_{i}\in (T \setminus S)} \actlit (T_i)$. Note that, according to Lemma \ref{lem:inadeq}, such a clause automatically makes every subset of $S$ as explored, which prunes the search space. So, it is said those subsets are \emph{dominated} by $S$.
  \end{itemize}
\end{itemize}


\begin{lemma}
\label{lem:map:sound}
When $map$ is satisfiable with model $M$, set $S = \bigcup_{a_i \in M} \actlit ^{-1} (a_i)$ is not subset (superset) of any
inadequate (adequate) explored sets of $\mathcal{P}(T)$.
\end{lemma}
\begin{proof}
Proof by contradiction. Case 1: Suppose there is an adequate set $Ex \subset S$ that has been already explored. Therefore, according to the definition, $map$ contains a clause $C = \bigvee_{T_{i}\in Ex} \neg {\actlit (T_i)}$, and since $Ex \subset S$, it is impossible for the model $M = \bigcup_{T_i \in S} \actlit (T_i)$ to satisfy $C$; hence, the assumption is false.

Case 2: Suppose there is an inadequate set $Ex$ such that $S \subset Ex$ and $Ex$ has been already explored. Therefore, according to the definition, $map$ contains a clause $C = \bigvee_{T_{i}\in (T \setminus S)} \actlit (T_i)$, and since $S \subset Ex$, it is impossible for the model $M = \bigcup_{T_i \in S} \actlit (T_i)$ to satisfy $C$; so, the assumption is false.

From Case 1 and Case 2, there is no model of $map$ whose corresponding set in $\mathcal{P}(T)$ is the subset (superset) of any inadequate (adequate) explored sets.
\end{proof}


\begin{lemma}
\label{lem:map:comp}
For $(I, T) \vdash P$, $map$ is unsatisfiable \emph{iff} all $S \in AIVC(P)$ and all $MIS$es of $T$ have been explored.
\end{lemma}
\begin{proof}
As with any CNF formula, $map$ is unsatisfiable \emph{iff} every complete assignment falsifies at least one of its clauses. Clauses in $map$ have especial structure;
according to definition, every clause contains a set of either entirely negative literals or entirely positive; when a set $S$ corresponding to a model $M$ of map is explored, if $S$ is adequate, it means that it contains necessary elements of $T$ to prove $P$, which
are added to $map$ as a new clause containing \emph{negative} activation literals of $S$. However, when
$S$ is inadequate, it means that $S$ lacks a correction set $C$ of $T$ so to prove $P$.
And, in this case, a new clause containing \emph{positive} activation literals of $C$ is added to $map$. Essentially, elements of correction are essential part of some proofs, so are elements in the adequate sets, which means eventually, negative literals added from adequate sets will have their own  positive form added from the correction sets(and vice versa).
Therefore, every complete assignment falsifies at least one clause in $map$ \emph{iff} every
member of $\mathcal{P}(T)$ is dominated by some adequate or inadequate set.
And, considering the definitions of $IVC$ and $MIS$,  every member of $\mathcal{P}(T)$ is dominated by some set \emph{iff} all $IVC$s and $MIS$es have been explored.
\end{proof}

\begin{lemma}
\label{lem:map:cc}
For $(I, T) \vdash P$, $map$ is unsatisfiable \emph{iff} every $S \in \mathcal{P}(T)$ has been explored.
\end{lemma}
\begin{proof}
Immediate from Lemma \ref{lem:map:comp}.
\end{proof}


In exploring the power set,
we first define several function; function \isadeq ~tells whether or not property $P$ is provable by a subset of $T$:
$$\isadeq : \mathcal{P}(T) \rightarrow \{true~(adequate), false~(inadequate)\}$$
\noindent Function \checksat determines if an
existentially quantified formula is satisfiable or not
\footnote{We assume readers are familiar with the boolean satisfiability problem, which is the problem of determining whether there exists an assignment that satisfies a given propositional formula. For more information, refer to \cite{cook1971complexity}.}:
$$\checksat : Boolean \rightarrow \{ true~(\sat), false~(\unsat) \}$$


\getivc ...

\blockup ...

\blockdown ...

... continue with the algorithms here

- say why the algorithm is fast; because we block down and up based on MIS and IVC, which prunes the maximum number of cases from the search space (map)


%\begin{algorithm}[t]
%  $k' \leftarrow 1$ \\
%  \While{$\checksat(\neg\iq_{k'}(T, P, P)) = \sat$} {
%    $k' \leftarrow k' + 1$ \\
%    }
%  \Return{$k'$} \\
%\caption{$\mink(T, P)$}
%\label{alg:minimize-k}
%\end{algorithm}


%\begin{theorem}
%\label{thm:minimal-hard}
%\end{theorem}
%\begin{proof}
%\end{proof}
