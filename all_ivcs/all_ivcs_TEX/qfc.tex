\section{Case Study}

\label{sec:qfc}
We demonstrate the applicability of the algorithms toward ``requirements coverage'' (or complete traceability as described by Murugesan~\cite{Murugesan16:renext}) on two large-scale models from a NASA Quad-redundant Flight Control System (QFCS)~\cite{NFM2015:backes}: the Flight Control System (FCS) (with 3423 equations and 2 properties) and the Flight Control Computer (FCC) (with 9953 equations and 9 properties).  The goal was to determine the scalability and utility of the approach for determining model redundancy and model coverage given the requirements. 

We performed our experiment by examining the conjunction of properties for each model.  The experiment was conducted with a timeout of 6 hours per model. Table \ref{tab:qfcs} shows the result of our study.  We examine {\em model coverage} of the requirements, using two measurements: {\em MAY} coverage, which examines which elements might be necessary for proof, and {\em MUST} coverage, which includes only equations necessary for all proofs.  We also calculate the number and diversity of IVCs, which is an indication of redundancy in the model for its requirements.


\iffalse
For each model. when the \aivcalg\ did not terminate before 4 hours, its runtime is shown with incomplete. \ela{we can explain a bit the numbers and table here, then get to coverage}

As mentioned in Section~\ref{sec:intro}, proof-based coverage analysis is one important application of all minimal IVCs. The goal of a
coverage metric is usually to assign a numeric score that de-
scribes how well properties cover the design. The majority of
the work on coverage metrics has focused on mutations, which
are “atomic” changes to the design, where the set of possible
mutations depends on the notation that is used. For large models, mutation-based coverage is quite inefficient. Having all minimal IVCs, we can easily formalize a family of proof-based metrics. For example, owe could define two distinct coverage metrics as follows:
\ela{choose two metric that I can calculate: we could choose must and may since you also talk about them in the introduction}
\fi

\begin{table}
  \caption{\aivcalg\ on the QFCS models}
  \vspace{-0.1in}
  \centering
\begin{tabular}{|c|c|c|}
  \hline
  % after \\: \hline or \cline{col1-col2} \cline{col3-col4} ...
  models & FCS & FCC \\
   \hline
   Initial proof time (min)  & 19.5 & 172 \\
   All-IVC runtime (h) & 6 & 6 \\
   \hline
   \# of IVCs  & 17 & 1   \\
   \hline
   IVC size by \ucbfalg\  & 1659 & 4302  \\
   min IVC size by \aivcalg & 1659 & 4302 \\
   avg IVC size by \aivcalg  & 1659.9 & 4302 \\
   max IVC size by \aivcalg  & 1660 & 4302 \\
  \hline

  coverage score of MAY & 0.62 & 0.43  \\
  coverage score of MUST & 0.47 & 0.43 \\
  \hline
\end{tabular}
\label{tab:qfcs}
\vspace{-0.2in}
\end{table}

%We ran the algorithm on each property separately, which left us with 11 complex problems to study.
%The FCS 3,423  104 prop
%fcc 9953 1511? prop


