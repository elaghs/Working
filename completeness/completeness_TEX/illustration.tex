\section{Illustration}

\newcommand{\allp}{\texttt{all\_p}}
\newcommand{\onp}{\texttt{on\_p}}
\newcommand{\offp}{\texttt{off\_p}}
\newcommand{\hystp}{\texttt{hyst\_p}}
\newcommand{\aonebelow}{\texttt{a1\_below}}
\newcommand{\atwobelow}{\texttt{a2\_below}}
\newcommand{\aoneabove}{\texttt{a1\_above}}
\newcommand{\atwoabove}{\texttt{a2\_above}}
\newcommand{\doion}{\texttt{doi\_on}}
\newcommand{\done}{\texttt{d1}}
\newcommand{\dtwo}{\texttt{d2}}
\newcommand{\abovehyst}{\texttt{above\_hyst}}
\newcommand{\inhibit}{\texttt{inhibit}}

We illustrate the idea of completeness metrics and the "score" of the different metrics on our altitude switch (ASW) example from Section~\ref{sec:example}.  In so doing, we hope to motivate the need for completeness metrics and also to justify our proof-based approaches.  

The ASW is responsible for turning on and off a device of interest, so we formulate two requirements that describe when the ASW should be {\em on} and when it should be {\em off}.  Our first (somewhat faulty) attempt at formalization is as follows:

%\begin{definition} {\emph{ASW Requirements Version 1} } 
{\smaller
\begin{verbatim}
on_p = (a1_below and a2_below) and not inhibit =>
    doi_on = true;
off_p = (a1_above and a2_above) and inhibit => 
    doi_on = false;   
all_p = on_p and off_p; 
\end{verbatim}
}
%\end{definition}
 
\noindent Informally, when both altimeters are below the threshold and not inhibited, then the DOI should be on (\onp), and when both altimeters are below the threshold and the ASW is inhibited, then the DOI should be off (\offp).  The property \allp\ is the conjunction of all requirements.  We can now apply each of the completeness notions: IVC, MAY, and MUST (which is the same as \nondetcov).  What we discover is that, for all three metrics, \allp\ only requires \texttt{\{below, d1, doi\_on\}}.  This is alarming, and somewhat puzzling, because one would think that at least the definitions of the `below' or `above' would be necessary.  However, because the specification used the model variables \aonebelow, \atwobelow, \aoneabove, and \atwoabove, the actual valuations of the thresholds do not matter.  This situation illustrates a classic specification problem: using computed variables in the antecedents of implications.\footnote{In this case, if the computation of the variables used in the antecedent is incorrect, then our property will not verify what it is expected to verify \mike{citation to one of our papers on specification here...}; note also that this does not mean the property is necessarily {\em vacuous}.}

We therefore modify our properties to use inputs and constants as antecedents and derive:
%\begin{definition} {\emph{ASW Requirements Version 2} }
{\smaller
\begin{verbatim}
on_p = ((alt1 < THRESHOLD) and (alt2 < THRESHOLD)) 
   and not inhibit => doi_on = true;
off_p = ((alt1 >= T_HYST) and (alt2 >= T_HYST)) 
   and inhibit => doi_on = false;
\end{verbatim}
}
% all_p = on_p and off_p;
%\end{definition}

\noindent In this version, we can see distinctions between the metrics.  \allp\ has two IVCs: \texttt{\{\{a1\_below, below, doi\_on, d1\}, \{a2\_below, below, doi\_on, d1\}\}}.  The reason that there are two IVCs is due to the \onp\ property: in the implementation, the DOI is turned on when either of the altimeters is below the threshold, while our property states that they both must be on.  
From domain experts, we determine that the requirement is correctly specified and that our implementation is a reasonable refinement, so there is no need to change the model or the property.  The MUST elements are the same as version 1: \texttt{\{below, doi\_on, d1\}}, because neither \aonebelow\ or \atwobelow\ is required for all proofs.  However, given the MUST elements, we can no longer construct a proof, because at one of these definitions is necessary for either proof.  The MAY elements contain both \aonebelow\ and \atwobelow.

We are still missing (in all metrics) the \abovehyst, \aoneabove, \atwoabove, and \dtwo\ variables, so the `above' thresholds are irrelevant to our properties.  Examining \offp, we realize that we have a specification error; the DOI should be off if either \inhibit\ is true or both altimeters are above the threshold, so we fix it: 

%\begin{definition} {\emph{ASW Requirements Version 2} }
{\smaller
\begin{verbatim}
off_p = ((alt1 >= T_HYST) and (alt2 >= T_HYST))
   or inhibit => doi_on = false;
\end{verbatim}
}
%on_p = ((alt1 < THRESHOLD) and (alt2 < THRESHOLD))
%   and not inhibit => doi_on = true;
%all_p = on_p and off_p;
%\end{definition}

\noindent Now the \allp\ requirement proof yields a single IVC that requires all variables except \{d2\}, so IVC = MAY = MUST.  Interestingly, the \offp\ proof requires both the lower altimeter thresholds even though the \onp\ proof does not; the reason is that if either of these is false, then \doion\ will be true.  To cover \{d2\}, we realize that there is no property covering the `no change' case where both altimeters are within the hysteresis range, so we further refine our requirements:

%\begin{definition} {\emph{ASW Requirements Version 2} }
{\smaller
\begin{verbatim}
hyst_p = not inhibit and
         (alt1 > THRESHOLD and alt2 > THRESHOLD) and
         (alt1 < T_HYST or alt2 < T_HYST) =>
   (doi_on = false -> doi_on = pre(doi_on))
all_p = on_p and off_p and hyst_p;
\end{verbatim}
}
%on_p = ((alt1 < THRESHOLD) and (alt2 < THRESHOLD))
%   and not inhibit => doi_on = true;
%all_p = on_p and off_p;
%\end{definition}
\noindent The final property states that if the antecedent conditions hold, then in the initial state, the \doion\ variable is assigned false, and in subsequent steps, it retains the same value as it previously had.

Now all model variables are required for the IVC, so the measures coincide, and we have a reasonably complete specification, so the measures appear to be useful.  However, the measures are certainly not foolproof; it turns out that using {\em only} the hysteresis property \hystp\ will {\em also} yield a ``complete'' result for all of the metrics.  We will examine this situation further in the discussion section~\mike{add this!}.  

\mike{How would we define a metric that would flag the model as incomplete?  Model transformation would do it: if we added separate variables for each assignment of doi\_on, then any of the metrics would flag the \hystp\ spec as incomplete.}
