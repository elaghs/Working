\chapter{Experiments}
\label{ch:experiment}

\newcommand{\takeaway}[1]{
\vspace{12pt}
\noindent\fbox{\parbox{\textwidth}{#1}}
\vspace{6pt}
}
We would like to evaluate the different algorithms presented in Chapter \ref{ch:ivc}.
 The use of \texttt{JKind} allows additional dimensions to our investigation: as mentioned before, it supports two different inductive algorithms: $k$-induction and PDR, and a ``fastest'' mode, that runs both algorithms in parallel.  Also, \texttt{JKind} supports multiple back-end SMT solvers including \texttt{Z3}~\cite{DeMoura08:z3}, \texttt{Yices}~\cite{Dutertre06:yices}, \texttt{MathSAT}~\cite{Cimatti2013:MathSAT}, and \texttt{SMTInterpol}~\cite{Christ2012:SMTInterpol}.

This chapter is organized in two parts. First we evaluate IVCs and their relationship to model checking algorithms.  There are two dominant model checking algorithms in modern solvers: PDR and $k$-induction, and we would like to determine (1) whether the choice of inductive algorithm affects the size of the IVCs calculated by \ucalg, (2) whether different solvers are more or less efficient at producing approximate minimal IVCs (\ucalg), and (3) whether running different solvers/algorithms leads to a diversity of solutions obtained by \ucalg .

Second, using the \ucalg\ algorithm as a baseline, we evaluate the performance the IVC algorithms presented in Chapter 3 in Section \ref{sec:exp1}. Broadly speaking, we are interested in the following aspects: (4) The runtime efficiency of all IVC algorithms, (5) The efficacy (in terms of minimality) of the \ucalg\ algorithm, (6) factors impacting performance of computing all IVCs.  In terms of the sixth question, experimental data was used to construct the online algorithm; an accurate accounting of the relative cost of SAT and UNSAT problems is necessary to create an efficient algorithm for this problem.

Finally, we examine the relationship between inductive validity cores and bounded validity cores in Section \ref{sec:exp3}.  There are many cases where systems become to large or complex to find an inductive proof; in such cases it is still possible to perform a bounded proof (in terms of the number of steps).  Bounded validity cores are always underapproximations of some IVC, so can be used to witness traceablity and adequacy relationships between properties and implementation elements.  Several questions are of interest: (1) how similar are BVCs of different depths to IVCs?  (2) how quickly do they tend to converge to a specific size?, and (3) is the converged set the same as one of the MIVCs?  This initial experiment is designed to lay the groundwork for future explorations of the relationship between these two concepts.

The initial experiments allow us to choose an optimal configuration for \texttt{JKind} based on which we can run our major experiments. The major experiments are conducted to evaluate the efficiency and efficacy the IVC algorithms presented in Chapter \ref{ch:ivc}.
In summary, we are interested in the following aspects:

\begin{itemize}
  \item Evaluating the performance of \ucalg\ algorithm and \aivcalg ,
  \item Evaluating the efficacy (minimality) of the \ucalg\ algorithm outcome,
  \item Effective factors on the performance of finding all minimal IVCs,
  \item Evaluating the online approach for generating all minimal IVCs,
  \item Studying validity cores in bounded model checking.
\end{itemize}
\noindent For this purpose, we organize the major experiments in different categories; Section \ref{sec:exp1} is about the evaluation of our two key proposed algorithms: \ucalg\ and \aivcalg . Then in Section \ref{sec:exp2}, we examine our approach for calculating all minimal IVCs in the online manner as described in \ref{sec:onaivc}. Finally in Section \ref{sec:exp3}, we investigate a couple of interesting research questions related to bounded validity cores.

\section{On the Relationship between Model Checking Algorithms and IVCs}
\label{sec:exprinit}
In this section we study the effect of different model checking algorithms/solvers on the performance and accuracy of the \ucalg\ algorithm. We started from a suite of 700 Lustre models developed
as a benchmark suite by Hagen and Tinelli~\cite{Hagen08:FMCAD}. We augmented this suite
with 81 additional models from recent verification projects including
avionics and medical devices~\cite{QFCS15:backes,hilt2013}. Most of
the benchmark models from~\cite{Hagen08:FMCAD} are small (with 6-40 equations) and contain a range of hardware benchmarks and
software problems involving counters. The additional models are much
larger: with over 300 up to 10000 equations. We added the new
benchmarks to better check the scalability for the tools, especially
with respect to the brute force algorithm.
%
%\mike{MORE HERE...stats on size, reasons for add'l models.}
Each benchmark model has a single property to analyze.  For our purposes, we are only interested in models with a {\em valid} property (though it is perhaps worth noting that there is no additional computation---and thus no overhead---using the \texttt{JKind} IVC options for {\em invalid} properties).  In our benchmark set, 295 models yield counterexamples, and 10 additional models are neither provable nor yield counterexamples in our test configuration (see next paragraph for configuration information).  The benchmark suite therefore contains 476 models with valid properties, which we use as our initial test subjects.

For each test model, we computed \ucalg\ in 12+1 configurations: the
twelve configurations were the cross product of all solvers \{\texttt{Z3},
\texttt{Yices}, \texttt{MathSAT}, \texttt{SMTInterpol}\} and inductive algorithms
\{$k$-induction, PDR, fastest\}, and the remaining (+1) configuration
was an instance of \bfalg\ run on \texttt{Yices}, which is the default solver in
JKind.
 In addition, for each of the 12 configurations, we ran an
instance of \texttt{JKind} without IVC to examine overhead. The initial experiments
were run on an Intel(R) i5-2430M, 2.40GHz, 4GB memory machine, with a
1 hour timeout for each analysis on any model. The data gathered for
each configuration of each model included the time required to check
the model without IVC, with IVC, and also the set of elements in the
computed IVC.\footnote{The benchmarks, all raw experimental results,
  and computed data are available on \cite{expr}.}

Note that not all analysis problems were solvable with all algorithms: for all solvers, $k$-induction (without IVC) was unable to solve 172 of the examples.  When comparing minimality of different solving algorithms, we only considered cases where both algorithms provided a solution.

For this study, we are interested in the following research questions:
\begin{itemize}
  \item \textbf{RQ1:} Does the choice of SMT solver affect the performance of \ucalg ?
  \item \textbf{RQ2:} Does the choice of SMT solver or algorithm used to produce a proof
(k-induction or PDR) matter in terms of the minimality of the IVCs generated by \ucalg ?
  \item \textbf{RQ3:} Do different solvers and algorithms
lead to different minimal cores for \ucalg ?
\end{itemize}

\vspace{0.1in}
\subsubsection{RQ1}
First, we examine the performance overhead of the \ucalg\ algorithm over the time necessary to find a proof using inductive model checking.  To examine this question, we use the default {\em fastest} option of JKind which terminates when either the $k$-induction or PDR algorithm finds a proof.  To measure the performance overhead of the \ucalg\ algorithm, we execute it over the proof generated by the {\em fastest} option.

Since the \ucalg\ algorithm uses the UNSAT core facilities of the
underlying SMT solver, the performance is dependent on the efficiency
of this part of the solver. Looking at Tables~\ref{tab:runtime-ucalg-solvers}
and~\ref{tab:overhead-ucalg-solvers}, it is possible to examine both the
computation time for analysis using the four solvers under evaluation
and the overhead imposed by the \ucalg\ algorithm.
Figure~\ref{fig:perf-solvers} allows a visualization of the runtime for
the \ucalg\ algorithm running different solvers. The lines are sorted individually based on the running time of the \ucalg\ for each model. The data suggests that
\texttt{Yices} (the default solver in \texttt{JKind}) and \texttt{Z3} are the most performant
solvers both in terms of computation time and overhead.

\begin{figure*}
  \centering
  \includegraphics[width=\textwidth]{figs/performance_solvers.png}
    \vspace{-0.1in}
  \caption{\ucalg\ performance on different solvers}
  \label{fig:perf-solvers}
\end{figure*}

\begin{table}
  \caption{\ucalg\ runtime with different solvers}
  \centering
  \begin{tabular}{ |c||c|c|c|c| }
    \hline
     runtime (sec) & min & max & mean & stdev \\[0.5ex]
    \hline\hline
 %   JSupport & 2.381 & 165.157 & 21.533 & 23.533 \\[0.5ex]
    Z3   & 0.005 & 2.335 & 0.192 & 0.355 \\[0.5ex]
    Yices &   0.014  & 13.297   & 0.589 & 1.473 \\[0.5ex]
    SMTInterpol& 0.029 & 19.254 &  1.396 & 2.991 \\[0.5ex]
    MathSAT & 0.011 & 86.421 &  3.071 & 10.403 \\[0.5ex]
    \hline
  \end{tabular} \\
  \label{tab:runtime-ucalg-solvers}
\end{table}

\begin{table}
  \caption{Overhead of \ucalg\ computations using different solvers}
  \centering
  \begin{tabular}{ |c||c|c|c|c| }
    \hline
     solver & min & max & mean & stdev \\[0.5ex]
    \hline
    Z3   & 0.73\% & 84.13\% & 17.38\% & 16.92\% \\[0.5ex]
    Yices &   0.17\%  & 351.47\%   & 52.20\% & 54.50\% \\[0.5ex]
   SMTInterpol& 1.46\% & 175.75\% &  46.81\% & 37.35\%\\[0.5ex]
    MathSAT & 0.78\% & 955.52\% &  80.21\% & 112.92\%\\[0.5ex]
    \hline
  \end{tabular}
  \label{tab:overhead-ucalg-solvers}
\end{table}

\vspace{0.1in}
\subsubsection{RQ2}
As described in Chapter~\ref{ch:ivc}, the \ucalg\
algorithm is not guaranteed to produce minimal cores due in part to
the role of invariants used in producing a proof; as $k$-induction and
PDR use substantially different invariant generation algorithms, it is
likely that the set of necessary invariants for proofs are dissimilar,
and that this would in turn affect the number of model elements required for
the proof.  It is possible that one or the other algorithm is more likely
to yield smaller invariant sets.  In addition, differences in the choice of the
UNSAT core algorithms in the different solvers could affect the size of the
generated core. However, our algorithm already performs a minimization
step on UNSAT cores, and thus the only differences would be due to one
algorithm leading to a different minimal core than another.

As mentioned, $k$-induction is unable to solve all of the analysis problems; therefore we include only models that are solvable using {\em both} $k$-induction and PDR by {\em all solvers}, 304 models in all.  Examining the aggregate data in Table~\ref{tab:minimality-algorithm-solvers}, we can see the sizes of cores produced by different algorithms and solvers.
\vspace{0.1in}
\subsubsection{RQ3}
In this section, we examine the issue of diversity:
do different solvers and algorithms lead to {\em different} minimal
cores? This is both a function of the models and the solution
algorithms: for certain models, there is only one possible minimal IVC
set, whereas other models might have many. Given that there are
multiple solutions, the interesting question is whether using
different solvers and algorithms will lead to different solutions.
The reason diversity is considered is that it has substantial relevance to
some of the uses of the tool, e.g., for constructing multiple traceability
matrices from proofs (see Section~\ref{sec:traceability}).
Note that our exploration in this experiment is not
exhaustive, but only exploratory, based on the IVCs returned by different
algorithms and tools; we leave exhaustive exploration of
IVCs for future work.

%Given diversity of results, we may wish to
%distinguish {\em must} traceability elements from {\em may}
%traceability elements across a set of diverse solutions, and consider
%more systematic explorations of diversity in future work.

To measure diversity of the generated IVCs, we use Jaccard distance:
\begin{definition}{\emph{Jaccard distance:}}
  \label{def:dj}
  $d_J(\small{A}, \small{B}) = 1 - \frac{|A \cap B|}{|A \cup B|} ,\\ 0 \leq d_J(\small{A}, \small{B}) \leq 1$
\end{definition}
\noindent Jaccard distance is a standard metric for comparing finite
sets (assuming that both sets are non-empty) by comparing the size of
the intersection of two sets over its union. For each model in the
benchmark, the experiments generated 13 potentially different IVCs. Therefore, we
obtained $\binom{13}{2} = 78$ combinations of pairwise distances per
model. Then, minimum, maximum, average, and standard deviation of the
distances were calculated (Figure~\ref{fig:jacdis}), by which, again,
we calculated these four measures among all models. As seen in
Table~\ref{tab:jaccard-avg}, on average, the Jaccard distance between
different solutions is small, but the maximum is close to 1, which
indicates that even for our exploratory analysis, there are models for
which the tools yield substantially diverse solutions. The diversity
between solutions is represented graphically in
Figure~\ref{fig:jacdis}, where for each model, we present the min,
max, and mean pairwise Jaccard distance of the solutions produced by algorithm
\ucalg\ for each model, sorted by the mean distance.

\begin{table}
  \caption{Pairwise Jaccard distances among all models}
  \centering
  \begin{tabular}{ |c|c|c|c| }
    \hline
     min & max & mean & stdev \\[0.5ex]
    \hline
    %sample size = 4196
     0.0   & 0.878 & 0.026 & 0.059 \\[0.5ex]
    \hline
  \end{tabular}
  \label{tab:jaccard-avg}
\end{table}

\begin{figure*}
  \centering
  \vspace{3mm}
  \includegraphics[width=\textwidth]{figs/jacdis2.png} \\
  \vspace{-0.1in}
  \caption{Pairwise Jaccard distance between IVCs}\label{fig:jacdis}
\end{figure*}

\iffalse

To measure the overall similarity among all sets, instead of a pairwise comparison, we used \emph{frequent pattern mining} \cite{han2007frequent}. To define an overall similarity among all sets of support of a given model\footnote{Note that all models in the benchmarks are single property; hence, instead of saying a set of support of a given \emph{property}, we just refer it as the support set of the \emph{model} while explaining the experimental results.}, we calculated a \emph{core} support set for each model in the benchmark, which can be considered as a closed frequent pattern; a core set of model $M$, denoted by $C_M$, is defined as:
\begin{definition}
  \label{def:core}
  $C_M = \bigcap_{i=1}^{13} s_{Mi},   \hspace{9pt} s_{Mi} \in S_M$
\end{definition}

Based on this notion, overall dissimilarity, denoted by $D_{J\{M\}}$, is defined as follows:

\begin{definition}
  \label{def:dis}
  $D_{J\{M\}} =  \frac{\sum_{i=1}^{12}d_J(s_{Mi}, C_M)}{12},   \hspace{9pt} s_{Mi} \in S_M$
\end{definition}

Since our goal is to measure the diversity or dissimilarity among sets computed by \texttt{ReduceSupport}, in \ref{def:dis}, we exclude the set generated by \ucbfalg . In Fig~\ref{fig:jacdis}, the \emph{overall distance} line shows $D_{J\{M\}}$ per model, which can be analyzed from the following hypotheses:
\begin{itemize}
  \item H0: variety of obtained sets of support is high (average $D_{J\{M\}}$ of 0.2)
  \item H1: variety of obtained sets of support is small (average $D_{J\{M\}}$ less than 0.2)
\end{itemize}
Table~\ref{tab:variety} shows that, with an effect size of 0.79, H0 can be rejected.
\begin{table}
  \caption{$D_{J\{M\}}$ among all models}
  \centering
  \begin{tabular}{ |c|c|c|c|c|c| }
    \hline
     min & max & mean & stdev & ES & p-value\\[0.5ex]
    \hline
    %sample size = 395
     0.0   & 0.879 & 0.099 & 0.141 & 0.72 & < 0.00001 \\[0.5ex]
    \hline
  \end{tabular}
  \label{tab:variety}
\end{table}
\fi
%summarized as follows:
%\begin{itemize}
%  \item minimum $D_{J\{M\}}$ among all models: 0.0
%  \item maximum $D_{J\{M\}}$ among all models: 0.879
%  \item average $D_{J\{M\}}$ among all models: 0.096
%  \item standard deviation of $D_{J\{M\}}$ among all models: 0.132
%\end{itemize}
%if we sum all elements of all cores together, that PDR has an smaller core size in aggregate than $k$-induction.
%However, the data is noisy, and to examine \textbf{RQ2.1} systematically, we construct a hypothesis that PDR will, in general, equal or outperform $k$-induction on an arbitrary model:

%\mike{ADD HYPOTHESIS/NULL HYPOTHESIS HERE}

%Although the aggregate data suggests that PDR will yield a smaller core (on average) than $k$-induction, this claim is not supported for a given model with significance.

%we already perform a linear scan of the cores generated by the SMT solver to remove unnecessary conjuncts

\begin{table}
  \caption{Aggregate IVC sizes produced by \ucalg\ using different inductive algorithms and solvers}
  \centering
  \begin{tabular}{ |c|c|c|c| }
    \hline
     solver & PDR & $k$-induction & \textbf{total} \\
    \hline
      Z3 & 2378 & 2379 & 4757 \\
      Yices & 2384 & 2376 & 4760 \\
      MathSAT & 2375 & 2369 & 4744 \\
      SMTInterpol & 2378 & 2368 & 4746 \\
    \hline
      \textbf{total} & 9515 & 9492 &   \\
    \hline
  \end{tabular}
  \label{tab:minimality-algorithm-solvers}
\end{table}



\section{On the efficiency and efficacy of different ``offline'' IVC algorithms}
\label{sec:exp1}
\input{expr1}

\section{On the relative performance of offline and online algorithms for all IVCs.}
\label{sec:exp2}
We are interested in examining the performance of algorithms to compute minimal IVCs.
We examine \textbf{Grow-Shrink}, the algorithm presented in Section \ref{sec:onaivc}, and the two state-of-the-art algorithms: \textbf{Offline MARCO} \ref{sec:offaivc}, and \textbf{Online MARCO} \ref{sec:onaivc} that performs a shrink step prior to returning a solution to ensure minimality.
%\old{We examine three algorithms:
%\textbf{Offline MARCO}, the algorithm from~\cite{Ghass17AllIVCs}, \textbf{Online MARCO}, a variant of the algorithm from~\cite{Ghass17AllIVCs} that performs a shrink step prior to returning a solution to ensure minimality, and \textbf{Grow-Shrink}, the algorithm described in this paper.}
We investigate the following research questions: (RQ1:) For the large models where the complete MIVC enumeration is intractable,
how many MIVCs are found within the given time limit?  (RQ2:) For the tractable models, i.e. models in which all MIVCs are found, how much time is required to complete the enumeration of MIVCs?  Finally, we are interested in how many solver calls are necessary for the enumeration.  Thus, we add (RQ3:) What is the (average) number of solver calls with result adequate/inadequate required by evaluated online algorithms to produce individual MIVCs?

\paragraph{Experimental Setup}:  We start from a benchmark suite that is a superset of the benchmarks used in the previous experiments. This suite contains 660 models, and includes all models that yield a valid result (530 in total) from previous Lustre model checking papers~\cite{Hagen08:FMCAD,piskac2016} and 130 industrial models yielding valid results derived from an infusion pump system \cite{hilt2013} and other sources \cite{piskac2016,NFM2015:backes}.
As this paper is concerned with analysis problems involving multiple MIVCs, we include only models that had more than 4 MIVCs (46 models in total).  To consider problems with many IVCs, we took those models and mutated them, constructing 20 mutants for each model. The mutants varied both in the number and in the size of individual MIVCs.
We added the mutants that still yielded valid results and have more than 5 MIVCs (384 in total) back to the benchmark suite.
Thus, the final suite contains 430 Lustre models. The original benchmarks and our augmented benchmark are available online\footnote{\url{https://github.com/elaghs/benchmarks}}.
%\cite{bench}.

For each test model, we configured \texttt{JKind} to use the \texttt{Z3} solver and the ``fastest'' mode of \texttt{JKind} (which involves running the $k$-induction and PDR engines in parallel and terminating when a solution is found). The experiments were run on a  3.50GHz  Intel(R) i5-4690 processor 16 GB memory machine running Linux with a 30 minute timeout.  All experimental data is available online\footnote{\url{https://github.com/jar-ben/online-mivc-enumeration}}.
%~\cite{expr}.




\begin{figure}[!t]
\centering
\begin{minipage}{.4\textwidth}
\centering
\includegraphics[scale=0.8]{./plots/found_mivcs.pdf}%
\captionof{figure}{Number of MIVCs produced by online algorithms.}%
\label{res:found_mivcs}
\end{minipage}\hfill
\begin{minipage}{.52\textwidth}
\centering
\includegraphics[scale=0.8]{./plots/time_to_complete.pdf}%
\captionof{figure}{Runtime for tractable benchmarks for all algorithms in a log scale.}%
\label{res:time_to_complete}
\end{minipage}
\end{figure}



\subsection{Experimental Results}
In this section, we examine the experimental results to address the research questions.

%\vspace{-5pt}
\paragraph{RQ1 and RQ2:}
Data related to the first two research questions are shown in Figures~\ref{res:found_mivcs} and~\ref{res:time_to_complete}.
Figure~\ref{res:found_mivcs} describes the number of MIVCs found be the two online algorithms in the intractable benchmarks, i.e. the benchmarks where the algorithms did not complete the computation within the time limit. There are 33 such benchmarks. The Grow-Shrink substantially outperforms Online MARCO in the majority of the benchmarks, finding an average of 55\% additional MIVCs.

Figure~\ref{res:time_to_complete} describes the time for each algorithm needed to complete the computation in the case of 397 tractable benchmarks.
Grow-Shrink is on average only 1.08 times slower than Offline MARCO,
yet as previously discussed, has the advantage of returning guaranteed MIVCs, rather than approximate MIVCs. It is on average 1.50 times faster than Online MARCO.


%It is much faster than the Online MARCO algorithm.


%\vspace{-5pt}
\paragraph{RQ3:}  For RQ3, we examined the number of required calls to the solver per MIVC.  For this question, we used the 33 models that contained a large number of MIVCs ($>$70) in order to show the solver efficiency as the number of MIVCs increased.  A point with coordinates $(x,y)$ states that the algorithm needed to perform $y$ solver calls (on average) in order to produce (find) the first $x$ MIVCs. We grouped the calls in terms of the number of calls that returned {\em adequate} vs. {\em inadequate} results.  It is evidenced by the results in Figure~\ref{res:checks}, the new algorithm improves upon Online MARCO as the number of MIVCs becomes larger.

\begin{figure}[!t]
\centering
\begin{subfigure}{.5\textwidth}
  \centering
  \includegraphics[scale=0.8]{./plots/adequate_checks_per_mivc_70.pdf}
  \caption{Checks with result "adequate".}
  \label{res:adequate_checks}
\end{subfigure}%
\begin{subfigure}{.5\textwidth}
  \centering
  \includegraphics[scale=0.8]{./plots/inadequate_checks_per_mivc_70.pdf}
  \caption{Checks with result "inadequate".}
  \label{res:inadequate_checks}
\end{subfigure}
\caption{Average number of performed adequacy checks required to produce individual MIVCs. Note that Figure (b) is in a log scale.}
\label{res:checks}
\end{figure}

The improvement in the number of \emph{inadequate} calls is due the novel shrinking and growing procedures.
Each (approximately) maximal inadequate subset found by the growing procedure allows to save (up to exponentially) many inadequate calls during subsequent executions of the shrinking procedure.
Indeed, the Grow-Shrink algorithm performed on average only 353 inadequate calls to output the first 70 MIVCs, whereas the online MARCO needed to perform 7775 calls to output the same number of MIVCs.

The improvement in the number of \emph{adequate} calls is not so significant as in the case of inadequate calls. Yet, since the adequate calls are usually much more time consuming than inadequate ones, even a slight saving in the number of adequate calls might significantly speed up the whole computation. The Grow-Shrink algorithm saves adequate calls due to the usage of the shrinking queue and due to the invariants that are maintained by the queue. In particular, shall two comparable sets appear in the queue, only the smaller is left. Thus, the algorithm avoids shrinking of relatively large sets and saves some adequate calls.



\section{On the relationship of BVCs and MIVCs}
\label{sec:exp3}
We would like to to observe and study how validity cores evolves over unrolling the transition relation. It is interesting to see how quickly validity cores from a bounded proof converges to an actual minimal IVC.

Note that the purpose of our experiments on BVCs is mostly to point out some research directions. Further studies may even can make use of BVCs in verification problems. It can be the case that a valid property is hard or impossible for an inductive model checker to prove. In such cases, looking at the history of BVC runs may give us some confidence about the correctness of the property. The experimental results show that when we reach one of the actual MIVCs, the BVC algorithm then constantly generates the same cores as depth of exploration increases. That is to say, when the BVC runs begin generating stable cores that do not change as depth changes, it may imply we might have already seen all the reachable states, and implicitly known they are safe. Although this hypothesis is by no means guaranteed to hold, it may be worth further investigations.

\subsection{Experimental Setup}
  We perform our experiments on the same benchmark suite with 660 models introduced in Section \ref{sec:expsetup}. The experiment is conducted with a maximum depth of 10 and one hour timeout; i.e., for each model, if unrolling to depth 10 takes more than one hour, the \bvcalg\ algorithm will terminate. We capture $\bvc _{k}$ for $ 0 \leq k \le 10$, then compare each \bvc\ of depth $k$ to see how they change during unrolling. Then, the final bounded validity cores obtained from at the maximum\footnote{Maximum depth in this experiment is 10. For most of the models, it is possible to reach this depth in less than an hour.}
  reachable depth in one hour, denoted by $\bvc _{max}$ , are considered as our final cores. These cores are compared with all the MIVCs gathered in Section \ref{subsec:res} to see if they match up with any of the actual minimal IVCs.

Research questions we would like to answer in this study are as follows:
\begin{itemize}
  \item \textbf{RQ1:} How many of the final \bvc s match one of the \mivc s?
  %How many of the final BVCs do match up with one of the MIVCs? For how many of the models does the algorithm time out?
  \item \textbf{RQ2:} How do \bvc s evolve as the analysis depth changes?
  %At what rate does size of the BVCs change? Does the size of the cores increase with the depth?
  \item \textbf{RQ3:} Is there a relationship between size and structure of models and the size of \bvc s and the rate at which they converge with a \mivc?
  %How close is $\bvc _{max}$ to an actual MIVC? Is there any relationship with the size of the models and convergence of the BVCs?
\end{itemize}

\vspace{0.1in}
\subsubsection{RQ1}
The result of the experiments show that $\bvc _{max}$ is the same as one of the MIVCs for 474 models out of 660. For 27 of the models, $\bvc _{max}$ was not subset of any MIVCs (had additional elements, also none of the MIVCs was a subset of the $\bvc _{max}$)\footnote{We will explain the reason in \textbf{RQ2}}. However, $\bvc _{max}$ was a subset of one of the MIVCs in 159 of the models.

We performed the experiments with \texttt{Z3} and \texttt{Yices} solvers. UNSAT core generation in \texttt{Z3} is faster than \texttt{Yices} in the current implementation of \texttt{JKind}. Using \texttt{Z3}, 12 of the models did not reach depth 10 in one hour. With \texttt{Yices}, 18 of the models timed out. An interesting fact is that we had models that did not reach $\bvc _{10}$, but their $\bvc _{max}$ was the same as one of the MIVCs. For example, one of the models containing 571 design elements only reached to $\bvc _{2}$, but $\bvc _{2}$ was the same as one of the MIVCs.
The BVC size for that model at different depths is as follows:\footnote{This particular model is named ``steam\_boiler\_no\_arr1.lus'' in our benchmarks. You can see the results and model in our experimental directories \cite{expr}.}

$|\bvc _{0}| = 6$, $|\bvc _{1}| = 11$, $|\bvc _{2}| = 128$

There are interesting case studies where from the initial depth, the BVC was the same as one of the MIVCs. For example, in our benchmark we have a model with 27 design elements\footnote{File ``car\_all\_e8\_856\_e2\_585.lus'' in our benchmark directory.}, for which $|\bvc _{i}| = 5$, $i \leq 0 \le 10$, and $\bvc _{0}$ is the same as its only one MIVC.

\vspace{0.1in}
\subsubsection{RQ2}
Our experimental results show that among 474 models for which $\bvc _{max}$ is the same as one of the MIVCs, the size of the BVCs were (nonstrictly) increasing 99.9\% of the time:
      $$ 0 \leq i \le max, |\bvc _{i}| \leq |\bvc _{i+1}|$$
      In other words, for only 12 of these models, the above relation did not hold.
      It is expected that bounded cores in each unrolling step (nonstrictly) increase as in each step more states are being reached and the cores required for the proof of the property is more likely to expand.
      We run the experiments over those 12 models with different solvers (once with \texttt{Z3} and once with \texttt{Yices}). The result of BVC runs for these models (on \texttt{Yices}) is shown in Table \ref{tab:bvc-abnormal}.

      It is interesting to see why those 12 models show different behavior. By looking at different case studies, we have found three main reasons that explain this anomaly.  The first explanation is that when a model has several distinct MIVCs, the bounded core could change during unrolling. However, the set of 12 models contain models that have only a single MIVC, so this cannot be the entire reason. Another explanation for such models is that MIVCs obtained from \aivcalg ~contained timeout loops; therefore, we do not have the exact minimal IVCs for those cases (for example model\#6 in Table \ref{tab:bvc-abnormal}).




      % (a) shows a picture containing all the models. Figure  \ref{fig:bvc-growth} (b) is the enlarged version for some of the smaller models, and Figure  \ref{fig:bvc-growth} (c) magnifies the parts for larger models.



\begin{table}
  \caption{BVC runs for the models with non-increasing behavior where $\bvc _{max}$ is the same as one of the MIVCs.}

  \centering
  \begin{tabularx}{\linewidth}{ |c||c|c|c|c|c|c|c|c|c|c||L|L|}
    \hline
    $|\bvc _{i}|$ ~/~ $i=$ & 0 & 1 & 2 & 3 & 4 & 5 & 6 & 7 & 8 & 9 & \small{model size} & \small{\#of MIVCs} \\[0.5ex]
    \hline\hline

    model\#1& 2 & 9 & 34 & 36& 28 & 28 & 28 & 28 & 28 & 28 & 70&1 \\[0.5ex]
    model\#2& 5 & 15 & 11& 11& 11 & 11 & 11 & 11 & 11 & 11 & 123 &1\\[0.5ex]
    model\#3& 8 & 9& 13& 33& 28& 40& 38& 41& 41& 41&57 &7 \\[0.5ex]
    model\#4& 2& 5& 8& 10& 12& 10& 10& 10& 10& 10 &64 &9\\[0.5ex]
    \small{model\#5 (\texttt{Yices})}&9& 24 & 84& 84& 82& 82& 82& 82& 82& 82&96 &1\\[0.5ex]
    \small{model\#5 (\texttt{Z3})}& 9& 24& 82& 82& 82& 82& 82& 82& 82& 82&96 &1\\[0.5ex]
    model\#6& 5& 6& 5& 7& 5& 7& 5& 7& 5& 7& 7 &1\\[0.5ex]
    model\#7& 5& 6& 6& 5& 5& 6& 6& 5& 5& 6&6 &1\\[0.5ex]
    \small{model\#8 (\texttt{Yices})}& 9& 12& 14& 28& 37& 36& 36& 36& 36& 36&103&1 \\[0.5ex]
    \small{model\#8 (\texttt{Z3})}&9& 12& 14& 28& 37& 37& 37& 37& 37& 37& 103&1\\[0.5ex]
    \small{model\#9 (\texttt{Yices})}& 2& 6& 10& 4& 4& 4& 4& 4& 4& 4 &64&1 \\[0.5ex]
    \small{model\#9 (\texttt{Z3})}& 2& 4& 4& 4& 4& 4& 4& 4& 4& 4 &64&1\\[0.5ex]
    model\#10& 2& 6& 8& 11& 7& 7& 7& 7& 7& 7 &64&1\\[0.5ex]
    model\#11& 4& 13& 32& 47& 61& 54& 54& 54& 54& 54 &103&8 \\[0.5ex]
 \small{model\#12 (\texttt{Yices})}& 8& 8& 21& 29& 39& 38& 38& 40& 41& 41&57&6\\[0.5ex]
  \small{model\#12 (\texttt{Z3})}& 8& 17& 21& 29& 32& 38& 38& 32& 32& 32&57&6 \\[0.5ex]
    \hline
  \end{tabularx} \\
%{Actual model names in Table \ref{tab:bvc-abnormal}\\model\#1: fast\_1\_e8\_751.lus \\model\#2: microwave05.lus\\mode\#3: DRAGON\_12.lus\\model\#4: Display\_Control-Gaurantee0
%  \\model\#5:fast\_2\_e8\_976.lus
%  \\model\#6: twisted\_counters.lus
%  \\model\#7: two\_counters.lus
%  \\model\#8: cruise\_controller\_04.lus
%  \\model\#9: Display\_Control-Gaurantee2.lus
%  \\model\#10: Display\_Control-Gaurantee1.lus
%  \\model\#11: cruise\_controller\_24.lus
%  \\model\#12: DRAGON\_13.lus}
\vspace{0.07in}
{\small{Actual model names in the benchmark, respectively: fast\_1\_e8\_751.lus, microwave05.lus, DRAGON\_12.lus, Display\_Control-Gaurantee0, fast\_2\_e8\_976.lus, twisted\_counters.lus, two\_counters.lus, cruise\_controller\_04.lus, Display\_Control-Gaurantee2.lus, Display\_Control-Gaurantee1.lus, cruise\_controller\_24.lus, DRAGON\_13.lus}}
  \label{tab:bvc-abnormal}
\end{table}

The third reason why the size of BVCs is not always increasing is more interesting.   This reason explains why in some cases $\bvc _{max}$ is not the subset of any of the MIVCs. This only has to do with the depth of bounded model checking. In some problems, when we are at the earlier steps of unrolling transition relation (i.e., lower depths), the property can be satisfied in different ways. In other words, a property may have multiple \bvc s at depth $k$, but as we advance towards the deeper bounds leading to a proof, the validity cores converge to a smaller subset. For example, consider the toy example in Figure \ref{fig:toybvc}. It shows a simple model containing two counters. The first counter ({\small{\texttt{counter1}}}) has the initial value of 0, and the second counter ({\small{\texttt{counter2}}}) starts off from 6. The property ({\small{\texttt{OK}}}) is either {\small{\texttt{counter1}}} is less than 5 or ({\small{\texttt{counter2}}}) is greater than 5. This property has only one MIVC, which is {\small{\texttt{\{counter2, OK\}}}}. However, before depth 5, this property can be satisfied into ways with {\small{\texttt{\{counter1, OK\}}}} and {\small{\texttt{\{counter2, OK\}}}}. You can see the output of the BVC engine for this example in Figure \ref{fig:toyo1}.

\begin{figure}
 \centering
  \includegraphics[width=0.8\columnwidth]{figs/toybvc.jpg}
  %\vspace{-0.1in}
  \caption{A toy example that shows multiple BVCs at earlier depths for a property with single MIVC}
  \vspace{0.1in}
  \label{fig:toybvc}
\end{figure}

\begin{figure}
 \centering
  \includegraphics[width=0.8\columnwidth]{figs/toyo1.jpg}
  %\vspace{-0.1in}
  \caption{BVCs for the property in Figure \ref{fig:toybvc}}
  \vspace{0.1in}
  \label{fig:toyo1}
\end{figure}



Let us take a look at one of the models that is small enough to display its results in a reasonable amount of space. This model is named \emph{ex3\_e8\_381\_e7\_224} in our benchmarks (Figure \ref{fig:expl}). It only has one MIVC ({\small{\texttt{\{V19\_late, V64\_incr, V63\_diff, V65\_PC, OK\}}}}), and \jkind is able to prove its property in less than a second. In addition, for this model, there is no timeout issue in the inner loops of \aivcalg\ algorithm. Using \texttt{Yices} in our experiments, $\bvc _{max}$ is not the subset of the only MIVC that this model has. However if we had just increased the depth by 1, from $\bvc _{10}$ on, the BVC would have become the same as the MIVC. For this model, up to depth 10, we have two ways of satisfying the property (we have two bounded validity cores {\small{\texttt{\{V19\_late, V64\_incr, V63\_diff, V65\_PC, OK\}}}} and {\small{\texttt{\{V20\_early, V64\_incr, V63\_diff, V65\_PC, OK\}}}}), but after depth 10, the property is satisfied with only one validity core ({\small{\texttt{\{V19\_late, V64\_incr, V63\_diff, V65\_PC, OK\}}}}). Figure \ref{fig:explout} shows the output of the \jkind BVC engine over this model up to depth 15.

 \begin{figure}
 \centering
  \includegraphics[width=0.9\columnwidth]{figs/expl.jpg}
  %\vspace{-0.1in}
  \caption{Model ex3\_e8\_381\_e7\_224 as a case study}
  \vspace{0.1in}
  \label{fig:expl}
\end{figure}

\begin{figure}
 \centering
  \includegraphics[width=0.9\columnwidth]{figs/explout.png}
  %\vspace{-0.1in}
  \caption{BVC runs for model ex3\_e8\_381\_e7\_224}
  \vspace{0.1in}
  \label{fig:explout}
\end{figure}


\vspace{0.1in}
\subsubsection{RQ3}
In order to show how quickly BVCs change and converge to an actual MIVC, we chose $\bvc _{0}$, $\bvc _{3}$, and $\bvc _{max}$  runs and plot the size of the cores. Mostly for models with less than 200 design elements, size of BVCs did not change much from depth 3 to 9. For the larger models there is some difference between the size of $\bvc _{3}$ and $\bvc _{max}$ (Figure \ref{fig:bvc-growth}).


 \begin{figure}
 \centering
  \includegraphics[width=.85\columnwidth]{figs/bvcmax.png}
  %\vspace{-0.1in}
  \caption{Size of BVCs at depth 3 and max}
  \vspace{0.1in}
  \label{fig:bvc-growth}
\end{figure}


We calculated the difference of $\bvc _{max}$ of each model with its MIVCs. Part of the results is described in \textbf{RQ1}. If $\bvc _{max}$  is the subset of one of the MIVCs, we calculated the difference between those two, and if not, $\bvc _{max}$  is compared with one of the MIVCs of the model, selected randomly.  Note that it is possible for $\bvc _{max}$ to be a subset of more than one of the MIVCs. In our calculation, we randomly selected the first MIVC containing $\bvc _{max}$.
%Figure \ref{fig:dif-bvc} visualizes the results.
Figure \ref{fig:bvc-size} shows the size of the models versus the size differences of MIVCs and BVCs
%line from Figure \ref{fig:dif-bvc}
in logarithmic scale.

% \begin{figure}
% \centering
%  \includegraphics[width=\columnwidth]{figs/bvc_dif_yices.png}
%  %\vspace{-0.1in}
%  \caption{Difference between $\bvc _{max}$ and MIVCs}
%  \vspace{0.1in}
%  \label{fig:dif-bvc}
%\end{figure}

 \begin{figure}
 \centering
  \includegraphics[width=\columnwidth]{figs/bvc_modelsize_yices.png}
  %\vspace{-0.1in}
  \caption{Difference between $\bvc _{max}$ and MIVCs vs the model size}
  \vspace{0.1in}
  \label{fig:bvc-size}
\end{figure}

\vspace{0.1in}
\subsubsection{Discussion}
Experimental results show that in many cases, bounded validity cores can be as accurate as actual minimal IVCs. The abnormal behavior in some cases showed that we cannot make a strong claim about the relationships between BVCs and IVCs. One observation is that at deeper bounds, we can have more accurate bounded validity cores. The more accurate bounded cores are, the more useful information we have to evaluate completeness and adequacy of proofs.

It may be possible to make use of other techniques to evaluate the accuracy of the bounded cores. For example, in case of non-increasing BVCs, we may build different abstractions for the model using different BVCs, and try to prove the property over abstracted models. If property fails over one abstraction, we can easily rule out the bounded core used for that abstraction. For example, in Figure \ref{fig:toybvc}, if we terminate the BVC generation at depth 4, we need to decide which of {\small{\texttt{\{counter1, OK\}}}} and {\small{\texttt{\{counter2, OK\}}}} is necessary for an inductive proof (i.e., is the subset of our MIVC). In order to do so, if we build a new abstract model using BVC {\small{\texttt{\{counter1, OK\}}}} (Figure \ref{fig:absbvc} (a)), and try to re-prove the property, property will be violated, which tells us   {\small{\texttt{counter2}}} was necessary for the proof of our property. In the same way, if we build an abstract model using BVC  {\small{\texttt{\{counter2, OK\}}}} (Figure \ref{fig:absbvc} (b)) and try to re-prove the property, we will see the property is still valid, which means  {\small{\texttt{counter1}}} is not in MIVC of the property.

  \begin{figure}
 \centering
  \includegraphics[width=\textwidth]{figs/absbvc.jpg}
  %\vspace{-0.1in}
  \caption{Building abstraction using BVCs}
  \vspace{0.1in}
  \label{fig:absbvc}
  \vspace{0.1in}
\end{figure}


Another observation is that if we calculate \emph{all} BVCs at a given depth for a valid property, there should be at least one BVC, which is the subset of one of the MIVCs. As we advance towards deeper bounds when BVCs stay the same, one may even conclude that BVC has already converged to an actual minimal IVC. However, this conclusion is not always accurate. It is possible that final MIVC is a superset of multiple BVCs at different bounds. Consider the toy example in Figure \ref{fig:toybvc}. If the initial value of {\small{\texttt{counter2}}} changes from 6 to 3, the MIVC of the property will be {\small{\texttt{\{counter1, counter2, OK\}}}} because the correctness of {\small{\texttt{OK}}} is dependent on both counters. However, if we obtain the BVCs for this new model (Figure \ref{fig:toybvc2}), you will see that BVC up to depth 5 is  {\small{\texttt{\{counter1, OK\}}}}, thereafter becomes {\small{\texttt{\{counter2, OK\}}}} and will not change (Figure \ref{fig:toyo2}).

\begin{figure}
 \centering
  \includegraphics[width=0.70\columnwidth]{figs/toybvc2.jpg}
  %\vspace{-0.1in}
  \caption{A toy example where MIVC is the superset of multiple unique BVCs}
  \vspace{0.1in}
  \label{fig:toybvc2}
\end{figure}

\begin{figure}
 \centering
  \includegraphics[width=0.8\columnwidth]{figs/toyo2.jpg}
  %\vspace{-0.1in}
  \caption{BVCs for the property in Figure \ref{fig:toybvc2}}
  \vspace{0.1in}
  \label{fig:toyo2}
\end{figure} 
