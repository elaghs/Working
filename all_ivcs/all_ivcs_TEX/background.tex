\section{Preliminaries}
\label{sec:background}

\newcommand{\bool}[0]{\mathit{bool}}
\newcommand{\reach}[0]{\mathit{R}}
\newcommand{\ite}[3]{\mathit{if}\ {#1}\ \mathit{then}\ {#2}\ \mathit{else}\ {#3}}

%\subsection{Transition Systems and Safety Properties}
This section defines the terms and concepts used in the paper. Given a state space $U$, a transition system $(I,T)$ consists of an
initial state predicate $I : U \to \bool$ and a transition step
predicate $T : U \times U \to \bool$. 
We define the notion of
reachability for $(I, T)$ as the smallest predicate $\reach : U \to
\bool$ which satisfies the following formulas:
\begin{gather*}
  \forall u.~ I(u) \Rightarrow \reach(u) \\
  \forall u, u'.~ \reach(u) \land T(u, u') \Rightarrow \reach(u')
\end{gather*}
A safety property $P : U \to \bool$ is a state predicate. A safety
property $P$ holds on a transition system $(I, T)$ if it holds on all
reachable states, i.e., $\forall u.~ \reach(u) \Rightarrow P(u)$,
written as $\reach \Rightarrow P$ for short. When this is the case, we
write $(I, T)\vdash P$. To generalize the notion of transition system, especially in SMT-based model checking, it is assumed that the transition relation has the structure of a top-level conjunction. This assumption gives us a structure that we can easily manipulate. Given $T(u, u') = T_1(u, u') \land \cdots \land T_n(u, u')$ we will write $T = T_1 \land \cdots \land T_n$ for short.
By further abuse of notation,
$T$ is identified with the set of its top-level conjuncts. Thus, $x \in
T$ means that $x$ is a top-level conjunct of $T$, and $S
\subseteq T$ means all top-level conjuncts of $S$ are top-level
conjuncts of $T$. When a top-level conjunct $x$ is removed from $T$, it is written as $T \setminus \{x\}$. Such a transition system can easily encode our example model in Section~\ref{sec:example}.  We assume each equation defines a conjunct within the transition system which we will denote by the variable assigned, so $T = \{$ {\small \texttt{a1\_below, a2\_below, c1, c2, below, alarm, p1, p2, doi\_on}} $\}$.

\begin{definition}{\emph{Minimal Inductive Validity Core (IVC):}}
  \label{def:minimal-ivc}
  $S \subseteq T$ for $(I, T)\vdash P$ is minimal Inductive Validity Core, denoted $IVC(P, S)$, iff
  $(I, S) \vdash P \wedge (\neg\exists S' \subset S .\quad (I, S') \vdash P) $.
\end{definition}

For brevity, we use term ``IVC'' refers to a minimal IVC. Note that there is always at least one IVC, and there could be many IVCs for a given property, corresponding to different proofs. To capture that notion, we define \emph{all IVCs ($AIVC$)} for a property as an association to all its IVCs.

$$ AIVC(P) \equiv  \{\ S~|~S \subseteq T \land  IVC(P, S)\} $$

 
%
%\begin{figure}
%\begin{gather*}
%I(s_0) \Rightarrow P(s_0) \\[-2pt]
%%
%\vdots \\[2pt]
%%
%I(s_0) \land T(s_0, s_1) \land \cdots \land T(s_{k-2}, s_{k-1})
%\Rightarrow P(s_{k-1}) \\[2pt]
%%
%P(s_0) \land T(s_0, s_1) \land \cdots \land P(s_{k-1}) \land
%T(s_{k-1}, s_k) \Rightarrow P(s_k)
%\end{gather*}
%\caption{$k$-induction formulas: $k$ base cases and one inductive
%  step}
%\label{fig:k-induction}
%\end{figure}
 
 
