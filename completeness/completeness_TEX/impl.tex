\section{Implementation}
\label{sec:impl}

To implement our method, we have made use of the technique proposed in \cite{Ghass16} for the efficient computation of IVCs, which are the same as elements of a support set. In \cite{Ghass16}, there are two detailed algorithms for computing minimal support sets: (1) Algorithm \ucbfalg that computes a truly minimal support set for a given property, which is expensive. (2) Algorithm \ucalg that computes support sets fairly close to minimal much more efficiently. In this section, we provide a high-level representation of these algorithms.

Intuitively, Algorithms \ref{alg:uc},
first obtains an inductive invariant from the inductive proof of
property $r$ (line 1 in Algorithm \ref{alg:uc}). Then, using function $map\_to\_design$, it
computes the (closely) minimal support set of $r$ \footnote{The way that support set
(or inductive validity cores) can be produced is described in \cite{Ghass16}. Here we abstract away the details.}.

Since \ucalg might not produce an \emph{exact} minimal support set,
Algorithm \ref{alg:ucbf}, \ucbfalg, includes a minimization process that
operates on the result of \ucalg.

Algorithm \ref{alg:must} is also an efficient way of computing the \emph{must} set of a given property using \ucalg. Note that, as discussed in Section \ref{sec:method}, the
output of \ucalg (as well as \ucbfalg) for $\Gamma \vdash R$ pinpoints the covered design elements
based on Definition \ref{def:coverage-ivc}.
In other words, it is equal to
$C = \{\varphi | \varphi \in \Gamma \wedge  \zeta_{ivc} \preccurlyeq \varphi \}$.
And, the output of \mustalg represents covered design artifacts according to Definition \ref{def:coverage1}, which is the same as $C = \{\varphi | \varphi \in \Gamma \wedge \zeta_{nc} \preccurlyeq \varphi \}$.

\begin{algorithm}
  \SetKwInOut{Input}{input}
  \SetKwInOut{Output}{output}
  \Input{$\Gamma \vdash r$}
  \Output{Support set for $\Gamma \vdash r$}
  \BlankLine
  $Inv \leftarrow get\_inductive\_invariant(\Gamma, r)$ \\
  $S \leftarrow map\_to\_design (\Gamma, Inv)$ \\
  \Return{S}
\caption{An abstract representation of \ucalg \cite{Ghass16}}
\label{alg:uc}
\end{algorithm}

\begin{algorithm}
  \SetKwInOut{Input}{input}
  \SetKwInOut{Output}{output}
  \Input{$\Gamma \vdash r$}
  \Output{Minimal support set for $\Gamma \vdash r$}
  \BlankLine
  $S \leftarrow \ucalg(\Gamma \vdash r)$ \\
  \For{$x \in S$} {
    \If{$S\setminus\{x\} \vdash r$}{
      $S \leftarrow S\setminus \{x\}$
    }
  }
  \Return{S}
\caption{An abstract representation of \ucbfalg \cite{Ghass16}}
\label{alg:ucbf}
\end{algorithm}


\begin{algorithm}
  \SetKwInOut{Input}{input}
  \SetKwInOut{Output}{output}
  \Input{$\Gamma \vdash r$}
  \Output{Must set for $\Gamma \vdash r$}
  \BlankLine
  $S \leftarrow \ucalg(\Gamma \vdash r)$ \\
  $M \leftarrow \varnothing$ \\
  \For{$x \in S$} {
    \If{$S\setminus\{x\} \nvdash r$}{
      $M = M \cup \{x\}$
    }
  }
  \Return{M}
\caption{\mustalg: an algorithm to compute $MUST(r)$ for a given $r$}
\label{alg:must}
\end{algorithm}

We have implemented the three algorithms introduced in this section in JKind \cite{jkind}, which is an infinite-state model checker for safety properties. JKind proves safety properties using multiple cooperative engines in parallel including k-induction \cite{SheeranSS00}, property directed reachability \cite{Een2011:PDR}, and template-based lemma generation \cite{Kahsai2011}. JKind accepts
Lustre programs written over the theory of linear integer and real
arithmetic. In the back-end, JKind uses an SMT solver such as
Z3 \cite{DeMoura08:z3}, Yices \cite{Dutertre06:yices},
MathSAT \cite{Cimatti2013:MathSAT}, or SMTInterpol \cite{Christ2012:SMTInterpol}.
JKind works on multiple properties simultaneously. When a
property is proven and IVC generation is enabled, an additional
parallel engine executes Algorithm \ref{alg:uc} to generate a nearly minimal
IVC (or support) set.

Although the minimality of $IVC$ sets makes $\zeta_{ivc}$ accurate
in terms of both preserving provability and not having false positives, the exact implementation of $\zeta_{ivc}$ is based on the \ucbfalg algorithm, which is as nearly expensive as the \mustalg algorithm for $\zeta_{nc}$. To alleviate this issue, we came up with an efficient implementation for $\zeta_{ivc}$, i.e. \ucalg \cite{Ghass16},
 which is an over-approximation and might not be always accurate in terms of minimality.

 As mentioned in Section \ref{sec:impl}, IVC sets are derived from inductive invariants; in other words, they are built upon the proof of the validity of a given property. One interesting fact about proofs
  is that a given property could be proved from different proof paths. That's why we defined $AIVC(r)$ in Section \ref{sec:background}. The $AIVC$ set gives a clear picture of various ways a property is satisfied. By getting all the IVC sets for all requirements of the system and categorizing them, one can find if there are target artifacts that do not trace to any requirement: set $\bigcap \{IRR (r) | r \in \Delta \}$.  If this set is non-empty, it is a possible indication of ``gold plating" or missing requirements. That is to say, it helps to assess if the requirements of the system describe all the behaviors of the system. Being able to measure the coverage of requirements over the model is crucial in the safety critical system domain.

Very recent, as yet unpublished, work has focused on the
generation of all IVC sets (IVC sets), whose preliminary evaluation
shows the overhead in discovering all IVC sets is a linear in the
number of unique IVC set in the problem multiplied by the cost
for finding a proof for a single IVC set. For complex models, such
as the ones described in \cite {QFCS15:backes} and \cite{hilt2013}, it has been possible to
find $AIVC$ for individual properties in a matter of minutes.
Based on our preliminary results we expect computing $AIVC$ to be computationally feasible for complex models. In
addition, we believe that it is possible to use the information
from the set of all IVCs to more efficiently produce minimal
IVCs than the \ucbfalg algorithm.
