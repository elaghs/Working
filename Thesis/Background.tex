\chapter{Preliminaries}
\label{ch:background}
The idea of inductive validity cores is applicable to the context of symbolic model checking using inductive proof methods. After proving the correctness of a given property, we extract a minimal portion of the system (model) necessary for the proof of the property, which is what we call IVCs. IVCs determine why the property is satisfied by the system. Since this information is obtained from the inductive proofs, we call it \emph{inductive} validity core. With minimal IVCs, we are able to abstract away the part of the system irrelevant to the proof of the property. This chapter mentions some background we need for a formal description of the IVC notion.

\section{Safety Verification}

Given a state space $U$, a transition system $(I,T)$ consists of an
initial state predicate $ I : U \to \bool $ and a transition step
predicate $ T : U \times U \to \bool $.
We define the notion of
reachability for $(I, T)$ as the smallest predicate $\reach : U \to
\bool$ which satisfies the following formulas:

\begin{gather*}
  \forall u.~ I(u) \Rightarrow \reach(u) \\
  \forall u, u'.~ \reach(u) \land T(u, u') \Rightarrow \reach(u')
\end{gather*}

A safety property $P : U \to \bool$ is a state predicate. A safety
property $P$ holds on a transition system $(I, T)$ if it holds on all
reachable states, i.e., $\forall u.~ \reach(u) \Rightarrow P(u)$,
written as $\reach \Rightarrow P$ for short. When this is the case, we
write $(I, T)\vdash P$.

For an arbitrary transition system $(I, T)$, computing reachability
can be very expensive or even impossible. Thus, we need a more
effective way of checking if a safety property $P$ is satisfied by the
system. The key idea is to over-approximate reachability. If we can
find an over-approximation that implies the property, then the
property must hold. Otherwise, the approximation needs to be refined.

A good first approximation for reachability is the property itself.
That is, we can check if the following formulas hold:
\begin{gather}
  \forall u.~ I(u) \Rightarrow P(u)
  \label{eq:1-ind-base} \\
  \forall u, u'.~ P(u) \land T(u, u') \Rightarrow P(u')
  \label{eq:1-ind-step}
\end{gather}
If both formulas hold then $P$ is {\em inductive} and holds over the
system. If (\ref{eq:1-ind-base}) fails to hold, then $P$ is violated
by an initial state of the system. If (\ref{eq:1-ind-step}) fails to
hold, then $P$ is too much of an over-approximation and needs to be
refined.

One way to refine our over-approximation is to add additional lemmas
to the property of interest. For example, given another property $L :
U \to bool$ we can consider the extended property $P'(u) = P(u) \land
L(u)$, written as $P' = P \land L$ for short. If $P'$ holds on the
system, then $P$ must hold as well. The hope is that the addition of
$L$ makes formula (\ref{eq:1-ind-step}) provable because the
antecedent is more constrained. However, the consequent of
(\ref{eq:1-ind-step}) is also more constrained, so the lemma $L$ may
require additional lemmas of its own. Finding and proving these
lemmas is the means by which property directed reachability (PDR)
strengthens and proves a safety property~\cite{Een2011:PDR}.

Another way to refine our over-approximation is to use use {\em
  $k$-induction} which unrolls the property over $k$ steps of the
transition system. For example, 1-induction consists of formulas
(\ref{eq:1-ind-base}) and (\ref{eq:1-ind-step}) above, whereas
2-induction consists of the following formulas:
\begin{gather*}
\forall u.~ I(u) \Rightarrow P(u) \\
\forall u, u'.~ I(u) \land T(u, u') \Rightarrow P(u') \\
\forall u, u', u''.~ P(u) \land T(u, u') \land P(u') \land T(u',
  u'') \Rightarrow P(u'')
\end{gather*}
That is, there are two base step checks and one inductive step check.
In general, for an arbitrary $k$, $k$-induction consists of $k$
base step checks and one inductive step check as shown in
Figure~\ref{fig:k-induction} (the universal quantifiers on $u_i$ have
been elided for space). We say that a property is $k$-inductive if it
satisfies the $k$-induction constraints for the given value of $k$.
The hope is that the additional formulas in the antecedent of the
inductive step make it provable.

\begin{figure}
\begin{gather*}
I(u_0) \Rightarrow P(u_0) \\[-2pt]
%
\vdots \\[2pt]
%
I(u_0) \land T(u_0, u_1) \land \cdots \land T(u_{k-2}, u_{k-1})
\Rightarrow P(u_{k-1}) \\[2pt]
%
P(u_0) \land T(u_0, u_1) \land \cdots \land P(u_{k-1}) \land
T(u_{k-1}, u_k) \Rightarrow P(u_k)
\end{gather*}
\caption{$k$-induction formulas: $k$ base cases and one inductive
  step}
\label{fig:k-induction}
\end{figure}

In practice, inductive model checkers often use a combination of the
above techniques. Thus, a typical conclusion is of the form ``$P$ with
lemmas $L_1, \ldots, L_n$ is $k$-inductive''.


%%% Local Variables:
%%% mode: latex
%%% TeX-master: "main.tex"
%%% End

%%  LocalWords:  bool reachability \texttt{JKind} Lustre PDR Yices MathSAT ok
%%  LocalWords:  SMTInterpol dataflow init
