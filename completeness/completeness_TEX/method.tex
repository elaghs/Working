\section{Method}
\label{sec:method}

We believe Definition \ref{def:coverage1} characterizes coverage in a way which is both expensive to compute and difficult to satisfy (i.e. it usually leads to low coverage scores). This section proposes a novel notion of coverage which is not only more efficient and practical to compute but also  is an immediate guidance of what is necessary for specification.

First, let state the relationship between the most recent coverage notions in formal verification and the idea of support sets.

\begin{lemma}
  \label{lem:sos-must-may}
 $\forall S \in SOS(r)$, $S$ can be partitioned into two disjoint sets $S_1$ and $S_2$
  such that $S_1 = MUST(r)$ and $S_2 \subseteq MAY(r)$.
\end{lemma}
\begin{proof}
 From the definition of $ASOS$ and $MUST$, $MUST(r) \subseteq S$, for $(r, S) \in SOS$. And
 by definition, $MUST(r) \cap MAY(r) = \varnothing$. Following the definition of $SOS$
 and $MAY$, $S = (MAY(r) \cap S) \cup MUST(r)$.
\end{proof}
\vspace{2mm}

\begin{lemma}
  \label{lem:must-not-enough}
  If $MAY(r) \neq \varnothing$, then $MUST(r) \nvdash r$.
\end{lemma}
\begin{proof}
 Immediate from Lemma \ref{lem:sos-must-may}, and the definition of $SOS$ and provability.
\end{proof}
\vspace{2mm}

\begin{lemma}
  \label{lem:must-coverage}
  Given function $\psi_{sm}(r, \varphi)$, $\forall \varphi \in \Gamma$ iff
  $\psi_{sm} (r) \preccurlyeq \varphi$ then  $\varphi \in MUST(r)$.

\end{lemma}
\begin{proof}
 From the definition of $MUST(r), \forall \varphi \in MUST (r)$, $\forall S \subseteq \Gamma$. $(r, S) \in SOS \Rightarrow \varphi \in SOS(r, S)$,
 which implies $f_m (S \setminus \{ \varphi \}) \nvdash r$.
 On the other hand, considering the definition of $\psi_{sm}$, if
 $f_m (S \setminus \{ \varphi \}) \nvdash r$ then $\varphi$ is necessary to prove $r$ (Lemma \ref{lem:must-not-enough}), which means $\varphi \in MUST(r)$.
\end{proof}
\vspace{2mm}

In light of Lemma \ref{lem:must-coverage}, the coverage score of specification $r$ obtained based on $\psi_{sm}$ can be also calculated by
$$\frac{|MUST(r)|}{|\Gamma|}$$
Therefore, for set of requirements $\Delta$, the coverage score is computed by $$\frac{|MUST(R)|}{|\Gamma|},\xspace  R = \bigwedge_{i} {r_i \in \Delta}$$

\begin{definition} {\emph{Justifiable notion of coverage:}} \\
  \label{def:coverage-ivc}
  Given $R = \bigwedge_{i} {r_i \in \Delta}$ and $S \in ASOS(R)$, justifiable coverage is formalized with function $\psi$ such that  $\forall \varphi \in S$. $\psi (R) \preccurlyeq \varphi$
  and $\forall \lambda \notin S$. $\psi (R) \nprec \lambda$.
\end{definition}
\vspace{2mm}

For the sake of simplicity, we refer to the coverage function
formalized in Definition \ref{def:coverage-ivc} as $\psi_{sos}$.

We call Definition \ref{def:coverage-ivc} \emph{justifiable} because, with a set of the model elements marked as covered by $\psi_{sos}$, every $r \in \Delta$ is provable, whereas a set of covered elements obtained from $\psi_{sm}$ is not sufficient to reconstruct the proofs of requirements in $\Delta$.\footnote{Throughout the paper, when a coverage function is justifiable, like $\psi_{sos}$, it is said that it preserves provability of the requirement.}
This leads the coverage score for $\psi_{sos}$ to be usually higher than the score for $\psi_{sm}$. Coverage score for $\psi_{sos}$ can be calculated with $$\frac{|S|}{|\Gamma|}$$

\begin{theorem}
\label{thm:cov-must}
Given $C = \{\varphi | \varphi \in \Gamma \wedge \psi_{sm} \preccurlyeq \varphi \}$
and \\ $R = \bigwedge_{i} {r_i \in \Delta}$, $C \nvdash R$.
\end{theorem}
\begin{proof}
Immediate from Lemma \ref{lem:must-not-enough} and Lemma \ref{lem:must-coverage}.
\end{proof}
\vspace{2mm}
Theorem \ref{thm:cov-must} demonstrates that $\psi_{sm}$ does not preserve provability.
With Theorem \ref{thm:cov-sos} and Corollary \ref{cor:cov-sos},
we show $\psi_{sos}$ maintains provability of all the requirements.
Theorem \ref{thm:sos-r} proves the sound relationship
between $\psi_{sos}$ and minimal support sets. 
It emphasizes that $\psi_{sos}$ is accurate meaning that it does not result in false positives 
(i.e. does not assign \emph{actual} uncovered elements to covered) because $SOS$ is \emph{minimal}.

\begin{theorem}
\label{thm:cov-sos}
Given $C = \{\varphi | \varphi \in \Gamma \wedge  \psi_{sos} \preccurlyeq \varphi \}$
and \\ $R = \bigwedge_{i} {r_i \in \Delta}$, $C \vdash R$.
\end{theorem}
\begin{proof}
By Definition, $\psi_{sos} \preccurlyeq \varphi_i$ implies that $\exists S_i \in ASOS(R)$. $\varphi_i \in S_i$.
In addition, since $(R, S_i) \in SOS \Rightarrow S_i \vdash R$, and the union of such $S_i$ sets
forms $C$,
$C \vdash R$.
\end{proof}
\vspace{2mm}

\begin{coroll}
\label{cor:cov-sos}
Given $C = \{\varphi | \varphi \in \Gamma \wedge  \psi_{sos} \preccurlyeq \varphi \}$, \\
$\forall r_i \in \Delta$. $C \vdash r_i$.
\end{coroll}
\begin{proof}
Immediate from Theorem \ref{thm:cov-sos}.
\end{proof}
\vspace{2mm}

\begin{theorem}
\label{thm:sos-r}
If $\psi_{sos} \preccurlyeq \varphi$ then $\exists r \in \Delta$, $S \in ASOS(r)$.
$S \setminus \{\varphi \} \nvdash r$.
\end{theorem}
\begin{proof}
Immediate from Corollary \ref{cor:cov-sos} and the definition of $ASOS$ and provability.
\end{proof}
\vspace{2mm}
To illustrate concepts described in this section, we use the the example in Fig. \ref{fig:ex};
using $\psi_{sos}$, either \{{\tt P}, {\tt c2}, {\tt c3}\} or
 \{{\tt P}, {\tt x}, {\tt c3}\} will be considered as covered. Note that
 $\psi_{sos}$ cannot detect {\tt c2} and {\tt x} as covered simultaneously.
 The fact that which of them will be marked as covered depends on
 the path chosen by the solver to prove {\tt P}. In other words,
 if the proof of {\tt P} is established through the elements of \{{\tt P}, {\tt x}, {\tt c3}\},
 then {\tt x} is covered. Otherwise, since there are only two proof paths (or $SOS({\tt P})$-es)
 here, {\tt c2} would be considered covered. However, this is less restrictive than $\psi_{sm}$ whereby only $MSUT({\tt P})$ is covered. In Section \ref{sec:discussion}, we will provide some other coverage notions to address this issue.
 Another important difference between $\psi_{sm}$ and $\psi_{sos}$ is that, unlike $\psi_{sm}$,
 $\psi_{sos}$ preserves provability. In our example, no matter which $SOS$ will be used;
 in both cases, the set of covered elements by $\psi_{sos}$ is sufficient to re-construct a proof of {\tt P}.
 However, as you can see, the covered set of elements obtained from $\psi_{sm}$ is not
 enough to establish a proof for the validity of {\tt P}.
