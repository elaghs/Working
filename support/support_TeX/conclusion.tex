\section{Conclusions \& Future Work}
\label{sec:conc}

In this paper, we have defined the notion of an inductive validity core (IVC) which
appears to be a useful measure in relation to a valid safety property
for inductive model checking. We have presented a novel algorithm for
computing IVCs that are nearly minimal and have shown that full
minimality is undecidable in many settings. Our algorithm is
applicable to all forms of inductive SAT/SMT-based model checking
including $k$-induction, property directed reachability (PDR), and
interpolation-based model checking.

We have implemented our IVC algorithm as part of the open source model
checker JKind. We have shown that the algorithm requires only a
moderate overhead and produces nearly minimal IVCs in practice.
Moreover, the produced IVCs are fairly stable with respect to
underlying proof engines ($k$-induction and PDR) and back-end SMT
solvers (Yices, Z3, MathSAT, SMTInterpol).

Our work has recently been integrated into the AADL/AGREE tool suite~\cite{QFCS15:backes,hilt2013} in two ways: first, it can be used to automatically compute traceability information between high-level and low-level requirements in compositional proofs.  Second, it has been added into a symbolic simulator for AGREE tool suite where it is used to explain conflicts when the simulator is not able to compute a "next state" for a set of chosen constraints.  

In future work, we plan to examine how the generated traceability matrices differ from those produced by human experts and by automated heuristic approaches.  A recently-started pilot project for formal system architecture design at Rockwell Collins is examining the traceability information produced by the AGREE tool using IVCs.  We also will examine the impact of multiple distinct IVCs on traceability research.  An initial paper on this work, which we call {\em complete traceability} has been accepted to the RE@Next! track of the Requirements Engineering conference~\cite{Murugesan16:renext}.  We do not as yet have a general idea why results are diverse, and are interested in exploring this further (does it indicate fault-tolerance? requirements redundancy?).  For completeness, we will compare our approach against other approaches towards measuring completeness of requirements (such as those in~\cite{chockler_coverage_2003, Kupferman:2006:SCF, kupferman_theory_2008}) both theoretically and also empirically, in terms of computation time and similarity of results on our benchmark corpus.

Finally, we plan to investigate algorithms for exploring the space of IVCs such as finding distinct minimal IVCs or finding minimum (rather than minimal) IVCs. Such algorithms could be used to automatically measure the redundancy of systems with respect to their safety properties, and give a better indication of requirements diversity.  

%% \begin{itemize}
%%     \item Write this at the end.
%%     \ela{We can add something about fault tolerance for future work.
%%     if we have all sets of support, and their intersection is empty, we have redundancy.
%%     we can talk about algorithm for minimum support set...}
%% \end{itemize}

%%  LocalWords:  IVC IVCs PDR Yices MathSAT SMTInterpol
