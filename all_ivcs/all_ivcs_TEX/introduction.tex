\section{Introduction}
\label{sec:intro}

Different notions of coverage have been well defined in software testing, however, in formal verification, it is very complex to define and compute this notion. 
Usually, coverage techniques in the property-based verification try to measure the quality of the specification with regards to the completeness of a set of properties. 
In fact, the goal is to point out unspecified behaviors, hence the idea behind most of the existing work is to address the question of `have we specified enough properties (requirements)?' 
Since the coverage notions are usually  and over-approximation, achieving a high coverage does not guarantee there will be no missing behavior. However, when the coverage is low, techniques will definitely reveal some unspecified cases \cite{claessen2007coverage}. 