\section{Implementation}
\label{sec:impl}

We have implemented all of the inductive validity core algorithms from the previous section in an industrial model checker called \jkind~\cite{jkind},
which verifies safety properties of infinite-state synchronous systems.
These tools operate over the Lustre language~\cite{Halbwachs91:lustre}, which we briefly illustrate below.

\subsection{Lustre and IVCs}
\label{sec:lustre}

Lustre~\cite{Halbwachs91:lustre} is a synchronous dataflow language
used as an input language for various model checkers. The textual
models in Figures~\ref{fig:ex-before}, \ref{fig:ex-after}, and \ref{fig:asw} are
written in Lustre. We will use model in Figure~\ref{fig:ex-before} as
a running example in this section. For our purposes, a Lustre program
consists of 1) input variables, {\tt x} in the example, 2) output
variables, {\tt a}, {\tt b}, and {\tt y} in the example, and 3) an
equation for each output variable. A Lustre program runs over discrete
time steps. On each step, the input variables take on some values and
are used to compute values for the output variables on the same step.
In addition, equations may refer to the previous value of a variable
using the {\tt pre} operator. This operator is underspecified in the
first step, so the arrow operator, {\tt ->}, is used to guard the
{\tt pre} operator. In the first step the expression {\tt e1 -> e2}
evaluates to {\tt e1}, and it evaluates to {\tt e2} in all other steps.

We interpret a Lustre program as a model specification by considering
the behavior of the program under all possible input traces. Safety
properties over Lustre can then be expressed as Boolean expressions in
Lustre. A safety property holds if the corresponding expression is
always true for all input traces. For example, the property for
Figure~\ref{fig:ex-before} is {\tt y >= 0}, which is a valid property.

It is straightforward to translate this interpretation of Lustre into
the traditional initial and transition relations. We will show this by
continuing with the example in Figure~\ref{fig:ex-before}. First we
introduce a new Boolean variable \textit{init} into the state space to denote
when the system is in its initial state, the state of the system prior
to initialization. In the initial state, all other variables are
completely unconstrained which models the underspecification of the \texttt{pre}
operator during the first step. The model is then:
\begin{align*}
  &I((x, a, b, y, \mathit{init})) = \mathit{init} \\
  &T((x, a, b, y, \mathit{init}), (x', a', b', y', \mathit{init'})) = \\
  &\hspace{1.5cm} (a' = f(x', \ite{init}{0}{y})) \land~ \\
  &\hspace{1.5cm} (b' = \ite{a' \geq 0}{a'}{-a'}) \land~ \\
  &\hspace{1.5cm} (y' = b' + (\ite{init}{0}{y})) \land ~\\
  &\hspace{1.5cm} \neg\mathit{init'}
\end{align*}
Note that $f$ is unspecified in Figure~\ref{fig:ex-before} and so also
in $T$. In a real system, $f$ would be defined in the Lustre model and
expanded in $T$. A safety property such as {\tt y >= 0} is translated
into $\mathit{init} \lor (y \geq 0)$. Nested uses of arrow and pre
operators are handled by introducing new output variables for nested
expressions, though such details are unimportant for our purposes.

Each equation in the Lustre program is translated into a single
top-level conjunct in the transition relation. This is very convenient
as the IVC of a Lustre property can be reported in terms of the output
variables whose equations are part of the IVC. Equivalently, the
interpretation of an IVC for a Lustre property is that any output
variable that is not part of the IVC can be turned into an input
variable, its equation thrown away, while preserving the validity of
the property. Thus the granularity of the IVC analysis is determined
by the granularity of the Lustre equations and can be adjusted by
introducing auxiliary variables for subexpressions if desired.  The idea of granularity will be discussed in more detail in Section~\ref{sec:granularity}.

\subsection{\jkind}

\jkind~\cite{jkind} is an infinite-state model checker for safety
properties. \jkind\ proves safety properties using multiple cooperative
engines in parallel including $k$-induction~\cite{SheeranSS00},
property directed reachability~\cite{Een2011:PDR}, and template-based
lemma generation~\cite{Kahsai2011}. \jkind\ accepts Lustre programs
written over the theories of linear integer arithmetic, linear real arithmetic, and uninterpreted functions. In the
back-end, \jkind\ uses an SMT solver such as Z3~\cite{DeMoura08:z3},
Yices~\cite{Dutertre06:yices}, MathSAT~\cite{Cimatti2013:MathSAT}, or
SMTInterpol~\cite{Christ2012:SMTInterpol}.

\jkind\ works on multiple properties simultaneously.  When a property is
proved and IVC generation is enabled, an additional parallel engine
executes one of the {\ucalg, \bfalg, \ucbfalg}~algorithms \cite{Ghass16} to generate an (approximately) minimal IVC.  To implement the \aivcalg\ procedure, we have extended \jkind\ with a new engine that implements Algorithm \ref{alg:aivc}.
We use the \jkind\ IVC generation engine to implement the \getivc\ procedure in  Algorithm \ref{alg:aivc}.


%It worth mentioning that we could have employed the \ucbfalg ~algorithm for this purpose as well.
%One might say that since \ucbfalg ~guarantees minimality, it would help to the \aivcalg ~algorithm terminate more quickly.
%However, as we will show in the experiments, on the one hand, the \ucbfalg ~algorithm is very expensive. On the other hand, the \ucalg ~algorithm is not only fast, but it also generates \ivc s that are
%quite close to the \mivc s computed by \ucbfalg.

As previously discussed, one issue that needs to
be handled in any implementation of the \ucalg, \ucbfalg, and \aivcalg~algorithms involves the undecidability of the model checking problem;
from Theorem~\ref{thm:minimal-hard}, the problem of determining whether or not any $S \subset T$ is adequate is undecidable for infinite-state systems. Thus, we set  timeouts for the model checking algorithm for each iteration of the \aivcalg\ procedure.
To do so, we measure the time required to prove the property over the original model (\emph{proof-time}), and the time required to calculate the first
(approximate) IVC using \ucalg\ (\emph{\ucalg-time}).
The timeout we set for each iteration of the \ucbfalg\ and \aivcalg\ algorithms is ($30$ sec  $+\ 5\ \times$ (\emph{proof-time} $+$ \emph{\ucalg-time})).
%
If a timeout occurs during computation, we report a warning to the user that our results are not guaranteed to be minimal.  %It is important to note that by increasing the timeout, it is possible that
%in some cases smaller IVCs can be generated, but the general problem will remain due
%to the undecidability of the model checking problem.
Complete implementation details on \jkind and the implementation of IVCs can be found in \cite{elathesis}.


\subsection{AGREE}

The {\em Assume Guarantee Reasoning Environment} (\agree) tool is
an open-source compositional verification tool that proves properties
of hierarchically-composed models in the Architectural Analysis and
Design Language (AADL) language~\cite{NFM2012:CoGaMiWhLaLu,QFCS15:backes,hilt2013}.
\jkind as the default model checker for AGREE.  \agree
refers to both an embedded language annex in the Architectural
Analysis and Design Language (AADL) and to a plugin for the OSATE AADL
Integrated Development Environment. The \agree annex annotates the
AADL model with formal requirements, and the plugin reasons about
these requirements. The purpose of \agree is to model behavioral
requirements of an embedded system using formal assume/guarantee
contracts using synchronous observers~\cite{Halbwachs91:lustre}, which can operate either in discrete or continuous time. The plugin generates Lustre specifications that are checked by \jkind.


This tool suite allows a model-based approach to system construction in which compositional proofs are used to automatically establish satisfaction arguments.
Given a component contract at a particular level of the AADL model, we attempt to prove it using the contracts of the subcomponents at the next lower level of the model hierarchy.
\agree makes use of multiple \jkind features including MIVCs to show requirements traceability, as well as counterexample generation to check the consistency of an AADL
component's contract.



%The source code is publicly available on \cite{mygit}.
%This section points out some technical issues in the implementation.
%
%We have extended \jkind with a new engine that
% implements Algorithm \ref{alg:aivc}.
% First a property is proved and a single IVC set $S$ is computed
% by the IVC generation engine. Then, the set $S$ is sent to our engine.
% For efficiency, instead of initializing $map$ with $\top$,
% we initialize $map$ with $\bigvee_{T_{i}\in S} \neg {\actlit (T_i)}$,
% which prunes off set $S$ and all its supersets from $\mathcal{P}(T)$ at the beginning of the algorithm. If $S$ is minimal, then chances are a large
% subset of  $\mathcal{P}(T)$ is marked as explored even before
% the algorithm starts the search.

