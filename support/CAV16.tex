\documentclass {llncs}
\usepackage{tabularx,colortbl}
\usepackage[dvipsnames]{xcolor}
\usepackage{flushend}
\usepackage{cite}
\usepackage{amsmath}
%\usepackage{amsthm}
\usepackage{amssymb}
\usepackage{epsfig}
\usepackage{stmaryrd}
\usepackage{url}
\usepackage{multirow}
\usepackage{latexsym}
%\usepackage{float}
\usepackage{graphics}
\usepackage{graphicx}
%\usepackage{enumitem}
\usepackage{comment}
\usepackage{longtable}
\usepackage{supertabular}
\usepackage{times}
\usepackage{listings}
\usepackage{subfigure}
\usepackage{color}
\usepackage{balance}


%\theoremstyle{Definition}
%\newtheorem{definition}{Definition}
%
%\theoremstyle{Theorem}
%\newtheorem{theorem}{Theorem}


%\newcommand{\definition}{\noindent \textbf{Definition} \citation{}}
%newcommand{\theorem}{\noindent \textbf{Theorem} \citation{}}


\newcommand{\mkeyword}[1]{\mbox{\texttt{#1}}}
\DeclareMathOperator{\kuop}{uop}
\DeclareMathOperator{\kbop}{bop}
\DeclareMathOperator{\kite}{ite}
\DeclareMathOperator{\kpre}{pre}
\DeclareMathOperator{\dom}{dom}
\DeclareMathOperator{\ktrue}{true}
\DeclareMathOperator{\kfalse}{false}
\DeclareMathOperator{\kselect}{select}
\DeclareMathOperator{\ran}{range}
\newcommand{\lbb}{[\![}
\newcommand{\rbb}{]\!]}
\newcommand{\expr}{\phi}
\newcommand{\exprS}{\Phi}
% this toggles between a tech-report version of the paper and the MEMOCODE version
%\newcommand{\TECHREPORT}{}
%\tolerance=1
%\emergencystretch=\maxdimen
%\hyphenpenalty=10000
%\hbadness=10000
\sloppypar



\begin{document}

 % Add this to every tex file, so that you can comment with a diff color%
 \definecolor{gold}{rgb}{0.90,.66,0}
 \definecolor{dgreen}{rgb}{0,0.6,0}
 \newcommand{\mike}[1]{\textcolor{red}{}}
 \newcommand{\fixed}[1]{\textcolor{red}{#1}}
 \newcommand{\ela}[1]{\textcolor{purple}{}}
\newcommand{\stateequiv}{\equiv_{s}}
\newcommand{\traceequiv}{\equiv_{\sigma}}
\newcommand{\ta}{\text{TA}}
\newcommand{\cta}{\text{TA$_{C}$}}
\newcommand{\tta}{\text{TA$_{T}$}}

%\newcommand{\TECHREPORT}{}


\title{An Efficient Way of Computing Set of Support\thanks{This work has been supported by XXX}}

%\author{
%\IEEEauthorblockN{Michael W. Whalen$^{\dag}$, Sanjai Rayadurgam$^{\dag}$, Elaheh Ghassabani$^{\dag}$, \\Anitha Murugesan$^{\dag}$, Oleg Sokolsky$^{*}$, Mats P.E. Heimdahl$^{\dag}$, Insup Lee$^{*}$}
%\IEEEauthorblockA{
%$^{\dag}$Department of Computer Science and Engineering,University of Minnesota\\
%%200 Union Street, Minneapolis, MN 55455\\
%\{whalen, rsanjai, ghassaba, anitha, heimdhal\}@cs.umn.edu\\
%$^{*}$Department of Computer and Information Science, University of Pennsylvania, \\
%%3330 Walnut St., Philadelphia, PA 19104\\
%\{sokolsky, lee\}@cis.upenn.edu}

\author{Michael W. Whalen\inst{1}, Andrew Gacek\inst{2}, Elaheh Ghassabani\inst{1}}

\institute{Department of Computer Science and Engineering\\
 University of Minnesota, 200 Union Street, Minneapolis, MN 55455,USA\\
\email{\{whalen,ghassaba\}@cs.umn.edu}
 \and
Rockwell Collins Advanced Technology Center\\
400 Collins Rd. NE, Cedar Rapids, IA, 52498, USA\\
\email{\{andrew.gacek\}@rockwellcollins.com}
}

\maketitle

\begin{abstract}
The idea of set of support aims to provide evidence and explanation for why a specific property is satisfied by a model. 
This core idea can have a lot of applications in software engineering. However, in order to make use of this idea, we have to inspect efficiency, accuracy, and usability of this notion. After implementation of this notion in the JKind model checker, we came up with some research questions to investigate the applicability of our approach and tool. 

This report provides a detailed description of the conducted experiments and their results. After explaining experimental design and set-up, each section reports some analytical results to answer a particular research question. We provide some technical detail as to how analyze the results.


\end{abstract}

\keywords{Auto-traceability, Set of Support, Completeness, Requirement Engineering}

\section{Introduction}
\label{sec:intro}
Most modern sequential model checking techniques for safety properties, including IC3/PDR~\cite{Een2011:PDR} and $k$-induction~\cite{SheeranSS00}, use a form of induction to establish proof.  These techniques are very powerful, and can often reason successfully over very larger or even infinite state spaces.  The proofs provided by these tools can provide rigorous evidence that a software or hardware system works as intended.

On the other hand, there many situations in which properties can be proved, but systems still will not perform as intended.  Issues such as vacuity~\cite{Kupferman03:Vacuity}, incorrect environmental assumptions~\cite{Whalen07:FMICS}, and errors either in English language requirements or formalization~\cite{Pike06:axioms} can all lead to failures of ``proved'' systems.  Thus, even if proofs are established, one must approach verification with skepticism.

Recently, Ghassabani et al.~\cite{Ghass16} introduced the idea of {\em Inductive Validity Cores} (IVCs) in order to provide additional information with proofs. IVCs offer proof explanation as to why a property is satisfied by a model in a formal and human-understandable way.  The idea lifts UNSAT cores~\cite{zhang2003extracting}
to the level of sequential model checking algorithms using induction.  Informally, if a model is viewed as a conjunction of constraints,
a minimal IVC (MIVC) is a set of constraints that is sufficient to construct a proof such that if any constraint is removed, the property is no longer valid.
IVCs and MIVCs can be used for several purposes, including performing traceability between specification and design elements, assessing model coverage, and explanation of unsatisfiable test obligations when using model checkers for test case generation. Ghassabani et al.~\cite{Ghass16} presented two algorithms: \ucalg, which computes an approximately minimal IVC that is computationally inexpensive, and \ucbfalg,
an algorithm that produces a
MIVC but is considerably more expensive to compute.
%
The IVC idea shares many similarities with approaches for computing minimal
invariant sets for inductive proofs (such as is performed for inductive proof certificates~\cite{piskac2016, ivrii2014small}), and in fact the \ucalg\ algorithm performs a minimal lemma set computation.  However, there is a substantive difference: to find a guaranteed minimal set of constraints, it is usually necessary to find new proofs involving {\em new lemmas} not used in the original proof, which accounts for the expense of the \ucbfalg\ algorithm.

It is often the case that there are multiple MIVCs for a given property.  In this case, the algorithms from~\cite{Ghass16} give, at best, an
incomplete picture of the traceability information associated with the proof.  Depending on the model and property to be analyzed, there is often substantial diversity between the IVCs used for proof, and there can be substantial difference in the size between the {\em minimal} IVC returned by the \ucbfalg\ algorithm and a {\em minimum} IVC, which is the (not necessarily unique) smallest MIVC.
 If {\em all} MIVCs can be found, then several additional analyses can be performed:
\begin{itemize}
    \item Coverage Analysis: MIVCs can be used to define coverage metrics by examining the percentage of model elements required for a proof.  However, since MIVCs are not unique, there are multiple, equally legitimate coverage scores possible.  Having \emph{all} MIVCs allows one to define additional metrics: coverage of MAY elements, coverage of MUST elements, as well as policies for the existing MIVC metric: e.g., choose the smallest MIVC. %\ela{I'm not sure if introducing MAY/MUST would make sense to the readers }
    \item Optimizing Logic Synthesis:  synthesis tools can benefit from MIVCs in the process of transforming an abstract behavior into a design implementation. A practical way of calculating MIVCs allows to find a minimum set of design elements (optimal implementation) for a certain behavior. Such optimizations can be performed at the different levels of synthesis.
    \item Impact Analysis: Given all MIVCs, it is possible to determine which requirements may be falsified by changes to the model.  This analysis allows for selective regression verification of tests and proofs: if there are alternate proof paths that do not require the modified portions of the model, then the requirement does not need to be re-verified.
    \item Robustness Analysis: As proposed by Murugesan et. al in~\cite{Murugesan16:renext}, it is possible partition the model elements into MUST and MAY sets based on whether they are in every MIVC or only some MIVCs, respectively.  This may allow insight into the relative importance of different model elements for property.  For example, if the MUST set is empty, then the requirement has been implemented in multiple ways, such as would be expected in a fault-tolerant system.  Moreover, examining the diversity of all MIVCs could lead to changes in how traceability
        ~\cite{COEST,cleland2007best}
     %~\cite{COEST,hayes2003improving,cleland2007best}
        is performed and managed in critical systems.
\end{itemize}
%\noindent In addition, the Requirements Engineering community is keenly interested in approaches to manage requirements traceability.  In most cases, it is assumed that there is a single ``golden'' set of trace links that describes how requirements are implemented in software~\cite{COEST,hayes2003improving,cleland2007best}.  However, if there are multiple MIVCs, then it is possible that there are several equally valid sets of trace links.  Examining the diversity of all MIVCs could lead to changes in how traceability is performed and managed in critical systems.

As far as commercial tools are concerned, we have found some of them that use the term \emph{proof-core} ~\cite{hanna2015formal, jasper_gold}, which sounds similar to the idea of a \emph{single} MIVC. However,
to the best of our knowledge, none of them offer the calculation of \emph{all} proof-cores.
Moreover, solutions provided by these tools are quite underspecified:
no formal description of the proof-core notion or algorithms are provided. In addition, no implementations or experimental results are provided, so it is not possible to compare their approach with IVCs.

In this paper, we propose a new method for computing \emph{all minimal} IVCs. In  recent  years,  a  number  of  efficient
algorithms  for  extracting  all MUSes  have  been proposed \cite{bacchus2015using, belov2012muser2, belov2013core, belov2012towards, nadel2014accelerated, liffiton2005max}.  In this paper, we adapt the recent work by Liffiton et al. \cite{marco2016fast} from the generation of MUSes from UNSAT-cores to all IVCs for inductive model checking.  This requires changing the underlying mechanisms that are used to construct candidate solutions and also changing the structure of the proof of correctness.  In addition, in our proof, we demonstrate that the approach terminates with all minimal IVCs even if the witness generator only generates approximately minimal IVCs (utilizing the ``fast'' \ucalg\ algorithm from~\cite{Ghass16}).  In our empirical results, this allows our algorithm to be quite efficient to the extent that in many cases, the cost of extracting all minimal IVCs is similar to the cost of finding a single guaranteed-minimal IVC, and on average is approximately 1.6x the cost of determining a single minimal IVC.
The contributions of the work are therefore as follows:
\begin{itemize}
    \item An algorithm for computing all minimal IVCs.
    \item A proof of correctness and completeness of the algorithm.
    \item An evaluation of the algorithm for performance and diversity of result sets against a benchmark suite.
    \item An industrial case study with over 10K design elements that demonstrates the practicality and usefulness of our technique.
\end{itemize}

%\ela{I think we need to make it clear that IVCs are different from MUSes, proof-certificates or minimal invariants, abstraction, slicing. Currently, the introduction doesn't say anything about these. You had an idea on having a table... Perhaps you want to include a discussion section?\\ Or, Maybe we could expand the introduction with these things and make it more motivating}

%\ela{Also, I think the contributions don't stand out. finding \emph{all} \textbf{minimal} IVCs itself is two contribution. I think minimality is important. Maybe we should stress on it a little bit more}

The rest of the paper is organized as follows.
Section \ref{sec:example} introduces a running example used to illustrate concepts and our method.
Section \ref{sec:background} covers the formal preliminaries for the approach.
In Section \ref{sec:allivcs}, we present our method for enumerating all minimal IVCs,
which is illustrated in
Section \ref{sec:illust}. In Section \ref{sec:impl}, we talk about implementation and evaluation of our method. Section \ref{sec:qfc} presents an industrial case study. Finally, Section \ref{sec:conc} mentions conclusions and future work. 

%\ifdefined\TECHREPORT
%\input{description}
%\fi
%\section{Illustrative Example}

\section{Motivating Example}
\label{sec:exmpl}

\begin{figure}
\includegraphics[width=\columnwidth]{figs/simulink.png}
\caption{Example model with property $y \geq 0$}
\label{fig:example}
\end{figure}

\begin{figure}
\includegraphics[width=\columnwidth]{figs/simulink-ivc.png}
\caption{Example model after IVC analysis}
\label{fig:example-ivc}
\end{figure}

One possibility for this is something like Figure~\ref{fig:example}.
Here we could show that the property $y \geq 0$ doesn't depend on the
function $f(u,v)$ in the model. The model after IVC analysis is shown
in Figure~\ref{fig:example-ivc}. The benefit of this example is that
it's visual and shows how this IVC generation is more than just
slicing since slicing would preserve $f(u,v)$. On the other hand, it
doesn't hit many of the other points Mike has asked for.

\begin{itemize}
    \item Not sure if this should go before or after the background section with a description of Lustre.
    \item Need a small but interesting example.  Andrew, do any of the models that you use as jkind tests
        function in this way?  It would be nice to look at what we have lying around; we need something 
        that requires invariants.
    \item It would also be good to have a few points of interest with the model-requirement pairing: 
    \item \quad   vacuity due to an overconstrained environment 
    \item \quad   definitions within the model that are irrelevant to the proof.
    \item Explain the model and the proof process.
\end{itemize}


\section{Preliminaries}
\label{sec:background}

\newcommand{\bool}[0]{\mathit{bool}}
\newcommand{\reach}[0]{\mathit{R}}
\newcommand{\ite}[3]{\mathit{if}\ #1\ \mathit{then}\ #2\ \mathit{else}\ #3}

\subsection{Transition Systems and Safety Properties}

Given a state space $S$, a transition system $(I,T)$ consists of an
initial state predicate $I : S \to \bool$ and a transition step
predicate $T : S \times S \to \bool$. We define the notion of
reachability for $(I, T)$ as a the smallest predicate $\reach : S \to
\bool$ which satisfies the following formulas:
\begin{equation*}
  \forall s.~ I(s) \Rightarrow \reach(s)
\end{equation*}
\begin{equation*}
  \forall s, s'.~ \reach(s) \land T(s, s') \Rightarrow \reach(s')
\end{equation*}
When the transition system is not obvious from context we will write
$\reach_{(I,T)}$ for the reachability predicate on the transition
system $(I,T)$.

A safety property $P : S \to \bool$ is a state predicate. A safety
property $P$ holds on a transition system $(I, T)$ if it holds on all
reachable states, i.e., $\forall s.~ \reach(s) \Rightarrow P(s)$,
written as $\reach \Rightarrow P$ for short.

For an arbitrary transition system $(I, T)$, computing reachability
can be very expensive or even impossible. Thus, we need a more
effective way of checking if a safety property $P$ is satisfied by the
system. The key idea is to over-approximate reachability. If we can
find an over-approximation that implies the property, then the
property must hold. Otherwise, the approximation needs to be refined.

A good first approximation for reachability is the property itself.
That is, we can check if the following formulas hold:
\begin{equation}
  \forall s.~ I(s) \Rightarrow P(s)
  \label{eq:1-ind-base}
\end{equation}
\begin{equation}
  \forall s, s'.~ P(s) \land T(s, s') \Rightarrow P(s')
  \label{eq:1-ind-step}
\end{equation}
If both formulas hold then $P$ is {\em inductive} and holds over the
system. If (\ref{eq:1-ind-base}) fails to hold, then $P$ is violated
by an initial state of the system. If (\ref{eq:1-ind-step}) fails to
hold, then $P$ is too much of an over-approximation and needs to be
refined.

One way to refine our over-approximation is to add additional lemmas
to the property of interest. For example, given another property $L :
S \to bool$ we can consider the extended property $P'(s) = P(s) \land
L(s)$, written as $P' = P \land L$ for short. If $P'$ holds on the
system, then $P$ must hold as well. The hope is that the addition of
$L$ makes formula (\ref{eq:1-ind-step}) provable because the
antecedent is more constrained. However, the consequent of
(\ref{eq:1-ind-step}) is also more constrained, so the lemma $L$ may
require additional lemmas of its own.

Another way to refine our over-approximation is to use use {\em
  $k$-induction} which to unrolls the property over $k$ steps of the
transition system. For example, 1-induction consists of formulas
(\ref{eq:1-ind-base}) and (\ref{eq:1-ind-step}) above, whereas
2-induction consists of the following formulas:
\begin{align*}
  \forall s.~ I(s) \Rightarrow P(s)
&&  \forall s, s'.~ I(s) \land T(s, s') \Rightarrow P(s')
\end{align*}
\begin{equation*}
  \forall s, s', s''.~ P(s) \land T(s, s') \land P(s') \land T(s', s'')  \Rightarrow P(s'')
\end{equation*}
That is, there are two base step checks and one inductive step check.
In general, for an arbitrary $k$, $k$-induction consists of the $k$
base step checks and one inductive step check as shown in
Figure~\ref{fig:k-induction}. We say that a property is $k$-inductive
if it satisfies the $k$-induction constraints for a the given value of
$k$. The hope is that the additional formulas in the antecedent of
the inductive step make it provable.

\begin{figure}
\begin{equation*}
  \forall s_0.~ I(s_0) \Rightarrow P(s_0)
\end{equation*}
\begin{center}
$\vdots$
\end{center}
\begin{equation*}
  \forall s_0, \ldots, s_{k-1}.~ I(s_0) \land T(s_0, s_1) \land \cdots
  \land T(s_{k-2}, s_{k-1}) \Rightarrow P(s_{k-1})
\end{equation*}
\begin{equation*}
  \forall s_0, \ldots, s_k.~ P(s_0) \land T(s_0, s_1) \land P(s_{k-1})
  \land T(s_{k-1}, s_k) \Rightarrow P(s_k)
\end{equation*}
\caption{$k$-induction formulas: $k$ base cases and one inductive
  step}
\label{fig:k-induction}
\end{figure}

In practice, we often use a combination of the above techniques. Thus,
a typical conclusion is of the form ``$P$ with lemmas $L_1, \ldots, L_n$
is $k$-inductive''.

\subsection{JKind and Lustre}

\marginpar{AJG: I think we should move this section to implementation.
  We can cut it down and also talk about what set-of-support means in
  Lustre vs general transition systems.}

JKind is an infinite-state model checker for safety properties. JKind
proves safety properties using multiple cooperative engines in
parallel including $k$-induction, property directed reachability
(PDR), and template-based lemma generation. JKind operates over
expressions in the theory of linear integer and real arithmetic. In
the back-end, JKind uses an SMT-solver such as Yices, Z3, CVC4,
MathSAT, or SMTInterpol.

\begin{figure}[t]
\begin{verbatim}
node main(x : int) returns (r : int; ok : bool);
let
  r = (0 -> pre r) + (if x > 0 then x else -x);
  ok = (r >= 0);
tel;
\end{verbatim}
  \caption{Example Lustre program}
  \label{fig:lustre-ex}
\end{figure}

The input language to JKind is Lustre, a synchronous dataflow language.
An example Lustre program is shown in Figure~\ref{fig:lustre-ex}. For
our purposes, a Lustre program consists of 1) some input variables,
{\tt x} in the example, 2) some output variables, {\tt r} and {\tt ok}
in the example, and 3) an equation for each output variable. A Lustre
program runs over discrete time steps. On each step, the input
variables take on some values and are used to compute values for the
output variables on the same step. In addition, equations may refer to
the previous value of a variable using the {\tt pre} operator. This
operator is undefined in the initial step, so the arrow operator, {\tt
  ->}, is used to guard such the {\tt pre} operator. In the initial
step the expression {\tt e1 -> e2} reduces to {\tt e1}, and it
reduces to {\tt e2} in all other steps.

We interpret a Lustre program as a model specification by considering
the behavior of the program under all possible input traces. Safety
properties over Lustre can then be expressed as Boolean output
variables in Lustre. A safety property holds if the corresponding
Boolean output variable is always true for all input traces. For
example, the program in Figure~\ref{fig:lustre-ex} represents an
integrator over the absolute value of the input variable. The output
variable {\tt ok} is a safety property of the system expressing that
the computed result is always non-negative. In this case, the property
is true.

It is easy to translate this interpretation of Lustre into the
traditional initial and transition relations. We will show this by
example using Figure~\ref{fig:lustre-ex}. First we introduce a new Boolean
variable $init$ into the state space to denote when the system is in
its initial step. Then we define,
\begin{align*}
  &I((x, r, \mathit{ok}, \mathit{init})) = \mathit{init} \\
  &T((x, r, \mathit{ok}, \mathit{init}), (x', r', \mathit{ok'},
  \mathit{init'})) = \\
  &\hspace{1cm} (r' = (\ite{init}{0}{r})) \land (\mathit{ok'} =
  (r' \geq 0)) \land \neg\mathit{init'}
\end{align*}
Each equation in the Lustre program is translated into a conjunct in
the transition relation. A safety property such as {\tt ok} is
translated into $\mathit{init} \lor \mathit{ok}$. Nested uses of arrow
and pre operators are handled by introducing new output variables for
nested expressions, though such details are unimportant for our
purposes.


%% Prevous plan:
%%
%% Symbolic transition systems (use material from Sheeran's "Induction using a SAT Solver" paper?)
%% Lustre language
%% UNSAT cores
%% jkind
%% more here?

%%% Local Variables:
%%% mode: latex
%%% TeX-master: "main.tex"
%%% End

%%  LocalWords:  bool reachability JKind Lustre PDR Yices MathSAT ok
%%  LocalWords:  SMTInterpol dataflow init


\section{Set of Support}
\label{sec:support}

\newcommand{\bq}{\textsc{BaseQuery}\xspace}
\newcommand{\iq}{\textsc{InductiveQuery}\xspace}
\newcommand{\fq}{\textsc{FullQuery}\xspace}

\newcommand{\mink}{\textsc{MinimizeK}\xspace}
\newcommand{\reduceinv}{\textsc{ReduceInvariants}\xspace}
\newcommand{\minsupport}{\textsc{MinimizeSupport}\xspace}

\newcommand{\checksat}{\textsc{CheckSat}\xspace}
\newcommand{\unsatcore}{\textsc{UnsatCore}\xspace}
\newcommand{\unsat}{\textsc{UNSAT}\xspace}
\newcommand{\sat}{\textsc{SAT}\xspace}

Given a transition system which satisfies a safety property $P$, we
want to know which parts of the system are necessary for satisfying
the safety property. One possible way of asking this is, ``What is the
most general version of this transition system that still satisfies
the property?'' The answer is disappointing. The most general system is
$I(s) = P(s)$ and $T(s, s') = P(s')$, i.e., you start in any state
satisfying the property and can transition to any state that still
satisfies the property. This answer gives no insight into the original
system because it has no connection to the original system. In this
section we introduce the notion of {\em set of support} which looks at
generalizing the original transition system while preserving a safety
property.

In order to talk about generalizing a transition system, we assume the
transition relation of the system has the structure of a top-level
conjunction. This assumption gives us a structure that we can easily
manipulate as we generalize the system. For ease of notation we will
write the transition system $T_1(s, s') \land \cdots \land T_n(s, s')$
as just $T_1 \land \cdots \land T_n$ or for short $\bigwedge T$ or
$\widehat T$ where $T = \{T_1, \ldots, T_n\}$. We will use a similar
notation of sets of invariants. We now define our notion of
generalization for transition systems.

\begin{definition}{\emph{Set of support:}}
  \label{def:set-of-support}
  Let $(I, \widehat T)$ be a transition system and let $P$ be a
  safety property with $(I, \widehat T)\vdash P$. We say $S \subseteq
  T$ is a {\em set of support} for $(I, \widehat T)\vdash P$ iff $(I,
  \widehat S) \vdash P$. When $I$, $T$, and $P$ can be inferred from
  context we will simply say $S$ is a set of support.
\end{definition}

\begin{definition}{\emph{Minimal Set of support:}}
  \label{def:minimal-set-of-support}
  A set of support $S$ for $(I, \widehat T)\vdash P$ is minimal iff
  there does not exist $M \subset S$ such that $M$ is a set of support
  for $(I, \widehat T)\vdash P$.
\end{definition}

\begin{lemma}
  \label{lem:set-of-support-monotonic}
  Let $(I, \widehat T)$ be a transition system and let $P$ be a
  safety property with $(I, \widehat T)\vdash P$. Let $S_1 \subseteq
  S_2 \subseteq T$. If $S_1$ is a set of support for $(I, \widehat
  T)\vdash P$ then $S_2$ is a set of support for $(I, \widehat T)\vdash P$.
\end{lemma}
\begin{proof}
  From $S_1 \subseteq S_2$ we have $\widehat S_2 \Rightarrow \widehat
  S_1$. Thus the reachable states of $(I, \widehat S_2)$ are a subset
  of the reachable states of $(I, \widehat S_1)$. \qed
\end{proof}

\begin{algorithm}[t]
  \SetKwInOut{Input}{input}
  \SetKwInOut{Output}{output}
  \Input{$(I, \widehat T)\vdash P$}
  \Output{Minimal set of support for $(I, \widehat T)\vdash P$}
  \BlankLine
  $S \leftarrow T$ \\
  \For{$x \in S$} {
    \If{$(I, \bigwedge (S\setminus\{x\})) \vdash P$}{
      $S \leftarrow S\setminus \{x\}$
    }
  }
  \Return{S}
\caption{Simple algorithm for computing a minimal set of support}
\label{alg:naive}
\end{algorithm}

This lemma gives us a simple, but inefficient algorithm for computing
a minimal set of support, Algorithm~\ref{alg:naive}. The resulting set
of this algorithm is obviously a set of support for $(I, \widehat
T)\vdash P$. The following lemma shows that it is also minimal.

\begin{lemma}
  The result of Algorithm~\ref{alg:naive} is a minimal set of support
  for $(I, \widehat T)\vdash P$.
\end{lemma}
\begin{proof}
  Let the result be $R$. Suppose towards contradiction that $R$ is not
  minimal. Then there a set of support $M$ with $M \subset R$. Take $x
  \in R\setminus M$. Since $x \in R$ it must be that during the
  algorithm $(I, \bigwedge(S\setminus\{x\}))\vdash P$ is not true for
  some set $S$ where $R \subseteq S$. We have $M \subset R \subseteq
  S$ and $x\not\in M$, thus $M \subseteq S\setminus \{x\}$. Since $M$
  is a set of support, Lemma~\ref{lem:set-of-support-monotonic} says
  that $S\setminus \{x\}$ is a set of support, and so $(I, \bigwedge
  (S\setminus\{x\}))\vdash P$. This is a contradiction, thus $R$ must
  be minimal.
\end{proof}

This algorithm has two problems in practice. First, checking if a
safety property holds is undecidable in general thus the algorithm may
never terminate even when the safety problem is easily provable over
the original transition system. Second, this algorithm is very
inefficient since it tries to re-prove the property multiple times.

\begin{algorithm}[t]
  \SetKwInOut{Input}{input}
  \SetKwInOut{Output}{output}
  \Input{$P$ with invariants $Q$ is $k$-inductive for $(I,
    \widehat T)$}
  \Output{Set of support for $(I, \widehat T)\vdash P$}
  \BlankLine
  $k \leftarrow \mink(I, \widehat T, P \land \widehat Q)$ \\
  $R \leftarrow \reduceinv_k(\widehat T, Q, P)$ \\
  \Return{$\minsupport_k(I, T, R)$}\\
\caption{Efficient algorithm for computing a nearly minimal set of support}
\label{alg:set-of-support}
\end{algorithm}

The key to a more efficient algorithm is to make better use of the
information that comes out of model checking. In addition to knowing
that $P$ holds on a system $(I, \widehat T)$, suppose we also know
something stronger: $P$ with the invariant set $Q$ is $k$-inductive
for $(I, \widehat T)$. This gives us the broad structure of a proof
for $P$ which allows us to reconstruct the proof over a modified
transition system. However, we must be careful since this proof
structure may be more than is actually needed to establish $P$. In
particular, $Q$ may contain unneeded invariants which could cause the
set of support for $P \land \widehat Q$ to be larger than the set of
support for $P$. Thus before computing the set of support we first try
to reduce the set of invariants to be as small as possible. This
operation is expensive when $k$ is large so as a first step we
minimize $k$. This is the motivation behind
Algorithm~\ref{alg:set-of-support}.

\begin{figure}
\begin{align*}
  &\bq_1(I, T, P) \equiv \forall s_0.~ I(s_0) \Rightarrow P(s_0) \\
%%%
  &\bq_{k+1}(I, T, P) \equiv \bq_k(I, T, P) \land~ \\
%
  &\hspace{10pt}\left(\forall s_0, \ldots, s_k.~ I(s_0) \land T(s_0,
  s_1) \land \cdots \land T(s_{k-1}, s_k) \Rightarrow P(s_k)\right)
  \\[5pt]
%%%
  &\iq_k(T, Q, P) \equiv (\forall s_0, \ldots, s_k.~\\
%
  &\hspace{10pt} Q(s_0) \land T(s_0,
  s_1) \land \cdots \land Q(s_{k-1}) \land T(s_{k-1}, s_k) \Rightarrow
  P(s_k)) \\[5pt]
%%%
  &\fq_k(I, T, P) \equiv \\
%
  &\hspace{10pt}\bq_k(I, T, P) \land \iq_k(T, P, P)
\end{align*}
\caption{$k$-induction queries}
\label{fig:queries}
\end{figure}

To describe the details of Algorithm~\ref{alg:set-of-support} we
define queries for the base and inductive steps of $k$-induction
(Figure~\ref{fig:queries}). Note, in $\iq(T, Q, P)$ we separate the
assumptions made on each step, $Q$, from the property we try to show
on the last step, $P$. We use this separation when reducing the set of
invariants.

We assume that our queries are checked by an SMT solver. That is, we
assume we have a function \checksat which determines if an
existentially quantified formula is satisfiable or not. In order to
efficiently manipulate our queries, we assume the ability to create
{\em activation literals} which are simply distinguished Boolean
variables. These activation literals are automatically held true when
calling \checksat. In the case when the formula is unsatisfiable we
assume we have a function \unsatcore which returns a minimal set of
the activation literals such that the formula is unsatisfiable with
those activation literals held true. In practice, SMT solvers often
return a non-minimal set, but we can minimize the set via repeated
calls to \checksat.

\begin{algorithm}[t]
  $k' \leftarrow 1$ \\
  \While{$true$} {
    \If{$\checksat(\neg\iq_{k'}(T, P, P)) = \unsat$} {
      \Return{$k'$} \\
    }
    $k' \leftarrow k' + 1$ \\
  }
\caption{$\mink(T, P)$}
\label{alg:minimize-k}
\end{algorithm}

The function $\mink(T, P)$ is defined in
Algorithm~\ref{alg:minimize-k}. This function assumes that $P$ is
$k$-inductive for $(I, T)$. It returns the smallest $k'$ such that $P$
is $k'$-inductive for $(I, T)$. We start checking at $k' = 1$ since
smaller values of $k'$ are much quicker to check than larger ones. The
checking must eventually terminate since $P$ is $k$-inductive. We also
only check the inductive query since we know the base query will be
true for all $k' \leq k$. Although we describe each query in
Algorithm~\ref{alg:minimize-k} separately, in practice they can be done
incrementally to improve efficiency.

\begin{algorithm}[t]
  $R \leftarrow \{P\}$ \\
  Create activation literals $a_1, \ldots, a_n$ \\
  $C \leftarrow \{a_1 \Rightarrow Q_1, \ldots, a_n \Rightarrow Q_n\}$ \\
  \While{$true$} {
    $\checksat(\neg\iq_k(T, \widehat C, \widehat R))$ \\
    \If{$\unsatcore() = \emptyset$}{
      \Return{R}
    }
    \For{$a_i \in \unsatcore()$}{
      $R \leftarrow R \cup \{Q_i\}$ \\
      $C \leftarrow C \setminus \{a_i \Rightarrow Q_i\}$ \\
    }
  }
\caption{$\reduceinv_k(T, \{Q_1, \ldots, Q_n\}, P)$}
\label{alg:reduce-invariants}
\end{algorithm}

The function $\reduceinv_k(T, \{Q_1, \ldots, Q_n\}, P)$ is defined in
Algorithm~\ref{alg:reduce-invariants}. This function assumes that $P
\land Q_1 \land \cdots \land Q_n$ is $k$-inductive for $(I, T)$. It
returns a set $R \subseteq \{P, Q_1, \ldots, Q_n\}$ such that
$\widehat R$ is $k$-inductive for $(I, T)$. Like \mink, this function
only checks the inductive query since each element of $R$ is an
invariant and therefore will always pass the base query. A significant
complication for reducing invariants is that some invariants may
mutually need each other, even though none of them is needed to prove
$P$. Thus in Algorithm~\ref{alg:reduce-invariants} we find a minimal
set of invariants needed to prove $P$, then we find a minimal set of
invariants to prove those invariants, and so on. We terminate when no
more invariants are needed to prove the properties in $R$. Even still,
this does not guarantee the result is minimal. For example, we may
find $P$ requires just $Q_1$, that $Q_1$ requires just $Q_2$, and that
$Q_2$ does not require any other invariants. This gives the result
$\{P, Q_1, Q_2\}$, but it may be that $Q_2$ alone is enough to prove
$P$ thus the original result is not minimal. Also note, we do not care
about the result of \checksat, only the \unsatcore that comes out of
it. Since $P \land Q_1 \land \cdots \land Q_n$ is $k$-inductive, we
know the \checksat call will always return \unsat.

\begin{algorithm}[t]
  Create activation literals $a_1, \ldots, a_n$ \\
  $T \leftarrow (a_1 \Rightarrow T_1) \land \cdots \land (a_n \Rightarrow T_n)$ \\
  $\checksat(\neg\fq_k(I, T, P))$ \\
  $R \leftarrow \emptyset$ \\
  \For{$a_i \in \unsatcore()$}{
    $R \leftarrow R \cup \{T_i\}$
  }
  \Return{R}
\caption{$\minsupport_k(I, \{T_1, \ldots, T_n\}, P)$}
\label{alg:minimize-support}
\end{algorithm}

The function $\minsupport_k(I, \{T_1, \ldots, T_n\}, P)$ is defined in
Algorithm~\ref{alg:minimize-support}. This function assumes that $P$
is $k$-inductive for $(I, \widehat T)$. It returns a minimal set of
support $R \subseteq \{T_1, \ldots, T_n\}$ such that $P$ is
$k$-inductive for $(I, \widehat R)$.

Our complete of support algorithm in
Algorithm~\ref{alg:set-of-support} does not guarantee a minimal set of
support. One reason is that \reduceinv does not guarantee a minimal
set of invariants. A larger reason is that we only consider the
invariants that the algorithm is given at the outset. It is possible
that there are other invariants which could lead to a smaller set of
support, but we do not search for them. In Section~\ref{sec:exprm}, we
show that in practice our algorithm is nearly minimal and much more
efficient that the naive algorithm.

%%% Local Variables:
%%% mode: latex
%%% TeX-master: "main.tex"
%%% End

%%  LocalWords:  Lustre iff TODO invariants Minimality BaseQuery
%%  LocalWords:  InductiveQuery FullQuery MinimizeK ReduceInvariants
%%  LocalWords:  MinimizeSupport CheckSat UnsatCore UNSAT


\section{Implementation}
\label{sec:impl}

The algorithm for efficiently computing IVCs can be found in a forthcoming FSE paper~\cite{Ghass16} and is implemented in the JKind \cite{jkind}, which is an infinite-state model checker for safety properties using multiple cooperative engines in parallel (such as k-induction and PDR). JKind accepts
Lustre programs written over the theory of linear integer and real
arithmetic. In the back-end, JKind uses an SMT solver such as
Z3, Yices, MathSAT, or SMTInterpol.
JKind works on multiple properties simultaneously. When a
property is proven and IVC generation is enabled, an additional
parallel engine executes the IVC generation algorithm to compute a minimal
IVC. We demonstrated the efficiency and precision of the approach using a set of Lustre models developed
as a benchmark suite for~\cite{Hagen08:FMCAD}, augmented with additional models from industrial projects (~\cite{QFCS15:backes,hilt2013}). The results show that our algorithm for computing IVCs is quite efficient even for industrial models with an average overhead of ~10\%. 

\section{Experiment}
\label{sec:experiment}

%\mike{What do we want to call our efficient algorithm: IVC?}

We would like to investigate both the {\em efficiency} and {\em
  minimality} of our three algorithms: the n{\"a}ive brute-force
algorithm (\bfalg), the UNSAT core-based algorithm (\ucalg), and the
combined UNSAT core followed by brute-force minimization algorithm
(\ucbfalg). Efficiency is computed in terms of wall-clock time: how
much overhead does the IVC algorithm introduce? Minimality is
determined by the size of the IVC: cores with a smaller number of
variables are preferred to cores with a larger number of variables.
Finally, we are interested in the {\em diversity} of solutions: how
often do different tools/algorithms generate different minimal IVCs?

The use of JKind allows additional dimensions to our investigation: it supports two different inductive algorithms: $k$-induction and PDR, and a ``fastest'' mode, that runs both algorithms in parallel.  In addition, JKind supports multiple back-end SMT solvers including Z3~\cite{DeMoura08:z3}, Yices~\cite{Dutertre06:yices}, MathSAT~\cite{Cimatti2013:MathSAT}, and SMTInterpol~\cite{Christ2012:SMTInterpol}.  We would like to determine whether the choice of inductive algorithm affects the size of the IVC, whether different solvers are more or less efficient at producing IVCs, and whether running different solvers/algorithms leads to {\em diversity} of IVC solutions.

Therefore, we investigate the following research questions:
\begin{itemize}
    \item \textbf{RQ1:} How expensive is it to compute inductive validity cores using the \bfalg, \ucalg, and \ucbfalg algorithms?
    \item \textbf{RQ2:} How close to minimal are the support sets computed by \ucalg as opposed to the (guaranteed minimal) \ucbfalg?  How do the sizes of IVCs compare to static slices of the model?
    \item \textbf{RQ3:} How much {\em diversity} exists in the solutions produced by different solver/induction algorithm configurations?
\end{itemize}

\subsection{Experimental Setup}
In this study, we started from a suite of 700 Lustre models developed
as a benchmark suite for~\cite{Hagen08:FMCAD}. We augmented this suite
with 82 additional models from recent verification projects including
avionics and medical devices~\cite{QFCS15:backes,hilt2013}. Most of
the benchmark models from~\cite{Hagen08:FMCAD} are small (10k or less,
with 6-40 equations) and contain a range of hardware benchmarks and
software problems involving counters. The additional models are much
larger: around 80k with over 300 equations. We added the new
benchmarks to better check the scalability for the tools, especially
with respect to the brute force algorithm.
%
%\mike{MORE HERE...stats on size, reasons for add'l models.}
Each benchmark model has a single property to analyze.  For our purposes, we are only interested in models with a {\em valid} property (though it is perhaps worth noting that there is no additional computation---and thus no overhead---using the JKind IVC options for {\em invalid} properties).  In our benchmark set, 295 models yield counterexamples, and 10 additional models are neither provable nor yield counterexamples in our test configuration (see next paragraph for configuration information).  The benchmark suite therefore contains 476 models with valid properties, which we use as our test subjects.

For each test model, we computed \ucalg in 12+1 configurations: the
twelve configurations were the cross product of all solvers \{Z3,
Yices, MathSAT, SMTInterpol\} and inductive algorithms
\{$k$-induction, PDR, fastest\}, and the remaining (+1) configuration
was an instance of \bfalg run on Yices, which is the default solver in
JKind. In addition, for each of the 12 configurations, we ran an
instance of JKind without IVC to examine overhead. The experiments
were run on an Intel(R) i5-2430M, 2.40GHz, 4GB memory machine, with a
1 hour timeout for each analysis on any model. The data gathered for
each configuration of each model included the time required to check
the model without IVC, with IVC, and also the set of elements in the
computed IVC.\footnote{The benchmarks, all raw experimental results,
  and computed data are available on \cite{expr}.}

Note that not all analysis problems were solvable with all algorithms: for all solvers, $k$-induction (without IVC) was unable to solve 172 of the examples.  When comparing minimality of different solving algorithms, we only considered cases where both algorithms provided a solution (as will be discussed in more detail in Section~\ref{sec:minimality}).

\iffalse
\begin{itemize}
    \item an algorithm to compute a truly minimal set of support, i.e. \texttt{JSupport}.
    \item given a LUS model, a static crawler which automatically marks all equations of a node in the initial support set of a property.
    \item some trackers that measure the verification time with/ without support computation.
   % \item some minor changes in the XML writers.
\end{itemize}

\mike{My thoughts on this section: mostly, it needs more structure: more information on the properties of the models: size, provenance, etc., a broken out subsection on the description of the experimental setup, etc}

\mike{I think we want to split out the results in another top-level section}

Experiment:
\begin{itemize}
    \item (Overview) describe research questions and goals.
    \item Experimental setup: tell me about the models: how many, how big are they?  Then, tell me about the experiment: the tool configurations, the machine used for test.
    \item Data generation: Describe what you measured for each model analysis.
\end{itemize}
\fi


%%  LocalWords:  minimality ive UNSAT IVC Minimality IVCs PDR Yices
%%  LocalWords:  MathSAT SMTInterpol RQ JSupport


\section{Related work}
\label{sec:related}

In recent years, extraction of Minimally Unsatisfiable Subformulas (MUSes) has been the focus of a lot of research work \cite{marques2010minimal, belov2012towards, ryvchin2011faster, belov2012computing, nadel2010boosting, ryvchin2011faster, }. Although algorithms proposed by such work can handle very large problems,
computing MUSes is still very resource-intensive task.
While some work aimed to provide a set of minimal unsatisfiable formulae, they define minimality in
a way that given a set of clauses \mathbb{M}, removing every member of \mathbb{M} makes it satisfiable
\cite{belov2012computing}. 
Such algorithms are often compared with each other. In this work, we compare a regular computation of minimal unsat-core against minimum unsat-core. In addition, our focus is not to provide a novel way of computing minimal unsat-core. Instead, we makes use of MUSes to efficiently compute a set of support in a model necessary for inductive proofs.

Nadel, in \cite{nadel2010boosting}, discusses a 
number of applications of MUS extraction in formal verification. 
Gupta et al. \cite{gupta2003iterative} and McMillan and Amla \cite{mcmillan2003automatic} introduced the use of unsatisfiable cores in proof-based abstraction engines. Their goal is to shrink the abstraction size by omitting the parts of the design that are irrelevant to the proof of the property
under verification. However, \cite{gupta2003iterative, mcmillan2003automatic} do not consider core minimization. To our knowledge, none of the existing work
has used MUS to provide support information that explains
the correctness of proofs provided by different inductive techniques
including PDR and k-induction.
negative result.

\cite{torlak2008finding} proposes an algorithm for finding MUSes of declarative specification implemented for the Alloy language. Alloy is a framework for describing high-level design of various systems, whose analyzer is a fully automatic constraint solver. Constraints are translated into propositional logic solved by a SAT solver; hence, the analysis considers only a finite number of values for each type. For this reason, even for a set of simple constraints, the analyzer is never able to prove the correctness of a property. A major difference between this work and ours is that we 
extract UNSAT cores from an inductive proof over a sequential model involving lemmas. In addition, Alloy mostly works based on SAT solving, instead of SMT solving. In our implementation, JKind supports a variety of powerful SMT solvers (such as Z3, Yices, Yices2, etc.).

\begin{itemize}
    \item MUS's : checked
    \item Work on Alloy: checked
    \item Work that Teme pointed us to : will be added
    \item Anything else Elaheh has found : \%60 checked
\end{itemize}



\section{Conclusions \& Future Work}
\label{sec:conclusion}

In this paper, we have defined a novel coverage notion for formal verification using
the IVC concept, a useful measure in relation to
a valid safety property for inductive model checking. We have shown that our method
 is computationally efficient while 
 being accurate about the covered parts of a given design. 
 We have referred to this accuracy as preserving provability, which means 
 that a set of elements considered covered by our algorithm is sufficient 
 to establish the validity proof for every requirement in the set of specifications.
 
 We have implemented
our algorithm as part of the open source model checker JKind. Using our approach, measuring coverage is quite possible as we have shown in our experiments.
 We also benchmarked our implementation and compared it with other techniques in the literature. 
 The experiments show that the computation imposes a small overhead to the verification process. We have described how the justifiable notion of coverage proposed in this paper can be used as a
means of quantifying requirements completeness.
 
 In addition, based on the idea of multiple support sets for a specification, we 
 have introduced and discussed some other complementary coverage notions in the context of formal verification. Finally, we are in the process of developing some efficient algorithms for exploring the space of IVCs, e.g., finding a
minimum, rather than minimal support set, or finding all support sets. Having such algorithms makes the utilization of other proposed coverage practical.
%ACKNOWLEDGMENTS are optional
\vspace{0.05in}
\textbf{Acknowledgments:}
We thank XXXX
\bibliographystyle{splncs}
\bibliography{../../../bibtex/crisys}

%\ifdefined\TECHREPORT
%\appendix
%
%\section{Appendix: Proof of Equivalence}
%\ifdefined\TECHREPORT
\label{appendix:traceequiv}
\textbf{Theorem \ref{thm:traceequiv}: Trace Equivalence}
\textit{
\input{agree-to-ta-thm}
}

\vspace{0.1in}

%\mike{Re-introduce $\phi$!  This might confuse people.}

\noindent \textbf{Proof: } (1) by construction.  Given an arbitrary trace $\sigma_{c}$ we construct an equivalent trace $\sigma_{t}$.  We construct $\sigma_{t}$ using induction over the natural numbers, assuming that we have constructed $\sigma_{t1} \ldots \sigma_{tk}$ and then extending the trace to $\sigma_{tk+1}$.

The proof decomposes into two cases.  First, we must show that the initial states match (base case).  By the initial state construction $l_{c} = l_{t-a} = (l_1^0,l_2^0,...,l_n^0)$ and $u_{c} = u_{t} = u_0$.  By construction, the domain of $\nu_{c}$ consists of $\dom(V_{A}) \cup \dom(v_{0})$, and
for $x \in \nu_{c}~.~\nu_{c}(x) = \nu_{t}(x) = \nu_{0}(x)$, and for $x \in V_{A}$, we have $\nu_{c}(x) = \nu_{t}(x) = \nu_{pc}(x) = \nu_{t}(x_{pre}) = V_{A0}(x)$.  We initialize $\nu_{t}(v_{sat})$ to true.
Finally, $o_{c} = \emptyset$ and $\{(v, false)~|~v \in V_{OL} \}$, so $(\forall (o,v_{o}) \in \mapoutputevent~.~(o \in o_{c} \iff \nu_{t}(v_{ol}) = true))$.

Suppose alternately that we have $k > 0$.  We show that we can extend the trace $\sigma_{t}$ to match $\sigma_{c}$ at step $k+1$.

Given state $\sigma_{ck}$, $\sigma_{ck+1}$ must be reached by one of the six transition rules in Definition~\ref{def:centa-semantics}.  Suppose CENTA rule (1) is used.  In this case, time advances by $d$.  But in this case, we can apply NTA rule 1 for $\sigma_{tk}$ for the same value of $d$.  This is immediate for any of the invariants for machines $\{\mathcal{A}_{1}, \mathcal{A}_{2}, \ldots, \mathcal{A}_{n}\}$ because from the pre-state equivalence $\stateequiv$ the states have the same valuations for locations, variables, and clocks.  For the valuation of $\mathcal{A}_{a}$, there is only one state ($l_{w}$) with invariant $(\lbb c_{\period} \leq \period \rbb, \lbb v_{sat} = \ktrue \rbb)$.  $v_{sat}$ is true in state $k$ and remains true in $k+1$ since no variables change value during a time update.  It remains to show that $u_{t}(c_{\period}) + d \leq \period$, which is straightforward since $u_{t}(c_{\period}) = u_{c}(c_{\period})$ and (1) contains a constraint: $u(c_{\period})+ d \leq \period$.  Therefore $\sigma_{ck+1} \stateequiv \sigma_{tk+1}$.


Suppose CENTA rule (2) is used.  In this case, a $\tau$ transition occurs in one of the machines $\{\mathcal{A}_{1}, \mathcal{A}_{2}, \ldots, \mathcal{A}_{n}\}$.  In this case, same transition can occur in the translated model using rule (2), yielding the same destination state, clock resets and variable valuations, so (a), (b), (c) are immediately satisfied.  Furthermore, $\agreestate$ is not modified by rule (2), so the definitions $v_{pc}$ and $o_{c}$ remain the same.  By $assigns\_ok$, it is also the case that no variables in the sets $V_{P}$ and $O_{E}$ or variable $v_{sat}$ will be modified, so (d), (e), and (f) are maintained, and $\sigma_{ck+1} \stateequiv \sigma_{tk+1}$.

Suppose CENTA rule (3) is used.  In this case, the reasoning is very similar to rule (2).

Suppose CENTA rule (4) is used.  In this case, we are latching an input signal into an input variable related to the AGREE contract.  By rule (4), there exists an event $(\alpha_{i}, v_{ie}) \in \mapinputevent$.  Therefore, we can apply rule (3) with the $E_I$ transition:
$(l_{w}, \alpha_{i}?, (\lbb \ktrue \rbb, \lbb \ktrue \rbb), \emptyset, \{ (v_{i}, \lbb \ktrue \rbb) \}, l_{w})$. The result of the application of rule (3) to the translation and rule (4) to the AGREE model perform the same variable modifications and clock resets, satisfying (a), (b), (c).  By $assigns\_ok$, no variables in the sets $V_{P}$ and $O_{E}$ or variable $v_{sat}$ will be modified, so (d), (e), and (f) are maintained and the state invariant for $l_{w}$ is maintained, and $\sigma_{ck+1} \stateequiv \sigma_{tk+1}$.

Suppose CENTA rule (5) is used.  This rule has the form:

$(\bar{l}_{c},u_{c}, \nu_{c}, (\nu_{pc}, \emptyset)) \rightarrow (\bar{l}_{c}, u'_{c}, \nu''_{c}, (\nu'_{c}, o'_{c}))$ if $\nu'_{c} \in Val^{O}(\nu_{c})$, $u_{c}(c_{\period}) = \period$, $C2S(\nu_{c}, \nu'_{c})$, $u'_{c} = u_{c} \oplus (c_{\period} \mapsto 0)$, and $o'_{c} = \outputevents(\nu'_{c})$.  We note that $\nu'_{c}$ is constructed from $\nu_{c}$ by nondeterministically assigning a value to each of the $m$ output variables from their types $Val^{O}(\nu_{c})$.  For the moment, we will call these additional assignments $\nu_{O} = \{(v_{0}, c_0), (v_1, c_1), \ldots, (v_m, c_m)\}$, and note that $\nu'_{c} = \nu_{c} \oplus \nu_{O}$.  In the construction of the translated automata, we create an assignment for {\em every} such valuation of outputs in the $Y_{TA}$ rule.  We choose the edge $e_{to}$ that has the matching assignment $\nu'_{c}$ from $Y_{TA}$: $Y_{tao}$.  This edge is defined in the translation as: $(l_{w}, \tau, (\lbb c_{\period} = \period \rbb, \bigwedge \{\lbb v_{o} = \kfalse\rbb~|~v_{o} \in \ran~\mapoutputevent \} ), \{c_{\period}\}, y, l_{w})$, where $y = Y_{tao} \concat Y_{TS} \concat Y_{TO} \concat Y_{TP} \concat Y_{TI} $.

We first note that the guard for $e_{to}$ is satisfied due to state equivalence on pre-states $s_{C}$ and $s_{T}$ (b) and (e).  We then examine transition post-states.  First, the valuations of $l_{c}$ and $l_{t-a}$ are unchanged in both rules and that the reset clocks are the same, satisfying equivalence parts (a) and (b) on the post-states.  To determine equivalence of variable maps, we first describe intermediate variable maps during evaluation of $y$, noting that $\nu'_{t} = y(\nu_{t})$ is equivalent to $\nu^{1}_{t} = Y_{tao}(\nu)$, $\nu^{2}_{t} = Y_{TS}(\nu^{1})$, $\nu^{3}_{t} = Y_{TO}(\nu^{2}_{t})$, $\nu^{4}_{t} = Y_{TP}(\nu^{3}_t)$, and $\nu'_{t} = Y_{TI}(\nu^{4}_t)$, and that each of the lists assign a disjoint set of variables.  Because $\nu_{t}^{1} = \nu_{t} \oplus \nu_{O}$, $(\forall x \in \dom~\nu'_{c}~.~\nu'_{c}(x) = \nu^{1}_{t}(x))$, satisfying (c).  By disjointness of assignments, (c) is also satisfied for $\nu_{t}'$.  Since $Y_{tao}$ does not assign any `pre' variables,  $(\forall x \in V_{A}~.~\nu_{pc}(x) = \nu^{1}_{t}(x_{pre}))$ holds.  From these equivalences of valuations of current and pre variables $\nu'_{c}$, $\nu_{pc}$ with $\nu^{1}_{t}$, we claim\footnote{A complete argument would require translation rules for replacing `pre' expressions and expression evaluation semantics; this is a lengthy but not difficult argument.} that $C2S(\nu_{pc},\nu'_{c}) = C2S^{*}(\nu^{1}_{t})$.  Therefore, $\nu^{2}_{t}(v_{sat}) = \nu^{'}_{t}(v_{sat}) = true$, so we satisfy (f) and the state invariant of $l_{w}$.  Next, we assign
output latch variables to match outputs in $\nu^{3}$ (satisfying (e)), and finally assign `pre' variables based on current valuations in $\nu^{4}$ (satisfying (d)). Finally, to satisfy (c) for $\nu_{t}'$ and $\nu_{c}''$, we reset all latched input variables to false using $Y_{TI}$.  Since variables other than latched inputs are unchanged, the properties (d) (e) (f) still hold.


Suppose finally that CENTA rule (6) is used.  The proof here is very similar (and symmetric) to the proof of rule (4).

Since $\sigma_{ck+1}$ must be derived from $\sigma_{ck}$ through one of the six CENTA rules, and we demonstrate that any rule CENTA application has an analogous NTA rule for the translated AGREE model, it is possible to extend $\sigma_{tk}$ to $\sigma_{tk+1}$ such that $\sigma_{ck+1} \stateequiv \sigma_{tk+1}$. %$\qed$

\textbf{Proof: }(2) By construction.  The proof is similar to the proof of (1). Given an arbitrary trace $\sigma_{t}$, an equivalent trace $\sigma_{c}$ is constructed by induction over the natural numbers. That is to say, we assume that $\sigma_{c1} \ldots \sigma_{ck}$ has been constructed, and then we are extending it to $\sigma_{ck+1}$. The base case is established in a similar way described for \ref{thm:traceequiv}. Given state $\sigma_{tk}$, state $\sigma_{tk+1}$ must be reached by one of the three rules in the definition of NTA. We show that, for each rule whereby $\sigma_{tk+1}$ is reached, we can construct $\sigma_{ck+1}$ using CENTA rules such that $\sigma_{ck+1} \stateequiv \sigma_{tk+1}$.\\
Suppose that we have reached $\sigma_{tk+1}$ using NTA rule (1); using this rule, $(\bar{l}_{t},u_{t}, \nu_{t}) \rightarrow (\bar{l}_{t}, u_{t}+d, \nu_{t})$ such that, $I(\bar{l_{t}})$ is satisfied after adding $d$ to $u_{t}$. Therefore, we know that, in $\sigma_{tk+1}$, for every $\mathcal{A}_{i} \in \{\mathcal{A}_1,\mathcal{A}_2, \ldots,\mathcal{A}_n,\mathcal{A}_a \}$, invariants are satisfied. Since the invariant of $\mathcal{A}_a$ is
$I = \{(l_{w},(\lbb c_{\period} \le \period \rbb, \lbb \nu_{sat} = true \rbb))\}$, we have
$u_{t}(c_{\period}) +d \le \period$. Then, here, with the same value of $d$, we can apply CENTA rule (1) to $\sigma_{ck}$.  Due to pre-state equivalence for every $\mathcal{A}_{i} \in \{\mathcal{A}_1,\mathcal{A}_2, \ldots,\mathcal{A}_n\}$, the states have the same valuation for locations, variables, and clocks. For $\mathcal{A}_a$, we have only one state $l_{w}$, where $\nu_{sat}$ remains $\ktrue$ because, during the time update, no variables have changed. As $u_{t}(c_{\period})=u_{c}(c_{\period})$, the clocks are the same in state $k+1$ after adding the same amount of $d$ to $c_{\period}$. Therefore, $\sigma_{ck+1} \stateequiv \sigma_{tk+1}$.


\input{appendix2} 
%\fi

%\section{Appendix: GPCA CENTA Model}
%\label{appendix:gpcacenta}
%\begin{figure}[!ht]
%\begin{center}
%\includegraphics[scale=0.6]{images/sampled_pca.PNG} %[trim = 0 2 0 0, clip=true]{Comp}
%\caption{GPCA AGREE Properties modeled as a Timed Automata} \label{fig:samplepca}
%\end{center}
%\end{figure}

%\balancecolumns

\end{document}
