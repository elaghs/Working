
Through formal definitions in the previous section we highlight two critical facets -- the semantics and completeness -- in establishing traceability. 
In practice, just being able to trace a requirement to a target artifact that satisfies it, say a line of code, is not useful~\cite{guo2015trace}; a holistic view of how that line of code in conjunction with other related lines of code satisfy 
the requirement provides meaningful information to perform analysis such as assessing the impact of a change~\cite{hull2010requirements}. 
By defining a trace from requirements to the set of target artifacts, we advocate that traceability be captured in a way that upholds its semantic rationale. Further, we also found that performing analysis, such as impact analysis, using a partial set of trace links may result in imprecise results and misplaced confidence about the system. 
By considering $MIVC$ trace links and complete traces for each requirement, 
we can assess analyses related to requirements satisfaction traceability on a proper semantic foundation. In this section, we elaborate on how the semantic foundations in the previous section help us understand, assess, and use traceability precisely.

Given {\em all} proofs of a particular property, it is possible to define additional, complementary coverage notions.  
To do so, we use the following categorization of the model elements based on \mivc ~and 
$AIVC$ relations for $P$:

\begin{itemize}
\item $MUST (P) = \bigcap AIVC(P)$
\item $MAY(P) = (\bigcup AIVC (P)) \setminus MUST (P)$
\item $IRR(P) = T \setminus (\bigcup AIVC (P))$
\end{itemize}

\noindent This categorization helps to identify the role and relevance of each design element in satisfying a property. Function $MUST$ specifies the parts of the model absolutely necessary for the property satisfaction.  Any change to these parts will affect provability of the property. On the other hand, any single element in $MAY (P)$, may be modified without affecting satisfaction of $P$(though modifying multiple elements may require re-proof). The $IRR$ denotes model elements that are irrelevant to the validity of $P$.

The categorization of the set of support to be useful in several analyses:

The $AIVC$ set improves understanding of how a change in the requirement will affect the target artifacts and vice versa. While the $AIVC$ of a requirement gives a clear picture of various ways a requirement is satisfied by the system, the categorization of target artifacts helps precisely assess and plan when and where the changes have to be implemented. The $MUST$ elements are those target artifacts that are highly likely to change with any change in the requirement, whereas not all $MAY$ elements may need to be changed.

If a requirement has elements only in its $MAY$ set, that is if $MUST$ set is empty 
($MUST(r) = \emptyset$), it indicates that the requirement has been (intentionally or unintentionally) implemented in independent ways, such as fault tolerant systems. For such requirements, one has to carefully analyze and decide if the target artifacts in all or one disjoint set needs to be changed. These analysis could be performed either from the perspective of one or all requirements of the system.

From the target artifact side, this categorization helps analyze the impact of changes to the artifact. Suppose we decide to change a target artifact in the $MAY$ set for a requirement. While one might think that it is safe to change this artifact since it does not affect that requirement's satisfaction, an examination of the $AIVC$ sets of other requirements helps identify if it is indeed safe to change that artifact. If it is present in the $MUST$ set for another requirement, then a change to this artifact will definitely impact the other requirement. However, if it is in the $MAY$ sets for all the requirements, then it is clearly safe to change. Hence, this categorization helps us to assess critical dependencies between the target artifacts and the satisfaction of requirements and thus enables a precise bi-directional impact analysis of a change.

Complete traceability can assist in tailoring verification and validation in systems. For instance, if several requirements have a certain target artifact in their $MUST$ set, say an particular assumption, it reveals the importance of focusing V\&V attention on that artifact. Along the same lines, for a system with a complex architecture (components that each have functionality) such as  system of systems, this categorization helps identify components that is critical to satisfy most requirements. This categorization helps plan verification strategies.

If we examine changing a target artifact that appears $MUST$ set for any requirement, then this requirement must be re-verified. However, if it appears in the $MAY$ set for the requirement, then we can instead remove any sets-of-support that contain the element; as long as there still exists at least one set of support for the requirement, no reverification is necessary for that requirement.

Further, the notion of complete traces helps to assess if requirements are satisfied by the system in an unintended manner. It is well known that issues such as vacuity~\cite{kupferman2003vacuity} can cause requirements to be satisfied in a trivial manner. Even for non-vacuous requirements, it can be the case that requirements can be satisfied using a much smaller portion of the system than intended because they are incorrectly specified.  By capturing all the sets of support and categorizing the items therein, it is possible to examine whether the MUST set corresponds to expectations on the system, or, if more rigor is required, to examine each set of support individually to see whether it matches the expectations of the architects and developers. 