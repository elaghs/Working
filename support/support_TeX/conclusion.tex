\section{Conclusions \& Future Work}
\label{sec:conc}

We have defined the notion of an inductive validity core (IVC) which
appears to be a useful measure in relation to a valid safety property
in inductive model checking. We have presented a novel algorithm for
computing IVCs that are nearly minimal and have shown that full
minimality is undecidable in many settings. Our algorithm is
applicable to all forms of inductive SAT/SMT-based model checking
including $k$-induction, property directed reachability (PDR), and
interpolation-based model checking.

We have implemented our IVC algorithm as part of the open source model
checker JKind. We have shown that the algorithm requires only a
moderate overhead and produces nearly minimal IVCs in practice.
Moreover, the produced IVCs are fairly stable with respect to
underlying proof engines ($k$-induction and PDR) and back-end SMT
solvers (Yices, Z3, MathSAT, SMTInterpol).

In ongoing work, we are using the notion of IVCs to perform vacuity
detection, completeness checking, tracability, and test case
generation failure analysis. We plan to investigate algorithms for
exploring the space of IVCs such as finding distinct minimal IVCs or
finding minimum (rather than minimal) IVCs. Such algorithms could be
used to automatically measure the redundancy of systems with respect
to their safety properties.

%% \begin{itemize}
%%     \item Write this at the end.
%%     \ela{We can add something about fault tolerance for future work. 
%%     if we have all sets of support, and their intersection is empty, we have redundancy.
%%     we can talk about algorithm for minimum support set...}
%% \end{itemize} 

%%  LocalWords:  IVC IVCs PDR Yices MathSAT SMTInterpol tracability
