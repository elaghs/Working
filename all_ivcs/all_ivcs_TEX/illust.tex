\section{Illustration}
\label{sec:illust}
To illustrate the $AIVC$ algorithm we use the example presented in Section \ref{sec:example} with $P = (\small{\texttt{p1 or p2}})$ .
%As inspired by the MARCO algorithm in \cite{marco2016fast}, we visualize $\mathcal{P}(\mathcal{A})$ as a lattice in a Hasse diagram (Figure~\ref{fig:lattice1}) to demonstrate the progress of the algorithm. As you can see, each level of the lattice contains sets with the same size linked to sets that are their immediate
%supersets (upper level regions) and subsets (lower level regions). Note that the power set is viewed as a Boolean formula, so each member in the lattice shows
% variable with $true$.
For better description, we view $T$ as an ordered set of its top-level conjuncts; i.e. $T = \{$ {\small \texttt{a1\_below, a2\_below, below, alarm, doi\_on, p1, p2}} $\}$.
The algorithm starts with creating activation literals for each $T_i \in T$. Let the ordered set of Boolean variables $\{ a_1, \ldots , a_7 \}$ be the corresponding literals to the elements of $T$ (e.g. $\actlit (\small{\texttt{a1\_below}}) = a_1$ and $\actlit (\small{\texttt{p2}}) = a_7$). Then, line 3 initializes $map$ with $\top$.

In the first iteration of the \texttt{while} loop, since $map$ is empty,
it is satisfiable, and a model for it can
be any subset of literals. So obviously, the first maximal model of $map$ contains all the literals, which means, in line 6, $M = \{a_1, \ldots, a_7\}$,
 and in line 7, $M = T$. Since $M$ is adequate for $P$,
 the \getivc ~module is called in line 9.
 Suppose the returned IVC by this function  is
 $S' = \{ \small{ \texttt{a1\_below},~\texttt{below}, ~\texttt{alarm},
~\texttt{doi\_on},~\texttt{p1}}\}$;
this set is added to $A$ in line 10, and thus it comes to adding a new clause to $map$ (line 11), which makes $map = (\neg a_1 \vee \neg a_3 \vee \neg a_4 \vee \neg a_5 \vee \neg a_6)$.
As discussed, this constraint marks all the supersets of $S'$ as explored and prune them off the search space.

For the second iteration, $map$ is still satisfiable,
so the algorithm gets to find a maximal model of it in line 5. Suppose this time, the maximal model makes $M = \{a_2, \ldots, a_7\}$,
which leads to $M = T \setminus \{\small{\texttt{a1\_below}}\} $ in line 7.
Since this $M$ is adequate for $P$, \getivc ~computes a new $IVC$ in line 9.
Let the new $IVC$ be $S' = \{ \small{ \texttt{a2\_below},~\texttt{below}, ~\texttt{alarm},
~\texttt{doi\_on},~\texttt{p2}}\}$; after adding this set to $A$,
it is time to constrain $map$ by a new clause in line 11,
which results in $map \leftarrow map \wedge (\neg a_2 \vee \neg a_3 \vee \neg a_4 \vee \neg a_5 \vee \neg a_7)$.

Although we found two $IVC$s of $P$, $map$ is still satisfiable meaning that the power set has not yet explored entirely.\footnote{Note that the algorithm terminates when all $IVC$s and $MIS$es are explored. Up to now, only $IVC$s have been explored.} So, we get into the third iteration of the \texttt{while} loop. Let the maximal model returned this time
yields to a
$M = T \setminus \{ \small{ \texttt{a1\_below},~\texttt{a2\_below}}\}$
 in line 7.
Now, $M$ is inadequate, so we get into line 13 updating $map$ as
$map \leftarrow map \wedge (a_1 \vee a_2)$.
Note that by adding this new clause to $map$,
all the subsets of $T \setminus \{ \small{ \texttt{a1\_below},~\texttt{a\_2below}}\}$
are removed from the search space.

The algorithm continues similar to the third iteration leading to $M$ (in line 7) and $map$ (in line 13) to be as follows:
\begin{itemize}
  \item Iteration 4:  $M = T \setminus \{ \small{ \texttt{alarm}}\}$, $map \leftarrow map \wedge a_4$

  \item Iteration 5: $M = T \setminus \{ \small{ \texttt{p1},~\texttt{p2}}\}$, $map \leftarrow map \wedge (a_6 \vee a_7)$

  \item Iteration 6: $M = T \setminus \{ \small{ \texttt{doi\_on}}\}$, $map \leftarrow map \wedge a_5$

  \item Iteration 7: $M = T \setminus \{ \small{ \texttt{below}}\}$, $map \leftarrow map \wedge a_3$

  \item Iteration 8: $M = T \setminus \{ \small{ \texttt{a1\_below},~\texttt{p2}}\}$, $map \leftarrow map \wedge (a_1 \vee a_7)$

  \item Iteration 9: $M = T \setminus \{ \small{ \texttt{a2\_below},~\texttt{p1}}\}$, $map \leftarrow map \wedge (a_2 \vee a_6)$
\end{itemize}
Finally, after the ninth iteration, $map$ becomes \unsat and the algorithm terminates.
It should be pointed out that $MIS$es and $IVC$s may be discovered in different orders from what explained here. The order by which sets are explored is
quite dependent on the maximal model returned in line 6 as well as the $IVC$ returned in line 9 because there could be several distinct maximal models ($MIS$es) and $IVC$s. Another thing to consider is the power set exploration problem. For this example with a $|T| = 7$ and $|\mathcal{P}(T)| = 2^7$, a brute force approach of power set exploration needs to look into  $2^7$ cases. However, the $AIVC$ algorithm only explored 9 cases to cover the entire power set. Even so, depending on the order by which $IVC$s and $MIS$es are discovered, the total cases needed to be explored by the algorithm maybe different.
% All in all, it is fair to say the $AIVC$ algorithm is  linear in the size of $T$.
%This issue could affect the performance of the algorithm as well.

