\subsection{Preliminaries}
\label{sec:background}

\newcommand{\bool}[0]{\mathit{bool}}
\newcommand{\reach}[0]{\mathit{R}}
\newcommand{\ite}[3]{\mathit{if}\ {#1}\ \mathit{then}\ {#2}\ \mathit{else}\ {#3}}

%\subsection{Transition Systems and Safety Properties}

Given a state space $S$, a transition system $(I,T)$ consists of an
initial state predicate $I : S \to \bool$ and a transition step
predicate $T : S \times S \to \bool$. We define the notion of
reachability for $(I, T)$ as the smallest predicate $\reach : S \to
\bool$ which satisfies the following formulas:
\begin{gather*}
  \forall s.~ I(s) \Rightarrow \reach(s) \\
  \forall s, s'.~ \reach(s) \land T(s, s') \Rightarrow \reach(s')
\end{gather*}
A safety property $P : S \to \bool$ is a state predicate. A safety
property $P$ holds on a transition system $(I, T)$ if it holds on all
reachable states, i.e., $\forall s.~ \reach(s) \Rightarrow P(s)$,
written as $\reach \Rightarrow P$ for short. When this is the case, we
write $(I, T)\vdash P$.

For an arbitrary transition system $(I, T)$, computing reachability
can be very expensive or even impossible. Thus, we need a more
effective way of checking if a safety property $P$ is satisfied by the
system. The key idea is to over-approximate reachability. If we can
find an over-approximation that implies the property, then the
property must hold. Otherwise, the approximation needs to be refined.

A good first approximation for reachability is the property itself.
That is, we can check if the following formulas hold:
\begin{gather}
  \forall s.~ I(s) \Rightarrow P(s)
  \label{eq:1-ind-base} \\
  \forall s, s'.~ P(s) \land T(s, s') \Rightarrow P(s')
  \label{eq:1-ind-step}
\end{gather}
If both formulas hold then $P$ is {\em inductive} and holds over the
system. If (\ref{eq:1-ind-base}) fails to hold, then $P$ is violated
by an initial state of the system. If (\ref{eq:1-ind-step}) fails to
hold, then $P$ is too much of an over-approximation and needs to be
refined.

One way to refine our over-approximation is to add additional lemmas
to the property of interest. For example, given another property $L :
S \to bool$ we can consider the extended property $P'(s) = P(s) \land
L(s)$, written as $P' = P \land L$ for short. If $P'$ holds on the
system, then $P$ must hold as well. The hope is that the addition of
$L$ makes formula (\ref{eq:1-ind-step}) provable because the
antecedent is more constrained. However, the consequent of
(\ref{eq:1-ind-step}) is also more constrained, so the lemma $L$ may
require additional lemmas of its own. Finding and proving these
lemmas is the means by which property directed reachability (PDR)
strengthens and proves a safety property.

Another way to refine our over-approximation is to use use {\em
  $k$-induction} which unrolls the property over $k$ steps of the
transition system. For example, 1-induction consists of formulas
(\ref{eq:1-ind-base}) and (\ref{eq:1-ind-step}) above, whereas
2-induction consists of the following formulas:
\begin{gather*}
\forall s.~ I(s) \Rightarrow P(s) \\
\forall s, s'.~ I(s) \land T(s, s') \Rightarrow P(s') \\
\forall s, s', s''.~ P(s) \land T(s, s') \land P(s') \land T(s',
  s'') \Rightarrow P(s'')
\end{gather*}
That is, there are two base step checks and one inductive step check.
In general, for an arbitrary $k$, $k$-induction consists of $k$
base step checks and one inductive step check as shown in
Figure~\ref{fig:k-induction} (the universal quantifiers on $s_i$ have
been elided for space). We say that a property is $k$-inductive if it
satisfies the $k$-induction constraints for the given value of $k$.
The hope is that the additional formulas in the antecedent of the
inductive step make it provable.

\begin{figure}
\begin{gather*}
I(s_0) \Rightarrow P(s_0) \\[-2pt]
%
\vdots \\[2pt]
%
I(s_0) \land T(s_0, s_1) \land \cdots \land T(s_{k-2}, s_{k-1})
\Rightarrow P(s_{k-1}) \\[2pt]
%
P(s_0) \land T(s_0, s_1) \land \cdots \land P(s_{k-1}) \land
T(s_{k-1}, s_k) \Rightarrow P(s_k)
\end{gather*}
\caption{$k$-induction formulas: $k$ base cases and one inductive
  step}
\label{fig:k-induction}
\end{figure}

In practice, inductive model checkers often use a combination of the
above techniques. Thus, a typical conclusion is of the form ``$P$ with
lemmas $L_1, \ldots, L_n$ is $k$-inductive''.


%%% Local Variables:
%%% mode: latex
%%% TeX-master: "main.tex"
%%% End

%%  LocalWords:  bool reachability JKind Lustre PDR Yices MathSAT ok
%%  LocalWords:  SMTInterpol dataflow init
