\chapter{Timeline and Conclusion}
Establishing a reference model for integrated process simulation---a process
situated within its expected project environment---is an important step toward
reducing process-adoption risk through \apriori process evaluation.
In this work, we propose creating and validating such a reference model to
capture the essential constructs and relationships underlying agile process
models situated within a project environment.
We plan to achieve this objective
by constructing a reference model from a number of agile practices and showing
it as the basis for credible simulation models by creating a simulation framework that can simulate concrete instances of the reference
model---validating the framework using metamorphic testing.  We will validate
the reference model's expressiveness by using it to encode a number of process
models.  Additionally, we will validate the simulation framework using
metamorphic testing, leveraging known relationships between input changes
and the corresponding change to the output currently captured in the literature.

%  by
% performing metamorphic testing on a simulation framework---leveraging known relationships between transformations on inputs and their expected, relative impact on the outputs to perform metamorphic testing on the simulation framework.
%
%
% We plan to achieve this objective
% by constructing a reference model from a number of agile practices, validating
% the model's expressiveness by using it to encode a number of process models, and
% showing it as the basis for credible simulation models by performing metamorphic
% testing on the simulation, leveraging known relationships between input changes
% and the corresponding change to the output.

The timeline for the proposed work is as follows:
\begin{description}
\item[Fall - Spring 2015:]  Develop simulation framework based on the reference
model.
\item[Fall 2015:]  Express agile process models in terms of reference models
\item[April - May 2015:]   Perform metamorphic testing and refine simulation
framework
\item[May - June 2015:]  Write dissertation
\item[July 2015:]  Complete requested edits
\item[Early August 2015:]  Defend dissertation
\item[August 2015:]  Complete final changes
\end{description}



Upon completion of the proposed research, we will have a verified reference
model for expressing simulatable, integrated process models and a
simulation framework for executing and evaluating models based on the reference
model.  Our simulation models expressed using our reference model will allow the
modeler to decoupling the product and process aspects of the overall
project.
%to The reference model will be able to capture agile processes and will support
% decoupling the product and process aspects of the overall project.
Further, the reference model will provide us with a means for expressing
integrated, agile process models---models of agile processes that take into account the
individuals on the team, the product under development, and the project
constraints.  This lays the groundwork for detailed, \apriori process evaluation
through simulation and---because of having to encode the process using the
reference model---provides some degree of manual analysis.  Such process
evaluation is important, not only for \apriori process selection and tailoring,
but also for developing new processes and improving our understanding of the
limitations of agile techniques.
