
\documentclass[10pt, conference]{IEEEtran}
% Add the compsocconf option for Computer Society conferences.
%
% If IEEEtran.cls has not been installed into the LaTeX system files,
% manually specify the path to it like:
% \documentclass[conference]{../sty/IEEEtran}





% Some very useful LaTeX packages include:
% (uncomment the ones you want to load)


% *** MISC UTILITY PACKAGES ***
%
%\usepackage{ifpdf}
% Heiko Oberdiek's ifpdf.sty is very useful if you need conditional
% compilation based on whether the output is pdf or dvi.
% usage:
% \ifpdf
%   % pdf code
% \else
%   % dvi code
% \fi
% The latest version of ifpdf.sty can be obtained from:
% http://www.ctan.org/tex-archive/macros/latex/contrib/oberdiek/
% Also, note that IEEEtran.cls V1.7 and later provides a builtin
% \ifCLASSINFOpdf conditional that works the same way.
% When switching from latex to pdflatex and vice-versa, the compiler may
% have to be run twice to clear warning/error messages.




\usepackage{cite}

\ifCLASSINFOpdf
\usepackage[pdftex]{graphicx}
  % declare the path(s) where your graphic files are
  % \graphicspath{{../pdf/}{../jpeg/}}
  % and their extensions so you won't have to specify these with
  % every instance of \includegraphics
  % \DeclareGraphicsExtensions{.pdf,.jpeg,.png}
\else
  % or other class option (dvipsone, dvipdf, if not using dvips). graphicx
  % will default to the driver specified in the system graphics.cfg if no
  % driver is specified.
\usepackage[dvips]{graphicx}
  % declare the path(s) where your graphic files are
  % \graphicspath{{../eps/}}
  % and their extensions so you won't have to specify these with
  % every instance of \includegraphics
  % \DeclareGraphicsExtensions{.eps}
\fi
% graphicx was written by David Carlisle and Sebastian Rahtz. It is
% required if you want graphics, photos, etc. graphicx.sty is already
% installed on most LaTeX systems. The latest version and documentation can
% be obtained at:
% http://www.ctan.org/tex-archive/macros/latex/required/graphics/
% Another good source of documentation is "Using Imported Graphics in
% LaTeX2e" by Keith Reckdahl which can be found as epslatex.ps or
% epslatex.pdf at: http://www.ctan.org/tex-archive/info/
%
% latex, and pdflatex in dvi mode, support graphics in encapsulated
% postscript (.eps) format. pdflatex in pdf mode supports graphics
% in .pdf, .jpeg, .png and .mps (metapost) formats. Users should ensure
% that all non-photo figures use a vector format (.eps, .pdf, .mps) and
% not a bitmapped formats (.jpeg, .png). IEEE frowns on bitmapped formats
% which can result in "jaggedy"/blurry rendering of lines and letters as
% well as large increases in file sizes.
%
% You can find documentation about the pdfTeX application at:
% http://www.tug.org/applications/pdftex





% *** MATH PACKAGES ***
%
\usepackage[cmex10]{amsmath}
\usepackage{amssymb}
\usepackage{stmaryrd}
\usepackage{amsthm}
\usepackage{algorithmic}
\usepackage{array}
%\usepackage{mdwmath}
%\usepackage{mdwtab}
\usepackage{eqparbox}
%\usepackage[tight,normalsize]{subfigure}
%\usepackage[font=normalsize]{caption}
%\usepackage{tabularx,colortbl}
\usepackage[dvipsnames]{xcolor}
\usepackage{flushend}
\usepackage{cite}
\usepackage{amsmath}
%\usepackage[font=footnotesize]{subfig}
%\usepackage[caption=false,font=footnotesize]{subfig}
\usepackage{fixltx2e}
\usepackage[ruled, vlined, linesnumbered]{algorithm2e}
\usepackage{stfloats}
\usepackage{url}
\usepackage{xspace}

\hyphenation{op-tical net-works semi-conduc-tor}
\newcommand{\mkeyword}[1]{\mbox{\texttt{#1}}}
\DeclareMathOperator{\kuop}{uop}
\DeclareMathOperator{\kbop}{bop}
\DeclareMathOperator{\kite}{ite}
\DeclareMathOperator{\kpre}{pre}
\DeclareMathOperator{\dom}{dom}
\DeclareMathOperator{\ktrue}{true}
\DeclareMathOperator{\kfalse}{false}
\DeclareMathOperator{\kselect}{select}
\DeclareMathOperator{\ran}{range}
\newcommand{\lbb}{[\![}
\newcommand{\rbb}{]\!]}
\newcommand{\expr}{\phi}
\newcommand{\exprS}{\Phi}

\begin{document}

\definecolor{gold}{rgb}{0.90,.66,0}
\definecolor{dgreen}{rgb}{0,0.6,0}
\newcommand{\mike}[1]{\textcolor{red}{#1}}
\newcommand{\fixed}[1]{\textcolor{purple}{#1}}
\newcommand{\andrew}[1]{\textcolor{green}{#1}}
\newcommand{\ela}[1]{\textcolor{blue}{#1}}
\newcommand{\stateequiv}{\equiv_{s}}
\newcommand{\traceequiv}{\equiv_{\sigma}}
\newcommand{\ta}{\text{TA}}
\newcommand{\cta}{\text{TA$_{C}$}}
\newcommand{\tta}{\text{TA$_{T}$}}

\newcommand{\bfalg}{{IVC\_BF}\xspace}
\newcommand{\ucalg}{{IVC\_UC}\xspace}
\newcommand{\ucbfalg}{{IVC\_UCBF}\xspace}
\newcommand{\mustalg}{{IVC\_MUST}\xspace}

\newtheorem{definition}{Definition}
\newtheorem{lemma}{Lemma}
\newtheorem{theorem}{Theorem}
\newtheorem{coroll}{Corollary}
%\newdef{lemma}{Lemma}
%\newdef{definition}{Definition}
%\newdef{theorem}{Theorem}
%\newdef{note}{Note}
%
% paper title
% can use linebreaks \\ within to get better formatting as desired
\title{Proof-based Coverage Metrics for Formal Verification \ela{I think we proposed several metrics not just one}}


% author names and affiliations
% use a multiple column layout for up to two different
% affiliations

\author{\IEEEauthorblockN{Authors Name/s per 1st Affiliation (Author)}
\IEEEauthorblockA{line 1 (of Affiliation): dept. name of organization\\
line 2: name of organization, acronyms acceptable\\
line 3: City, Country\\
line 4: Email: name@xyz.com}
\and
\IEEEauthorblockN{Authors Name/s per 2nd Affiliation (Author)}
\IEEEauthorblockA{line 1 (of Affiliation): dept. name of organization\\
line 2: name of organization, acronyms acceptable\\
line 3: City, Country\\
line 4: Email: name@xyz.com}
}

% conference papers do not typically use \thanks and this command
% is locked out in conference mode. If really needed, such as for
% the acknowledgment of grants, issue a \IEEEoverridecommandlockouts
% after \documentclass

% for over three affiliations, or if they all won't fit within the width
% of the page, use this alternative format:
%
%\author{\IEEEauthorblockN{Michael Shell\IEEEauthorrefmark{1},
%Homer Simpson\IEEEauthorrefmark{2},
%James Kirk\IEEEauthorrefmark{3},
%Montgomery Scott\IEEEauthorrefmark{3} and
%Eldon Tyrell\IEEEauthorrefmark{4}}
%\IEEEauthorblockA{\IEEEauthorrefmark{1}School of Electrical and Computer Engineering\\
%Georgia Institute of Technology,
%Atlanta, Georgia 30332--0250\\ Email: see http://www.michaelshell.org/contact.html}
%\IEEEauthorblockA{\IEEEauthorrefmark{2}Twentieth Century Fox, Springfield, USA\\
%Email: homer@thesimpsons.com}
%\IEEEauthorblockA{\IEEEauthorrefmark{3}Starfleet Academy, San Francisco, California 96678-2391\\
%Telephone: (800) 555--1212, Fax: (888) 555--1212}
%\IEEEauthorblockA{\IEEEauthorrefmark{4}Tyrell Inc., 123 Replicant Street, Los Angeles, California 90210--4321}}




% use for special paper notices
%\IEEEspecialpapernotice{(Invited Paper)}




% make the title area
\maketitle


\begin{abstract}
Requirements are the key part of any system design. And, the common goal in formal
verification is to verify the validity of the requirements.
Once the verification process is done, one perennial question is that have we specified enough requirements? In other words, oftentimes, we are in need of a method for judgment on the quality of a system
specification. This question is akin to the problem of
test suite adequacy in testing well-addressed with different coverage metrics.
However, in the context of property-based verification, coverage is more difficult to define and compute.
This paper proposes a novel coverage notion that determines
which parts of a design are necessary to establish the proof of a given specification,
whereby requirement adequacy is achieved when all the design artifacts are
essential in the satisfaction proof of the conjunction of all requirements.
We present and implemented an efficient algorithm for quantifying requirements completeness based
on inductive proofs of
safety properties for sequential systems integrated into state-of-the-art SAT-based Model Checking. We compare/ benchmark
 our method against the existing techniques in the literature. Besides, based on the idea of inductive validity cores, we sketch
more complementary coverage notions.

\end{abstract}

\begin{IEEEkeywords}
  coverage; requirements completeness; formal verification; SAT-based model checking;
  inductive proofs; inductive validity cores;
\end{IEEEkeywords}

\IEEEpeerreviewmaketitle

\mike{Note: start with a quote on requirements completeness by Boehm or other eminence?}

%\mike{Note: mine traceability paper for ideas: abstract notion of proof?   Distinction between proof and entailment?  }
%
%\mike{Start from (a variant of) the formalism from the RE paper.  Then use inductive validity cores as an {\em implementation} of this idea, again just like the RE paper.  Bring in the IVC formalism only when we describe it as an ``implementation'' of our general idea.}


\mike{There is a very close analogue between our notion of IVC (finding minimal model elements) and the notion of mutual vacuity for requirements (maximal strengthenings of requirements); we need to look into this. }

\mike{One interesting question: is there an analogue to the ``strongest passing formula'' work by Chockler as explored in Form Methods Syst Des (2013) 43:552–571 DOI 10.1007/s10703-013-0192-6, and a maximal model weakening?  Such a thing would involve changing logical operators within the model rather than simply removing equations.}

\mike{Our running example does not have many illustrations of multiple satisfactions...should I add a "mistake" to one of the requirements?  I did this to show differences between requirements }

\mike{We could have a better example if we did some kind of sensor fusion in the model - we could have a 'pick 2 of 3' kind of thing; lots of different satisfaction paths}

\section{Introduction}
\label{sec:intro}
Most modern sequential model checking techniques for safety properties, including IC3/PDR~\cite{Een2011:PDR} and $k$-induction~\cite{SheeranSS00}, use a form of induction to establish proof.  These techniques are very powerful, and can often reason successfully over very larger or even infinite state spaces.  The proofs provided by these tools can provide rigorous evidence that a software or hardware system works as intended.

On the other hand, there many situations in which properties can be proved, but systems still will not perform as intended.  Issues such as vacuity~\cite{Kupferman03:Vacuity}, incorrect environmental assumptions~\cite{Whalen07:FMICS}, and errors either in English language requirements or formalization~\cite{Pike06:axioms} can all lead to failures of ``proved'' systems.  Thus, even if proofs are established, one must approach verification with skepticism.

Recently, Ghassabani et al.~\cite{Ghass16} introduced the idea of {\em Inductive Validity Cores} (IVCs) in order to provide additional information with proofs. IVCs offer proof explanation as to why a property is satisfied by a model in a formal and human-understandable way.  The idea lifts UNSAT cores~\cite{zhang2003extracting}
to the level of sequential model checking algorithms using induction.  Informally, if a model is viewed as a conjunction of constraints,
a minimal IVC (MIVC) is a set of constraints that is sufficient to construct a proof such that if any constraint is removed, the property is no longer valid.
IVCs and MIVCs can be used for several purposes, including performing traceability between specification and design elements, assessing model coverage, and explanation of unsatisfiable test obligations when using model checkers for test case generation. Ghassabani et al.~\cite{Ghass16} presented two algorithms: \ucalg, which computes an approximately minimal IVC that is computationally inexpensive, and \ucbfalg,
an algorithm that produces a
MIVC but is considerably more expensive to compute.
%
The IVC idea shares many similarities with approaches for computing minimal
invariant sets for inductive proofs (such as is performed for inductive proof certificates~\cite{piskac2016, ivrii2014small}), and in fact the \ucalg\ algorithm performs a minimal lemma set computation.  However, there is a substantive difference: to find a guaranteed minimal set of constraints, it is usually necessary to find new proofs involving {\em new lemmas} not used in the original proof, which accounts for the expense of the \ucbfalg\ algorithm.

It is often the case that there are multiple MIVCs for a given property.  In this case, the algorithms from~\cite{Ghass16} give, at best, an
incomplete picture of the traceability information associated with the proof.  Depending on the model and property to be analyzed, there is often substantial diversity between the IVCs used for proof, and there can be substantial difference in the size between the {\em minimal} IVC returned by the \ucbfalg\ algorithm and a {\em minimum} IVC, which is the (not necessarily unique) smallest MIVC.
 If {\em all} MIVCs can be found, then several additional analyses can be performed:
\begin{itemize}
    \item Coverage Analysis: MIVCs can be used to define coverage metrics by examining the percentage of model elements required for a proof.  However, since MIVCs are not unique, there are multiple, equally legitimate coverage scores possible.  Having \emph{all} MIVCs allows one to define additional metrics: coverage of MAY elements, coverage of MUST elements, as well as policies for the existing MIVC metric: e.g., choose the smallest MIVC. %\ela{I'm not sure if introducing MAY/MUST would make sense to the readers }
    \item Optimizing Logic Synthesis:  synthesis tools can benefit from MIVCs in the process of transforming an abstract behavior into a design implementation. A practical way of calculating MIVCs allows to find a minimum set of design elements (optimal implementation) for a certain behavior. Such optimizations can be performed at the different levels of synthesis.
    \item Impact Analysis: Given all MIVCs, it is possible to determine which requirements may be falsified by changes to the model.  This analysis allows for selective regression verification of tests and proofs: if there are alternate proof paths that do not require the modified portions of the model, then the requirement does not need to be re-verified.
    \item Robustness Analysis: As proposed by Murugesan et. al in~\cite{Murugesan16:renext}, it is possible partition the model elements into MUST and MAY sets based on whether they are in every MIVC or only some MIVCs, respectively.  This may allow insight into the relative importance of different model elements for property.  For example, if the MUST set is empty, then the requirement has been implemented in multiple ways, such as would be expected in a fault-tolerant system.  Moreover, examining the diversity of all MIVCs could lead to changes in how traceability
        ~\cite{COEST,cleland2007best}
     %~\cite{COEST,hayes2003improving,cleland2007best}
        is performed and managed in critical systems.
\end{itemize}
%\noindent In addition, the Requirements Engineering community is keenly interested in approaches to manage requirements traceability.  In most cases, it is assumed that there is a single ``golden'' set of trace links that describes how requirements are implemented in software~\cite{COEST,hayes2003improving,cleland2007best}.  However, if there are multiple MIVCs, then it is possible that there are several equally valid sets of trace links.  Examining the diversity of all MIVCs could lead to changes in how traceability is performed and managed in critical systems.

As far as commercial tools are concerned, we have found some of them that use the term \emph{proof-core} ~\cite{hanna2015formal, jasper_gold}, which sounds similar to the idea of a \emph{single} MIVC. However,
to the best of our knowledge, none of them offer the calculation of \emph{all} proof-cores.
Moreover, solutions provided by these tools are quite underspecified:
no formal description of the proof-core notion or algorithms are provided. In addition, no implementations or experimental results are provided, so it is not possible to compare their approach with IVCs.

In this paper, we propose a new method for computing \emph{all minimal} IVCs. In  recent  years,  a  number  of  efficient
algorithms  for  extracting  all MUSes  have  been proposed \cite{bacchus2015using, belov2012muser2, belov2013core, belov2012towards, nadel2014accelerated, liffiton2005max}.  In this paper, we adapt the recent work by Liffiton et al. \cite{marco2016fast} from the generation of MUSes from UNSAT-cores to all IVCs for inductive model checking.  This requires changing the underlying mechanisms that are used to construct candidate solutions and also changing the structure of the proof of correctness.  In addition, in our proof, we demonstrate that the approach terminates with all minimal IVCs even if the witness generator only generates approximately minimal IVCs (utilizing the ``fast'' \ucalg\ algorithm from~\cite{Ghass16}).  In our empirical results, this allows our algorithm to be quite efficient to the extent that in many cases, the cost of extracting all minimal IVCs is similar to the cost of finding a single guaranteed-minimal IVC, and on average is approximately 1.6x the cost of determining a single minimal IVC.
The contributions of the work are therefore as follows:
\begin{itemize}
    \item An algorithm for computing all minimal IVCs.
    \item A proof of correctness and completeness of the algorithm.
    \item An evaluation of the algorithm for performance and diversity of result sets against a benchmark suite.
    \item An industrial case study with over 10K design elements that demonstrates the practicality and usefulness of our technique.
\end{itemize}

%\ela{I think we need to make it clear that IVCs are different from MUSes, proof-certificates or minimal invariants, abstraction, slicing. Currently, the introduction doesn't say anything about these. You had an idea on having a table... Perhaps you want to include a discussion section?\\ Or, Maybe we could expand the introduction with these things and make it more motivating}

%\ela{Also, I think the contributions don't stand out. finding \emph{all} \textbf{minimal} IVCs itself is two contribution. I think minimality is important. Maybe we should stress on it a little bit more}

The rest of the paper is organized as follows.
Section \ref{sec:example} introduces a running example used to illustrate concepts and our method.
Section \ref{sec:background} covers the formal preliminaries for the approach.
In Section \ref{sec:allivcs}, we present our method for enumerating all minimal IVCs,
which is illustrated in
Section \ref{sec:illust}. In Section \ref{sec:impl}, we talk about implementation and evaluation of our method. Section \ref{sec:qfc} presents an industrial case study. Finally, Section \ref{sec:conc} mentions conclusions and future work. 

%\ifdefined\TECHREPORT
%\input{description}
%\fi
\section{Running Example}
\label{sec:example}

%\begin{figure*}
%\begin{center}
%\includegraphics[width=0.8\textwidth]{figs/ex.png}
%\vspace{-0.1in}
%\caption{A Lustre model with property $P$}
%\label{fig:ex}
%\end{center}
%\end{figure*}

%% We put the image here so it shows up side-by-side with fig:ex-after
\begin{figure}[t]
\centering
\includegraphics[width=0.8\columnwidth]{figs/code.jpg}
%{\smaller
%\begin{verbatim}
%node asw(alt1, alt2: int; inhibit: bool)
%        returns (doi_on: bool);
%var
%   a1_below, a2_below, a1_above, a2_above,
%   below, above_hyst, d1, d2: bool;
%let
%   a1_below = (alt1 < THRESHOLD);        // (1)
%   a2_below = (alt2 < THRESHOLD);        // (2)
%   a1_above = (alt1 >= T_HYST);          // (3)
%   a2_above = (alt2 >= T_HYST);          // (4)
%   below = a1_below or a2_below;         // (5)
%   above_hyst = a1_above and a2_above;   // (6)
%   doi_on = if (below and not inhibit)   // (7)
%        then true else d1;
%   d1 = if (inhibit or above_hyst)       // (8)
%         then false else d2;
%   d2 = (false -> pre(doi_on));          // (9)
%tel;
%\end{verbatim}
%}
\vspace{-0.1in}
\caption{Altitude Switch Model}
\label{fig:asw}
\end{figure}

We will use a very simple system from the avionics domain to illustrate our approach. An Altitude Switch (ASW) is a hypothetical device that turns power on to another subsystem, the Device of Interest (DOI), when the aircraft descends below a threshold altitude and turns the power off again after we ascend over the threshold plus some hysteresis factor.  An implementation of an ASW containing two altimeters written in the Lustre language (simplified and adapted from~\cite{HCW02:ase-deviation}) is shown in Fig.~\ref{fig:asw}.  If the system is not ``inhibited'' by the user and either altimeter is below the constant THRESHOLD, then it turns on the DOI; else, if the system is inhibited or both altimeters are above the threshold plus the hysteresis factor (THRESHOLD + HYST), then the DOI is turned off, and if neither condition holds, then in the initial computation it is false and thereafter retains its previous value.  The notation \texttt{(false -> pre(doi\_on))} in equation (9) describes an initialized register in Lustre: in the initial state, the expression is \texttt{false}, and thereafter it is the previous value of \texttt{doi\_on}.  In the remainder of the paper, we will use this model to illustrate aspects of requirements completeness.  %Section \ref{sec:illust}.


\iffalse
\mike{replace with altitude switch example!}

We will use the model in Fig.~\ref{fig:ex} (a) as
a running example throughout the paper. This model is written in Lustre~\cite{Halbwachs91:lustre}, which is a synchronous dataflow language used as an input language for various model checkers. For our purposes, a Lustre program
consists of 1) input variables, {\tt in1} and {\tt in2} in the example, 2) output
variables, {\tt P} in the example, and 3) an
equation for each output variable. A Lustre program runs over discrete
time steps. On each step, the input variables take on some values and
are used to compute values for the output variables on the same step.
In addition, equations may refer to the previous value of a variable
using the {\tt pre} operator, like {\tt x} in the example. This operator is underspecified in the
first step, so the arrow operator, {\tt ->}, is used to guard the
{\tt pre} operator. In the first step the expression {\tt e1 -> e2}
evaluates to {\tt e1}, and it evaluates to {\tt e2} in all other steps. We interpret a Lustre program as a model specification by considering
the behavior of the program under all possible input traces. Safety
properties over Lustre can then be expressed as Boolean expressions in
Lustre. A safety property holds if the corresponding expression is
always true for all input traces. For example, the property for
Fig.~\ref{fig:ex} is {\tt P}, which is a valid property.

In the example, the structure of the model allows property {\tt P} to be proved in two ways.
Note that {\tt c1}, {\tt r1}, and {\tt r2} do not affect the validity of {\tt P}, which means
if we remove these equations making {\tt c1}, {\tt r1}, and {\tt r2} input, still {\tt P} will be valid.
In other words, no matter what value these equations have, any change in their value is not observable in the value of {\tt P}.
However, we always need equation {\tt c3} to prove {\tt P}. In addition to {\tt c3}, in order for {\tt P} to be valid either {\tt c2} or {\tt x} is required.

In the rest of the paper, we will refer to this example while explaining different coverage notions.
\fi

%% \begin{itemize}
%%     \item Not sure if this should go before or after the background section with a description of Lustre.
%%     \item Need a small but interesting example.  Andrew, do any of the models that you use as jkind tests
%%         function in this way?  It would be nice to look at what we have lying around; we need something
%%         that requires invariants.
%%     \item It would also be good to have a few points of interest with the model-requirement pairing:
%%     \item \quad   vacuity due to an overconstrained environment
%%     \item \quad   definitions within the model that are irrelevant to the proof.
%%     \item Explain the model and the proof process.
%% \end{itemize}

%%  LocalWords:  IVC


%\section{Preliminaries}
\label{sec:background}

\newcommand{\bool}[0]{\mathit{bool}}
\newcommand{\reach}[0]{\mathit{R}}
\newcommand{\ite}[3]{\mathit{if}\ #1\ \mathit{then}\ #2\ \mathit{else}\ #3}

\subsection{Transition Systems and Safety Properties}

Given a state space $S$, a transition system $(I,T)$ consists of an
initial state predicate $I : S \to \bool$ and a transition step
predicate $T : S \times S \to \bool$. We define the notion of
reachability for $(I, T)$ as a the smallest predicate $\reach : S \to
\bool$ which satisfies the following formulas:
\begin{equation*}
  \forall s.~ I(s) \Rightarrow \reach(s)
\end{equation*}
\begin{equation*}
  \forall s, s'.~ \reach(s) \land T(s, s') \Rightarrow \reach(s')
\end{equation*}
When the transition system is not obvious from context we will write
$\reach_{(I,T)}$ for the reachability predicate on the transition
system $(I,T)$.

A safety property $P : S \to \bool$ is a state predicate. A safety
property $P$ holds on a transition system $(I, T)$ if it holds on all
reachable states, i.e., $\forall s.~ \reach(s) \Rightarrow P(s)$,
written as $\reach \Rightarrow P$ for short.

For an arbitrary transition system $(I, T)$, computing reachability
can be very expensive or even impossible. Thus, we need a more
effective way of checking if a safety property $P$ is satisfied by the
system. The key idea is to over-approximate reachability. If we can
find an over-approximation that implies the property, then the
property must hold. Otherwise, the approximation needs to be refined.

A good first approximation for reachability is the property itself.
That is, we can check if the following formulas hold:
\begin{equation}
  \forall s.~ I(s) \Rightarrow P(s)
  \label{eq:1-ind-base}
\end{equation}
\begin{equation}
  \forall s, s'.~ P(s) \land T(s, s') \Rightarrow P(s')
  \label{eq:1-ind-step}
\end{equation}
If both formulas hold then $P$ is {\em inductive} and holds over the
system. If (\ref{eq:1-ind-base}) fails to hold, then $P$ is violated
by an initial state of the system. If (\ref{eq:1-ind-step}) fails to
hold, then $P$ is too much of an over-approximation and needs to be
refined.

One way to refine our over-approximation is to add additional lemmas
to the property of interest. For example, given another property $L :
S \to bool$ we can consider the extended property $P'(s) = P(s) \land
L(s)$, written as $P' = P \land L$ for short. If $P'$ holds on the
system, then $P$ must hold as well. The hope is that the addition of
$L$ makes formula (\ref{eq:1-ind-step}) provable because the
antecedent is more constrained. However, the consequent of
(\ref{eq:1-ind-step}) is also more constrained, so the lemma $L$ may
require additional lemmas of its own.

Another way to refine our over-approximation is to use use {\em
  $k$-induction} which to unrolls the property over $k$ steps of the
transition system. For example, 1-induction consists of formulas
(\ref{eq:1-ind-base}) and (\ref{eq:1-ind-step}) above, whereas
2-induction consists of the following formulas:
\begin{align*}
  \forall s.~ I(s) \Rightarrow P(s)
&&  \forall s, s'.~ I(s) \land T(s, s') \Rightarrow P(s')
\end{align*}
\begin{equation*}
  \forall s, s', s''.~ P(s) \land T(s, s') \land P(s') \land T(s', s'')  \Rightarrow P(s'')
\end{equation*}
That is, there are two base step checks and one inductive step check.
In general, for an arbitrary $k$, $k$-induction consists of the $k$
base step checks and one inductive step check as shown in
Figure~\ref{fig:k-induction}. We say that a property is $k$-inductive
if it satisfies the $k$-induction constraints for a the given value of
$k$. The hope is that the additional formulas in the antecedent of
the inductive step make it provable.

\begin{figure}
\begin{equation*}
  \forall s_0.~ I(s_0) \Rightarrow P(s_0)
\end{equation*}
\begin{center}
$\vdots$
\end{center}
\begin{equation*}
  \forall s_0, \ldots, s_{k-1}.~ I(s_0) \land T(s_0, s_1) \land \cdots
  \land T(s_{k-2}, s_{k-1}) \Rightarrow P(s_{k-1})
\end{equation*}
\begin{equation*}
  \forall s_0, \ldots, s_k.~ P(s_0) \land T(s_0, s_1) \land P(s_{k-1})
  \land T(s_{k-1}, s_k) \Rightarrow P(s_k)
\end{equation*}
\caption{$k$-induction formulas: $k$ base cases and one inductive
  step}
\label{fig:k-induction}
\end{figure}

In practice, we often use a combination of the above techniques. Thus,
a typical conclusion is of the form ``$P$ with lemmas $L_1, \ldots, L_n$
is $k$-inductive''.

\subsection{JKind and Lustre}

\marginpar{AJG: I think we should move this section to implementation.
  We can cut it down and also talk about what set-of-support means in
  Lustre vs general transition systems.}

JKind is an infinite-state model checker for safety properties. JKind
proves safety properties using multiple cooperative engines in
parallel including $k$-induction, property directed reachability
(PDR), and template-based lemma generation. JKind operates over
expressions in the theory of linear integer and real arithmetic. In
the back-end, JKind uses an SMT-solver such as Yices, Z3, CVC4,
MathSAT, or SMTInterpol.

\begin{figure}[t]
\begin{verbatim}
node main(x : int) returns (r : int; ok : bool);
let
  r = (0 -> pre r) + (if x > 0 then x else -x);
  ok = (r >= 0);
tel;
\end{verbatim}
  \caption{Example Lustre program}
  \label{fig:lustre-ex}
\end{figure}

The input language to JKind is Lustre, a synchronous dataflow language.
An example Lustre program is shown in Figure~\ref{fig:lustre-ex}. For
our purposes, a Lustre program consists of 1) some input variables,
{\tt x} in the example, 2) some output variables, {\tt r} and {\tt ok}
in the example, and 3) an equation for each output variable. A Lustre
program runs over discrete time steps. On each step, the input
variables take on some values and are used to compute values for the
output variables on the same step. In addition, equations may refer to
the previous value of a variable using the {\tt pre} operator. This
operator is undefined in the initial step, so the arrow operator, {\tt
  ->}, is used to guard such the {\tt pre} operator. In the initial
step the expression {\tt e1 -> e2} reduces to {\tt e1}, and it
reduces to {\tt e2} in all other steps.

We interpret a Lustre program as a model specification by considering
the behavior of the program under all possible input traces. Safety
properties over Lustre can then be expressed as Boolean output
variables in Lustre. A safety property holds if the corresponding
Boolean output variable is always true for all input traces. For
example, the program in Figure~\ref{fig:lustre-ex} represents an
integrator over the absolute value of the input variable. The output
variable {\tt ok} is a safety property of the system expressing that
the computed result is always non-negative. In this case, the property
is true.

It is easy to translate this interpretation of Lustre into the
traditional initial and transition relations. We will show this by
example using Figure~\ref{fig:lustre-ex}. First we introduce a new Boolean
variable $init$ into the state space to denote when the system is in
its initial step. Then we define,
\begin{align*}
  &I((x, r, \mathit{ok}, \mathit{init})) = \mathit{init} \\
  &T((x, r, \mathit{ok}, \mathit{init}), (x', r', \mathit{ok'},
  \mathit{init'})) = \\
  &\hspace{1cm} (r' = (\ite{init}{0}{r})) \land (\mathit{ok'} =
  (r' \geq 0)) \land \neg\mathit{init'}
\end{align*}
Each equation in the Lustre program is translated into a conjunct in
the transition relation. A safety property such as {\tt ok} is
translated into $\mathit{init} \lor \mathit{ok}$. Nested uses of arrow
and pre operators are handled by introducing new output variables for
nested expressions, though such details are unimportant for our
purposes.


%% Prevous plan:
%%
%% Symbolic transition systems (use material from Sheeran's "Induction using a SAT Solver" paper?)
%% Lustre language
%% UNSAT cores
%% jkind
%% more here?

%%% Local Variables:
%%% mode: latex
%%% TeX-master: "main.tex"
%%% End

%%  LocalWords:  bool reachability JKind Lustre PDR Yices MathSAT ok
%%  LocalWords:  SMTInterpol dataflow init


\section{Preliminaries}
\label{sec:background}
\newcommand{\satisfies}{\vdash_{\!\!s}}
\newcommand{\nsatisfies}{\nvdash_{\!\!s}}
\newcommand{\bool}[0]{\mathit{bool}}
\newcommand{\reach}[0]{\mathit{R}}
\newcommand{\ite}[3]{\mathit{if}\ {#1}\ \mathit{then}\ {#2}\ \mathit{else}\ {#3}}
\newcommand{\nondetcov}{\text{\sc Nondet-Cov}}
\newcommand{\mutcov}{\text{\sc Mutant-Cov}}

\subsection{Models, Requirements, and Provability}

Given a state space $S$, a transition system $(I,T)$ consists of an
initial state predicate $I : S \to \bool$ and a transition step
predicate $T : S \times S \to \bool$. We define the notion of
reachability for $(I, T)$ as the smallest predicate $\reach : S \to
\bool$ which satisfies the following formulas:
\begin{gather*}
  \forall s.~ I(s) \Rightarrow \reach(s) \\
  \forall s, s'.~ \reach(s) \land T(s, s') \Rightarrow \reach(s')
\end{gather*}
A safety property $P : S \to \bool$ is a state predicate. A safety
property $P$ holds on a transition system $(I, T)$ if it holds on all
reachable states, i.e., $\forall s.~ \reach(s) \Rightarrow P(s)$,
written as $\reach \Rightarrow P$ for short. When this is the case, we
write $(I, T)\vdash P$.
We assume the transition relation of the system has the structure of a top-level conjunction. This assumption gives us a structure that we can easily manipulate. Given $T(s, s') = T_1(s, s') \land \cdots \land T_n(s, s')$ we will write $T = T_1 \land \cdots \land T_n$ for short.
By further abuse of notation we will identify
$T$ with the set of its top-level conjuncts. Thus we will write $x \in
T$ to mean that $x$ is a top-level conjunct of $T$. We will write $S
\subseteq T$ to mean that all top-level conjuncts of $S$ are top-level
conjuncts of $T$.
We will write $T \setminus \{x\}$ to mean $T$
with the top-level conjunct $x$ removed. Such a transition system can easily encode our example model in Section~\ref{sec:example}.  We assume each equation defines a conjunct within the transition system which we will denote by the variable assigned, so $T = \{$ {\small \texttt{a1\_below, a2\_below, a1\_above, a2\_above, below, above\_hyst}} $\}$.
\ela{I leave this here for now. Maybe we should move it later to the place where we're talking about granularity}

\ela{since we are going to submit this paper to a HW oriented community, I think using properties or specification instead of requirements would be better?}

\begin{definition}{\emph{Provability:}}
Let $(I, T)$ be a transition system; given a set of safety properties $\Delta$ and $S \subseteq T$, $r \in \Delta$ is \emph{provable} by $S$ iff
$(I, S) \vdash r$.
\end{definition}

\begin{definition}{\emph{Inductive Validity Cores:}}
  \label{def:ivc}
  Given $(I, T)\vdash r$, $S \subseteq
  T$ is a set of {\em inductive validity cores} for $r$
  iff $r$ is provable by $S$;
\end{definition}

In examining provability, we are interested in {\em minimal} sets that satisfy a property $r$; tracing a property to the entire implementation     \ela{should we say target artifacts instead of implementation?}   is not particularly enlightening.

\begin{definition}{\emph{Minimal Inductive Validity Cores (IVC):}}
  \label{def:minimal-ivc}
  A set of inductive validity cores $S$ for $(I, T)\vdash r$ is minimal, denoted $IVC(r, S)$, iff
  $\neg\exists S' \subset S .\quad (I, S') \vdash r $
\end{definition}

For the sake of simplicity, hereafter, we refer a minimal set of inductive validity cores as an "IVC" set.
Note that there could be many IVC sets for a given property, corresponding to different proofs. To capture that notion, we define \emph{all IVC sets ($AIVC$)} for a property as an association to all its IVC sets.

$$ AIVC(r) \equiv  \{\ S~|~S \subseteq T \land  IVC(r, S)\} $$

%In the example in Fig. \ref{fig:asw}, as visualized in part (b),
%$AIVC ({\tt P}) = \{\{{\tt P}, {\tt c2}, {\tt c3}\}, \{{\tt P}, {\tt x}, {\tt c3}\}\}$.
\noindent The set of $AIVC$-s for all properties represents the complete traceability of the system. Establishing the $AIVC$ for a single property, one gets a clear picture of the all the model elements that are necessary to prove the property.

\subsection{Coverage and Mutations}
In general, specification completeness can be defined with
regard to the notion of coverage. In fact, the way that coverage
is formalized plays a key part in the strength/ effectiveness of
a method for the assessment of completeness. The goal of a coverage metric is usually to assign a numeric score that describes how well properties cover the design. The majority of the work on coverage metrics has focused on {\em mutations}, which are ``atomic'' changes to the design, where the set of possible mutations depends on the notation that is used.  A mutant is ``killed'' if one of the requirements that is satisfied by the original design is violated by the mutated design~\cite{chockler_coverage_2003,chockler2001practical,chockler2010coverage,Kupferman:2006:SCF,kupferman_theory_2008}.  There are Many different kinds of mutations that have been proposed, primarily focused on checking sequential bit-level hardware designs.  For these designs, {\em State-based} mutations flip the value of one of the bits in the state.  There are several variations depending on whether the flip is performed on a single state within a Kripke structure~\cite{hoskote1999coverage}, or in the description of the signal in the transition relation of the circuit~\cite{chockler2001practical}.  {\em Logic-based} mutations fix the value of a bit to constant zero or one, and can be used to determine whether requirements can find stuck-at faults.  {\em Syntactic} mutations~\cite{chockler_coverage_2003} remove states in a control flow graph representation of hardware.  Similarly, for software, it is possible to apply any of the ``standard'' source code mutation operators used for software testing~\cite{Andrews06:mutation} towards requirements coverage analysis.  Some examples of software mutations are:
\begin{enumerate}
    \item Replace an integer constant C by one of $\{0, 1, -1, C + 1, C - 1\}$.
    \item Replace an arithmetic, relational, logical, bitwise logical, increment/decrement, or arithmetic-assignment operator by another operator from the same class.
    \item Negate the decision in an if or while statement
    \item Delete a statement
\end{enumerate}

We assume each element $T_i \in T$ has a set of possible mutations associated with it.  Depending on the modeling formalism used, this may be the value of a gate or signal or an expression within a statement in a program.  We will further assume the existence of a mutation function $f_{m}$ that, given a model element, will return a finite set of mutations for that element.  We can then define the set of mutant models $M$ as follows:
\[
    M = \{ T_i \in T, m \in f_{m}(T_i)\ |\ T \setminus \{T_i\} \cup \{m\} \}
\]

\noindent and then define the mutation score for a set of properties $\Delta$ in the standard way:

\begin{definition} {\emph{Generalized mutation coverage.} } \\
\[
   \mutcov = \frac{ | \{m \in M(T)~|~ m \nvdash \Delta\} |}{|M(T)|}
\]
\end{definition}
\ela{the above definition is missing something. $m \nvdash \Delta$ isn't clear I think}

In our example in Figure~\ref{fig:asw}, applying the software mutations from~\cite{Andrews06:mutation} would involve manipulating the constants used in the definitions of \texttt{a1\_below, a2\_below, a1\_above, a2\_above}, swapping 'or' and 'and' in the definition of \texttt{below, above\_hyst}, or negating the conditions in the if/then/else statements.  Even for this small model, note that there are a large number of possible mutations: 57 given the set defined above, and that this number increases rapidly with the size of the program and the chosen set of mutations.

Of particular interest is the mutation that replaces a computed variable ({\em signal} in hardware) with a ``fresh'' input; this mutation is called a {\em nondeterminism mutation} with a coverage metric called (\nondetcov)~\cite{chockler2010coverage} and is discussed in~\cite{Kupferman:2006:SCF,kupferman_theory_2008,chockler2010coverage}.  If we use an equational transition system to assign the variables, then performing \nondetcov\ coverage an isomorphic operation to removing the defining equation from the set $T$ and checking whether provability is preserved.  In this case, we can dispense with the set $M$ and compute a mutation score much more simply. In one sense, the nondeterminism mutation is the {\em strongest} mutation because it introduces the most additional behaviors into the model, that is, any execution sequence constructed by modifying the assigning equation is also an execution sequence for a nondeterministic mutation.  Equivalently, given a set of universal properties, it is the easiest mutation to ``kill''.  For our example in Figure~\ref{fig:asw}, this mutation would lead to 10 mutations, one for each equation in the model.

\begin{definition} {\emph{Nondeterministic coverage.} }
\label{def:non-det}
$\forall r \in \Delta,\quad T_i \in T$, a nondeterministic coverage can be formalized as a boolean function $\zeta(r, T_i)$. This function maps $r$ and $T_i$ to $true$ iff $r$ covers $T_i$, denoted by $r \rightarrow_{\zeta} T_i$,
 otherwise it returns $false$, denoted by $r \nrightarrow_{\zeta} T_i$.
\end{definition}
\ela{check the above def. I think the notation that Mike said,
$T_i \in \zeta(r)$, makes $zeta$ a relation. but, it's actually a
boolean function with 2 arguments that only returns true or false: $T_i$ is either covered by $r$ or not}

\ela{I want this definition because it makes further formalizations and notations easier and simpler}

\ela{I call the following single mutation, because in addition to be non-det,
each time only one design element gets unconstrained}
\begin{definition} {\emph{Single mutation coverage ~\cite{chockler2010coverage}.} }
\label{def:single-mut}
$\forall r \in \Delta$, $T_i \in T$, $r$ covers $T_i$ iff
$(I, T) \vdash r$ and $(I, T \setminus \{T_i\}) \nvdash r$.
\end{definition}

For the sake of simplicity, we refer to the coverage function
formalized in Definition \ref{def:single-mut} as $\zeta_{sm}$. Using  $\zeta_{sm}$, the coverage score of specification $r$ is computed by
\[
   \frac{ | \{T_i \in T~|~ T \setminus \{T_i\} \nvdash r\} |}{|T|}
\]





\newcommand{\minproofcov}{\text{\sc MinProof-Cov}}

%\clearpage
\section{Proof-Based Metrics}
\label{sec:method}

We propose a new approach for measuring property completeness based on proof rather than mutation.  We first define notation, then describe different possible metrics given a set of {\em minimal proofs}.%\footnote{Section~\ref{sec:impl} describes how these proofs are discovered in practice.}
%\subsection{Coverage and Minimal Proofs}
%Alternatively, we can consider using the proofs themselves as a mechanism for determining adequacy of requirements.

\begin{definition} {\emph{IVC coverage (\ivccov):}} \\
\label{def:coverage-justi}
Given $S \in AIVC(P)$, $T_i$ is covered by $P$ via $S$ \emph{iff} $T_i \in S$.
%Given $S \in AIVC(P)$, $T_i \in T$ is covered by $P$ \emph{iff} $T_i \in S$,
%denoted by $T_i \in \ivccov (P, S)$
\end{definition}

%For the sake of simplicity, we refer to the coverage function
%formalized in Definition \ref{def:coverage-justi} as \ivccov\.
%
We call Definition \ref{def:coverage-justi} a \emph{proof-preserving} metric because, with a set of the model elements marked as covered by \ivccov, $P$ is provable.  Other notions, as will be discussed in Section~\ref{subsec:method-disc}, may yield subsets of the model that are insufficient to reconstruct the proof of the property.
%\footnote{\noindent ~Throughout the paper, when a coverage metric is justifiable, like \ivccov, we say that it preserves provability of the property.}
%Thus, the coverage score for \ivccov\ is often higher than the score for \nondetcov.
The coverage score for \ivccov\ can be calculated with: $$\frac{|S|}{|T|}$$
%Note that because minimal proofs are not unique, there are several possible coverage scores.
Because $P$ may have multiple \mivc s,  \ivccov\ metric can lead to various scores that belong to the following set:
\[
\{~\frac{ |S|}{|T|}~|~S \in AIVC(P)~\}
\]

\noindent Note that if an \mivc ~contains all model elements (i.e., the model is {\em completely covered}), then there is only one possible \mivc , so in this case there is no diversity of scores.

%the model is {\em completely covered}, on the other hand, then there is only one possible minimal set: the set of all elements.

Given {\em all} proofs of a particular property, it is possible to define additional, complementary coverage notions.  To do so, we use the following categorization of the model elements based on \mivc ~and $AIVC$ relations for $P$:
\begin{itemize}
 % \item  \textbf{$MUST$} contains model elements in all the \mivc s of $P$.
%      %$$ MUST_x = \{\forall i (S_xi \in \Sigma_x) \mid \bigcap S_xi \}$$
%      \[
%      MUST (P) = \bigcap AIVC(P)
%      \]
%
%  \item  \textbf{$MAY$} includes model elements that are used in some, but not all, \mivc s.
%      \[
%      MAY(P) = (\bigcup AIVC (P)) \setminus MUST (P)
%      \]
%
%  \item  \textbf{$IRR$} specifies model elements that are not in any of the possible \mivc s of $P$.
%  $$IRR(P) = T \setminus (\bigcup AIVC (P))$$
\item $MUST (P) = \bigcap AIVC(P)$
\item $MAY(P) = (\bigcup AIVC (P)) \setminus MUST (P)$
\item $IRR(P) = T \setminus (\bigcup AIVC (P))$
\end{itemize}

\noindent This categorization helps to identify the role and relevance of each design element in satisfying a property. Function $MUST$ specifies the parts of the model absolutely necessary for the property satisfaction.  Any change to these parts will affect provability of the property. On the other hand, any single element in $MAY (P)$, may be modified without affecting satisfaction of $P$(though modifying multiple elements may require re-proof). The $IRR$ denotes model elements that are irrelevant to the validity of $P$ \cite{Murugesan16:renext}.

Using the notions of $MAY$ and $MUST$, we can introduce additional coverage metrics.
%Since the primary goal of
% this paper has been to provide a complementary coverage notion in
%  formal verification, it is worth exploring other possible notions based on the idea of provability and $AIVC$, which is beneficial, as with testing, because if a coverage notion is an over-approximation, when the coverage
% is high, it does not necessarily mean the quality of
% the specification (or test suite) is high, or when it is an under-approximation, a low coverage score does not always mean the specification is of poor quality.

\begin{definition} {\emph{(\maycov):}}
  \label{def:comp-1}
 $T_i \in T$ is covered by $P$ \emph{iff} $T_i \in \maycov (P)$, where
   $\maycov (P) = \{T_i ~|~ \exists S \in AIVC(P)~.~T_i \in S \}$.
\end{definition}

\begin{definition} {\emph{(\mustcov):}}
  \label{def:mustcov}
 $T_i \in T$ is covered by $P$ \emph{iff} $T_i \in \mustcov (P)$, where
   $\mustcov (P) = \{T_i ~|~ \forall S \in AIVC(P)~.~T_i \in S \}$.
\end{definition}

The $\maycov$ notion aims to deal with the fact that a property $P$ may have
several distinct \mivc s. In such cases, \ivccov\ only looks at an arbitrary \mivc\
that may contain a subset of $MAY(P)$, which means, depending on
which \mivc\ it considers, every time it may report a different part of $MAY(P)$
as uncovered. However, \maycov\ resolves this issue reporting the entire set of $MAY(P)$ as covered, which also leads to higher coverage scores.  \mustcov\ takes the opposite view, considering a model element as covered only if it affects all the proofs of $P$.


It is still possible to build more relaxed coverage metrics in which coverage
is captured by looking at individual properties, rather than their conjunction.
%for example, in the definition of \ivccov , it is wise to look at $P$ as
%the conjunction of all properties. However,
We can, for example, describe a metric in which any element used by an \mivc ~for any property is considered covered.
%with this view,
%elements around IVCs that do not have common \emph{must}
%elements with others will be treated as uncovered while they are at least covered by one
% IVC of an individual property in the specification.
%
The next definition, \allcov, formalizes this notion.
\begin{definition} {\emph{(\allcov):}}
  \label{def:comp-2}
     Given a set of properties $\Delta$ over $T$, $T_i \in T$ is covered
   \emph{iff} $T_i \in \allcov (T)$, where
   $\allcov (T) = \{T_i ~|~ \exists P \in \Delta ,~ S \in AIVC(P).~T_i \in S \}$.
\end{definition}

%Considering $MAY$ and $MUST$ categorization, we can formalize another
%coverage metric that takes into account the \emph{must} set;
%however, such a metric is the same as \nondetcov\ as we discuss in the next sub-section.

\subsection{Discussion}
\label{subsec:method-disc}


Based on the categorization of elements, we will state some relationships about \mivc s so to compare different proof-based metrics proposed earlier.

\begin{lemma}
  \label{lem:must-not-enough}
  If $MAY(P) \neq \varnothing$, then $P$ is not provable by $MUST(P)$.
\end{lemma}
\begin{proof}
  $MAY(P) \neq \varnothing \Rightarrow  \exists T_i \in MAY(P).$
$T_i \in \bigcup AIVC(P) \wedge T_i \notin MUST(P)$,
which implies $\exists S \in AIVC(P).~ T_i \in S$.
Considering the fact that $S$ is minimal and
$MUST(P) \subset S$ (since $T_i \in S \wedge T_i \notin MUST(P)$),
 $\nexists S' \subset S.~ (I,S') \vdash P$,  which means $(I, MUST(P)) \nvdash P$.
\end{proof}
\vspace{2mm}

%\begin{lemma}
%    \label{lem:must-mustcov}
%    $T_i \in MUST(P) \Leftrightarrow T_i \in \mustcov(P)$
%\end{lemma}
%\begin{proof}
%Immediate from the definition of $MUST$ and \mustcov.
%\end{proof}

Now we focus on the relationship between non-deterministic mutation-based coverage and proof-based metrics. In Chockler et. al. \cite{chockler2010coverage}, each mutant design changes the type of a single node to \inputnode\ (see Section \ref{sec:background}).
Given a suitable encoding of the netlist, assigning a ``fresh'' input is an isomorphic operation to simply removing a $T_i$ from $T$. The mapping is as follows: the net-list becomes a conjunction
of equations, where each vertex becomes a variable $v_i \in U$, and where each non-input vertex becomes an assignment equation $T_i \in T$.
For example, given an AND-vertex $v_i$ with three input edges from other vertexes $\{v_a, v_b, v_c\}$, we would define an equation $T_i \in T$ of the form $(v_i = (v_a \wedge v_b \wedge v_c))$.
%
%As the variable is no longer constrained by a defining equation, it is effectively an %input.

Given this encoding, we can reframe the non-deterministic coverage proposed in \cite{chockler2010coverage} as follows:

\begin{definition} {\emph{Nondeterministic coverage (alternate specification) (\nondetcovalt) ~\cite{chockler2010coverage}.} }
\label{def:non-det-2}
$T_i \in T$ is covered by property $P$ \emph{iff} $T_i \in \nondetcovalt (P)$, where
$\nondetcovalt (P) = \{T_i~|~ (I, T) \vdash P \wedge (I, T \setminus \{T_i\}) \nvdash P\}$.
\end{definition}


%\begin{definition} {\emph{Nondeterministic coverage alternate definition (\nondetcovalt) ~\cite{chockler2010coverage}.} }
%\label{def:non-det-2}
%$T_i \in T$ is covered by property $P$ \emph{iff} $T_i \in \nondetcovalt (P)$, where
%$\nondetcovalt (P) = \{T_i~|~ (I, T) \vdash P \wedge (I, T \setminus \{T_i\}) \nvdash P\}$.
%\end{definition}
%
%\begin{lemma}
%    \label{lem:nondet-nondetaltcov}
%    $\nondetcov(P) = \nondetcovalt(P)$
%\end{lemma}
%\begin{proof}
%\mike{obvious?} \ela{   not so sure if obvious}
%\end{proof}

\noindent Given this definition, it becomes straightforward to define some additional properties.

\begin{lemma}
  \label{lem:must-coverage}
$T_i \in \nondetcovalt (P) \Leftrightarrow T_i \in \mustcov(P)$.
\end{lemma}
\begin{proof}
$T_i \in \nondetcovalt (P)$ means that $(I, T \setminus \{ T_i \}) \nvdash P$ then
%$T_i$ is necessary to prove $P$,  which means
$\forall S \subset T .~ T_i \notin S \Rightarrow (I, S) \nvdash P$.
Therefore, since $(I, T) \vdash P$, $T_i \in \bigcap AIVC(P)$, which means  $T_i \in MUST(P)$.
On the other hand, let $T_i \in MUST(P)$; then $\forall S \in AIVC(P).~ T_i \in S$.
By definition, any proof of $P$ is a superset of some minimal IVC in $AIVC(P)$.
Thus, any subset $S$ of $T$ leading to proof contains $T_i$.
Therefore, $T \setminus \{ T_i \}$ does not lead to a proof.
%On the other hand, by definition, $MUST(P)$ is the intersection of all IVCs.
%From the definition of $MUST$, removing a $T_i \in MUST(P)$ from $T$
%results in $ \bigcap AIVC(P) \setminus \{ T_i \} $.
%And since all IVCs in $AIVC$ are \emph{minimal} removing an element from all possible IVCs makes
% $P$ unprovable by every single of them:
% $\forall S \in AIVC(P),~ T_i \in \bigcap AIVC(P).~ (I, S \setminus \{ T_i \}) \nvdash P$. And, we know $S \subseteq T$, so $S \setminus \{ T_i \} \subseteq T \setminus \{ T_i \}$, which means the reachable states of
% $(I, T \setminus \{ T_i \})$ are a subset of the reachable states from
%   $(I, S \setminus \{ T_i \})$. Therefore,
%   $ (I, S \setminus \{ T_i \}) \nvdash P \Rightarrow (I, T \setminus \{ T_i \}) \nvdash P$.
\end{proof}
\vspace{2mm}

In light of Lemma \ref{lem:must-coverage}, the \nondetcovalt\ coverage score of specification $P$ can be also calculated by
$$\frac{|MUST(P)|}{|T|}$$
%Therefore, for set of properties $\Delta$, the coverage score is computed by $$\frac{|MUST(\Pi)|}{|T|},\quad  \Pi= \bigwedge_{i} {P_i \in \Delta}$$


%\mike{after all metrics presented, contrast them on the example.  Introduce the properties HERE and then discuss the coverage sets}
%
%\mike{Then, you can talk about justification, etc.}
\begin{coroll}
\label{cor:must-not-provable}
\nondetcovalt\ is not proof-preserving.
\end{coroll}
\begin{proof}
Immediate from Lemma \ref{lem:must-not-enough} and Lemma \ref{lem:must-coverage}
\end{proof}
\vspace{2mm}
\begin{coroll}
\label{cor:ivc-provable}
\ivccov\ is proof-preserving.
\end{coroll}
\begin{proof}
Immediate from Definition~\ref{def:minimal-ivc} and Definition \ref{def:coverage-justi}
\end{proof}
\vspace{2mm}

%It should be pointed out that \ivccov\ is accurate meaning that it does not result in false positives. In other words, since IVCs are \emph{minimal}, \ivccov\ does not mark
%any \emph{actual} uncovered element as covered.

To conclude this section, we should mention that one can define many more proof-based coverage metrics based on the $\mivc /AIVC$ idea. Metrics that make use of the $AIVC$ relation are computationally more expensive to compute than \ivccov\ although they might be easier to satisfy (i.e., result in higher coverage scores).

%\ela{please read the following paragraph and improve it. I've been trying to justify why we only have implementation for \ivccov \\}
The proposed coverage metrics can be ranked in terms of their scores as follows:
$$\nondetcovalt\ \leq \ivccov\ \leq \maycov\ \leq \allcov$$
\ivccov\ and \nondetcovalt\ are equivalent when all elements within the model are covered: if all model elements are MUST elements, then there can only be one \mivc , and this \mivc ~uses all of the model elements.   In the implementation and experiments, we will focus on the \ivccov\ and \nondetcovalt\ metrics.  Both metrics are fairly rigorous and can be computed reasonably efficiently.  The equivalence of \mustcov\ and \nondetcovalt\ allows us to compare our algorithms against state-of-the-art mutation based coverage.


%However, we will show \ivccov\ is a lot cheaper to compute
%and in terms of rigor, it is neither too hard (like \nondetcovalt)
%nor too easy (like \allcov) to satisfy.
%Let $\Delta$ be a set of properties over $T$. When we define a new property $P = \bigwedge p_i$, where $p_i \in \Delta$, possible \mivc s of $P$ are the parts of the models that affect the proof of every property $p_i \in \Delta$. In this way, we are able to identify portions of the model that are constrained by some properties without having to calculate the $AIVC$ relation for individual properties, which will significantly reduce computational cost.
%\footnote{Next section will illustrate this idea.} For these reasons, in Section \ref{sec:impl},
%we consider computation of \ivccov\
%and in order to benchmark it against the existing methods,
%we also provide implementation of \mustcov\ which is the same as \nondetcovalt .

%Based on our preliminary evaluation, we believe that metrics based on
%$AIVC$ relation (like \maycov\ and \allcov) are approximately as computationally expensive as \nondetcov.\footnote{\noindent ~The reason is that \nondetcov\ computes the must set which is also based on $AIVC$ relation. However, in terms of preserving provability, a set of design elements marked as covered by \allcov\ and \maycov\ are
%sufficient to reconstruct the proof of the properties.}
%In the following sections, we first illustrate how the different metrics measure coverage of our ASW example with some sample requirements, and then perform a larger experiment with the \nondetcov\ and \ivccov\ metrics.

%So, in order to examine the proof-based metrics, Section \ref{sec:impl} considers the implementation of two major notions: \nondetcov\ and
% \ivccov ; because \nondetcov\ is based on a recent work in the literature,
% and among all the other proposed notions, \ivccov\ is the
% one that does not take into account $AIVC$.
 %Besides, in terms of coverage score, \ivccov\ is not too easy (or hard) to satisfy.  

\input{Illustration}

\section{Implementation}
\label{sec:impl}

The algorithm for efficiently computing IVCs can be found in a forthcoming FSE paper~\cite{Ghass16} and is implemented in the JKind \cite{jkind}, which is an infinite-state model checker for safety properties using multiple cooperative engines in parallel (such as k-induction and PDR). JKind accepts
Lustre programs written over the theory of linear integer and real
arithmetic. In the back-end, JKind uses an SMT solver such as
Z3, Yices, MathSAT, or SMTInterpol.
JKind works on multiple properties simultaneously. When a
property is proven and IVC generation is enabled, an additional
parallel engine executes the IVC generation algorithm to compute a minimal
IVC. We demonstrated the efficiency and precision of the approach using a set of Lustre models developed
as a benchmark suite for~\cite{Hagen08:FMCAD}, augmented with additional models from industrial projects (~\cite{QFCS15:backes,hilt2013}). The results show that our algorithm for computing IVCs is quite efficient even for industrial models with an average overhead of ~10\%. 

\section{Experiment}
\label{sec:experiment}

%\mike{What do we want to call our efficient algorithm: IVC?}

We would like to investigate both the {\em efficiency} and {\em
  minimality} of our three algorithms: the n{\"a}ive brute-force
algorithm (\bfalg), the UNSAT core-based algorithm (\ucalg), and the
combined UNSAT core followed by brute-force minimization algorithm
(\ucbfalg). Efficiency is computed in terms of wall-clock time: how
much overhead does the IVC algorithm introduce? Minimality is
determined by the size of the IVC: cores with a smaller number of
variables are preferred to cores with a larger number of variables.
Finally, we are interested in the {\em diversity} of solutions: how
often do different tools/algorithms generate different minimal IVCs?

The use of JKind allows additional dimensions to our investigation: it supports two different inductive algorithms: $k$-induction and PDR, and a ``fastest'' mode, that runs both algorithms in parallel.  In addition, JKind supports multiple back-end SMT solvers including Z3~\cite{DeMoura08:z3}, Yices~\cite{Dutertre06:yices}, MathSAT~\cite{Cimatti2013:MathSAT}, and SMTInterpol~\cite{Christ2012:SMTInterpol}.  We would like to determine whether the choice of inductive algorithm affects the size of the IVC, whether different solvers are more or less efficient at producing IVCs, and whether running different solvers/algorithms leads to {\em diversity} of IVC solutions.

Therefore, we investigate the following research questions:
\begin{itemize}
    \item \textbf{RQ1:} How expensive is it to compute inductive validity cores using the \bfalg, \ucalg, and \ucbfalg algorithms?
    \item \textbf{RQ2:} How close to minimal are the support sets computed by \ucalg as opposed to the (guaranteed minimal) \ucbfalg?  How do the sizes of IVCs compare to static slices of the model?
    \item \textbf{RQ3:} How much {\em diversity} exists in the solutions produced by different solver/induction algorithm configurations?
\end{itemize}

\subsection{Experimental Setup}
In this study, we started from a suite of 700 Lustre models developed
as a benchmark suite for~\cite{Hagen08:FMCAD}. We augmented this suite
with 82 additional models from recent verification projects including
avionics and medical devices~\cite{QFCS15:backes,hilt2013}. Most of
the benchmark models from~\cite{Hagen08:FMCAD} are small (10k or less,
with 6-40 equations) and contain a range of hardware benchmarks and
software problems involving counters. The additional models are much
larger: around 80k with over 300 equations. We added the new
benchmarks to better check the scalability for the tools, especially
with respect to the brute force algorithm.
%
%\mike{MORE HERE...stats on size, reasons for add'l models.}
Each benchmark model has a single property to analyze.  For our purposes, we are only interested in models with a {\em valid} property (though it is perhaps worth noting that there is no additional computation---and thus no overhead---using the JKind IVC options for {\em invalid} properties).  In our benchmark set, 295 models yield counterexamples, and 10 additional models are neither provable nor yield counterexamples in our test configuration (see next paragraph for configuration information).  The benchmark suite therefore contains 476 models with valid properties, which we use as our test subjects.

For each test model, we computed \ucalg in 12+1 configurations: the
twelve configurations were the cross product of all solvers \{Z3,
Yices, MathSAT, SMTInterpol\} and inductive algorithms
\{$k$-induction, PDR, fastest\}, and the remaining (+1) configuration
was an instance of \bfalg run on Yices, which is the default solver in
JKind. In addition, for each of the 12 configurations, we ran an
instance of JKind without IVC to examine overhead. The experiments
were run on an Intel(R) i5-2430M, 2.40GHz, 4GB memory machine, with a
1 hour timeout for each analysis on any model. The data gathered for
each configuration of each model included the time required to check
the model without IVC, with IVC, and also the set of elements in the
computed IVC.\footnote{The benchmarks, all raw experimental results,
  and computed data are available on \cite{expr}.}

Note that not all analysis problems were solvable with all algorithms: for all solvers, $k$-induction (without IVC) was unable to solve 172 of the examples.  When comparing minimality of different solving algorithms, we only considered cases where both algorithms provided a solution (as will be discussed in more detail in Section~\ref{sec:minimality}).

\iffalse
\begin{itemize}
    \item an algorithm to compute a truly minimal set of support, i.e. \texttt{JSupport}.
    \item given a LUS model, a static crawler which automatically marks all equations of a node in the initial support set of a property.
    \item some trackers that measure the verification time with/ without support computation.
   % \item some minor changes in the XML writers.
\end{itemize}

\mike{My thoughts on this section: mostly, it needs more structure: more information on the properties of the models: size, provenance, etc., a broken out subsection on the description of the experimental setup, etc}

\mike{I think we want to split out the results in another top-level section}

Experiment:
\begin{itemize}
    \item (Overview) describe research questions and goals.
    \item Experimental setup: tell me about the models: how many, how big are they?  Then, tell me about the experiment: the tool configurations, the machine used for test.
    \item Data generation: Describe what you measured for each model analysis.
\end{itemize}
\fi


%%  LocalWords:  minimality ive UNSAT IVC Minimality IVCs PDR Yices
%%  LocalWords:  MathSAT SMTInterpol RQ JSupport


In this section, we examine our experimental results to address the above research questions over the performance of our algorithm.

\begin{figure}[t]
 \centering
  \includegraphics[width=\textwidth]{figs/performance.jpg}
  \label{fig:performance}
  \vspace{-0.2in}
  \caption{Runtime of \aivcalg, \ucbfalg, and \ucalg ~algorithms}
\end{figure}
\vspace{0.1in}
\textbf{RQ1)} We measured the performance overhead of the algorithms over the time
necessary to find a proof using inductive model checking. Fig. \ref{fig:performance}
 allows a visualization of the  overhead  of the \aivcalg ~algorithm  in  comparison  with \ucalg ~and \ucbfalg.
 In the figure, the models are ranked along the x-axis by the number of IVCs found by \ucalg ~per model.
 Table \ref{tab:runtime} and Table \ref{tab:overhead} also provide a summary of the computation time and the overhead of different algorithms.
 As it can be seen, the \ucalg ~algorithm imposes a negligible overhead to the proof time and is quite fast, whereas \ucbfalg ~algorithm adds a substantial penalty in order to find a single IVC.
 The \aivcalg ~algorithm is able to outperform the \ucbfalg ~in a lot of cases,
 or perform approximately the same.
 Note that the \aivcalg ~algorithm could have had better (worse) performance
 if timeout had been set lower (higher), which caused the average runtime (overhead) of the \aivcalg ~shown in Table \ref{tab:runtime} (Table \ref{tab:overhead}) to be $\thicksim 3$ times more than \ucbfalg .


\begin{table}
  \caption{Runtime of different computations}
   \vspace{-0.1in}
  \centering
  \begin{tabular}{ |c||c|c|c|c| }
    \hline
      runtime (sec)& min & max & mean & stdev \\[0.5ex]
    \hline\hline
    \emph{\small{proof-time}}    & 0.047 & 14.617 & 1.299 & 1.940 \\[0.5ex]
    \aivcalg    & 10.125 & 2375.058& 58.884 & 256.529 \\[0.5ex]
    \ucbfalg &   0.248 & 1323.515 &  17.247& 104.838\\[0.5ex]
    \ucalg&  0.0  & 1.422  & 0.084 & 0.184 \\[0.5ex]
    \hline
  \end{tabular}
  \label{tab:runtime}
\end{table}

\begin{table}
  \caption{Overhead of different algorithms}
   \vspace{-0.1in}
  \centering
  \begin{tabular}{ |c||c|c|c|c| }
    \hline
     algorithm & min & max & mean & stdev \\[0.5ex]

    \hline
    \aivcalg   & 13.642\% & 101034.615\% & 2544.399\% & 7764.159\% \\[0.5ex]
    \ucbfalg &   14.092\% & 111124.432\% &  882.018\% & 1512.071\%\\[0.5ex]
    \ucalg&  0.00\%  & 100.00\%   & 10.226\% & 11.718\% \\[0.5ex]
    \hline
  \end{tabular}
  \label{tab:overhead}
\end{table}

%\takeaway{Computing all minimal Inductive Validity Cores with the \aivcalg ~algorithm is as nearly expensive as computing one single minimal Inductive Validity Core with the \ucbfalg  ~algorithm.
%\ela{Ela: is that fair to say??}}
\vspace{0.1in}
\textbf{RQ2)} The structure of the model and specification can play a part in how well \aivcalg ~performs.
Therefore, we would like to examine whether or not there is a relationship between the performance and the size of the model, proof-time, and the diversity of IVCs. A graph showing the size of each model (determined by the number of equations in the model) and the number of IVCs
 along with the running time of \aivcalg ~and normal verification time is shown in Fig \ref{fig:modelsize}. In the figure, the models are ranked along the x-axis by their size. The picture shows that as models get larger, it is more likely for the \aivcalg ~algorithm to take more time to complete. However, there is no straightforward relationship between the performance and the number of IVCs. It can be expected that the running time of the \aivcalg ~algorithm goes higher when verification takes more time.

 \begin{figure}[t]
 \centering
  \includegraphics[width=\textwidth]{figs/size.jpg}
  \label{fig:modelsize}
  \vspace{-0.2in}
  \caption{Runtime of different computations along with the number of IVCs}
\end{figure}

\ela{do we want this table 3? I commented the table description in case you want to use it...}
%Table \ref{tab:ivcsize} compares
%the  minimum,  maximum,  average,
%and standard deviation of the size of the IVCs computed by the different algorithms.
%For the \aivcalg ~algorithm, the  minimum,  maximum,  average,
%and standard deviation of the size of the IVCs per model is calculated, and then again, these four measures are calculated among all models.
%As for the \texttt{minimum IVC} row, the four measures are calculated among the size of the minimum IVC generated by \aivcalg ~for each model.
%The size of IVCs computed by \ucalg ~and \ucbfalg ~are quite close to each other. It means the \ucalg ~algorithm computes IVCs that are very close to the minimal ones obtained from the \ucbfalg , which makes the \ucalg ~algorithm a reasonable choice for the \getivc ~procedure in Algorithm \ref{alg:aivc}
%although it does not guarantee minimality.
%Fig. \ref{fig:ratio} also demonstrates that average cost of \aivcalg ~per IVC is very close to average cost of finding one IVC by \ucbfalg. Given the fact that the size of IVCs generated by \ucalg ~is very close to the ones generated by \ucbfalg, makes the \ucalg ~algorithm
%more efficient for the \getivc ~procedure.
\begin{table}
  \caption{Size of IVCs from different computations}
   \vspace{-0.1in}
  \centering
  \begin{tabular}{ |c||c|c|c|c| }
    \hline
     algorithm & min & max & mean & stdev \\[0.5ex]

    \hline
    \aivcalg   & 1 & 159 & 12.462 & 1.1684 \\[0.5ex]
    \ucalg   & 1 & 141 & 12.754 & 16.000 \\[0.5ex]
    \ucbfalg &   1 & 141 &  12.185 & 16.107\\[0.5ex]
    \texttt{minimum IVC} & 1  & 134  & 12.078 & 15.550 \\[0.5ex]
    \hline
    \end{tabular}
  \label{tab:ivcsize}
\end{table}

 %\begin{figure}
% \centering
%  \includegraphics[width=\textwidth]{figs/ratio.jpg}
%  \label{fig:ratio}
%  \vspace{-0.2in}
%  \caption{Runtime of \aivcalg ~along divided by the number of IVCs vs the runtime of other computations}
%\end{figure}
\subsubsection {Discussion}
\ela{rewrite...\\}
\textbf{RQ4)} In the benchmarks, there have been
several models containing undecidable cases which affected the average
performance of \aivcalg ~reported in Table \ref{tab:runtime}. Moreover, such cases also influence the minimality of the IVCs computed by \aivcalg. It is also possible that an iteration of
the \texttt{while} loop in Algorithm \ref{alg:aivc} times out while
the adequacy of the subset under examination is decidable in general.
We were interested in determining how often it is possible to come across such cases in our benchmarks. Size of IVCs obtained from different algorithms is shown in Fig. \ref{fig:min}.
As you can see, there are 14 cases among 476 models for which the size of
 minimum IVC computed by the \aivcalg ~is bigger than the size of the minimal IVC
 generated by the \ucbfalg ~algorithm.
 \begin{figure}
 \centering
  \includegraphics[width=\textwidth]{figs/min.jpg}
  \label{fig:min}
  \vspace{-0.2in}
  \caption{Size of IVCs obtained from different computations}
\end{figure}


\section{Discussion}
\label{sec:discussion}

In Section \ref{sec:method}, we have proposed a new coverage notion that
is more practical to use. Although the minimality of $SOS$ makes $\psi_{sos}$ accurate
 both in terms of preserving provability and not having false positives, as discussed, the exact implementation of $\psi_{sos}$ is based on the \ucbfalg algorithm, which is as nearly expensive as the \mustalg algorithm for $\psi_{sm}$.
 To alleviate this issue, we came up with an efficient implementation for $\psi_{sos}$, i.e. \ucalg \cite{Ghass16}, 
 which is an over-approximation and might not be always accurate in terms of minimality. 
 However, the idea of IVCs or support sets makes
it possible to have other coverage metrics in formal verification, which is beneficial, as with testing, because if a
coverage notion is an over-approximation, when the coverage
 is high, it does not necessarily mean the quality of
 the specification (or test suite) is high, or when it is an under-approximation, a low coverage does not always mean the specification is of poor quality.

 As mentioned in Section \ref{sec:impl}, support sets are derived from an inductive invariant; in other words, they are built upon the proof of the validity of a given property. One interesting fact about proofs
  is that a given property could be proved from different proof paths. That's why we defined $ASOS(r)$ in Section \ref{sec:background}. The $ASOS$ set gives a clear picture of various ways a property is satisfied. By getting all the support sets for all requirements of the system and categorizing them, one can find if there are target artifacts that do not trace to any requirement: set $\bigcap \{IRR (r) | r \in \Delta \}$.  If this set is non-empty, it is a possible indication of ``gold plating" or missing requirements. That is to say, it helps to assess if the requirements of the system describe all the behaviors of the system. Being able to measure the coverage of requirements over the model is crucial in the safety critical system domain.

Very recent, as yet unpublished, work has focused on the
generation of all support sets (IVC sets), whose preliminary evaluation
shows the overhead in discovering all support sets is a linear in the
number of unique support set in the problem multiplied by the cost
for finding a proof for a single support set. For complex models, such
as the ones described in \cite {QFCS15:backes} and \cite{hilt2013}, it has been possible to
find $ASOS$ for individual properties in a matter of minutes.
Based on our preliminary results we expect computing $ASOS$ to be computationally feasible for complex models. In
addition, we believe that it is possible to use the information
from the set of all IVCs to more efficiently produce minimal
IVCs than the \ucbfalg algorithm.

Since the primary goal of
 this paper has been to provide a complementary coverage notion in
  formal verification, having an efficient technique to practically compute $ASOS$, it is worth exploring other possible notions based on the idea of provability and $ASOS$.

\begin{definition} {\emph{Complementary coverage notion 1:}}
  \label{def:comp-1}
   $ R = \bigwedge_{i} {r_i \in \Delta}, \varphi \in \Gamma,  \psi (R) \preccurlyeq \varphi$ iff $ \exists S
   \in ASOS(R)$. $\varphi \in S$.
\end{definition}

\begin{figure}
\begin{center}
\includegraphics[width=\columnwidth]{figs/disc.png}
\vspace{-0.1in}
\caption{A visual example to illustrate other coverage notions with $ASOS$}\label{fig:disc}
\end{center}
\end{figure}

Consider Fig. \ref{fig:disc} (a); it represents two distinct support sets for $ R = \bigwedge_{i} {r_i \in \Delta}$. If \ucalg returns support set $\{ A, B, C, D\}$, then $\psi_{sos}$ will
report elements $\{ F, G, H\}$ as uncovered, while they are not. However,
the coverage function proposed in Definition \ref{def:comp-1} will mark all the elements in the example as covered.

\begin{definition} {\emph{Complementary coverage notion 2:}}
  \label{def:comp-2}
   $\forall \varphi \in \Gamma,  \psi (R) \preccurlyeq \varphi$ iff $ \exists S
   \in ASOS(r), r \in \Delta$. $\varphi \in S$.
\end{definition}

Another example provided in Fig. \ref{fig:disc} (b) shows a design with
three requirements $\{ r_1, r_2, r_3\}$. As you can see, each requirement points to its
own support sets; for example, $r_3$ has two support sets,
which means $ASOS(r_3) = \{ \{B, H\}, \{I, J\}\}$. In this example,
function $\psi_{sos}$ marks $\{I, J\}$ as uncovered, while they are actually covered in a way by $r_3$.
However, Definition \ref{def:comp-2} will consider them as covered by $r_3$.
Note that $\psi_{sm}$ is even much harder to satisfy; for instance, in Fig. \ref{fig:disc},
it will only consider $\{ A, B\}$ as covered.

These two definitions proposed in this section are computationally much more expensive than the metric in Definition \ref{def:coverage-ivc}. However, they are easier to satisfy. A preliminary evaluation shows that they are as nearly expensive as the metric computed by the \mustalg algorithm for Definition \ref{def:coverage1}.
In terms of preserving provability, a set of design elements marked as covered by Definition \ref{def:comp-1} and \ref{def:comp-2} are
sufficient to reconstruct the proof of any requirement in $\Delta$.



\section{Related work}
\label{sec:related}

In recent years, extraction of Minimally Unsatisfiable Subformulas (MUSes) has been the focus of a lot of research work \cite{marques2010minimal, belov2012towards, ryvchin2011faster, belov2012computing, nadel2010boosting, ryvchin2011faster, }. Although algorithms proposed by such work can handle very large problems,
computing MUSes is still very resource-intensive task.
While some work aimed to provide a set of minimal unsatisfiable formulae, they define minimality in
a way that given a set of clauses \mathbb{M}, removing every member of \mathbb{M} makes it satisfiable
\cite{belov2012computing}. 
Such algorithms are often compared with each other. In this work, we compare a regular computation of minimal unsat-core against minimum unsat-core. In addition, our focus is not to provide a novel way of computing minimal unsat-core. Instead, we makes use of MUSes to efficiently compute a set of support in a model necessary for inductive proofs.

Nadel, in \cite{nadel2010boosting}, discusses a 
number of applications of MUS extraction in formal verification. 
Gupta et al. \cite{gupta2003iterative} and McMillan and Amla \cite{mcmillan2003automatic} introduced the use of unsatisfiable cores in proof-based abstraction engines. Their goal is to shrink the abstraction size by omitting the parts of the design that are irrelevant to the proof of the property
under verification. However, \cite{gupta2003iterative, mcmillan2003automatic} do not consider core minimization. To our knowledge, none of the existing work
has used MUS to provide support information that explains
the correctness of proofs provided by different inductive techniques
including PDR and k-induction.
negative result.

\cite{torlak2008finding} proposes an algorithm for finding MUSes of declarative specification implemented for the Alloy language. Alloy is a framework for describing high-level design of various systems, whose analyzer is a fully automatic constraint solver. Constraints are translated into propositional logic solved by a SAT solver; hence, the analysis considers only a finite number of values for each type. For this reason, even for a set of simple constraints, the analyzer is never able to prove the correctness of a property. A major difference between this work and ours is that we 
extract UNSAT cores from an inductive proof over a sequential model involving lemmas. In addition, Alloy mostly works based on SAT solving, instead of SMT solving. In our implementation, JKind supports a variety of powerful SMT solvers (such as Z3, Yices, Yices2, etc.).

\begin{itemize}
    \item MUS's : checked
    \item Work on Alloy: checked
    \item Work that Teme pointed us to : will be added
    \item Anything else Elaheh has found : \%60 checked
\end{itemize}



\section{Conclusions \& Future Work}
\label{sec:conclusion}

In this paper, we have defined a novel coverage notion for formal verification using
the IVC concept, a useful measure in relation to
a valid safety property for inductive model checking. We have shown that our method
 is computationally efficient while 
 being accurate about the covered parts of a given design. 
 We have referred to this accuracy as preserving provability, which means 
 that a set of elements considered covered by our algorithm is sufficient 
 to establish the validity proof for every requirement in the set of specifications.
 
 We have implemented
our algorithm as part of the open source model checker JKind. Using our approach, measuring coverage is quite possible as we have shown in our experiments.
 We also benchmarked our implementation and compared it with other techniques in the literature. 
 The experiments show that the computation imposes a small overhead to the verification process. We have described how the justifiable notion of coverage proposed in this paper can be used as a
means of quantifying requirements completeness.
 
 In addition, based on the idea of multiple support sets for a specification, we 
 have introduced and discussed some other complementary coverage notions in the context of formal verification. Finally, we are in the process of developing some efficient algorithms for exploring the space of IVCs, e.g., finding a
minimum, rather than minimal support set, or finding all support sets. Having such algorithms makes the utilization of other proposed coverage practical.

%\section*{Acknowledgment}
%
%
%The authors would like to thank...
%more thanks here


\bibliographystyle{IEEEtran}
\bibliography{biblio}
\end{document}


