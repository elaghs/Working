\section{Introduction}
\label{sec:intro}

Symbolic model checking using induction-based techniques such as IC3/PDR and k-induction can often determine whether safety properties hold of complex finite or infinite-state systems.  Model checking tools are attractive both because they are automated, requiring little or no interaction with the user and, if the answer to a correctness query is negative, they provide a counterexample to the satisfaction of the property.  However, when it comes to the valid properties, oftentimes, tools do not provide any useful information as to why a given property holds. It is well known that issues such as vacuity and inconsistencies can cause verification to succeed despite errors in a property specification or in the model. Therefore, the level of feedback provided by the tool to the user matters. What we would like to provide is traceability information, an inductive validity core (IVC), that explains the proof. We introduce a generic and efficient mechanism for extracting supporting information, similar to an UNSAT core, from the proofs of safety properties using inductive techniques such as PDR and k-induction. Our technique allows efficient and precise extraction of IVCs even in the presence of auxiliary lemmas.
Once generated, the IVCs can be used for many purposes in the verification process, including vacuity detection, requirement completeness checking, traceability, and inconsistency detection. This paper briefly explains the IVC idea along with some useful applications of it. The rest of the paper is organized as follows. Section \ref{sec:method} explains the notion of IVCs. Section \ref{sec:impl} briefly talks about its implementation. In Section \ref{sec:disc}, we discuss our ongoing research activities on IVCs talking about its applications. Finally, Section \ref{sec:conclusion} concludes our work.

