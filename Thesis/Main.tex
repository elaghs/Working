%%%%%%%%%%%%%%%%%%%%%%%%%%%%%%%%%%%%%%%%%%%%%%%%%%%%%%%%%%%%%%%%%%%%%%%%%%%%%%%%
% DEFINE DOCUMENT TYPE
%%%%%%%%%%%%%%%%%%%%%%%%%%%%%%%%%%%%%%%%%%%%%%%%%%%%%%%%%%%%%%%%%%%%%%%%%%%%%%%%
\documentclass[11pt, oneside]{mnthesis}


%%%%%%%%%%%%%%%%%%%%%%%%%%%%%%%%%%%%%%%%%%%%%%%%%%%%%%%%%%%%%%%%%%%%%%%%%%%%%%%%
% IMPORT PACKAGES
%%%%%%%%%%%%%%%%%%%%%%%%%%%%%%%%%%%%%%%%%%%%%%%%%%%%%%%%%%%%%%%%%%%%%%%%%%%%%%%%
\usepackage{epic,eepic,units}
\usepackage{hyperref}
\usepackage{url}
%\usepackage{tikz}
%\usetikzlibrary{arrows,shapes,positioning}
% From:
% http://tex.stackexchange.com/questions/12703/how-to-create-fixed-width-table-columns-with-text-raggedright-centered-raggedlef
\usepackage{array}
\newcolumntype{L}[1]{>{\raggedright\let\newline\\\arraybackslash\hspace{0pt}}p{#1}}
\newcolumntype{C}[1]{>{\centering\let\newline\\\arraybackslash\hspace{0pt}}p{#1}}
\newcolumntype{R}[1]{>{\raggedleft\let\newline\\\arraybackslash\hspace{0pt}}p{#1}}

%\usepackage{longtable}
%\usepackage{mathrsfs}
%\usepackage{multirow}
%\usepackage{bigstrut}
%\usepackage{amssymb}
%\usepackage{graphicx}
\usepackage{cite}
\usepackage{paralist}
\usepackage[stable]{footmisc}   %Lets us use footnotes in section headers.
% \usepackage[style={square, numbers}]{natbib}
% \usepackage{amsmath}
% \usepackage{amssymb}
% \usepackage{amsthm}
% \usepackage{caption}
% \usepackage[vertfit]{breakurl}
\usepackage{listings, multicol}
\usepackage{enumitem}
\usepackage{subcaption}  % subfigures
\usepackage{bibentry}
% \usepackage[section, numberedsection=autolabel, nonumberlist]{glossaries}

%Dealing with dutch names:
% http://tex.stackexchange.com/questions/40747/bibtex-handling-of-the-dutch-van-name-prefix-with-natbib
\DeclareRobustCommand{\VAN}[2]{#2}
\usepackage {graphicx}
\usepackage[cmex10]{amsmath}
\usepackage{amssymb}
\usepackage{stmaryrd}
\usepackage{amsthm}
\usepackage{algorithmic}
\usepackage{array}
%\usepackage{mdwmath}
%\usepackage{mdwtab}
\usepackage{eqparbox}
%\usepackage[tight,normalsize]{subfigure}
%\usepackage[font=normalsize]{caption}
%\usepackage{tabularx,colortbl}
\usepackage[dvipsnames]{xcolor}
\usepackage{flushend}
\usepackage{cite}
\usepackage{amsmath}
%\usepackage[font=footnotesize]{subfig}
%\usepackage[caption=false,font=footnotesize]{subfig}
\usepackage{fixltx2e}
\usepackage[ruled, vlined, linesnumbered]{algorithm2e}
\usepackage{stfloats}
\usepackage{url}
\usepackage{xspace}
\theoremstyle{definition}
\hyphenation{op-tical net-works semi-conduc-tor}
\newcommand{\mkeyword}[1]{\mbox{\texttt{#1}}}
\DeclareMathOperator{\kuop}{uop}
\DeclareMathOperator{\kbop}{bop}
\DeclareMathOperator{\kite}{ite}
\DeclareMathOperator{\kpre}{pre}
\DeclareMathOperator{\dom}{dom}
\DeclareMathOperator{\ktrue}{true}
\DeclareMathOperator{\kfalse}{false}
\DeclareMathOperator{\kselect}{select}
\DeclareMathOperator{\ran}{range}
\newcommand{\lbb}{[\![}
\newcommand{\rbb}{]\!]}
\newcommand{\expr}{\phi}
\newcommand{\exprS}{\Phi}
\newcommand{\bool}[0]{\mathit{bool}}
\newcommand{\reach}[0]{\mathit{R}}
\newcommand{\ite}[3]{\mathit{if}\ {#1}\ \mathit{then}\ {#2}\ \mathit{else}\ {#3}}

\definecolor{gold}{rgb}{0.90,.66,0}
\definecolor{dgreen}{rgb}{0,0.6,0}
\newcommand{\mike}[1]{\textcolor{red}{#1}}
\newcommand{\fixed}[1]{\textcolor{purple}{#1}}
\newcommand{\andrew}[1]{\textcolor{green}{#1}}
\newcommand{\ela}[1]{\textcolor{blue}{#1}}
\newcommand{\stateequiv}{\equiv_{s}}
\newcommand{\traceequiv}{\equiv_{\sigma}}

\newcommand{\bfalg}{\texttt{\small{IVC\_BF}}}
\newcommand{\ucalg}{\texttt{\small{IVC\_UC}}}
\newcommand{\ucbfalg}{\texttt{\small{IVC\_UCBF}}}
\newcommand{\mustalg}{\texttt{\small{IVC\_MUST}}}

\newcommand{\nondetcov}{\text{\sc Nondet-Cov}}
\newcommand{\nondetcovalt}{\text{\sc Nondet-Cov$^{*}$}}
\newcommand{\ivccov}{\text{\sc IVC-Cov}}
\newcommand{\maycov}{\text{\sc May-Cov}}
\newcommand{\mustcov}{\text{\sc Must-Cov}}
\newcommand{\allcov}{\text{\sc Model-Cov}}
\newcommand{\mutcov}{\text{\sc Mutant-Cov}}


\newcommand{\ivc}{\textit{IVC}\xspace}
\newcommand{\mivc}{\textit{MIVC}\xspace}

\newcommand{\bq}{\textsc{BaseQuery}\xspace}
\newcommand{\iq}{\textsc{IndQuery}\xspace}
\newcommand{\fq}{\textsc{FullQuery}\xspace}

\newcommand{\mink}{\textsc{MinimizeK}\xspace}
\newcommand{\reduceinv}{\textsc{ReduceInvariants}\xspace}
\newcommand{\minivc}{\textsc{MinimizeIvc}\xspace}

\newcommand{\checksat}{\textsc{CheckSat}}
\newcommand{\isadeq}{\textsc{CheckAdq}}
\newcommand{\actlit}{\textsc{ActLit}}
\newcommand{\unsatcore}{\textsc{UnsatCore}\xspace}
\newcommand{\unsat}{\texttt{UNSAT}\xspace}
\newcommand{\sat}{\texttt{SAT}\xspace}

\newcommand{\getivc}{\textsc{GetIVC}}
\newcommand{\getmodel}{\textsc{GetLiteralsFromMaxModel}}
\newcommand{\aivcalg}{\texttt{\small{All\_IVCs}}}
\newcommand{\blockup}{\textsc{BlockUp}}
\newcommand{\blockdown}{\textsc{BlockDown}}
\newcommand{\mis}{\textit{MIS}}
\newcommand{\mcs}{\textit{MCS}}

\newtheorem{observation}{Observation}
\renewcommand{\theobservation}{\arabic{observation}.}% #.

\newtheorem{lemma}{Lemma}
\newtheorem{definition}{Definition}
\newtheorem{corollary}{Corollary}
\newtheorem{theorem}{Theorem}
\newtheorem{note}{Note}

%%%%%%%%%%%%%%%%%%%%%%%%%%%%%%%%%%%%%%%%%%%%%%%%%%%%%%%%%%%%%%%%%%%%%%%%%%%%%%%%
% ADDITIONAL PREABLE MATERIAL
%%%%%%%%%%%%%%%%%%%%%%%%%%%%%%%%%%%%%%%%%%%%%%%%%%%%%%%%%%%%%%%%%%%%%%%%%%%%%%%%
\hypersetup{
    colorlinks,
%     %Set the link colors.
%     citecolor=black,
%     filecolor=blue,
%     linkcolor=blue,
%     urlcolor=blue
    %To turn off color, comment out the block above and uncomment the block
    % below.
    citecolor=black,
    filecolor=black,
    linkcolor=black,
    urlcolor=black
}

% Suppresses inline citation page numbers if they are defined using \cite.
    \let\oldcite\cite
    \renewcommand\cite[2][]{\oldcite{#2}}

% \setcounter{secnumdepth}{5}

% % \appendix{Glossary}
% \begin{description}
% \item[process]\label{def:process}  ``set of interrelated or interacting
%     activities which transforms inputs into outputs''~\cite{_iso9000_2005}
% \item[work package]\label{def:work_package}  small pieces of functionality or
%     supporting documentation to be completed
% \end{description}

\newglossaryentry{def:brownfield_project}
{
    name={brownfield project},
    description={a project that is based on or must coexist with one or more
            legacy systems} 
}

\newglossaryentry{def:challenged_project}
{
    name={challenged project},
    description={a project that completes ``late, over-budget, and/or with less
            than required features or functions''~\cite[p.1]{_chaos_2013}; with
            agile techniques, typically only schedule and cost concerns are
            measured} 
}


\newglossaryentry{def:failed_project}
{
    name={failed project},
    description={a project that was unable to complete because it was 
        ``cancelled prior to completion or delivered and never 
        used''~\cite[p.1]{_chaos_2013}} 
}


\newglossaryentry{def:greenfield_project}
{
    name={greenfield project},
    description={a project that is not constrained by any prior work, such as
            legacy systems} 
}

\newglossaryentry{def:method-family}
{
    name={method-family},
    description={the process type; a group of processes that can be
        characterized by a set of well-defined properties (e.g., agile,
        traditional)},
    plural={method-families}
}


\newglossaryentry{def:oracle}
{
    name={oracle},
    description={the standard by which we can compare the output of a
        system being evaluated} 
}


\newglossaryentry{def:process}
{
    name=process,
    description={``set of interrelated or interacting
                activities which transforms inputs into
                outputs''~\cite{_iso9000_2005}},
    user1={_iso9000_2005},
    plural=processes 
}


\newglossaryentry{def:process_characterization}
{
    name={process characterization},
    description={determining the properties of the project and product to be
        completed and establishing the project's goals (such as focusing on
        quality)~\cite{xu_using_2008}}
}


\newglossaryentry{def:process_model}
{
    name={process model},
    description={an abstraction of a process; this may be either a model that
        represents a concrete process or a pre-tailored form of a model (e.g.,
        scrum, extreme programming, spiral model)}
}


\newglossaryentry{def:project_network}
{
    name={project network},
    description={the set of \glspl{def:work_package} and their
        interdependencies}
}


\newglossaryentry{def:successful_project}
{
    name={successful project},
    description={a completed project that was ``delivered on time, on budget,  
        with required features and functions''~\cite[p.1]{_chaos_2013}}
}


\newglossaryentry{def:triple_constraint}
{
    name={triple-constraint},
    description={a set of metrics---cost, schedule, and scope (including
        quality)---commonly used to evaluate project success; also a method for
        balancing project priorities (limiting one of the metrics results in
        changes---increased cost, increased schedule, and decreased scope---in
        one or both of the other constraints)} 
}


\newglossaryentry{def:work_breakdown_structure}
{
    name={work breakdown structure},
    description={}
}


\newglossaryentry{def:work_package}
{
    name={work package},
    description={small pieces of functionality or supporting documentation to be
                 completed}
}





% 
% 
%     \item[High-level Process Model Selection]  Select the high-level process
%             model (e.g. Scrum or XP for the agile method-family) as the basis
%             for further specialization.
%     \item[Tailoring]  Adapting the process model to meet the
%             specific needs or constraints of the organization, product, and
%             project.
%     \item[Process Evaluation (Verification and Validation)]  Verifying and
%         validating that the process (or process model instance) meets the needs
%         of the team and is ``consistent with the project goals and
%         environment''~\cite{xu_using_2008}.  This can be performed in a number
%         of ways.
%             \textbf{TODO -- this part should be its own (sub)section.}
%     \item[Adoption (Execution)]  Implementing the process within the
%             organization to complete the project.  Depending on the process,
%             this may involve \textit{in~situ} process analysis and improvements.
%     \item[Post-mortem Analysis]  Evaluating the process after it has been
%             executed to improve later projects.
% \makeglossaries

% \input{my_definitions}

% FROM:
% http://tex.stackexchange.com/questions/4297/how-to-place-a-full-citation-in-the-abstract-using-bibtex
% \newcommand{\ignore}[1]{}
% \newcommand{\nobibentry}[1]{{\let\nocite\ignore\bibentry{#1}}}
% % apsrev entries in the text need definitions of these commands
% \newcommand{\bibfnamefont}[1]{#1}
% \newcommand{\bibnamefont}[1]{#1}
\nobibliography*

%%%%%%%%%%%%%%%%%%%%%%%%%%%%%%%%%%%%%%%%%%%%%%%%%%%%%%%%%%%%%%%%%%%%%%%%%%%%%%%%
% STRUCTURE SET-UP
%%%%%%%%%%%%%%%%%%%%%%%%%%%%%%%%%%%%%%%%%%%%%%%%%%%%%%%%%%%%%%%%%%%%%%%%%%%%%%%%
\newcommand{\apriori}{\textit{a~priori} }
\newcommand{\Apriori}{\textit{A~priori} }


\linespread{1.3}

%%%%%%%%%%%%%%%%%%%%%%%%%%%%%%%%%%%%%%%%%%%%%%%%%%%%%%%%%%%%%%%%%%%%%%%%%%%%%%%%
% BEGIN DOCUMENT
%%%%%%%%%%%%%%%%%%%%%%%%%%%%%%%%%%%%%%%%%%%%%%%%%%%%%%%%%%%%%%%%%%%%%%%%%%%%%%%%
\begin{document}

%%%%%%%%%%%%%%%%%%%%%%%%%%%%%%%%%%%%%%%%%%%%%%%%%%%%%%%%%%%%%%%%%%%%%%%%%%%%%%%%
% title.tex - Set up the beginning of thesis.
%%%%%%%%%%%%%%%%%%%%%%%%%%%%%%%%%%%%%%%%%%%%%%%%%%%%%%%%%%%%%%%%%%%%%%%%%%%%%%%%
% For a  PhD give the command \phd. Default is masters
%\degree (normally Doctor of Philosophy or Master of Science)
%\initials (normally Ph.D. or M.S.)
%\ms % use if for a Master of Science thesis
\phd % use if for a Ph.D. dissertation
% \draft
\topicProposaltrue

\title{\textbf{Inductive Validity Cores for Formal Verification}}
\author{Elaheh Ghassabani}
\campus{University of Minnesota}
\program{Computer Science and Engineering}
% \degree{Doctor of Philosophy}
\director{Micael W. Whalen and Mats P. E. Heimdahl}

% Optionally specify the month and year.
\submissionmonth{Oct} % defaults to current month.
\submissionyear{2017} % defaults to current year.

%Comment out below on final copy
\abstract{%Increasing usage of computer systems in safety-critical applications demands the utmost care in their specification, design, and implementation.  Formal verification is a useful method for mathematical/systematic examination of the requirements a system must meet. However, due to the complexity of modern systems, safety analysis is challenging. One of the most successful and powerful methods for formal verification is symbolic model checking.
Symbolic model checkers are useful tools for the purpose of hardware/software verification that can construct proofs of properties over
very complex models. However, the results reported by the tool
when a proof succeeds do not generally provide much insight to
the user. It is often useful for users to have traceability information related to the proof: which portions of the model were necessary to construct it.  This traceability information can be used to diagnose a variety of modeling problems such as overconstrained axioms and underconstrained properties, and can also be used to measure {\em completeness} of a set of requirements over a model.

We propose the notion of {\em inductive validity cores} (IVCs), which are intended to trace a property to a \emph{minimal} set of model elements necessary for proof. Besides minimality, computing \emph{all} minimal IVCs of a given property is
an interesting problem that provides several useful analyses, including
regression analysis for testing/proof, determination of the minimum (as
opposed to minimal) number of model elements necessary for proof, the
diversity examination of model elements leading to proof, and analyzing fault
tolerance.
We, fisrt, propose a method to efficiently compute a single IVC within a model necessary for inductive proofs of safety properties for sequential systems.  The algorithm is based on the UNSAT core support built into current SMT solvers and a novel encoding of the inductive problem to try to generate a minimal inductive validity core.
Then, we propose an efficient method for finding \emph{all minimal} IVCs of a
given property. Finally, we introduce several useful applications of this idea. 
%We prove our algorithms are correct, and describe their implementation in the JKind model checker for Lustre models.
%We then present an experiment in which we benchmark the algorithms in terms of speed, diversity of produced cores, and minimality, with promising results.
}
%\words{331}    % number of words in the abstract
%\copyrightpage % Do you want copyright protection?
%\acknowledgements{This thesis is a milestone in four joyful years of work with amazing UMN Critical Systems Group (CriSys). My experience at UMN has been nothing short of spectacular. Since my first day on August 20th, 2014 I have felt at home at UMN with meeting my warm-hearted and caring advisors and friends. I have been given unique opportunities and taken
advantage of them. This includes being part of the cutting-edge projects while serving as a graduate research assistant at CriSys, attending a lot of top-tier conferences, meeting with the best researchers in my field, and on top of all, learning from many distinguished and knowledgeable professors at UMN. I would like to extend thanks to the many people who so generously contributed to the work presented in this thesis.

Special mention goes to my enthusiastic supervisor, Michael Whalen. My PhD has been an amazing experience and I thank Michael, not only for his tremendous academic support, but also for giving me so many wonderful opportunities. I would like to express my sincere gratitude to him for the continuous support of my Ph.D study and related research, for his patience, motivation, and immense knowledge.

Similar, profound gratitude goes to Mats Heimdahl, who has been a truly supportive co-advisor. I am particularly indebted to him for his constant faith in my work and for his support. I have very fond memories of my time with Michael and Mats, who have been like my family all these years. Their guidance helped me in all the time of research and writing of this thesis. I could not have imagined having a better advisor and mentor for my graduate studies.

Besides my advisors, I would like to thank the rest of my thesis committee: Prof. Stephen McCamant, Prof. Marc Riedel, and Prof. Eric Van Wyk, for their insightful comments and encouragement to widen my research from various perspectives.

I thank my fellow labmates for the stimulating discussions, for the sleepless nights we were working together before deadlines, and for all the fun we have had in the last four years. Also I thank the wonderful people in other organizations I had the opportunity to work with; in particular, I am grateful to Dr. Andrew Gacek and Jaroslav Bendik.

Last but not least, I would like to thank my loving family: my parents and my sisters for supporting me spiritually throughout writing this thesis and my my life in general. }
\acknowledgements{This thesis is a milestone in four joyful years of work with amazing UMN Critical Systems Group (CriSys). My experience at UMN has been nothing short of spectacular. Since my first day on August 20th, 2014 I have felt at home at UMN with meeting my warm-hearted and caring advisors and friends. I have been given unique opportunities and taken
advantage of them. This includes being part of the cutting-edge projects while serving as a graduate research assistant at CriSys, attending a lot of top-tier conferences, meeting with the best researchers in my field, and on top of all, learning from many distinguished and knowledgeable professors at UMN. I would like to extend thanks to the many people who so generously contributed to the work presented in this thesis.

Special mention goes to my enthusiastic supervisor, Michael Whalen. My PhD has been an amazing experience and I thank Michael, not only for his tremendous academic support, but also for giving me so many wonderful opportunities. I would like to express my sincere gratitude to him for the continuous support of my Ph.D study and related research, for his patience, motivation, and immense knowledge.

Similar, profound gratitude goes to Mats Heimdahl, who has been a truly supportive co-advisor. I am particularly indebted to him for his constant faith in my work and for his support. I have very fond memories of my time with Michael and Mats, who have been like my family all these years. Their guidance helped me in all the time of research and writing of this thesis. I could not have imagined having a better advisor and mentor for my graduate studies.

Besides my advisors, I would like to thank the rest of my thesis committee: Prof. Stephen McCamant, Prof. Marc Riedel, and Prof. Eric Van Wyk, for their insightful comments and encouragement to widen my research from various perspectives.

I thank my fellow labmates for the stimulating discussions, for the sleepless nights we were working together before deadlines, and for all the fun we have had in the last four years. Also I thank the wonderful people in other organizations I had the opportunity to work with; in particular, I am grateful to Dr. Andrew Gacek and Jaroslav Bendik.

Last but not least, I would like to thank my loving family: my parents and my sisters for supporting me spiritually throughout writing this thesis and my my life in general. }
% \dedication{\input{dedication}}

% Use a special preface
%\extra{\chapter*{Author Declaration}
 \addcontentsline{toc}{chapter}{Author Declaration}
Some of the material presented within has previously been published in the
following papers:

\begin{itemize}
  \item \bibentry{de_silva_reference_2015}
\end{itemize}

\flushleft
All the work contained within represents the original contribution of the
author.}

% The \beforepreface command actually causes insertion of the title,
% abstract, signature, and copyright pages into the new document.
\beforepreface

% Define the text to go before the table of contents
\figurespage
\tablespage

% The \afterpreface command actually causes insertion of the
% contents, list of figures, etc. into the new document.
\afterpreface
%%%%%%%%%%%%%%%%%%%%%%%%%%%%%%%%%%%%%%%%%%%%%%%%%%%%%%%%%%%%%%%%%%%%%%%%%%%%%%%%





%%%%%%%%%%%%%%%%%%%%%%%%%%%%%%%%%%%%%%%%%%%%%%%%%%%%%%%%%%%%%%%%%%%%%%%%%%%%%%%%
% CONTENT
%%%%%%%%%%%%%%%%%%%%%%%%%%%%%%%%%%%%%%%%%%%%%%%%%%%%%%%%%%%%%%%%%%%%%%%%%%%%%%%%
% Input the sectons here using \input{fileName}
%%%
%\input{draft3}
% %Increasing usage of computer systems in safety-critical applications demands the utmost care in their specification, design, and implementation.  Formal verification is a useful method for mathematical/systematic examination of the requirements a system must meet. However, due to the complexity of modern systems, safety analysis is challenging. One of the most successful and powerful methods for formal verification is symbolic model checking.
Symbolic model checkers are useful tools for the purpose of hardware/software verification that can construct proofs of properties over
very complex models. However, the results reported by the tool
when a proof succeeds do not generally provide much insight to
the user. It is often useful for users to have traceability information related to the proof: which portions of the model were necessary to construct it.  This traceability information can be used to diagnose a variety of modeling problems such as overconstrained axioms and underconstrained properties, and can also be used to measure {\em completeness} of a set of requirements over a model.

We propose the notion of {\em inductive validity cores} (IVCs), which are intended to trace a property to a \emph{minimal} set of model elements necessary for proof. Besides minimality, computing \emph{all} minimal IVCs of a given property is
an interesting problem that provides several useful analyses, including
regression analysis for testing/proof, determination of the minimum (as
opposed to minimal) number of model elements necessary for proof, the
diversity examination of model elements leading to proof, and analyzing fault
tolerance.
We, fisrt, propose a method to efficiently compute a single IVC within a model necessary for inductive proofs of safety properties for sequential systems.  The algorithm is based on the UNSAT core support built into current SMT solvers and a novel encoding of the inductive problem to try to generate a minimal inductive validity core.
Then, we propose an efficient method for finding \emph{all minimal} IVCs of a
given property. Finally, we introduce several useful applications of this idea. 
%We prove our algorithms are correct, and describe their implementation in the JKind model checker for Lustre models.
%We then present an experiment in which we benchmark the algorithms in terms of speed, diversity of produced cores, and minimality, with promising results.

\chapter{Introduction}
\label{ch:intro}
Software has become an integral part of our daily life and is being used in various environments (such as homes, hospitals, factories) and application areas (such as medical devices, aircraft flight control, weapons, and nuclear systems) where failure could lead to loss of life, financial loss, or environmental damage. It is vital to verify the soundness and safety of such critical applications. \emph{Formal verification}, the process of mathematically proving or disproving the correctness of a system with respect to certain requirements or properties, is increasingly applied to critical systems to ensure they work in all cases.

One of the most successful and powerful methods for formal verification is symbolic model checking. Symbolic model checking using induction-based techniques such as IC3/PDR~\cite{Een2011:PDR} and $k$-induction~\cite{SheeranSS00} can often determine whether safety properties hold of complex finite or infinite-state systems.  Model checking tools are attractive both because they are automated, requiring little or no interaction with the user, and if the answer to a correctness query is negative, they provide a counterexample to the satisfaction of the property.  These counterexamples can be used both to illustrate subtle errors in complex hardware and software designs~\cite{hilt2013,McMillan99:compositional, Miller10:CACM} and to support automated test case generation~\cite{Whalen13:OMCDC, You15:dse}.

In the event that a property is proved, however, it is not always clear what level of assurance should be invested in the result. It is well known that issues such as vacuity~\cite{Kupferman03:Vacuity} can cause verification to succeed despite errors in a property specification or in the model.
Even for non-vacuous specifications, it is possible to over-constrain the specification of the {\em environment} in the model such that the implementation will not work in the actual operating environment.
Given that these kinds of analyses are performed for safety- and security-critical software, these issues can lead to overconfidence in the behavior of the fielded system.
In such cases, a system or subsystem component will not exhibit the expected behavior in its intended operating environment, which may bring about catastrophic losses. Achieving a proof in verification of safety requirements is not enough: careful scrutiny of the property of interest, the model, and the assumptions (axioms) used during proof must be performed.  For these reasons, the level of feedback provided by the tools to the user in the event of a proof is important.

Inductive Validity Cores (IVCs) offer an explanation as to why a property is satisfied by a model in a formal and human-understandable way. Informally, if a model is viewed as a conjunction of constraints,
a minimal IVC (MIVC) is a set of constraints that is sufficient to construct a proof such that if any constraint is removed, the property is no longer valid.
IVCs and MIVCs can be used for several purposes, including performing traceability between specification and design elements, assessing model coverage, and explaining unsatisfiable test obligations when using model checkers for test case generation.

%At issue is the level of feedback provided by the tool to the user.

\section{Objectives and Significance}
\label{sec:obj}
%For decidable problems, the result of (safety) verification shows if the property of interest is valid or violated. In case of violation, tools will generate a counter example
%that shows an unsafe scenario by which the user is able to understand why the property does not hold.
In this thesis, we are specifically concerned with the scenarios where a model checker establishes the correctness proof of a given property.
 When it comes to verification, if the answer to a correctness query is positive, most tools provide no further information.  The \textit{objective of this dissertation} is to provide traceability information that explains a proof, in much the same way that a counterexample explains a negative result.
Such an explanation should be both formal and human-understandable. This research will add to the usability of the symbolic model checkers by equipping the tools with a mechanism to show why a proved property is valid.

Reasoning about the proofs is not a new idea: UNSAT cores~\cite{zhang2003extracting}
provide the same kind of information for individual SAT or
SMT queries, and this approach has been lifted to bounded analysis
for Alloy in~\cite{Torlak08:cores}.
What we propose is a generic and efficient
mechanism for extracting supporting information, similar to an UNSAT
core, from the proofs of safety properties using inductive techniques
such as PDR and $k$-induction. Because many
properties are not themselves inductive, these proof techniques
introduce lemmas as part of the solving process in order to strengthen
the properties and make them inductive. Our approach, which we call {\em inductive validity cores} (IVCs), allows efficient, accurate, and precise extraction of model elements necessary even in the presence of such auxiliary lemmas. The idea lifts UNSAT cores~\cite{zhang2003extracting}
to the level of sequential model checking algorithms using induction.  Informally, if a model is viewed as a conjunction of constraints,
a minimal IVC (MIVC) is a set of constraints that is sufficient to construct a proof such that if any constraint is removed, the property is no longer valid.


The IVC idea facilitates several useful system analyses/engineering tasks. Specifically, it is useful when the validity of a safety requirement has been established by the model checker. In this case, IVCs provide usable information both formal and human-understandable that explains why the requirement is satisfied. Such information is valuable in analyzing safety-critical systems and can be used for many purposes in the software verification process, including at least the following:
\begin{description}
    \item[Vacuity detection:] The idea of syntactic vacuity detection (checking whether all subformulae within a property are necessary for its satisfaction) has been well studied~\cite{Kupferman03:Vacuity}.   However, even if a property is not syntactically vacuous, it may not require substantial portions of the model.  This in turn may indicate that either a.) the model is incorrectly constructed or b.) the property is weaker than expected.  We have seen several examples of this mis-specification in our verification work, especially when variables computed by the model are used as part of antecedents to implications.
%    \item[Completeness checking:] Closely related to vacuity detection is the idea of {\em completeness checking}, e.g., are all atoms in the model necessary for at least one of the properties proven about the model?  Several different notions of completeness checking have been proposed~\cite{chockler_coverage_2003, kupferman_theory_2008}, but these are very expensive to compute, and in some cases, provide an overly strict answer (e.g., checking can only be performed on non-vacuous models for~\cite{kupferman_theory_2008}).
    \item[Traceability:] Certification standards for safety-critical systems (e.g.,~\cite{DO178C, MOD:00-55}) usually require {\em traceability matrices} that map high-level requirements to lower-level requirements and (eventually) leaf-level requirements to code or models.  Current traceability approaches involve either manual mappings between requirements and code/models~\cite{SimulinkTraceability} or a heuristic approach involving natural language processing~\cite{Keenan12:Tracelab}.  Both of these approaches tend to be inaccurate.  For functional properties that can be proven with inductive model checkers, inductive validity cores can provide accurate traceability matrices with no user effort.
    \item[Symbolic Simulation / Test Case Generation:]  Model checkers are now often used for symbolic simulation and structural-coverage-based test case generation~\cite{SimulinkDesignVerifier,Whalen13:OMCDC}.  For either of these purposes, the model checker is supposed to produce a witness trace for a given coverage obligation using a ``trap property'' which is expected to be falsifiable.  In systems of sufficient size, there is often ``dead code'' that cannot ever be reached.  In this case, a proof of non-reachability is produced, and the IVC provides the reason why this code is unreachable.
\end{description}
\noindent Nevertheless, to be useful for these tasks, the generation
process must be efficient and the generated IVC must be
accurate and precise (that is, sound and close to minimal).  The requirement for accuracy is obvious; otherwise the ``minimal'' set of model elements is no longer sufficient to produce a proof, so it no longer meets our IVC definition.  Minimality is important because (for traceability) we do not want unnecessary model elements in the trace matrix, and (for completeness) it may give us a false level of confidence that we have enough requirements.

%\ela{should we add this: (?)}
In addition, %\fixed{ a property can have as many unique minimal IVC sets as the possible paths through which it can be proved. Therefore,}
we are also interested in {\em diversity}:  how many different IVCs can be computed for a given property and model? Requirements engineers often talk about ``the traceability matrix'' or ``the satisfaction argument''.  If proofs are regularly diverse, then there are potentially many equally valid traceability matrices, and this may lead to changes in traceability research.
It is often the case that there are multiple MIVCs for a given property.  In this case, computing a single IVC provides, at best, an incomplete picture of the traceability information associated with the proof.  Depending on the model and property to be analyzed, there is often substantial diversity between the IVCs used for proof, and there can also be a substantive difference in the size of a {\em minimal} IVC and a {\em minimum} IVC, which is the (not necessarily unique) smallest MIVC.
 If {\em all} MIVCs can be found, then several additional analyses can be performed:
\begin{description}
    \item [Coverage Analysis:] Closely related to vacuity detection is the idea of {\em completeness checking}, e.g., are all atoms in the model necessary for at least one of the properties proven about the model?  Several different notions of completeness checking have been proposed~\cite{chockler_coverage_2003, kupferman_theory_2008}, but these are very expensive to compute, and in some cases, provide an overly strict answer (e.g., checking can only be performed on non-vacuous models for~\cite{kupferman_theory_2008}). MIVCs can be used to define coverage metrics by examining the percentage of model elements required for a proof.  However, since MIVCs are not unique, there are multiple, equally legitimate coverage scores possible.  Having \emph{all} MIVCs allows one to define additional metrics: coverage of MAY elements, coverage of MUST elements, as well as policies for the existing MIVC metric: e.g., choose the smallest MIVC.
    \item [Optimizing Logic Synthesis:]  synthesis tools can benefit from MIVCs in the process of transforming an abstract behavior into a design implementation. A practical way of calculating all MIVCs allows to find a minimum set of design elements (optimal implementation) for a certain behavior. Such optimizations can be performed at different levels of synthesis.
    \item [Impact Analysis:] Given all MIVCs, it is possible to determine which requirements may be falsified by changes to the model.  This analysis allows for selective regression verification of tests and proofs: if there are alternate proof paths that do not require the modified portions of the model, then the requirement does not need to be re-verified.
    \item [Robustness Analysis:] It is possible to partition the model elements into MUST and MAY sets based on whether they are in every MIVC or only some MIVCs, respectively.  This may allow insight into the relative importance of different model elements for the property.  For example, if the MUST set is empty, then the requirement has been implemented in multiple ways, such as would be expected in a fault-tolerant system.
        %Moreover, examining the diversity of all MIVCs could lead to changes in how traceability~\cite{cleland2007best} is performed and managed in critical systems.
\end{description}

The Requirements Engineering community is keenly interested in approaches to manage requirements traceability.  In most cases, it is assumed that there is a single ``golden'' set of trace links that describes how requirements are implemented in software~\cite{COEST,hayes2003improving,cleland2007best}.
With computing a \emph{single minimal IVC}, we are able to automatically establish one accurate traceability matrix. However, if there are \emph{multiple} MIVCs, then it is possible that there are several equally valid sets of trace links.  Examining the diversity of \emph{all }MIVCs could lead to changes in how traceability is performed for critical systems.


\subsection{Use in Research \& Systems Development}

Three of the important concerns in \emph{certification} of critical systems are: conformance, traceability, and adequacy.  Conformance involves determining whether a system meets its requirements: formal verification tools have excellent support for conformance.  However, most formal verification tools do not provide support for traceability and adequacy.  IVCs could be a mechanism by which formal verification tools address these concerns.

For example, airborne software must undergo a rigorous software development process to ensure its airworthiness. This process is governed by DO-178C: Software Considerations in Airborne Systems and Equipment Certification and when formal methods tools are used, DO-333: Formal Methods Supplement to DO-178C and DO-278A \cite{DO178C}.
%DO-178C proposes a rigorous software development process that starts with an abstract requirements artifact that is iteratively refined into a software designs, source code, and finally, object code, and a set of {\em objectives} that should be met by critical avionics software.  Two of the key tenets of this process are traceability and adequacy; that is, each refinement of an artifact must be traceable to the artifact if was derived from. Further, each refinement must be shown not to introduce functionality not present in the artifact from which it was derived (adequacy). For example, DO-178C objectives A-3.6 (traceability of high-level requirements to system requirements) and A-4.6 (traceability of software design to high-level requirements) specifically require applicants to demonstrate bi-directional traceability.
DO178C currently uses a variety of metrics to determine adequacy of requirements, but much of the effort involves code-level testing.  Test suites are derived from requirements and used to test the software and measured using different structural coverage test metrics.  If code-level test suites do not achieve full coverage, then an analysis is performed to determine whether there are missing requirements and test cases.  The kind of structural coverage required (e.g., statement, branch, MCDC) for adequate testing is driven by the criticality of the software in question.

With the idea of IVCs, we propose a set of proof-based coverage metrics suitable for analyzing requirements competentness. Then, we will have the utility of the proposed approach evaluated by an industrial partner.


\section{Contributions}
%First, we propose a new method for extracting a single IVC from the inductive proof of a given property. This method is intended to be very efficient, imposing a negligible overhead on the verification process.
%Then, we propose a new method for computing \emph{all} MIVCs that is {\em always} minimal for decidable model checking problems and {\em usually} (and detectably) minimal for model-checking problems that are generally undecidable. We evaluate the usability of our idea by examining its different applications.
 Inductive validity cores have potential software engineering uses in several phases of the development cycle. However, efficient and effective generation strategies must be proposed to achieve these benefits. The contributions of the work are as follows:
\begin{itemize}
    \item \emph{Efficient techniques for extracting inductive validity cores from inductive proofs of safety properties over sequential models involving lemmas:} The thesis provides a formalization of techniques for computing inductive validity cores, and efficient algorithms for computing approximately minimal IVCs from proofs.  {\em Efficient} in this context means that the computation time required is a small fraction of the time required to compute the original proof.
    \item \emph{Efficient algorithms for computing all minimal IVCs from inductive proofs of safety properties over sequential models involving lemmas:} depending on the model and the property specification, the property of interest may be satisfied through different proof paths, which could results in multiple distinct IVCs. This thesis formalizes techniques for producing all inductive validity cores.  It explores methods that are sound and reasonably efficient for computing all IVCs.  It is not possible to guarantee completeness due to decidability issues, but we present algorithms that are complete for decideable problems and that will report possibly incomplete results to the user in situations in which a complete solution may not be possible.
%    \item
    %\item An evaluation of the algorithm for performance and diversity of result sets against a benchmark suite.
   \item \emph{A family of coverage metrics for formal verification based on \emph{minimal} Inductive Validity Cores (MIVCs) that evaluate requirements adequacy:} we present a new approach to coverage analysis in formal verification which is much more efficient than previously proposed mutation-based analyses. Our goal is to provide a set of metrics that offer a range of levels of rigor that can be tailored to the criticality of the software. We discuss the relationship between proof-based metrics and mutation-based metrics, including a proof of equivalence between non-deterministic mutation coverage and one of our proposed proof-based metrics.
         % Currently, there is not any practical approach that can address this issue. It has been always a great challenge for the designers to know if they have considered enough system requirements. The goal of coverage metrics in this context is to provide a mechanism by which we can explain if in the verification of a given model (implementation), enough properties have been verified because erroneous system behaviours not captured by any property will remain undiscovered.
%\item \emph{A discussion of :} the notion of coverage in formal verification is relatively new, compared to testing. Existing techniques in this area are mostly adapted from testing inspired by the idea of mutations, which are very expensive and inefficient. Mutations are atomic changes made to the model. The general idea in these approaches is to reduce the coverage problem into verification problem. To do so, each property has to be verified against all possible mutated models. In realistic problems, a model could have too many mutants to verify, which is quite impractical, while our proof-based coverage metrics are intended to be efficient and practical enough for coverage analysis.
\item \emph{A new notion of proof-based auto-traceability based on IVCs:} requirements traceability is the primary application of IVCs. Currently, this task is performed manually without any formal analysis, which takes a lot of effort and yet is not accurate. With IVCs, we present the notion of complete traceability, by which requirement traceability can be performed automatically and accurately driven from the proofs of the properties.
    \item  \emph{A study of the relationship between inductive validity cores and bounded validity cores (BVCs).}    IVCs are derived from inductive proofs.  In some cases, proving safety properties over complex systems is often very expensive or infeasible. In these cases, engineers have to rely on bounded proofs.  Bounded validity cores explain bounded proofs in the same way that inductive validity cores explain inductive proofs.  By definition, such cores are smaller than inductive cores as they explain partial proofs.  However, there are important relationships to be studied between bounded and inductive cores: how quickly do bounded cores converge to IVC sets?  Can bounded cores be used as a notion of completeness or traceability?
    
\item \emph{Implementation of all the techniques:} the correctness of the techniques is proved/discussed formally, while their efficacy is evaluated via substantial experiments. To this end, we have implemented all our methods in an open source model checker. The implementation and experimental results are publicly available. To this  end, we have chosen an industrial model checker called \texttt{JKind} ~\cite{jkind},
which verifies safety properties of infinite-state synchronous systems.
It accepts Lustre programs \cite{Halbwachs91:lustre} as input. In JKind, verification is supported by multiple ``proof engines'' that execute in parallel, including $k$-induction,
property directed reachability (PDR), and lemma generation engines that attempt to prove
multiple properties in parallel. To implement the engines,
JKind emits SMT problems using the theories of linear integer and real arithmetic. \texttt{JKind} supports the \texttt{Z3}, \texttt{Yices}, \texttt{MathSAT}, \texttt{SMTInterpol}, and \texttt{CVC4} SMT solvers as back-ends.  We have extended \texttt{JKind} with new engines that implement our IVC generation algorithms.
%    In terms of finding a single IVC, we will evaluate the efficiency, minimality, and robustness of the IVC generation process. As for the all IVCs generation method, we will use a large benchmark containing industrial case studies, evaluating the overhead of the process over the verification time. In addition, we will perform an experiment that compares our proof-based metrics against a state of the art mutation-based notion of completeness\footnote{The mutation-based techniques will be explained in Chapter \ref{ch:rel}. The state of the art of these techniques will be described in detail in Chapter \ref{ch:prop}}.
\item \emph{An initial examination of how IVCs can be used to meet certification objectives:}  critical software systems must undergo a rigorous software development process to ensure their correctness. This process is usually governed by an standard such as DO-178C \cite{DO178C}. We would like to examine the usefulness of the IVCs in providing satisfaction arguments that formally show how a system meets the certification objectives.

\end{itemize}

%\section{Evaluation}
%We have performed a substantial evaluation that shows that the practicality and efficiency of our technique. For this purpose, we have collected a large set of benchmarks including academic and industrial cases from different sources such as ~\cite{Hagen08:FMCAD, piskac2016} \cite{hilt2013} \cite{piskac2016, NFM2015:backes}. We have selected only benchmark problems consisting of a Lustre model with
%properties that \texttt{JKind} could prove with a 3-hour timeout.
%Experiments are run in a configuration with the \texttt{Z3} solver and the ``fastest'' mode of \texttt{JKind} (which involves running the $k$-induction and PDR engines in parallel and terminating when a solution is found).
%
%
%We would like to evaluate the cost of computing one single IVC using a brute-force
%algorithm and our algorithms. We are interested in examining the {\em efficacy} and {\em efficiency} of generating all minimal IVCs, as compared to algorithms for computing a {\em single minimal} IVC.  We would also like to know how performance is affected by the size of models and number of minimal IVCs.  Next, we are also interested in examining the minimality of the cores found by the algorithms.  %If the AIVC algorithm is similarly efficient to \ucbfalg\ then several analyses can be performed that would not be possible with a single \mivc\computed from the \ucbfalg\ algorithm.
%%
%%
%Therefore, we investigate the following research questions:
%\begin{itemize}
%\item \textbf{RQ1:} How expensive is it to compute a minimal IVC?
%\begin{itemize}
%  \item \textbf{RQ1.1:} If the cost is high, can we approximate minimality efficiently and effectively?
%   \begin{itemize}
%    \item \textbf{RQ1.1.1:} If so, how close to minimal are the IVCs obtained by the approximate approach as opposed to the guaranteed minimal IVCs computed by an exact algorithm?
%  \end{itemize}
%\end{itemize}
%  \item \textbf{RQ2:} How expensive is it to compute all minimal IVCs compared to one minimal IVC?
%  \item \textbf{RQ3:} How is the verification time of algorithms affected by the baseline proof time and the number of IVCs that can be found for a property?
%   \item \textbf{RQ4:} How do the sizes of minimal IVCs compare to static slices of the model?
%\end{itemize}
%
%Our method for computing all MIVCs is inspired by a recent work in the domain of satisfiability analysis \cite{marco2016fast}. One interesting direction is to devise similar MIVC enumeration algorithms based on other studies on MUSes extraction such as \cite{nadel2014accelerated}.
%Another interesting future direction for this project is to parallelize the enumeration process: it is certainly possible to ask for multiple distinct maximal models to be solved in parallel.
%%, though this may result in unnecessary work performed by some of the parallel solvers.
%
%We also plan to investigate additional applications of the idea.  When performing {\em compositional verification}, the All-IVCs technique may be able to determine {\em minimal component sets} within an architecture that can satisfy a given set of requirements, which may be helpful for design-space exploration and synthesis. Finally, we are interested in adapting the notion of (all) validity cores for \emph{bounded} model checking for quantifying how much of models have been explored by bounded analysis.
%
%Upon completion of the proposed research, we will have our IVCs computation algorithms integrated in the \texttt{JKind} model checker. The implementation will be benchmarked and evaluated rigorously. The usefulness of the IVC idea will be shown by utilizing its applications into different projects.


\section{Chapters}
This thesis is organized in 7 chapters. Chapter \ref{ch:background} mentions some formal notations and background and broadly discusses related work. Chapter \ref{ch:ivc} describes the notion of IVC, minimal IVC, all minimal IVCs, and BVCs while providing some algorithms for each of them. The correctness of the algorithms are formally established in this chapter.
In Chapter \ref{ch:impl}, we describe the implementation of the proposed techniques and algorithms.  In Chapter \ref{ch:experiment}, we evaluate our techniques through a set of substantial experiments.
In this chapter we introduced some uses of the IVCs. Chapter \ref{ch:apps} shows how IVCs could be used in different areas. Finally, Chapter \ref{ch:con} concludes this thesis and draw future research directions.

%\section{State of the Art}
%Our work builds on top of a substantial foundation provided by special tools known as constraint solvers. Constraint solving is a powerful mathematical method that allows the computer to solve a problem formulated by the user. Many verification problems can be reduced to constraint satisfaction problems and solved with tools known as SMT  solvers.  A lot of useful formal methods are built on top of SMT solvers, such as model checking algorithms, abstraction techniques, and proof-certificate generation. There is significant amount of valuable research on these topics in the literature. Although such reasoning techniques are helpful, they are not expressive enough to provide good insights into the quality of a system or specification. With the IVC idea, we are able to bridge the gap between verification techniques and the user insight into the results provided by the tools. The goal behind this idea is different from existing applications of constraint solving. The IVC idea shares many similarities with existing approaches for computing proof certificates, and in fact the IVC algorithm performs this computation as well. However, there is a substantive difference; to find a guaranteed minimal set of certificates, it is usually necessary to find new proofs involving new invariants not used in the original proof, which existing techniques do not deal with.



%% We put the image here so it shows up side-by-side with fig:ex-after
%\begin{figure}[t]
%\centering
%\includegraphics[width=\columnwidth]{figs/simulink.png}
%{\smaller
%\begin{verbatim}
%node filter(x : real) returns (a, b, y : real);
%let
%  a = f(x, 0.0 -> pre y);
%  b = if a >= 0.0 then a else -a;
%  y = b + (0.0 -> pre y);
%tel;
%\end{verbatim}
%}
%\vspace{-0.1in}
%\caption{Model with property $y \geq 0$, before IVC analysis}
%\label{fig:ex-before}
%\end{figure} 
\chapter{Preliminaries and Related Work}

\label{sec:related}
Inductive validity cores aim to bridge the gap between verification techniques and the user insight into the results provided by the tools. The goal behind this idea is to have expressive verification results that help the engineers to evaluate the quality of a system or specification.

Broadly, IVCs can be compared with several existing methods such as invariant minimization, minimal unsatisfiable subformula, and slicing. In this section, first we compare IVCs with these techniques.

Several major uses of IVCs are in requirements traceability, checking adequacy and vacuity. In the current chapter, we discuss existing approaches in the literature used for these purposes.




\section{Symbolic Model Checking}
\label{ch:background}
The idea of inductive validity cores is applicable to the context of symbolic model checking using inductive proof methods. After proving the correctness of a given property, we extract a minimal portion of the system (model) necessary for the proof of the property, which is what we call IVCs.
Correctness can be expressed in terms of \emph{safety} and \emph{liveness} properties. Safety properties state that nothing bad ever happens, while liveness properties specifying that something good eventually happens. 

IVCs determine why a \emph{safety} property is satisfied by the system. Since this information is obtained from the inductive proofs, we call it \emph{inductive} validity core. With minimal IVCs, we are able to abstract away the part of the system irrelevant to the proof of the property. This section mentions some background on symbolic model checking.

Given a state space $U$, a transition system $(I,T)$ consists of an
initial state predicate $ I : U \to \bool $ and a transition step
predicate $ T : U \times U \to \bool $.
We define the notion of
reachability for $(I, T)$ as the smallest predicate $\reach : U \to
\bool$ which satisfies the following formulas:

\begin{gather*}
  \forall u.~ I(u) \Rightarrow \reach(u) \\
  \forall u, u'.~ \reach(u) \land T(u, u') \Rightarrow \reach(u')
\end{gather*}

A safety property $P : U \to \bool$ is a state predicate. A safety
property $P$ holds on a transition system $(I, T)$ if it holds on all
reachable states, i.e., $\forall u.~ \reach(u) \Rightarrow P(u)$,
written as $\reach \Rightarrow P$ for short. When this is the case, we
write $(I, T)\vdash P$.

For an arbitrary transition system $(I, T)$, computing reachability
can be very expensive or even impossible. Thus, we need a more
effective way of checking if a safety property $P$ is satisfied by the
system. The key idea is to over-approximate reachability. If we can
find an over-approximation that implies the property, then the
property must hold. Otherwise, the approximation needs to be refined.

A good first approximation for reachability is the property itself.
That is, we can check if the following formulas hold:
\begin{gather}
  \forall u.~ I(u) \Rightarrow P(u)
  \label{eq:1-ind-base} \\
  \forall u, u'.~ P(u) \land T(u, u') \Rightarrow P(u')
  \label{eq:1-ind-step}
\end{gather}
If both formulas hold then $P$ is {\em inductive} and holds over the
system. If (\ref{eq:1-ind-base}) fails to hold, then $P$ is violated
by an initial state of the system. If (\ref{eq:1-ind-step}) fails to
hold, then $P$ is too much of an over-approximation and needs to be
refined.

One way to refine our over-approximation is to add additional lemmas
to the property of interest. For example, given another property $L :
U \to bool$ we can consider the extended property $P'(u) = P(u) \land
L(u)$, written as $P' = P \land L$ for short. If $P'$ holds on the
system, then $P$ must hold as well. The hope is that the addition of
$L$ makes formula (\ref{eq:1-ind-step}) provable because the
antecedent is more constrained. However, the consequent of
(\ref{eq:1-ind-step}) is also more constrained, so the lemma $L$ may
require additional lemmas of its own. Finding and proving these
lemmas is the means by which property directed reachability (PDR)
strengthens and proves a safety property~\cite{Een2011:PDR}.

Another way to refine our over-approximation is to use use {\em
  $k$-induction} which unrolls the property over $k$ steps of the
transition system. For example, 1-induction consists of formulas
(\ref{eq:1-ind-base}) and (\ref{eq:1-ind-step}) above, whereas
2-induction consists of the following formulas:
\begin{gather*}
\forall u.~ I(u) \Rightarrow P(u) \\
\forall u, u'.~ I(u) \land T(u, u') \Rightarrow P(u') \\
\forall u, u', u''.~ P(u) \land T(u, u') \land P(u') \land T(u',
  u'') \Rightarrow P(u'')
\end{gather*}
That is, there are two base step checks and one inductive step check.
In general, for an arbitrary $k$, $k$-induction consists of $k$
base step checks and one inductive step check as shown in
Figure~\ref{fig:k-induction} (the universal quantifiers on $u_i$ have
been elided for space). We say that a property is $k$-inductive if it
satisfies the $k$-induction constraints for the given value of $k$.
The hope is that the additional formulas in the antecedent of the
inductive step make it provable. 
In practice, inductive model checkers often use a combination of the
above techniques. Thus, a typical conclusion is of the form ``$P$ with
lemmas $L_1, \ldots, L_n$ is $k$-inductive''.

\begin{figure}
\begin{gather*}
I(u_0) \Rightarrow P(u_0) \\[-2pt]
%
\vdots \\[2pt]
%
I(u_0) \land T(u_0, u_1) \land \cdots \land T(u_{k-2}, u_{k-1})
\Rightarrow P(u_{k-1}) \\[2pt]
%
P(u_0) \land T(u_0, u_1) \land \cdots \land P(u_{k-1}) \land
T(u_{k-1}, u_k) \Rightarrow P(u_k)
\end{gather*}
\caption{$k$-induction formulas: $k$ base cases and one inductive
  step}
\label{fig:k-induction}
\end{figure}




Liveness properties involve temporal operators of \emph{always eventually}, denoted by $GF$, and intuitively have infinite-length counterexamples. In order to verify such properties with symbolic model checking, there are a couple of well-known techniques. One approach is to reduce a liveness problem to a safety problem \cite{Schuppan:2006}, where a suitable counterexample-detection logic is used by duplicating state elements,
 $U_{copy} = \{ u_c | u \in U \}$. It non-deterministically
samples the design state and tries to find a valid counterexample
scenario with finding a state repetition
loop during which the behavior of the liveness and fairness
conditions are observed. 
Although converting liveness to safety makes it possible to use existing safety
verification algorithms, such translation is usually impractical due to substantially increasing problem size. Another technique is bounded liveness checking, $k$-liveness, \cite{Schuppan:2006}, which proves the absence of a liveness failure up to a certain bound;
i.e. for a property $GF P$, instead of proving $\neg P$ cannot happen infinitely often, it tries to prove $\neg P$ does not occur after $k$ steps. $k$-liveness checks liveness as a sequence of safety checks increasing $k$ incrementally.
An improved version of $k$-liveness is to perform a sequence of
safety queries as necessary to find a large-enough bound to
avoid spurious failures by counting the maximum number of times that $\neg P$ can occur \cite{claessen2012liveness}. Adapting any of these methods for liveness checking makes it possible to tackle the problem with existing inductive algorithms for safety verification.


%%% Local Variables:
%%% mode: latex
%%% TeX-master: "main.tex"
%%% End

%%  LocalWords:  bool reachability \texttt{JKind} Lustre PDR Yices MathSAT ok
%%  LocalWords:  SMTInterpol dataflow init

Unbounded model checking is performed through inductive proof methods such as $k$-induction~\cite{SheeranSS00} and IC3/PDR~\cite{Een2011:PDR}.
The PDR is currently the dominant unbounded model checking technique. In the past few years, several variations of this algorithm have been published \cite{hoder2012generalized, vizel2014interpolating, jovanovic2016property, gurfinkelk}.
The original idea in PDR is to compute  a
safe  inductive  invariant by strengthening the property using inductive couter-examples without unrolling the transition relation, while a classical implementation of
$k$-induction tries to find an inductive invariant through
$k$-step unrolling of a transition relation. Symbolic model checkers usually employ these proof methods, using an SMT/SAT solver in the backend. For example, \texttt{JKind} \cite{jkind} is an SMT-based model checker for safety properties that uses parallel cooperating engines including $k$-induction, PDR, and template-based invariant generation.

Another form of symbolic evaluation is performed through bounded model checking.
The goal of bounded model checking is to decide if a given program reaches an error within at
most $k$ unfolding of the transition relation. Although bounded model checkers do not provide a full proof of correctness, they are useful to discover bugs. For example, \texttt{CBMC} \cite{cbmc} checks array bounds (buffer overflows), absence of
null de-references, and assertions. The \texttt{Alloy analyzer} \cite{alloy} is another bounded model checker that checks temporal formulas specified using LTL. This tool also has a core extraction capability based on UNSAT cores. \texttt{JBMC} \cite{jbmc} is a Bounded Model Checker for Java programs that checks runtime exceptions and user-definded assertions. \texttt{LLBMC} is another bounded model checker for finding bugs in C/C++ programs, mainly intended for checking low-level system code. By exploiting the UNSAT core generation mechanism, we will be able to determine bounded validity cores using these tools\footnote{See \ref{sec:muses} and \ref{sec:bvc}.}.

\subsection{Compositional Reasoning}
Complex systems are usually composed from libraries of components. The specification of such systems are decomposed into properties of each individual component. Then, compositional verification is employed to ensure the correctness of the top level properties, while integrating components \cite{NFM2012:CoGaMiWhLaLu}. Previously Murugesan et al. demonstrated ~\cite{hilt2013} a model-based approach to system construction in which compositional proofs are used to to establish satisfaction arguments. To cope with the complexity of modeling and scalability of verification of large systems,
they proposed an approach in which systems can be decomposed into subsystems, modeled individually and verified compositionally. The decomposition of system into subsystems induces the need to decompose the requirements of the system `flowed down'' to each subsystem that is then modeled and verified.

Given an architectural model of the system (decomposition of system into components) in which each component (including the system) is endowed with its own set of requirements,
 they used a tool suite called AGREE~\cite{NFM2012:CoGaMiWhLaLu} -- a reasoning framework based on assume-guarantee reasoning -- to compositionally verify whether system level requirements are established as a logical consequence of the component level requirements and the system level assumptions.
 Using AGREE they were able to verify large and complex systems efficiently. AGREE partitions the task of verification along the architectural lines of the system. Stating from the leaf level, it systematically verifies if the parent level requirements hold as a logical consequence of its immediately child component requirements in the given architecture.

To verify the requirements, AGREE uses the JKind~\cite{jkind} model checker. The underlying SMT solver in JKind automatically constructs proofs to establish satisfaction of requirements in the model. A proof can be visualized as a derivation tree where the leaves of the tree are axioms -- elements of the model such as components requirements, interface connections, system assumptions -- and each interior node represents the application of an inference rule that leads to proving the system requirement. If the solver encounters a violation of a requirement while constructing the proof, it reports it along with a counterexample - a concrete path of execution that explains the violation. On the other hand, when the proof is successfully constructed, the tool reports that the requirement is satisfied.

An evidence in this context is nothing but an
explanation about which parts of the model (the component requirements and system assumptions) the
model checker used to prove the system level requirement. Since the solvers typically abstract away the proof it creates, with IVCs, we develop a technique to query the solver to excavate the axioms that were used as part of the proof. The IVC helps explain how the solver reported the satisfaction of the requirement, that is comparable to the counterexample explains the negative result.



\subsection{Commercial Model Checkers}
Several commercial tools produce~\emph{proof-cores}~\cite{hanna2015formal, jasper_gold}, which we believe to be similar to IVCs/MIVCs, but are not presented at a level of formality to perform a precise comparison.  However, to the best of our knowledge, none of these tools offer to calculate \emph{all} proof-cores. Besides, the proof-core provided by these tools is usually used for internal analyses the tool performs such as coverage measurement. Therefore, the cores are not intended to be returned to the user in a clear way representing the actual design elements or a portion of the model. Moreover, these tools usually skip the minimization process, so their computed cores are not minimal.

In general, solutions provided by the commercial tools are quite underspecified:
no formal description of the proof-core notion or algorithms are provided. In addition, no implementations or experimental results are provided, so we are not able to benchmark our techniques against those tools. However, our work can also be useful towards the support of this capability in future editions of these tools.

\section{Slicing}

Program  Slicing  is  a  well-known  decomposition  technique  that  maintains a
set of program statements  relevant to  the computation  of a  selected  function, called a slicing criterion. Generally speaking, given a slicing criterion, a slice is defined
 as any subset of the program which maintains the original effect of the program on the criterion \cite{Weiser97}, which is called an executable slice \cite{Androutsopoulos}.
  Slicing has many applications including optimizing program models
   for the purpose of verification using model checking \cite{Androutsopoulos, Jhala:2005, Dwyer:2006}.

Slicing is usually performed based on reachability analysis in program
dependence graphs (PDGs). PDG nodes and edges show program states and dependence\footnote{data dependence or control dependence}, respectively. PDGs are specifically useful in \emph{static slicing}, where
a slice is independent of the inputs,
 and maintains the program effects on the criterion
correctly for all possible executions. Alternatively,
\emph{dynamic slicing} executes a path through the program, computing the statements which have an impact on the criterion for that
specific execution \cite{Androutsopoulos}. Dynamic slicing is very useful in debugging, while static slicing is more attractive as an aid to verification.

For a given slicing criterion, static slices can be constructed from a backward or forward analysis. In the backward approach, the statements of the program that does not have any effect on the criterion are sliced away. However, the forward approach slices away those statements not affected by the criterion.
Our work can be viewed as a more accurate form of backwards static slicing starting from a requirement ~\cite{Tip95asurvey}. Slicing can determine the cone of influence (COI) for a given property, while IVCs are a subset of COI.

In fact, to start the verification process and IVC computation, we fisrt perform {\em backwards slicing} from the formula that defines the property of interest of the model. This step is to speed up the verification process.
 Then, IVCs are computed from the proof of the property over the sliced model.  The slice produced is smaller and more accurate than a static slice of the formula~\cite{Weiser:1981:slicing}, but guaranteed to be a sound slice for the formula for all program executions, unlike dynamic slicing~\cite{Agrawal:1990:slicing}.  Predicate-based slicing has been used~\cite{Li04:slicing} to try to minimize the size of a dynamic slice.  Our approach may have utility for some concerns of program slicing (such as model understanding) by constructing simple ``requirements'' of a model and using the tool to find the relevant portions of the model.


\section{UNSAT cores and Minimal UNSAT Subformulae}
\label{sec:muses}
Satisfiability Modulo (the) Theory (SMT) solvers are powerful tools to decide the
satisfiability of boolean combinations
of propositional atoms in the language of a specific theory. 
A Boolean formula is satisfiable if there is a consistent assignment of values $true$ or $false$ to its variables in such a way that the formula evaluates to $true$. If the formula evaluates to $false$ for all possible variable assignments, then it is unsatisfiable. For example, the formula
 $a \wedge \neg a$ is unsatisfiable because there is no possible 
 assignment for $a$ to make the formula 
 evaluate to $true$. 
 On the other hand the formula
  $a \vee b$ is satisfiable when at least 
  either $a$ or $b$ 
  is assigned to $true$. An assignment that makes the formula $true$ is called a model. Solvers return a satisfiable model in case the formula is satisfiable: $a = true$ is one possible model for our example.

For a given unsatisfiable problem instance, solvers try to generate a proof of unsatisfiability. It is usually more useful to have a minimum proof of unsatisfiability. Such a proof is dependent on identifying a sub-set of clauses that make the problem unsatisfiable. Solvers are usually capable of reporting such sub-sets in the proof, which is known as \emph{UNSAT core}. However, the generated unsat core is not guaranteed to be minimal.

Every propositional logic formula can be transformed into an equivalent conjunctive normal form (CNF) using the laws of Boolean algebra. So, a satisfiability problem can be given to a solver as a set of queries. A formula is in CNF, if it is a conjunction of clauses (or a single clause). Each clause is disjunction of positive or negative literals (i.e. a variable or the negation of a variable). So each CNF formula can be formulated as a finite set of clauses (or constraints). Then we assume there is a function $\checksat(F)$ which determines if $F$ is satisfiable or not. 
When $F$ is unsatisfiable, we assume we
have a function $\unsatcore()$ which returns a minimal subset of the
constraints such that the formula is satisfiable without them. 

\begin{definition}{\emph{Minimal Unsatisfiable Subformulas(MUSes):}}
  \label{def:mus}
  Let $C$ be a finite set of constraints.
  $U \subseteq C$ be its unsatisfiable subset. 
  A constraint $c \in U$ is an 
  unsatisfiable core for $U$ if $U \setminus \{c\}$ is satisfiable.
  A set of all unsatisfiable cores of $U$ constitute 
  an MUS for $C$. 
  Note that such a set is not necessarily unique, and $C$ can have several distinct MUSes.
\end{definition}

Our work builds on top of a substantial foundation building
MUSes from UNSAT cores~\cite{Cimatti2007:UNSAT}, including \cite{marques2010minimal, belov2012towards, ryvchin2011faster, belov2012computing, nadel2010boosting}.  Recent algorithms can handle very large problems, but computing MUSes is still a resource-intensive task.  While some work is aimed at providing a set of minimal unsatisfiable formulae, minimality is usually defined such that given a set of clauses $M$, removing any member of $M$ makes it satisfiable \cite{belov2012computing}.  The step of producing minimal invariants for proofs has been investigated in depth by Ivrii et al. in~\cite{Ivrii14:invariants}.

In  recent  years,  a  number  of  efficient algorithms  for  extracting minimal UNSAT subformulae (MUSes)  have  been proposed \cite{liffiton2005max},
most of which are focused on computing a single MUS  \cite{bacchus2015using, belov2012muser2, belov2013core, belov2012towards, nadel2014accelerated}. A general algorithm for extracting a single MUS is shown in \ref{alg:mus}.

\begin{algorithm}[t]
  \SetKwInOut{Input}{input}
  \SetKwInOut{Output}{output}
  \Input{an unsatisfiable set of constraints $C$}
  \Output{MUS $M$}
  \BlankLine
  $M \leftarrow C$\\
  \For{$c \in M$} {
    \If{$\checksat(M \setminus c) = \unsat$}{
      $M \leftarrow M \setminus \{c\}$
    }
  }
  \Return{M}
\caption{SAT domain single MUS extraction algorithm}
\label{alg:mus}
\end{algorithm}

As mentioned, an unsatisfiable problem can have several distinct MUSes. Although the problem of finding all MUSes is even harder than finding one MUS, there is some strong research in the literature focusing on this problem. For example, Recent work by Liffiton et al. \cite{marco2016fast} proposed an efficient algorithm to generate MUSes, called MARCO.
Another work by Bendik et al. \cite{bendk16} tries to address this problem in the domains where minimization process is rather expensive.
These algorithms can be used in our work in order to develop a new algorithm for computing all minimal IVCs. This will require changing the underlying mechanisms that are used to construct candidate solutions and also changing the structure of the proof of correctness. The technique used in MARCO can be adapted to our work for computing all minimal IVCs. Algorithm \ref{alg:marco} is an abstract version of MARCO. This algorithm uses a symbolic representation of the
power set of $C$, $\mathcal{P} (C)$,  cleverly exploiting isomorphism between finite power sets
and Boolean algebras. 
A brute-force approach to calculate all MUSes is basically explore all subsets of $C$ and determine if they are MUS or not. In the power set exploration process, subsets whose satisfiability is not known
yet are called \emph{unexplored}; i.e. initially, all subsets in $\mathcal{P} (C)$ are unexplored, and at the end, satisfiability of every subset is known; i.e. all are \emph{explored}.
In MARCO, subsets of
 $C = \{c_1 ,c_2 , \dots ,c_n \}$ are encoded using a set of Boolean variables (literals)
$A = \{a_1 ,a_2 , \dots, a_n \}$ such that every constraint $c_i \in C$ is assigned to a Boolean variable $a_i \in A$. Let assume we have a function $Lit: C \rightarrow A$ that returns a corresponding literal of $c_i \in C$.

Then the algorithm maintains a Boolean formula from literals of $A$, called $map$, to represent the set
of unexplored subsets of $C$.  $map$ is initially $true$.
MARCO iteratively explores $\mathcal{P} (C)$ to find each MUS. In each iteration, an unexplored subset $C' \subseteq C$ is non-deterministically selected by finding
a model of $map$. If $C'$ is satisfiable, it is grown
to a maximal satisfiable subset (MSS); Set $S \in C$ is MSS if $\forall c \in C \setminus \{S\}, (S \cup \{c\})$ is unsatisfiable. All subsets of a maximal satisfiable subset are also satisfiable. So, in this case, $map$ is updated in a way to block those subsets from future computation by marking them as \emph{explored} (line \ref{line:marco:sat}).
On the other hand, if $C'$ is $\unsat$, it is shrunk to a MUS. 
When an MUS is found then all of its supersets are guaranteed to be unsatisfiable. So, $map$ is updated in a way 
to mark all those supersets as \emph{explored} in order to block them from future exploration (line \ref{line:marco:unat}).
The grow/shrink procedure can be carried out by any algorithm for
a single MSS/MUS extraction which makes MARCO applicable to arbitrary
constraint satisfaction domain.\footnote{In Algorithm \ref{alg:marco} we abstracted these procedures. The shrink is basically the loop in Algorithm \ref{alg:mus}. A similar approach can be taken to perform the grow procedure: constraints are added one by one in a loop to check for unsatisfiability. We will explain these procedures further in Chapter \ref{ch:ivc}.}

\begin{algorithm}[t]
  \SetKwInOut{Input}{input}
  \SetKwInOut{Output}{output}
  \Input{an unsatisfiable set of constraints $C$}
  \Output{set of all MUSes $AM$}
  \BlankLine
  $AM \leftarrow \empty$
  $map \leftarrow \top$ \\
   \While{$\checksat (map) = \sat$}  {
   $C' \leftarrow$ find an unexplored subset of $C$ using a model of $map$
    \If{$\checksat(C') = \sat$}{
      $M \leftarrow$ grow $(C')$\\
      $map \leftarrow map \wedge (\bigvee_{c_i \notin M} Lit (c_i))$  \label{line:marco:sat}
    }
    \Else{
    $M \leftarrow$ shrink $(C') $\\
    $AM \leftarrow AM \cup M$ \\
    $map \leftarrow map \wedge (\bigvee_{c_i \in M} \neg Lit (c_i))$ \label{line:marco:unat}
    }
  }
  \Return{$AM$}
\caption{MARCO algorithm for computing all MUSes}
\label{alg:marco}
\end{algorithm}


UNSAT cores and MUSes are used for many different activities within
formal verification. Gupta et al. \cite{gupta2003iterative} and
McMillan and Amla \cite{mcmillan2003automatic} introduced the use of
unsatisfiable cores in proof-based abstraction engines. Their goal is
to shrink the abstraction size by omitting the parts of the design
that are irrelevant to the proof of the property under verification.
However this work is for finite systems in the domain of SAT solving,
 and the abstractions built are not intended to be returned to the user.
 We design our algorithms for IVC computation for
 infinite systems with the support of the state of the art of the SMT solvers. In addition, for IVC computation, the goal is to provide meaningful results to the user.

\section{Proof/Lemma Minimization}
The IVC idea shares many similarities with approaches for computing minimal invariant sets for inductive proofs (such as is performed for inductive proof certificates~\cite{piskac2016, Ivrii14:invariants}).
A proof certificate is an artifact embodying a proof of the
claim that can then be validated by a trusted checker.
Given a safety property $P$, an formula $Q$ is a $k$ inductive strengthening of $P$ if
$P \rightarrow Q$, and $Q$ is $k$ inductive. 
%For $i = 1,2$, if $Q_i$ is $k_i$
%inductive strengthening of $P_i$, then $(Q_1 \wedge Q_2)$
%is $k$ inductive strengthening of $(P_1 \wedge P_2)$, where $k = max(k_1, k_2)$.
Formula $Q$ is a certificate for property $P$ if $Q$ is
a $k$ inductive strengthening of $P$.
Certificates need to be concise and efficient to check
by an independent tool or method. In particular, 
checking a certificate should not take more
time than proving the original property. Mebsout and Tinelli presented a method for extracting and verifying proof certificates \cite{piskac2016} implemented in \texttt{Kind 2} model checker. 
\texttt{Kind 2} performs verification by running different engines concurrently. Specifically, it employs number of auxiliary invariant generation engines to
discover and pass along auxiliary invariants that might be
helpful in proving a property of interest. Then these invariants are considered as
safety certificates for the property. 
To simplify the certificates, they either reduce $k$ or the size/complexity of the certificate formula.
After obtaining a $k$ inductive invariant $Q$, \texttt{Kind 2} starts with reducing $k$ before simplifying the formula. It will replay the inductive
step for $Q$ for values $i < k$ , following one
of three different strategies:
\begin{itemize}
  \item forward enumeration: all values of $1 \le i \leq k' < k$  are checked, and $k'$ is the first where $k'$ inductiveness holds.
  \item binary search: the interval $[1 , k ]$ is divided into 
  subintervals $[1 , k' ]$ and
$[ k' + 1 , k ]$ of similar size. Then, the first or
the second interval is recursively considered depending on whether $Q$ is $k'$ inductive
or not.
  \item backward enumeration: all values of $i$ from $k$ down to 1 are checked, and it stops
when $k'$ inductiveness does not hold anymore.
\end{itemize}

To simplify the certificate, $Q$ is converted into a set of subformulae.  
Iteratively, each subformula is removed from a set and it is checked if $Q$ is still $k$ inductive or not. In this case subformulae not needed to prove
$Q$ are pruned away. Finally, to verify a certificate, it needs to be proved that $Q$ is
a $k$ inductive strengthening of the original property.

Our IVC algorithm also needs to find a minimal lemma set.  However, there is a substantive difference: to find a minimal set of constraints, it is usually necessary to find new proofs involving {\em new lemmas} not used in the original proof, which accounts for the expense of finding an accurate minimal IVC. This process will be explained in Chapter \ref{ch:ivc}. Note that our lemma minimization technique used in IVC generation is actually more efficient than the above technique.

\section{Traceability}
\emph{Requirements traceability} can be defined as

\begin{quotation}
\textit{``the ability to describe and follow the life of a requirement, in both forwards and backwards direction (i.e., from its origins, through its development and specification, to its subsequent deployment and use, and through all periods of on-going refinement and iteration in any of these phases).''}~\cite{gotel}.
\end{quotation}

This topic has been of great interest in research and practice for several decades. Intuitively, it concerns establishing relationships, called \emph{trace links}, between the requirements and one or more artifacts of the system.
Among the several different development artifacts and the relationships that can be established from/to the requirements, being able to establish trace links from requirements to artifacts that realize or \emph{satisfy} those requirements---particularly
to entities within those artifacts called \emph{target artifacts}~\cite{gotel2012traceability}---has been enormously useful in practice. For instance, it helps analyze the impact of changes in one artifact on the other, assess the quality of the system, aid in creating assurance arguments for the system, etc. In this work, we focus our attention to this subset of requirement traceability, that we call \emph{Requirements Satisfaction Traceability.}

Instead of just recording the trace links from each requirement to the target artifacts, \emph{Satisfaction Arguments}~\cite{zave1997four} offer a semantically rich way to establish them. Originally proposed by Zave and Jackson~\cite{zave1997four}, a satisfaction argument demonstrates how the behaviors of the system and its environment together satisfy the requirements. From a traceability perspective, these arguments help establish
 trace links (the \emph{satisfied by} relationship) between the requirements and those parts of the system and environment (the target artifacts) that were necessary to satisfy the requirements; We call those target artifacts a \emph{set of support} for that requirement. This set of support is the same as a minimal inductive validity core obtained from the correctness proof of the requirement.

\section{Requirements Adequacy}
Determining completeness of properties has also been extensively studied. Certification standards such as DO-178C~\cite{DO178C} require that requirements-derived tests achieve some level of structural coverage (MC/DC, decision, statement) depending on the criticality level of the software, in order to approximate completeness.  If coverage is not achieved, then additional requirements and tests are added until coverage is achieved. Recent work by Murugesan~\cite{murugesan2015we} and Schuller~\cite{schuler_assessing_2011} attempted to combine test coverage metrics with requirements to determine completeness.  Chockler~\cite{chockler_coverage_2003} defined the first completeness metrics directly on formal properties based on mutation coverage.  Later work by Kupferman et al.~\cite{Kupferman:2006:SCF} defines completeness as an extension of vacuity to elements in the model.  We present an alternative approach that uses the proof directly, which we expect to be considerably less expensive to compute.


\subsection{Coverage and Mutations}

Different notions of coverage have been well defined in software testing. However, in formal verification, it is very complex to define and compute this notion.
Usually, coverage techniques in the property-based verification try to measure the quality of the specification in regard to the completeness of a set of properties.

Coverage in verification was introduced in \cite{hoskote1999coverage, katz1999have}. Hoskote et al. \cite{hoskote1999coverage} suggested a state-based metric in model checking based on FSM mutations, which are small atomic changes to the design. Then, the method for measuring coverage is to model check a given property for each mutant design.
Later in \cite{chockler_coverage_2003}, Chockler et al. provided corresponding notions of metrics used in simulation-based verification for formal verification. In fact, they improved the same idea of mutation-based coverage where each mutation is generated to check if a specific
design element is necessary for the proof of the property.
 However, the proposed algorithm is both computationally expensive and approximately linear
 in the number of mutations. Note that most of the mutation-based metrics, including \cite{kupferman_theory_2008, chockler2001practical}, are focused on finite state systems and hardware systems.

In general, specification completeness can be defined with
regard to the notion of coverage. In fact, the way that coverage
is formalized plays a key part in the strength/effectiveness of
a method for the assessment of completeness. The goal of a coverage metric is usually to assign a numeric score that describes how well properties cover the design. The majority of the work on coverage metrics has focused on {\em mutations}, which are ``atomic'' changes to the design, where the set of possible mutations depends on the notation that is used.  A mutant is ``killed'' if one of the properties that is satisfied by the original design is violated by the mutated design~\cite{chockler_coverage_2003,chockler2001practical,chockler2010coverage,Kupferman:2006:SCF,kupferman_theory_2008}.  There are many different kinds of mutations that have been proposed, primarily focused on checking sequential bit-level hardware designs.  For these designs, {\em state-based} mutations flip the value of one of the bits in the state.  There are several variations depending on whether the flip is performed on a single state within a Kripke structure~\cite{hoskote1999coverage}, or in the description of the signal in the transition relation of the circuit~\cite{chockler2001practical}.  {\em Logic-based} mutations fix the value of a bit to constant zero or one, and can be used to determine whether properties can find stuck-at faults.  {\em Syntactic} mutations~\cite{chockler_coverage_2003} remove states in a control flow graph representation of hardware.  Similarly, for software, it is possible to apply any of the ``standard'' source code mutation operators used for software testing~\cite{Andrews06:mutation} towards requirements coverage analysis.  Some examples of software mutations are \cite{Budd:1980}:
\begin{enumerate}
    \item Replace an integer constant $C$ by one of $\{0, 1, -1, C + 1, C - 1\}$,
    \item Replace an arithmetic, relational, logical, bitwise logical, increment/decrement, or arithmetic-assignment operator by another operator from the same class,
    \item Negate the decision in an \texttt{if} or \texttt{while} statement,
    \item Delete a statement.
\end{enumerate}

Mutation-based approaches are often impractically expensive to compute; even for small models, there are many possible mutations and we deal with too many verification problems.  The number of single-mutation programs is roughly the product of the leaf elements of the program abstract syntax tree (AST) and the size of the chosen set of mutations, which can lead to an impractical number of verification problems.

\newcommand{\andnode}{{\sc and}}
\newcommand{\invnode}{{\sc inv}}
\newcommand{\inpnode}{{\sc inp}}
\newcommand{\regnode}{{\sc reg}}
\newcommand{\mutnode}{{\sc mut}}
\newcommand{\inputnode}{{\sc input}}

Mutations for hardware are discussed in~\cite{chockler2010coverage,Kupferman:2006:SCF,kupferman_theory_2008}. A more recent work in \cite{chockler2010coverage} performs coverage analysis through interpolation \cite{mcmillan2003interpolation}. This work is also based on design-dependent mutations \cite{chockler_coverage_2003}, where a design is considered as a net-list with nodes of types \{ \andnode, \invnode, \regnode, \inputnode \}.
Each mutant design changes the type of a single node to \inputnode . When property $\phi$ satisfied by the original net-list fails on the mutant design, it is said that a mutant is discovered for $\phi$.
Then, the coverage metric for $\phi$ is defined as the fraction of the discovered mutants, based on which the coverage of a set of properties is measured as the fraction of mutants discovered by at least one property.
To decrease the cost of computation, coverage analysis is performed at several stages; first, all the nodes that do not appear in the resolution proof of a given property are marked as \emph{not-covered}, and the rest of the nodes are marked as \emph{unknown}. Then, for the unknown nodes, the basic mutation check is performed: if a corresponding mutant design violates the property, it will be considered as \emph{covered}. Otherwise, the algorithm tries to drive an inductive invariant to prove that the node is not covered. Finally, an interpolant-based model checking is applied to the nodes that are still unknown.

Most of the mutation-based coverage techniques can be put into the category of falsity coverage, where we mutate the design and see if the property is still valid or not.
In this way, we understand if that mutant was necessary (covered) for (by) that property. 
It is important to note that some mutations reduce the state space of the system to
be explored; in this case, any universal property that was true of the original system
must, by definition, be true of the mutated system. This is where falsity coverage is not effective, and the notion of \emph{vacuity coverage} comes into play. Falsity coverage asks weather the mutant FSM still satisfies the property, while vacuity coverage checks if the mutant FSM satisfies the property vacuously. In vacuity coverage, first, we makes sure the property is non-vacuous. Then, we mutate the design. If the property is
vacuous afterwards, it means that the mutant was necessary for that property.


%\mike{end placement}

A similar notion to IVCs outlined in a patent~\cite{hanna2015formal}, which sketches a family of {\em proof core}-based metrics for use in hardware verification.  While the approach described by the patent is general, it is quite underspecified and it is not possible to compare their approach and ours. In addition, in commercial hardware verification frameworks do different forms of coverage analysis: Cadence JasperGold~\cite{jasper_gold} does some form of proof core coverage and Synopsys VC Formal~\cite{Synopsys_VC_formal} does a mutation-based coverage approach.  These coverage measurement approaches may be similar to the metrics we introduce but are not described in sufficient depth to be compared.

A different approach to measure coverage involves checking whether each output signal is fully constrained by the specification \cite{das2005formal, claessen2007coverage, grosse2007estimating}. For example, in \cite{claessen2007coverage}, authors propose a design-independent coverage analysis where missing properties are identified by unconstrained output signals. Given a property list and a specific computed signal $s$ (usually drawn from the circuit outputs), if there is a trace with a point in time when $s$ is not constrained to be a single value by the set of properties and the input trace, then the property set is incomplete. Alternately, given two traces that differ only in the value of signal $s$ at a particular time step, if both traces satisfy property $P$, then $s$ is not covered by $P$.
 The work in \cite{haedicke2012guiding} refines this notion of coverage by providing a numeric score for each incompletely covered signal $s$.  Such metrics are very rigorous but can lead to overspecification: the specification must completely define the input/output function of the implementation.

%a coverage metric that computes a numerical value to describe how much of the circuit behavior is constrained by a given set of properties. This methods investigates, given property $\phi$ and a specific output $s$, if there exist two traces $\sigma_{1}$ and $\sigma_{2}$ that: (1) $\sigma_{1} \vDash \phi$ and $\sigma_{2} \vDash \phi$ (2) $\forall$ signals $s' \neq s, \forall t. \sigma_{1}(t, s') = \sigma_{2}(t, s')$ (3) $\exists t. \sigma_{1}(t, s) \neq \sigma_{2}(t, s)$. This method was implemented in SMV model checker \cite{smv}.

Another technique to measure requirements completeness is to employ several surrogate models; for example, Zowghi and Gervasi~\cite{zowghi2003three} use refinement to show {\em relative completeness} with respect to a {\em domain} model, which describes the behavior of the real world, irrespective of change induced by software.  In their model, each iteration of refinement of requirements and domain models must be sufficient to prove the requirements of the previous iteration.  However, this idea has two problems: first it provides no notion of absolute completeness, and second, it requires construction of a domain model, which is often difficult and/or expensive to construct.

Outside the context of formal verification, many authors have theorised and empirically validated conceptual model completeness, which are mostly dependent on a subjective judgement \cite{drechsler2012completeness, firesmith2005your, chang2007finding,katta2013investigating, zowghi2003three, espana2009evaluating}.



\section{Vacuity Detection}
Another potential use of our work is for ``semantic'' vacuity detection.  A standard definition of vacuity is syntactic and defined as follows~\cite{Kupferman:2006:SCF}: {\em A system K satisfies a formula $\phi$ vacuously iff $K \vdash \phi$ and there is some subformula $\psi$ of $\phi$ such that $\psi$ does not affect $\phi$ in K}.  Vacuity has been extensively studied~\cite{Gurfinkel:2012:RVB,Chockler2008,DBLP:Ben-DavidK13,Kupferman:2006:SCF,Chockler:2007,Beer1997} considering a range of different temporal logics and definitions of ``affect''.  On the other hand, our work can be used to consider a broader definition of vacuity.  Even if all subformulae are required (the property is not syntactically vacuous), it may not require substantial portions of the model, and so may be provable for vacuous reasons.  The problem is exacerbated when the modeling and property language are the same (as in JKind), because whether a subformula is considered part of the model or part of the property, from the perspective of checking tools, can be unclear.

Torlak et al. in~\cite{Torlak08:cores} finds MUSes of Alloy
specifications, and considers semantic vacuity.
 Alloy models are only analyzed up to certain
size bounds, however, and in general are unable to prove properties
for arbitrary models. Also, because we are extracting information from
proofs, it is possible to use IVCs for additional purposes (proof
explanation and completeness checking).

\section{Safety Standards}
Due to the complexity of computer systems and our reliance on them, it is of the utmost importance that the development of these systems proceeds in a way that minimizes development errors. There are a couple of safety standards that focus on safety critical components, including DO-178C \cite{DO178C}, MOD-0053 \cite{standard2007standard}, and ISO 26262 \cite{iso201126262}. Production of a functional safety case is usually a requirement for compliance with a specific standard, which brings opportunities and challenges to safety practitioners and researchers. In this section we briefly describes the objectives of these standards and how IVCs can be related to this area.

Software Considerations in airborne systems and equipment are usually regulated by another certification: DO-178C \cite{DO178C}, which demands a rigorous software development process.   There are a couple of key components in DO-178C that are related to our purpose; first is to ensure the low-level requirements are in compliance with the high-level safety requirements. That is, each refinement must be shown not to introduce functionality not present in the artifact from which it was derived (adequacy).
Another component of DO-178C is coverage analysis at the two levels: requirements-based analysis and structural analysis.
After requirements-based testing, which ensures the software in the target computer will satisfy the high-level requirements, the purpose of coverage analysis in DO-178C is to determine how well this type of testing verified the implementation of the software requirements. Then, the structural coverage analysis is to determine which code structure was not exercised by the requirements-based test procedure.
DO178C currently uses a variety of metrics to determine adequacy of requirements, but much of the effort involves code-level testing.  Test suites are derived from requirements and used to test the software, and measured using different structural coverage test metrics.
If code-level test suites do not achieve full coverage, then an analysis is performed to determine whether there are missing requirements and test cases.  The kind of structural coverage required (e.g., statement, branch, MCDC) for adequate testing is driven by the criticality of the software in question.
Traceability is another explicitly defined component of DO-178C; that is, low-level requirements must be traceable to the high-level requirements if was derived from. Further, two other traceability objectives in DO-178C are (1) traceability of high-level requirements to system requirements and (2) traceability of software design to high-level requirements, which specifically require applicants to demonstrate bi-directional traceability.

MOD-0053 \cite{standard2007standard} is a defense standard that provides safety management requirements for defense systems, which are designed to be applied in different phases of the development process of MOD\footnote{Ministry of Defence} projects. A key component of this standard is a
\emph{Safety Case}, which demonstrates how safety will be, is being and has been, achieved and maintained. Then, to summarize a Safety Case and document safety management activities, Safety Case Reports should be provided. These reports are treated as safety assurance that shows safety is managed effectively. Hazard analysis is another important component of MOD-0053, for which enough evidence must be provided to show adequate hazards are identified and managed properly. Therefore, traceability in this standard is a key component defined at two levels: (1) traceability  between each safety
requirement and the source of that requirement, and (2) traceability of the safety
risk/hazards management to hazards and accidents. Requirements traceability in MOD-0053 should be established in both ways: (1) to trace each requirement to the
part of the code which implements it, and (2) to trace from any part of the
code, back through the software design/specification, to the requirement. Currently, traceability checks are performed via traceability matrices built manually, however, MOD-0053 recommends that traceability be provided between the formal arguments and the software
requirement, which will help to check if all the requirements have
been verified, and to ensure that the implications of changes to requirements can be assessed. In addition to traceability, this standard requires an assessment of the veracity and completeness of the software Safety Case, which we believe all these objectives can be achieved/facilitated by IVCs.

ISO 26262 \cite{iso201126262} is a common standard used in the automotive industry that allows to measure how safe a system will be in service. This standard provides guidance in different steps of the product development process to manage functional safety of a system at the hardware and software levels.
One important component of ISO 26262 is Automotive Safety Integrity Levels (ASILs), by which each component is assigned to an acceptable risk level determined at the beginning of the development process. The goal is to analyze the system functionalities with respect to possible hazards. Each requirement is assigned a class of criticality from A to D, where D has the most safety critical processes and strictest testing regulations. In ISO 26262, qualification of software components demand testing not only under normal operating conditions, but also in the presence of faults so to determine how system reacts to abnormal inputs.  ISO 26262 has other important components like \emph{test tool qualification}, which are not closely related to the context of verification.

In order to meet the objectives of the safety standards, developers have to put a lot of manual effort into providing acceptable evidence such as assurance cases, traceability matrices, and requirements adequacy. We claim that such safety analyses can be automated with the IVC notion, and we would like to study how we can achieve that. Since IVCs are derived from the formal proofs, they will make much more accurate safety evidence than those created manually.


%\chapter{Proposed Approach}

The idea of inductive validity cores was introduced in Chapter \ref{ch:intro}. 
This idea is applicable to the context of symbolic model checking using inductive proof methods. The idea is, after proving the correctness of a given property, to extract a minimal portion of the system (model) necessary for the proof of the property. In other words, we would like to determine why the property is satisfied by the system. Since this information is obtained from the inductive proofs, we call it \emph{inductive} validity core. With minimal IVCs, we are able to abstract away the part of the system irrelevant to the proof of the property.

This chapter, first, mentions some required background, then provides a formal description of the IVC notion. Using this formalization, we explain how IVCs can be used in traceability and adequacy checking.

\section{Background}

\newcommand{\satisfies}{\vdash_{\!\!s}}
\newcommand{\nsatisfies}{\nvdash_{\!\!s}}
\newcommand{\bool}[0]{\mathit{bool}}
\newcommand{\reach}[0]{\mathit{R}}
\newcommand{\ite}[3]{\mathit{if}\ {#1}\ \mathit{then}\ {#2}\ \mathit{else}\ {#3}}
\newcommand{\ivc}{\textit{IVC}}
\newcommand{\mivc}{\textit{MIVC}}

This section presents formalizations of transition systems, inductive validity cores, and background information on mutation-based coverage metrics.  Although we focus the formalism below on safety properties, the approach is able to handle liveness properties through reduction to safety properties, as is performed by, e.g., K-liveness.

Given a state space $U$, a transition system $(I,T)$ consists of an
initial state predicate $I : U \to \bool$ and a transition step
predicate $T : U \times U \to \bool$. We define the notion of
reachability for $(I, T)$ as the smallest predicate $\reach : U \to
\bool$ which satisfies the following formulas:
\begin{gather*}
  \forall s.~ I(u) \Rightarrow \reach(u) \\
  \forall u, u'.~ \reach(u) \land T(u, u') \Rightarrow \reach(u')
\end{gather*}
A safety property $P : U \to \bool$ is a state predicate that holds on a transition system $(I, T)$ if it holds on all
reachable states, i.e., $\forall u.~ \reach(u) \Rightarrow P(u)$,
written as $\reach \Rightarrow P$ for short. When this is the case, we
write $(I, T)\vdash P$. We assume the transition relation has the structure of a top-level conjunction. This assumption gives us a structure that we can easily manipulate. Given $T(u, u') = T_1(u, u') \land \cdots \land T_n(u, u')$ we will write $T = T_1 \land \cdots \land T_n$ for short.
By further abuse of notation,
$T$ is identified with the set of its top-level conjuncts. Thus, $x \in
T$ means that $x$ is a top-level conjunct of $T$, and $S
\subseteq T$ means all top-level conjuncts of $S$ are top-level
conjuncts of $T$. When a top-level conjunct $x$ is removed from $T$, it is written as $T \setminus \{x\}$.


%\begin{definition}{\emph{Inductive Validity Core:}}
%  \label{def:ivc}
%  Given $(I, T)\vdash P$, $S \subseteq
%  T$ is an {\em Inductive Validity Core} for $P$
%  \emph{iff} $(I, S) \vdash P$.
%\end{definition}
%
%In examining provability, we are interested in {\em minimal} sets that satisfy a property $P$; tracing a property to the entire model is not particularly enlightening.
\section{Inductive Validity Cores}

\begin{definition}{\emph {Inductive Validity Core (\ivc):}}
  \label{def:ivc}
  $S \subseteq T$ for $(I, T)\vdash P$ is an Inductive Validity Core,
  denoted by $\ivc(P, S)$, iff $(I, S) \vdash P $.
\end{definition}

In examining provability, we are interested in minimal sets
that satisfy a property P; tracing a property to the entire model
is not particularly enlightening.  Fortunately, \ivc s have
the following monotonicity property: given $(I, T)\vdash P$, $\forall S_1 \subseteq S_2 \subseteq T$. $IVC(P, S_1) \Rightarrow IVC(P, S_2)$.  We next introduce the notion of {\em minimal} inductive validity cores.

\begin{definition}{\emph{Minimal Inductive Validity Core (\mivc):}}
  \label{def:minimal-ivc}
  $S \subseteq T$ is a minimal Inductive Validity Core,
  denoted by $\mivc(P, S)$, iff ~$\ivc(P, S) \wedge \forall T_i \in S.~ (I, S\setminus\{ T_i \}) \nvdash P$.
\end{definition}

Note that given $(I, T) \vdash P$, $P$ always has at least one \mivc , which implies \mivc s are not necessarily unique.
For example, take $I = a \land b$, $T = a' \land b'$, and $P = a \lor
b$. Then both $\{a'\}$ and $\{b'\}$ are \mivc s for $(I, T)\vdash P$. To capture this fact, the \emph{all \mivc s ($AIVC$)} relation has been introduced \cite{Murugesan16:renext}.
$$ AIVC(P) \equiv  \{\ S~|~S \subseteq T \land  \mivc(P, S)\} $$
%In the example in Figure \ref{fig:asw}, as visualized in part (b),
%$AIVC ({\tt P}) = \{\{{\tt P}, {\tt c2}, {\tt c3}\}, \{{\tt P}, {\tt x}, {\tt c3}\}\}$.
\noindent

 %We adapt the recent work by Liffiton et al. \cite{marco2016fast} from the generation of MUSes from UNSAT-cores to all IVCs for inductive model checking.  This requires changing the underlying mechanisms that are used to construct candidate solutions and also changing the structure of the proof of correctness.  In addition, we demonstrate that the approach can terminate with all minimal IVCs even if the witness generator only generates approximately minimal IVCs (utilizing a ``fast''  algorithm for a single IVC computation).


\section{Traceability}
Given {\em all} proofs of a particular property, we are able to categorize the model elements based on \mivc ~and
$AIVC$ relations for $P$:

\begin{itemize}
\item $MUST (P) = \bigcap AIVC(P)$
\item $MAY(P) = (\bigcup AIVC (P)) \setminus MUST (P)$
\item $IRR(P) = T \setminus (\bigcup AIVC (P))$
\end{itemize}

\noindent This categorization helps to identify the role and relevance of each design element in satisfying a property. Function $MUST$ specifies the parts of the model absolutely necessary for the property satisfaction.  Any change to these parts will affect provability of the property. On the other hand, any single element in $MAY (P)$, may be modified without affecting satisfaction of $P$(though modifying multiple elements may require re-proof). The $IRR$ denotes model elements that are irrelevant to the validity of $P$.

The $AIVC$ set improves understanding of how a change in the requirement will affect the target artifacts and vice versa. While the $AIVC$ of a requirement gives a clear picture of various ways a requirement is satisfied by the system, the categorization of target artifacts helps precisely assess and plan when and where the changes have to be implemented. The $MUST$ elements are those target artifacts that are highly likely to change with any change in the requirement, whereas not all $MAY$ elements may need to be changed.

If a requirement has elements only in its $MAY$ set, that is if $MUST$ set is empty
($MUST(r) = \emptyset$), it indicates that the requirement has been (intentionally or unintentionally) implemented in independent ways, such as fault tolerant systems. For such requirements, one has to carefully analyze and decide if the target artifacts in all or one disjoint set needs to be changed. These analysis could be performed either from the perspective of one or all requirements of the system.

From the target artifact side, this categorization helps analyze the impact of changes to the artifact. Suppose we decide to change a target artifact in the $MAY$ set for a requirement. While one might think that it is safe to change this artifact since it does not affect that requirement's satisfaction, an examination of the $AIVC$ sets of other requirements helps identify if it is indeed safe to change that artifact. If it is present in the $MUST$ set for another requirement, then a change to this artifact will definitely impact the other requirement. However, if it is in the $MAY$ sets for all the requirements, then it is clearly safe to change. Hence, this categorization helps us to assess critical dependencies between the target artifacts and the satisfaction of requirements and thus enables a precise bi-directional impact analysis of a change.

Complete traceability can assist in tailoring verification and validation in systems. For instance, if several requirements have a certain target artifact in their $MUST$ set, say an particular assumption, it reveals the importance of focusing V\&V attention on that artifact. Along the same lines, for a system with a complex architecture (components that each have functionality) such as  system of systems, this categorization helps identify components that is critical to satisfy most requirements. This categorization helps plan verification strategies.

\section{Coverage Metrics in Formal Verification}

Mutations for hardware are discussed in~\cite{chockler2010coverage,Kupferman:2006:SCF,kupferman_theory_2008}.  In most of this work, the hardware model is formalized as an and-inverter-graph net-list, a graph representation of circuits in which vertices are \{\andnode, \invnode\} gates or memory elements (registers) and edges are connections between gates (signals).  Formally, A net-list $N$ is a directed graph $(V_N,E_N, \tau_N)$ where $V_N$ is a finite set of vertices, $E_N \subseteq V_N \times V_N$ is the set of directed edges and $\tau_N : V_N \rightarrow \{$\andnode, \invnode, \regnode, \inpnode$\}$ maps a node to its type, where \andnode\ is an ``and'' gate, \invnode\ is an inverter, \regnode\ is a register, and \inpnode\ is a primary input.  The in-degree of a vertex of type \andnode\ is at least two, of type \invnode\ and \regnode\ is exactly one and of type \inpnode\ is zero. Any cycle in $N$ must contain at least one \regnode\ node~\cite{chockler2010coverage}.

Given this representation, it is possible to discuss mutations of a single vertex: either stuck-at-zero, stuck-at-one, or nondeterministic.  This mutation is performed by changing the vertex to \mutnode, where \mutnode\ can be a ``fresh'' input for \emph{non-deterministic mutations}, or fixed to 0 or 1 for stuck-at mutations. Formally the semantics of a mutant net-list is defined as a new labeling function:
\[ \tau^{v}_{N}(u) = \begin{cases}
    \tau(v) & \textrm{ if $v \neq u$} \\
    \textrm{\mutnode}   & \textrm{ if $v = u$}
\end{cases}  \]
\noindent
%To mutate a vertex $v_i$, a  multiplexer is added to $N$. Then, the  edges  of  the  net-list are modified such  that  the  tails  of  all  the edges directed from
%$v_i$ are changed to the output of the multiplexer, which replaces
%$E_N$ with a new set of edges $E^{v_i}_M$ in the mutated net-list.
Let  $N = (V_N,E_N, \tau_N)$ be the original net-list; to mutate a vertex $v_i$ using $\tau^{v_i}_{N}$, the  edges that had tails pointing to $v_i$ are removed,
 which replaces $E_N$ with a new set of edges $E^{v_i}_M$ in the mutated net-list.
When property $P$ satisfied by $N$ fails on the mutant net-list $(V_N, E^{v_i}_N, \tau^{v_i}_{N})$, which is obtained from the \emph{non-deterministic} mutation of $v_i$, it is said that a mutant is discovered for $P$ (or $v_i$ is covered by $P$).
We assume a function $TR : N \rightarrow T$ that returns the corresponding transition relation of a net-list.
Given this representation, and the initial state $I$, we can define nondeterministic coverage as follows:

\begin{definition} {\emph{Nondeterministic coverage (\nondetcov) ~\cite{chockler2010coverage}.} }
\label{def:non-det}
Given $N = (V_N,E_N, \tau_N)$,
$v_i \in V_N$ is covered by property $P$ \emph{iff} $v_i \in \nondetcov (P)$, where
$\nondetcov (P) = \{ v_i~|~ v_i \in V_N \wedge (I, TR(N)) \vdash P \wedge (I, TR((V_N, E^{v_i}_N, \tau^{v_i}_{N}))) \nvdash P \}$.
\end{definition}
Using  $\nondetcov$, the coverage score of specification $P$ is computed by
\[
   \frac{ | \nondetcov (P) |}{|V_N|}
\]


We propose a new approach for measuring property completeness based on proof rather than mutation.  We first define notation, then describe different possible metrics given a set of {\em minimal proofs}.%\footnote{Section~\ref{sec:impl} describes how these proofs are discovered in practice.}
%\subsection{Coverage and Minimal Proofs}
%Alternatively, we can consider using the proofs themselves as a mechanism for determining adequacy of requirements.

\begin{definition} {\emph{IVC coverage (\ivccov):}} \\
\label{def:coverage-justi}
Given $S \in AIVC(P)$, $T_i$ is covered by $P$ via $S$ \emph{iff} $T_i \in S$.
%Given $S \in AIVC(P)$, $T_i \in T$ is covered by $P$ \emph{iff} $T_i \in S$,
%denoted by $T_i \in \ivccov (P, S)$
\end{definition}

%For the sake of simplicity, we refer to the coverage function
%formalized in Definition \ref{def:coverage-justi} as \ivccov\.
%
We call Definition \ref{def:coverage-justi} a \emph{proof-preserving} metric because, with a set of the model elements marked as covered by \ivccov, $P$ is provable.
%\footnote{\noindent ~Throughout the paper, when a coverage metric is justifiable, like \ivccov, we say that it preserves provability of the property.}
%Thus, the coverage score for \ivccov\ is often higher than the score for \nondetcov.
The coverage score for \ivccov\ can be calculated with: $$\frac{|S|}{|T|}$$
%Note that because minimal proofs are not unique, there are several possible coverage scores.
Because $P$ may have multiple \mivc s,  \ivccov\ metric can lead to various scores that belong to the following set:
\[
\{~\frac{ |S|}{|T|}~|~S \in AIVC(P)~\}
\]

\noindent Note that if an \mivc ~contains all model elements (i.e., the model is {\em completely covered}), then there is only one possible \mivc , so in this case there is no diversity of scores.
Using the notions of $MAY$ and $MUST$, we can introduce additional coverage metrics, which will be part of our research.

Now we focus on the relationship between non-deterministic mutation-based coverage and proof-based metrics. In Chockler et. al. \cite{chockler2010coverage}, each mutant design changes the type of a single node to \inputnode .
Given a suitable encoding of the netlist, assigning a ``fresh'' input is an isomorphic operation to simply removing a $T_i$ from $T$. The mapping is as follows: the net-list becomes a conjunction
of equations, where each vertex becomes a variable $v_i \in U$, and where each non-input vertex becomes an assignment equation $T_i \in T$.
For example, given an AND-vertex $v_i$ with three input edges from other vertexes $\{v_a, v_b, v_c\}$, we would define an equation $T_i \in T$ of the form $(v_i = (v_a \wedge v_b \wedge v_c))$.
%
%As the variable is no longer constrained by a defining equation, it is effectively an %input.

Given this encoding, we can reframe the non-deterministic coverage proposed in \cite{chockler2010coverage} as follows:

\begin{definition} {\emph{Nondeterministic coverage (alternate specification) (\nondetcovalt) ~\cite{chockler2010coverage}.} }
\label{def:non-det-2}
$T_i \in T$ is covered by property $P$ \emph{iff} $T_i \in \nondetcovalt (P)$, where
$\nondetcovalt (P) = \{T_i~|~ (I, T) \vdash P \wedge (I, T \setminus \{T_i\}) \nvdash P\}$.
\end{definition}

\ivccov\ and \nondetcovalt\ are equivalent when all elements within the model are covered: if all model elements are MUST elements, then there can only be one \mivc , and this \mivc ~uses all of the model elements.   In the implementation and experiments, we will focus on the \ivccov\ and \nondetcovalt\ metrics.  Both metrics are fairly rigorous and can be computed reasonably efficiently. We use the \nondetcovalt\  to benchmark our proof-based metrics against the state-of-the-art mutation based coverage.










%
% Mathematically, if we think of the argument as a proof in which the requirement is the claim, then the set of support is the set of axioms (or clauses) that were necessary to prove the claim, and the trace links are means to associate the claim to those clauses. Such proofs are not, in general, unique, and often there are multiple sets of clauses that could be used to construct a proof.  The existing traceability literature does not discuss multiple alternative satisfaction arguments or sets of trace links for one system design.
%
%In our opinion, in the context of satisfaction arguments and {\em satisfied by} trace links, it is beneficial to distinguish between trace links to \textit{``a''} set of support (containing the clauses needed to make a satisfaction argument) vs. \textit{``the''} sets of support (all clauses needed to make all possible satisfaction arguments).
%
%
%Establishing trace links to all sets of support (all MIVCs), that we call \emph{complete} traceability, provides insight into the elements of the set of support---elements that are necessary for any satisfaction argument, elements that are needed for some satisfaction arguments, and elements that do not contribute to the satisfaction of the requirement of interest.  We categorize the elements as \emph{Must}, \emph{May} and \emph{Irrelevant} support elements for each requirement.
%
%
%While precise and complete traceability is beneficial but has been considered impossible to establish in practice~\cite{stravsunskas2002traceability}, our hypothesis is that, in the realm of model based development (MBD), it can be achieved in an automated and efficient manner. We base our hypothesis on the fact that in MBD, the artifacts - both models and requirements - are captured using some form of formal notation and sophisticated tools automatically verify if the requirements are satisfied in the models. We believe that the mathematics underlying the verification tools can be leveraged to establish traceability. In this section, we briefly explain some of the prior work that lead us to pursue our hypothesis.


\section{Related work}
\label{sec:related}

In recent years, extraction of Minimally Unsatisfiable Subformulas (MUSes) has been the focus of a lot of research work \cite{marques2010minimal, belov2012towards, ryvchin2011faster, belov2012computing, nadel2010boosting, ryvchin2011faster, }. Although algorithms proposed by such work can handle very large problems,
computing MUSes is still very resource-intensive task.
While some work aimed to provide a set of minimal unsatisfiable formulae, they define minimality in
a way that given a set of clauses \mathbb{M}, removing every member of \mathbb{M} makes it satisfiable
\cite{belov2012computing}. 
Such algorithms are often compared with each other. In this work, we compare a regular computation of minimal unsat-core against minimum unsat-core. In addition, our focus is not to provide a novel way of computing minimal unsat-core. Instead, we makes use of MUSes to efficiently compute a set of support in a model necessary for inductive proofs.

Nadel, in \cite{nadel2010boosting}, discusses a 
number of applications of MUS extraction in formal verification. 
Gupta et al. \cite{gupta2003iterative} and McMillan and Amla \cite{mcmillan2003automatic} introduced the use of unsatisfiable cores in proof-based abstraction engines. Their goal is to shrink the abstraction size by omitting the parts of the design that are irrelevant to the proof of the property
under verification. However, \cite{gupta2003iterative, mcmillan2003automatic} do not consider core minimization. To our knowledge, none of the existing work
has used MUS to provide support information that explains
the correctness of proofs provided by different inductive techniques
including PDR and k-induction.
negative result.

\cite{torlak2008finding} proposes an algorithm for finding MUSes of declarative specification implemented for the Alloy language. Alloy is a framework for describing high-level design of various systems, whose analyzer is a fully automatic constraint solver. Constraints are translated into propositional logic solved by a SAT solver; hence, the analysis considers only a finite number of values for each type. For this reason, even for a set of simple constraints, the analyzer is never able to prove the correctness of a property. A major difference between this work and ours is that we 
extract UNSAT cores from an inductive proof over a sequential model involving lemmas. In addition, Alloy mostly works based on SAT solving, instead of SMT solving. In our implementation, JKind supports a variety of powerful SMT solvers (such as Z3, Yices, Yices2, etc.).

\begin{itemize}
    \item MUS's : checked
    \item Work on Alloy: checked
    \item Work that Teme pointed us to : will be added
    \item Anything else Elaheh has found : \%60 checked
\end{itemize}


\section{Inductive Validity Cores}
\label{sec:ivc}
\newcommand{\ivc}{\textit{IVC}\xspace}
\newcommand{\mivc}{\textit{MIVC}}
\newcommand{\aivc}{\textit{AIVC}}
\newcommand{\must}{\textit{MUST}}
\newcommand{\may}{\textit{MAY}}

\newcommand{\bq}{\textsc{BaseQuery}\xspace}
\newcommand{\iq}{\textsc{IndQuery}\xspace}
\newcommand{\fq}{\textsc{FullQuery}\xspace}

\newcommand{\mink}{\textsc{MinimizeK}\xspace}
\newcommand{\reduceinv}{\textsc{ReduceInvariants}\xspace}
\newcommand{\minivc}{\textsc{MinimizeIvc}\xspace}

\newcommand{\checksat}{\textsc{CheckSat}}
\newcommand{\isadeq}{\textsc{CheckAdq}}
\newcommand{\actlit}{\textsc{ActLit}}
\newcommand{\unsatcore}{\textsc{UnsatCore}\xspace}
\newcommand{\unsat}{\texttt{UNSAT}\xspace}
\newcommand{\sat}{\texttt{SAT}\xspace}

\newcommand{\getivc}{\textsc{GetIVC}}
\newcommand{\getmodel}{\textsc{GetLiteralsFromMaxModel}}
\newcommand{\aivcalg}{\texttt{\small{All\_IVCs}}}
\newcommand{\blockup}{\textsc{BlockUp}}
\newcommand{\blockdown}{\textsc{BlockDown}}
\newcommand{\mis}{\textit{MIS}}
\newcommand{\mcs}{\textit{MCS}}

Given a transition system that satisfies a safety property $P$, we
want to know which parts of the system are necessary for satisfying
the safety property. One possible way of asking this is, ``What is the
most general version of this transition system that still satisfies
the property?'' The answer is disappointing. The most general system is
$I(u) = P(u)$ and $T(u, u') = P(u')$, i.e., you start in any state
satisfying the property and can transition to any state that still
satisfies the property. This answer gives no insight into the original
system because it has no connection to the original system. In this
section we introduce the notion of {\em inductive validity core} (IVC)
which looks at generalizing the original transition system while
preserving a safety property.

We assume the transition relation has the structure of a top-level conjunction.  Given $T(u, u') = T_1(u, u') \land \cdots \land T_n(u, u')$ we will write $T = \bigwedge_{i=1..n}T_i$ for short.
By further abuse of notation,
$T$ is identified with the set of its top-level conjuncts. Thus, $T_i \in
T$ means that $T_i$ is a top-level conjunct of $T$, and $S
\subseteq T$ means all top-level conjuncts of $S$ are top-level
conjuncts of $T$. When a top-level conjunct $T_i$ is removed from $T$, we write $T \setminus \{T_i\}$. Such a transition system can easily encode our example model in Section~\ref{sec:example}, where each equation defines a conjunct within $T$ that we will denote by the variable assigned; so, $T = \{$ {\small \texttt{a1\_below, a2\_below, a1\_above, a2\_above, below, above\_hyst, doi\_on, d1, d2}} $\}$.

\begin{definition}{\emph{Inductive Validity Core (\ivc):}}
  \label{def:ivc}
  Let $(I, T)$ be a transition system and let $P$ be a
  safety property with $(I, T)\vdash P$.
  We say $S \subseteq T$ for $(I, T)\vdash P$ is an Inductive Validity Core,
  denoted by $\ivc(P, S)$, iff $(I, S) \vdash P $.
  When $I$, $T$, and $P$ can be inferred from
  context we will simply say $S$ is an inductive validity core.
\end{definition}

\begin{definition}{\emph{Minimal Inductive Validity Core (\mivc):}}
  \label{def:minimal-ivc}
  $S \subseteq T$ is a minimal Inductive Validity Core,
  denoted by $\mivc(P, S)$, iff ~
  $\ivc(P, S) \wedge \forall T_i \in S.~ (I, S\setminus\{ T_i \}) \nvdash P$.
\end{definition}

Note that, given $(I, T) \vdash P$, $P$ always has at least one \mivc, and it may also have many distinct {\mivc}s corresponding to different proof paths. To capture the latter, the \emph{all {\mivc}s ($\aivc$)} relation has been introduced in \cite{Murugesan16:renext}.
\begin{definition}{\emph{All {\mivc}s ($\aivc$):}}
    \label{def:allivcs}
    Given $(I, T) \vdash P$, $\aivc(P)$ is the set of all \mivc s for $P$:
    $$ \aivc(P) \equiv  \{\ S~|~S \subseteq T \land  MIVC(P, S)\} $$
\end{definition}

Inductive validity cores have the following monotonicity property.

\begin{lemma}
  \label{lem:ivc-monotonic}
  Let $(I, T)$ be a transition system and let $P$ be a safety property
  with $(I, T)\vdash P$. Let $S_1 \subseteq S_2 \subseteq T$. If $S_1$
  is an inductive validity core for $(I, T)\vdash P$ then $S_2$ is an
  inductive validity core for $(I, T)\vdash P$.
\end{lemma}
\begin{proof}
  From $S_1 \subseteq S_2$ we have $S_2 \Rightarrow S_1$. Thus the
  reachable states of $(I, S_2)$ are a subset of the reachable states
  of $(I, S_1)$.
\end{proof}

Fig.~\ref{fig:ivcs} illustrates these notions by a graphical representation of minimal IVCs for property $P = ({\small{\texttt{on\_p}}})$ in the example presented in Section~\ref{sec:example}. As shown in the picture, this property has two distinct \mivc s, which means the model satisfies $P$ in two different ways:  {\small \texttt{\{\{a1\_below, below, doi\_on\}, \{a2\_below, below, doi\_on\}\}}}, This is because in the implementation, the DOI is turned on when either of the altimeters is below the threshold, while our property states that they both must be below.
Note that there is a subset of model elements, $\{{\small \texttt{a1\_above, a2\_above, above\_hyst, d1, d2}}\}$, that does not show up in $\aivc(P)$. Elements in such a subset
do not affect the satisfaction of $P$.  For comparison, note that a backwards static slice starting from {\small{\texttt{on\_p}}} will include the entire model.
%In the complete ASW model in~\cite{HCW02:ase-deviation} there are additional properties that use these elements, but they are not necessary for the discussion in this paper.

\begin{figure}[t]
 \centering
  \includegraphics[width=0.9\columnwidth]{figs/ivcs.jpg}
  \vspace{-0.1in}
  \caption{Graphical representation of \mivc s for the model in Fig.~\ref{fig:asw}
  with  $P = ({\small \texttt{on\_p}})$}
  \label{fig:ivcs}
  %\vspace{-0.2in}
\end{figure}

%Distinct IVCs may have common elements, and the intersection of all \mivc s is called the \emph{must} set for $P$.

Generally, an IVC computation technique aims to determine, for any subset $S \subseteq T$, whether $P$ is provable by $S$. Then, a minimal subset that satisfies $P$ is seen as a minimal proof explanation called a minimal Inductive Validity Core.


\subsection{Algorithms for computing an inductive validity core}
%\label{subsec:ivcalg}
\input{ivcalg}



\subsection{Algorithm for computing all minimal inductive validity cores}
%\label{sec:allivcs}
\section{Method}
\label{sec:allivcs}
 
\newcommand{\ucalg}{IVC\_UC\xspace}  

\newcommand{\mink}{\textsc{MinimizeK}\xspace}
\newcommand{\reduceinv}{\textsc{ReduceInvariants}\xspace}
\newcommand{\minivc}{\textsc{MinimizeIvc}\xspace}

\newcommand{\checksat}{\textsc{CheckSat}\xspace}
\newcommand{\unsatcore}{\textsc{UnsatCore}\xspace}
\newcommand{\unsat}{\textsc{UNSAT}\xspace}
\newcommand{\sat}{\textsc{SAT}\xspace}

As mentioned, the contribution of this paper is to provide an efficient and complete method for calculating $AIVC$ of a property. To this end, we first introduce several additional notions and definitions, most of which are inspired by the MUS enumeration technique proposed in \cite{marco2016fast}.

\begin{definition}{\emph{Maximal Irrelevant Subset (MIS):}}
  \label{def:mis}
  $S \subset T$ for $(I, T) \vdash P$ is a Maximal Irrelevant Subset (MIS) iff 
  $(I, S) \nvdash P$ and $\forall T_i \in T\setminus S.~ (I, S\cup{T_i}) \vdash P$.  
\end{definition}

\begin{definition}{\emph{Minimal Correction Set (MCS):}}
  \label{def:mcs}
  $S \subset T$ for $(I, T) \vdash P$ is a Minimal Correction Set (MCS) iff
  $(I, T \setminus S) \nvdash P$ and $\forall T_i \in S.~ (I, (T \setminus S)\cup \{T_i\}) \vdash P$.
\end{definition}

%It should be mentioned that minimality and maximality are about minimum or maximum cardinality subsets. 
Note that $MCS$ is more of syntactic sugar that specifies sets that can also be specified by $MSS$; i.e. for any $MIS$ of $T$, there is a corresponding $MCS$ such that adding any element of that $MCS$ to the $MIS$, makes the property provable by the $MIS$. 
And, that's why it is called the ``minimal correction'' set. 

The technique for enumerating all IVCs is a generalization of exploring the power set of $T$ (denoted by $ \mathcal{P}(T) $).
Basically, the algorithm needs to explore the provability of a 
given property by any subset of $T$, which may sound impractical. 
However, using some facts about $IVC$s and $MIS$es we can have a complete
enumeration algorithm that only needs to explore a (small) subset of $T$ 
in order to calculate $AIVC$:
\begin{itemize}
  \item Given property $P$ and every subset $S$ of $\mathcal{P}(T)$, we have either $(I, S) \vdash P$ or $(I, S) \nvdash P$. In the former case, we say $S$ is \textbf{adequate} (to prove $P$); in the latter, we say that $S$ is \textbf{inadequate} (for the proof of P).
  \item If a given subset $S$ is adequate (inadequate), then all of its supersets (subsets) are (in)adequate as well. 
\end{itemize}

In exploring the power set, 
we define two functions $MAP$ and $\mathcal{A}$
from subsets to truth values. 
Function $\mathcal{A}$ tells whether or not property $P$ is provable by a subset of $T$:
$$\mathcal{A} : \mathcal{P}(T) \rightarrow \{true~(adequate), false~(inadequate)\}$$  
\noindent Function $MAP$ is defined to guide the exploration algorithm. 
If we consider the power set as a lattice in a Hasse diagram, 
$MAP$ tells us which parts of the lattice have been explored:
$$MAP: \mathcal{P}(T) \rightarrow \{true~(unexplored), false~(explored)\}$$
\noindent The core of an algorithm for computing $AIVC(P)$ is to choose an \emph{unexplored} subset $S$ of $\mathcal{P}(T)$ and examine whether or not $S$ is adequate for $P$. If so, then compute an $S' \subseteq S$ such that $IVC(P, S')$.

To choose 

\begin{definition}{\emph{Unexplored subset problem:}}
  \label{def:usp}
  
\end{definition}





 

%\begin{algorithm}[t]
%  $k' \leftarrow 1$ \\
%  \While{$\checksat(\neg\iq_{k'}(T, P, P)) = \sat$} {
%    $k' \leftarrow k' + 1$ \\
%    }
%  \Return{$k'$} \\
%\caption{$\mink(T, P)$}
%\label{alg:minimize-k}
%\end{algorithm}
 

%\begin{theorem}
%\label{thm:minimal-hard}
%\end{theorem}
%\begin{proof}
%\end{proof}


\section{Illustration}
\label{sec:illust}
To illustrate the \aivcalg ~algorithm we use the example presented in Section \ref{sec:example} with $P = (\small{\texttt{on\_p}})$ .
%As inspired by the MARCO algorithm in \cite{marco2016fast}, we visualize $\mathcal{P}(\mathcal{A})$ as a lattice in a Hasse diagram (Fig.~\ref{fig:lattice1}) to demonstrate the progress of the algorithm. As you can see, each level of the lattice contains sets with the same size linked to sets that are their immediate
%supersets (upper level regions) and subsets (lower level regions). Note that the power set is viewed as a Boolean formula, so each member in the lattice shows
% variable with $true$.
For better description, we view $T$ as an ordered set of its top-level conjuncts; i.e. $T = \{$ {\small \texttt{a1\_below, a2\_below, a1\_above, a2\_above, one\_below, both\_above, doi\_on, on\_p}} $\}$.
The algorithm starts with creating activation literals for each $T_i \in T$. Let the ordered set of Boolean variables $\{ a_1, \ldots , a_8 \}$ be the corresponding literals to the elements of $T$ (e.g. $\actlit (\small{\texttt{a1\_below}}) = a_1$ and $\actlit (\small{\texttt{on\_p}}) = a_8$). Then, line 3 initializes $map$ with $\top$.

In the first iteration of the \texttt{while} loop, since $map$ is
empty, it is satisfiable, and a model for it can be any subset of
literals. So obviously, the first maximal model of $map$ contains all
the literals, which means, in line~\ref{alg:aivc:assignm}, $M = \{a_1,
\ldots, a_8\}$, and in line~\ref{alg:aivc:assigns}, $S = T$. Since $S$
is adequate for $P$, the \getivc ~module is called in
line~\ref{alg:aivc:getivc}. Suppose the returned \mivc\ by this function
is $S' = \{ \small{ \texttt{a1\_below},~\texttt{one\_below},
  ~\texttt{doi\_on},~\texttt{on\_p}}\}$; this set is added to $A$ in
line~\ref{alg:aivc:addset}, and thus it comes to adding a new clause
to $map$ (line~\ref{alg:aivc:aadd}), which makes $map = (\neg a_1 \vee
\neg a_5 \vee \neg a_7 \vee \neg a_8)$. As discussed, this constraint
marks all the supersets of $S'$ as blocked and prunes them off the
search space.

For the second iteration, $map$ is still satisfiable,
so the algorithm gets to find a maximal model of it in line~\ref{alg:aivc:maxsat}. Suppose this time, the maximal model makes $M = \{a_2, \ldots, a_8\}$,
which leads to $S = T \setminus \{\small{\texttt{a1\_below}}\} $ in line~\ref{alg:aivc:assigns}.
Since this $S$ is adequate for $P$, \getivc ~computes a new \mivc\ in line~\ref{alg:aivc:getivc}.
Let the new \mivc\ be $S' = \{ \small{ \texttt{a2\_below},~\texttt{one\_below}, ~\texttt{doi\_on},~\texttt{on\_p}}\}$; after adding this set to $A$,
it is time to constrain $map$ by a new clause in line~\ref{alg:aivc:addset},
which results in $map \leftarrow map \wedge (\neg a_2 \vee \neg a_5 \vee \neg a_7 \vee \neg a_8)$.

After two iterations, $map$ is still satisfiable.\footnote{Note that the algorithm terminates when all $IVC$s and $MIS$es are explored. Up to now, only $IVC$s have been explored.} The maximal model returned is now $M = T \setminus \{ \small{ \texttt{a1\_below},~\texttt{a2\_below}}\}$ in line~\ref{alg:aivc:assignm}.
In this case, $M$ is inadequate, so we get into line~\ref{alg:aivc:iadd} updating $map$ as
$map \leftarrow map \wedge (a_1 \vee a_2)$. Note that by adding this new clause to $map$,
all the subsets of $T \setminus \{ \small{ \texttt{a1\_below},~\texttt{a\_2below}}\}$
are removed from the search space.

The algorithm continues similar to the third iteration leading to $M$ (in line~\ref{alg:aivc:assignm}) and $map$ (in line~\ref{alg:aivc:iadd}) to be as follows:
\mike{Elaheh: Fill these in correctly!}
\begin{itemize}
  \item Iteration 4:  $M = T \setminus \{ \small{ \texttt{alarm}}\}$, $map \leftarrow map \wedge a_4$

  \item Iteration 5: $M = T \setminus \{ \small{ \texttt{p1},~\texttt{p2}}\}$, $map \leftarrow map \wedge (a_6 \vee a_7)$

  \item Iteration 6: $M = T \setminus \{ \small{ \texttt{doi\_on}}\}$, $map \leftarrow map \wedge a_5$

  \item Iteration 7: $M = T \setminus \{ \small{ \texttt{below}}\}$, $map \leftarrow map \wedge a_3$

  \item Iteration 8: $M = T \setminus \{ \small{ \texttt{a1\_below},~\texttt{p2}}\}$, $map \leftarrow map \wedge (a_1 \vee a_7)$

  \item Iteration 9: $M = T \setminus \{ \small{ \texttt{a2\_below},~\texttt{p1}}\}$, $map \leftarrow map \wedge (a_2 \vee a_6)$
\end{itemize}
Finally, after the ninth iteration, $map$ becomes \unsat and the algorithm terminates.
Note that $MIS$es and $IVC$s may be discovered in different orders from what explained here. The order by which sets are explored is
quite dependent on the maximal model returned in line~\ref{alg:aivc:maxsat} as well as the $IVC$ returned in line~\ref{alg:aivc:getivc} because there could be several distinct maximal models ($MIS$es) and $IVC$s. For this example with a $|T| = 8$ and $|\mathcal{P}(T)| = 2^8$, a brute force approach of power set exploration needs to look into  256 cases. However, the \aivcalg ~algorithm only explored 9 cases to cover the entire power set. Even so, depending on the order by which $IVC$s and $MIS$es are discovered, the total cases to explore by the algorithm may change.
% All in all, it is fair to say the \aivcalg ~algorithm is  linear in the size of $T$.
%This issue could affect the performance of the algorithm as well.



\section{Implementation}
\label{sec:impl}

The algorithm for efficiently computing IVCs can be found in a forthcoming FSE paper~\cite{Ghass16} and is implemented in the JKind \cite{jkind}, which is an infinite-state model checker for safety properties using multiple cooperative engines in parallel (such as k-induction and PDR). JKind accepts
Lustre programs written over the theory of linear integer and real
arithmetic. In the back-end, JKind uses an SMT solver such as
Z3, Yices, MathSAT, or SMTInterpol.
JKind works on multiple properties simultaneously. When a
property is proven and IVC generation is enabled, an additional
parallel engine executes the IVC generation algorithm to compute a minimal
IVC. We demonstrated the efficiency and precision of the approach using a set of Lustre models developed
as a benchmark suite for~\cite{Hagen08:FMCAD}, augmented with additional models from industrial projects (~\cite{QFCS15:backes,hilt2013}). The results show that our algorithm for computing IVCs is quite efficient even for industrial models with an average overhead of ~10\%. 
%
\section{Experiment}
\label{sec:experiment}

%\mike{What do we want to call our efficient algorithm: IVC?}

We would like to investigate both the {\em efficiency} and {\em
  minimality} of our three algorithms: the n{\"a}ive brute-force
algorithm (\bfalg), the UNSAT core-based algorithm (\ucalg), and the
combined UNSAT core followed by brute-force minimization algorithm
(\ucbfalg). Efficiency is computed in terms of wall-clock time: how
much overhead does the IVC algorithm introduce? Minimality is
determined by the size of the IVC: cores with a smaller number of
variables are preferred to cores with a larger number of variables.
Finally, we are interested in the {\em diversity} of solutions: how
often do different tools/algorithms generate different minimal IVCs?

The use of JKind allows additional dimensions to our investigation: it supports two different inductive algorithms: $k$-induction and PDR, and a ``fastest'' mode, that runs both algorithms in parallel.  In addition, JKind supports multiple back-end SMT solvers including Z3~\cite{DeMoura08:z3}, Yices~\cite{Dutertre06:yices}, MathSAT~\cite{Cimatti2013:MathSAT}, and SMTInterpol~\cite{Christ2012:SMTInterpol}.  We would like to determine whether the choice of inductive algorithm affects the size of the IVC, whether different solvers are more or less efficient at producing IVCs, and whether running different solvers/algorithms leads to {\em diversity} of IVC solutions.

Therefore, we investigate the following research questions:
\begin{itemize}
    \item \textbf{RQ1:} How expensive is it to compute inductive validity cores using the \bfalg, \ucalg, and \ucbfalg algorithms?
    \item \textbf{RQ2:} How close to minimal are the support sets computed by \ucalg as opposed to the (guaranteed minimal) \ucbfalg?  How do the sizes of IVCs compare to static slices of the model?
    \item \textbf{RQ3:} How much {\em diversity} exists in the solutions produced by different solver/induction algorithm configurations?
\end{itemize}

\subsection{Experimental Setup}
In this study, we started from a suite of 700 Lustre models developed
as a benchmark suite for~\cite{Hagen08:FMCAD}. We augmented this suite
with 82 additional models from recent verification projects including
avionics and medical devices~\cite{QFCS15:backes,hilt2013}. Most of
the benchmark models from~\cite{Hagen08:FMCAD} are small (10k or less,
with 6-40 equations) and contain a range of hardware benchmarks and
software problems involving counters. The additional models are much
larger: around 80k with over 300 equations. We added the new
benchmarks to better check the scalability for the tools, especially
with respect to the brute force algorithm.
%
%\mike{MORE HERE...stats on size, reasons for add'l models.}
Each benchmark model has a single property to analyze.  For our purposes, we are only interested in models with a {\em valid} property (though it is perhaps worth noting that there is no additional computation---and thus no overhead---using the JKind IVC options for {\em invalid} properties).  In our benchmark set, 295 models yield counterexamples, and 10 additional models are neither provable nor yield counterexamples in our test configuration (see next paragraph for configuration information).  The benchmark suite therefore contains 476 models with valid properties, which we use as our test subjects.

For each test model, we computed \ucalg in 12+1 configurations: the
twelve configurations were the cross product of all solvers \{Z3,
Yices, MathSAT, SMTInterpol\} and inductive algorithms
\{$k$-induction, PDR, fastest\}, and the remaining (+1) configuration
was an instance of \bfalg run on Yices, which is the default solver in
JKind. In addition, for each of the 12 configurations, we ran an
instance of JKind without IVC to examine overhead. The experiments
were run on an Intel(R) i5-2430M, 2.40GHz, 4GB memory machine, with a
1 hour timeout for each analysis on any model. The data gathered for
each configuration of each model included the time required to check
the model without IVC, with IVC, and also the set of elements in the
computed IVC.\footnote{The benchmarks, all raw experimental results,
  and computed data are available on \cite{expr}.}

Note that not all analysis problems were solvable with all algorithms: for all solvers, $k$-induction (without IVC) was unable to solve 172 of the examples.  When comparing minimality of different solving algorithms, we only considered cases where both algorithms provided a solution (as will be discussed in more detail in Section~\ref{sec:minimality}).

\iffalse
\begin{itemize}
    \item an algorithm to compute a truly minimal set of support, i.e. \texttt{JSupport}.
    \item given a LUS model, a static crawler which automatically marks all equations of a node in the initial support set of a property.
    \item some trackers that measure the verification time with/ without support computation.
   % \item some minor changes in the XML writers.
\end{itemize}

\mike{My thoughts on this section: mostly, it needs more structure: more information on the properties of the models: size, provenance, etc., a broken out subsection on the description of the experimental setup, etc}

\mike{I think we want to split out the results in another top-level section}

Experiment:
\begin{itemize}
    \item (Overview) describe research questions and goals.
    \item Experimental setup: tell me about the models: how many, how big are they?  Then, tell me about the experiment: the tool configurations, the machine used for test.
    \item Data generation: Describe what you measured for each model analysis.
\end{itemize}
\fi


%%  LocalWords:  minimality ive UNSAT IVC Minimality IVCs PDR Yices
%%  LocalWords:  MathSAT SMTInterpol RQ JSupport


\section{Applications}
\label{sec:apps}
As described in Section \ref{sec:intro}, IVCs can facilitate several engineering tasks in different phases of a system development process.  In this section, we focus on two of the most important: traceability and coverage.

\iffalse
\fi

\subsection{Automated Proof-Based Traceability}

\emph{Requirements traceability} can be defined as \\
\begin{quotation}
\textit{``the ability to describe and follow the life of a requirement, in both forwards and backwards direction (i.e., from its origins, through its development and specification, to its subsequent deployment and use, and through all periods of on-going refinement and iteration in any of these phases).''}~\cite{gotel}. \\
\end{quotation}

%\mike{This is really jarring...we need to more immediately place this into our framework}

Traceability is concerned with establishing relationships, called \emph{trace links}, between the requirements and one or more artifacts (design elements) of the system.
Among the several different development artifacts and the relationships that be can established from/to the requirements, being able to establish trace links from requirements to artifacts that realize or \emph{satisfy} those requirements---particularly to entities within those artifacts called \emph{target artifacts}~\cite{gotel2012traceability}---has been enormously useful in practice.
For instance, it helps analyze the impact of changes in one artifact on the other, assess the quality of the system, aid in creating assurance arguments for the system, etc.

There is substantial interest within the Requirements Engineering research community towards automating the construction and maintenance of traceability links~\cite{hayes2003improving, egyed2002automating,cleland2007best}.
%In fact our work on IVCs was originally driven by the goal of automatically generating a subset of these trace links.
There are many kinds of trace links that may have to do with functional correctness, performance, architectural qualities, user understanding, and many other criteria.
%
%To that end, there are repositories such as the Data sets published at  Center of Excellence for Software Traceability~\cite{COEST} containing many example systems, each with a reasonably complete set of requirements and target artifacts and with trace links constructed by groups of experts.  It is then possible to benchmark automated and semi-automated traceability approaches against vetted sets of trace links.
%
We focus our attention to this subset of requirement traceability called {\em Satisfaction Arguments}~\cite{Hammond01:WiW} that are used to determine the portions of a design or model that are necessary to satisfy a functional requirement.  IVCs automatically provide such arguments accurately and without human effort.  In addition, we can automatically generate expected artifact types, such as traceability matrices for these kind of relationships (see Figures~\ref{fig:propertyset1} and \ref{fig:propertyset4}).  

%It is also the case, when computing all IVCs, that we can provide additional insight.  As far as we are aware, none of the existing Satisfaction Argument literature discusses the issue that there are often multiple satisfaction arguments between a requirement and its implementation.  Given all IVCs, it is possible to perform more accurate impact analysis and define multiple notions of requirements adequacy, as we will see in the following sections.

\iffalse
This has important ramifications for other forms of automatic trace link generation as pursued by the requirements engineering community.
For traceability research, the standard measures for examining the performance of different approaches is in terms of {\em precision} and {\em recall} against the ``gold standard'' set of traceability links that exist in predefined repositories.  Our concern is that, for requirements satisfaction traceability, there are often many such sets of valid links, as we have explored in this paper, so these metrics may be misleading.  One can envision situations in which the gold standard pursues one set of support and the automated approach pursues another, leading to low precision and recall scores.  A close examination of traceability links and categorizations such as the ones we have explored may be useful to provide more accurate measurements of the quality of automated approaches.
\fi


\subsection{Coverage Analysis and Requirements Completeness}

For critical systems, it has been argued that formal methods
%, involving formalized requirements and proofs of implementation correctness,
should be applied to gain higher assurance than is possible with testing~\cite{Miller10:CACM,Rushby09:SEFM,Hardin09:Security}.  For these approaches, testing may still be performed, but the verification effort is primarily focused on performing proofs.  Unfortunately, proof-based approaches tend not to answer the question as to whether implementations have {\em additional functionality} that is not covered by requirements.  %Testing, despite its faults, can measure {\em structural coverage} to find untested functionality and can find some errors by {\em serendipity}, in which problems not directly related to the requirement under test are exposed.  Therefore, in formal verification approaches, it is even more important that requirements be complete.
%
Thus we are interested in notions of {\em coverage}: determining how well the requirements characterize the implementation model.
The goal of a {\em coverage metric} is usually to assign a numeric score that describes how well properties cover the design.

Relatively recently, techniques have been devised for analyzing completeness of requirements against formal implementation models, specified as transition systems or Kripke structures \cite{chockler2001practical,das2005formal, claessen2007coverage, grosse2007estimating,chockler_coverage_2003,chockler2010coverage,
Kupferman:2006:SCF,kupferman_theory_2008}.  These models are agnostic to the abstraction level of the implementation: implementations can be lower-level requirements, software architectures, or concrete implementations of system behavior.  The mechanism used is based on {\em mutation} and {\em proof}: is it possible to prove that the requirements still hold of the system after mutating the model in some way?  If so, then the requirements are incomplete with respect to that model element.


\iffalse
Mutations are ``atomic'' changes to the design, where the set of possible mutations depends on the notation that is used.  A mutant is ``killed'' if one of the properties that is satisfied by the original design is violated by the mutated design~\cite{chockler_coverage_2003,chockler2001practical,chockler2010coverage,Kupferman:2006:SCF,kupferman_theory_2008}.  There are many different kinds of mutations that have been proposed, primarily focused on checking sequential bit-level hardware designs.
For these designs, {\em State-based} mutations flip the value of one of the bits in the state.  There are several variations depending on whether the flip is performed on a single state within a Kripke structure~\cite{hoskote1999coverage}, or in the description of the signal in the transition relation of the circuit~\cite{chockler2001practical}.  {\em Logic-based} mutations fix the value of a bit to constant zero or one, and can be used to determine whether properties can find stuck-at faults.  {\em Syntactic} mutations~\cite{chockler_coverage_2003} remove states in a control flow graph representation of hardware.
Similarly, for software, it is possible to apply any of the ``standard'' source code mutation operators used for software testing~\cite{Andrews06:mutation} towards requirements coverage analysis.
Some examples of software mutations are \cite{Budd:1980}:
\begin{enumerate}
    \item Replace an integer constant $C$ by one of $\{0, 1, -1, C + 1, C - 1\}$,
    \item Replace an arithmetic, relational, logical, bitwise logical, increment/decrement, or arithmetic-assignment operator by another operator from the same class,
    \item Negate the decision in an \texttt{if} or \texttt{while} statement,
    \item Delete a statement.
\end{enumerate}

We assume each element $T_i \in T$ has a set of possible mutations associated with it.  Depending on the modeling formalism used, this may be the value of a gate or signal or an expression within a statement in a program.  We will further assume the existence of a mutation function $f_{m}$ that, given a model element, will return a finite set of mutations for that element.  We can then define the set of mutant models $M$ as follows:
\[
    M = \{ (T \setminus \{T_i\}) \cup \{m\} \ |\ T_i \in T, m \in f_{m}(T_i) \}
\]
\noindent and then define the mutation score for property $P$ in the standard way:
\begin{definition} {\emph{Generalized mutation coverage.} } \\
\[
   \mutcov = \frac{ | \{m~|~ m \in M~\land~(I, m) \nvdash P\} |}{|M|}
\]
\end{definition}

\noindent Note that without loss of generality, we consider a single property $P$, which can be viewed as the conjunction of all the properties of the model.

The state of the art of mutation-based coverage can be found in Chockler \textit{et al.} \cite{chockler2010coverage}, where a design is considered as a net-list with nodes of types {\small \texttt{\{AND, INVERTER, REGISTER, INPUT\}}}.
Each mutant design changes the type of a single node to {\small \texttt{INPUT}}. When property $\phi$ satisfied by the original net-list fails on the mutant design, it is said that a mutant is discovered for $\phi$. Then, the coverage metric for $\phi$ is defined as the fraction of the discovered mutants, based on which the coverage of a set of properties is measured as the fraction of mutants discovered by at least one property.
To decrease the cost of computation, coverage analysis is performed at several stages; first, all the nodes that do not appear in the resolution proof of a given property are marked as \emph{not-covered}, and the rest of the nodes are marked as \emph{unknown}. Then, for the unknown nodes, the basic mutation check is performed: if a corresponding mutant design violates the property, it will be considered as \emph{covered}.
\fi

Unfortunately, previous metrics based on mutation are expensive to compute, as they involve running many different verification efforts against mutant models.  For example, given a mutant generator developed for testing Lustre programs~\cite{jkind}, the 50 largest models in the benchmark suite each had more than 30,000 mutants.  While it is possible to approximate mutation coverage scores using sampling~\cite{Zhang13:sampling}, samples still often require 5\% or more of the possible mutants.  One can also restrict the set of possible mutations, but this runs the risk of biasing the program to miss certain kinds of faults.

Thus, we propose a new family of metrics for measuring property completeness based on proofs and \mivc s rather than mutation.  These metrics are based on the notions of \mivc s and MAY and MUST elements introduced in the previous section.  The metrics have benefits in that they are cheaper to compute than previous metrics (especially the \mivc\ metric) and have a formal basis provided by proofs.

For the purposes of this discussion of coverage, without loss of generality, we assume that the analysis is performed over a single property $P$ (which may represent the conjunction of all of the properties for the model).

%they can {\em underapproximate} the portion of the model necessary for proof
%can {\em underapproximate} which portions of an implementation model are necessary to produce a proof.
%completeness metrics can {\em underapproximate} which portions of a program are necessary to fulfill the requirements.  That is, if we construct a model consisting of only the required model elements as determined by the analysis, it is often no longer possible to prove the requirement.  Thus the feedback provided to the developer may be somewhat misleading.  In addition, the mutation-based analyses tend to be very computationally expensive.  For example, for model checkers, state of the art techniques have runtimes of (in the best case) several times more than is required for proof~\cite{chockler2010coverage}.


%\footnote{Section~\ref{sec:impl} describes how these proofs are discovered in practice.}
%\subsection{Coverage and Minimal Proofs}
%Alternatively, we can consider using the proofs themselves as a mechanism for determining adequacy of requirements.

\begin{definition} {\emph{IVC coverage (\ivccov):}} \\
\label{def:coverage-justi}
Given $S \in \aivc(P)$, $T_i$ is covered by $P$ via $S$ \emph{iff} $T_i \in S$.
\end{definition}

%We call Definition \ref{def:coverage-justi} a \emph{proof-preserving} metric because, with a set of the model elements marked as covered by \ivccov ,
%$P$ is provable.  %Other notions, as will be discussed,
%may yield subsets of the model that are insufficient to
%reconstruct the proof of the property.
%\footnote{\noindent ~Throughout the paper, when a coverage metric is justifiable, like \ivccov, we say that it preserves provability of the property.}
%Thus, the coverage score for \ivccov\ is often higher than the score for \nondetcov.
The coverage score for \ivccov\ can be calculated with: $$\frac{|S|}{|T|}$$  where $|T|$ is the number of conjuncts in $T$.
%Note that because minimal proofs are not unique, there are several possible coverage scores.
Because $P$ may have multiple \mivc s,
  \ivccov\ metric can lead to various scores that belong to the following set:
\[
\left\{~\frac{ |S|}{|T|}~\bigg|~S \in AIVC(P)~\right\}
\]

\noindent Note that if an \mivc ~contains all model elements (i.e., the model is {\em completely covered}), then there is only one possible \mivc , so in this case there is no diversity of scores. On the other hand, a set of states and signals can be considered covered or not covered depending on a particular \mivc\ we consider for coverage evaluation. To address this issue, we introduce additional coverage metrics using the notions of \may\ and \must.

%Since the primary goal of
% this paper has been to provide a complementary coverage notion in
%  formal verification, it is worth exploring other possible notions based on the idea of provability and $\aivc$, which is beneficial, as with testing, because if a coverage notion is an over-approximation, when the coverage
% is high, it does not necessarily mean the quality of
% the specification (or test suite) is high, or when it is an under-approximation, a low coverage score does not always mean the specification is of poor quality.

\begin{definition} {\emph{(\maycov):}}
  \label{def:comp-1}
 $T_i \in T$ is covered by $P$ \emph{iff} $T_i \in \maycov (P)$, where
   $\maycov (P) = \{T_i ~|~ \exists S \in \aivc(P)~.~T_i \in S \}$.
\end{definition}

\begin{definition} {\emph{(\mustcov):}}
  \label{def:mustcov}
 $T_i \in T$ is covered by $P$ \emph{iff} $T_i \in \mustcov (P)$, where
   $\mustcov (P) = \{T_i ~|~ \forall S \in \aivc(P)~.~T_i \in S \}$.
\end{definition}

The $\maycov$ notion considers a model element covered if it is found in any \mivc ; thus it is a weaker notion of coverage for models with multiple \mivc s, but has the benefit of being uniquely defined.  \mustcov\ takes the opposite view, considering a model element as covered only if it affects all the proofs of $P$.

\iffalse
Algorithm \ref{alg:must} is also an efficient way of computing the \emph{must} set of a given property using \ucalg. A different algorithm for computing $\must (P)$ is to first compute $\aivc (P)$ and then take the intersection of all sets in $\aivc (P)$.


\begin{algorithm}
  \SetKwInOut{Input}{input}
  \SetKwInOut{Output}{output}
  \Input{$(I, T) \vdash P$}
  \Output{Must set for $(I, T) \vdash P$}
  \BlankLine
  $S \leftarrow \ucalg((I, T) \vdash P)$ \\
  $M \leftarrow \varnothing$ \\
  \For{$x \in S$} {
    \If{$(I, T\setminus\{x\}) \nvdash P$}{
      $M = M \cup \{x\}$
    }
  }
  \Return{M}
\caption{\mustalg: an algorithm to compute $\must(P)$ for a given $P$}
\label{alg:must}
\end{algorithm}
\fi

\iffalse
It is still possible to build more relaxed coverage metrics in which coverage
is captured by looking at individual properties, rather than their conjunction.
%for example, in the definition of \ivccov , it is wise to look at $P$ as
%the conjunction of all properties. However,
We can, for example, describe a metric in which any element used by an \mivc ~for any property is considered covered.
%with this view,
%elements around IVCs that do not have common \emph{must}
%elements with others will be treated as uncovered while they are at least covered by one
% IVC of an individual property in the specification.
%
The next definition, \allcov, formalizes this notion.
\begin{definition} {\emph{(\allcov):}}
  \label{def:comp-2}
     Given a set of properties $\Delta$ over $T$, $T_i \in T$ is covered
   \emph{iff} $T_i \in \allcov (T)$, where
   $\allcov (T) = \{T_i ~|~ \exists P \in \Delta ,~ S \in \aivc(P).~T_i \in S \}$.
\end{definition}

\fi

\iffalse
Based on the categorization of elements, we will state a relationship about \mivc s in order to compare different proof-based metrics proposed earlier.

\begin{lemma}
  \label{lem:must-not-enough}
  If $\may(P) \neq \varnothing$, then $P$ is not provable by $\must(P)$.
\end{lemma}
\begin{proof}
  $\may(P) \neq \varnothing \Rightarrow  \exists T_i \in \may(P).$
$T_i \in \bigcup \aivc(P) \wedge T_i \notin \must(P)$,
which implies $\exists S \in \aivc(P).~ T_i \in S$.
Considering the fact that $S$ is minimal and
$\must(P) \subset S$ (since $T_i \in S \wedge T_i \notin \must(P)$),
 $\nexists S' \subset S.~ (I,S') \vdash P$,  which means $(I, \must(P)) \nvdash P$.
\end{proof}
\vspace{2mm}

\fi

%\begin{lemma}
%    \label{lem:must-mustcov}
%    $T_i \in \must(P) \Leftrightarrow T_i \in \mustcov(P)$
%\end{lemma}
%\begin{proof}
%Immediate from the definition of $\must$ and \mustcov.
%\end{proof}

\iffalse
Now we focus on the relationship between non-deterministic mutation-based coverage and proof-based metrics. In Chockler et. al.
~\cite{chockler2010coverage},
each mutant design changes the type of a single node to an input node .
Given a suitable encoding of the netlist, assigning a ``fresh'' input is an isomorphic operation to simply removing a $T_i$ from $T$. The mapping is as follows: the net-list becomes a conjunction
of equations, where each vertex becomes a variable $v_i \in U$, and where each non-input vertex becomes an assignment equation $T_i \in T$.
For example, given an AND-vertex $v_i$ with three input edges from other vertexes $\{v_a, v_b, v_c\}$, we would define an equation $T_i \in T$ of the form $(v_i = (v_a \wedge v_b \wedge v_c))$.
%
%As the variable is no longer constrained by a defining equation, it is effectively an %input.

Given this encoding, we can reframe the non-deterministic coverage proposed in \cite{chockler2010coverage} as follows:

\begin{definition} {\emph{Nondeterministic coverage (alternate specification) (\nondetcovalt) ~\cite{chockler2010coverage}.} }
\label{def:non-det-2}
$T_i \in T$ is covered by property $P$ \emph{iff} $T_i \in \nondetcovalt (P)$, where
$\nondetcovalt (P) = \{T_i~|~ (I, T) \vdash P \wedge (I, T \setminus \{T_i\}) \nvdash P\}$.
\end{definition}
\noindent Given this definition, it becomes straightforward to define some additional properties.

\begin{lemma}
  \label{lem:must-coverage}
$T_i \in \nondetcovalt (P) \Leftrightarrow T_i \in \mustcov(P)$.
\end{lemma}
\begin{proof}
$T_i \in \nondetcovalt (P)$ means that $(I, T \setminus \{ T_i \}) \nvdash P$ then
%$T_i$ is necessary to prove $P$,  which means
$\forall S \subset T .~ T_i \notin S \Rightarrow (I, S) \nvdash P$.
Therefore, since $(I, T) \vdash P$, $T_i \in \bigcap \aivc(P)$, which means  $T_i \in \must(P)$.
On the other hand, let $T_i \in \must(P)$; then $\forall S \in \aivc(P).~ T_i \in S$.
By definition, any proof of $P$ is a superset of some minimal IVC in $\aivc(P)$.
Thus, any subset $S$ of $T$ leading to proof contains $T_i$.
Therefore, $T \setminus \{ T_i \}$ does not lead to a proof.

\end{proof}
\vspace{2mm}

In light of Lemma \ref{lem:must-coverage}, the \nondetcovalt\ coverage score of specification $P$ can be also calculated by
$$\frac{|\must(P)|}{|T|}$$
%Therefore, for set of properties $\Delta$, the coverage score is computed by $$\frac{|\must(\Pi)|}{|T|},\quad  \Pi= \bigwedge_{i} {P_i \in \Delta}$$


%\mike{after all metrics presented, contrast them on the example.  Introduce the properties HERE and then discuss the coverage sets}
%
%\mike{Then, you can talk about justification, etc.}
\begin{corollary}
\label{cor:must-not-provable}
\nondetcovalt\ is not proof-preserving.
\end{corollary}
\begin{proof}
Immediate from Lemma \ref{lem:must-not-enough} and Lemma \ref{lem:must-coverage}
\end{proof}
\vspace{2mm}
\begin{corollary}
\label{cor:ivc-provable}
\ivccov\ is proof-preserving.
\end{corollary}
\begin{proof}
Immediate from Definition~\ref{def:minimal-ivc} and Definition \ref{def:coverage-justi}
\end{proof}
\vspace{2mm}

%It should be pointed out that \ivccov\ is accurate meaning that it does not result in false positives. In other words, since IVCs are \emph{minimal}, \ivccov\ does not mark
%any \emph{actual} uncovered element as covered.
\fi

\iffalse
Figure~\ref{fig:runtimeall} allows a visualization of the runtime of different coverage analyses
in comparison with the proof time, which indicates the overhead induced by each algorithm.
As can be seen, it is computationally cheap to find an
approximately minimal IVC using the algorithm \ucalg; however, finding a {\em guaranteed}
minimal IVC using the \ucbfalg\ algorithm is computationally expensive. The overhead of the \ucalg\ algorithm is on average 31\% over the baseline proof, as opposed to 2276\% for the \ucbfalg\ algorithm.
Therefore, in order to compute \ivccov, it is much more efficient to use \ucalg\ rather than the \ucbfalg\ algorithm.
In terms of comparing cost of coverage computation from \ivccov\ and \mustcov ,
the \mustcov\ computation imposes an average 4183\% runtime overhead on the verification time.
\fi

\iffalse
\begin{figure}
  \centering
  \includegraphics[width=\columnwidth]{figs/timing_cv.jpg}
  %\vspace{-0.2in}
  \caption{Runtime of different analyses}\label{fig:runtimeall}
\end{figure}
\fi

\begin{figure}
  \centering
  \includegraphics[width=\columnwidth]{figs/cv_size.jpg}
  %\vspace{-0.2in}
  \caption{Size of the set of covered elements by different algorithms}\label{fig:cvsize}
\end{figure}

When a coverage metric brings about lower coverage scores on average,
we say that the metric is harder to satisfy. To study this aspect of the proposed metrics, we first calculated the size of the output sets generated by each algorithm: on average, the ratio of the size of the sets generated by \ucalg\ to the size of the ones obtained from \ucbfalg\ is 1.08,
while this ratio for \mustalg\ to \ucbfalg\ is 0.93, which shows \mustalg\ is harder to satisfy.

Figure \ref{fig:cvsize} is a visualization of the size of the set of covered elements by different algorithms. Models over the x-axis are sorted based on the size of the minimal IVCs obtained from the \ucbfalg\ algorithm.  The graph shows the degree of under-approximation of a minimal proof set by \mustcov\ as well as the degree of over-approximation by \ucalg\ and \maycov.  The proposed coverage metrics can be ranked in terms of their scores as follows:
$$\mustcov \leq \ivccov\ \leq \maycov\ $$
\ivccov\ and \mustcov\ are equivalent when all elements within the model are covered.  If all model elements are part of an \mivc, then there is only one \mivc\ possible and all elements are \must elements.

%The equivalence of \mustcov\ and \nondetcovalt\ allows us to compare our algorithms against state-of-the-art mutation based coverage.


%For many analysis problems, this may be accurate enough to

%, which makes \ucalg a reasonable choice for computing \ivccov ~(rather than using \ucbfalg ).
%Therefore, minimality does not dramatically
%affect the coverage scores when \ivccov\ is computed by the \ucalg\ rather than \ucbfalg.
%However, \ucalg might report some elements as covered, while they are not because of the minimality issue.
%And, \mustalg reports some elements uncovered, while they are because it is not able to find \emph{may} elements.

Table~\ref{tab:cov-score} describes the aggregate of the coverage scores returned by the analyses.  Across all benchmarks, the min and max coverage scores are the same, and as expected, the average number of elements required is smallest for the \mustalg\ algorithm and largest the for \ucalg\ algorithm. 
At first glance, you might expect the average of coverage scores obtained from \maycov\ to be higher than the scores computed by \ivccov . However, this is not quite accurate; most of the models in our benchmark have only one \mivc . The \ivccov\ score for these models are calculated from the output of the \ucalg\ algorithm, which is not necessarily minimal. However, \maycov\ coverage scores are calculated from the output of \aivcalg\ algorithm, where minimality is guaranteed. Therefore, for the models with \emph{one} \mivc , sometimes \maycov\ yields lower coverage scores as apposed to \ivccov , which balances out the average scores by these metrics (Table \ref{tab:cov-score}).



\begin{table}
  \caption{Coverage scores of different algorithms across all models}
  \centering
  \begin{tabular}{ |c||c|c|c|c| }
    \hline
     score & min & max & mean & stddev \\[0.5ex]
    \hline\hline
    \small{\ivccov}\ with \ucalg &   0.002  & 1.0  & 0.475 & 0.302 \\[0.5ex]
    \small{\ivccov}\ with \ucbfalg&  0.002 & 1.0 &  0.445 & 0.291 \\[0.5ex]
    \mustcov & 0.002 & 1.0 &  0.417 & 0.291 \\[0.5ex]
    \maycov& 0.002 & 1.0 &  0.476 & 0.301 \\[0.5ex]
    \hline
  \end{tabular}
  \label{tab:cov-score}
\end{table}


\iffalse
To investigate the relationship between provability and different coverage notions,
we were interested in the number of models in the benchmark for which
\mustalg\ resulted in the sets not equal to an MIVC (i.e. models for which
\mustalg\ did not preserve provability).
Obviously properties are provable by 100\% of the IVCs computed by \ucalg\ (and \ucbfalg).
As for the \mustalg\ algorithm, the properties of 290 models in the benchmarks were not provable by the output of \mustalg. In practice, for larger models, \mustcov\ is more likely not to maintain provability,
 and since more than half of the models are small, 43\% may still not reveal the actual degree
 to which \mustcov\ underapproximates the covered parts of a model.
  The notion of proof preservation is appealing because it allows a concrete demonstration to the user of the irrelevance of portions of the implementation.  The IVC coverage notion also allows, in cases where there are multiple minimal satisfying sets, insight on multiple ways by which the model meets a requirement.
\fi

%To conclude this section, we should mention that one can define many more proof-based coverage metrics based on the $\mivc$s and $\aivc$s.  Metrics that make use of the $\aivc$ relation are computationally more expensive than \ivccov.


\iffalse
The size of sets computed by \ucalg\ is very close to the size
of the ones obtained from \ucbfalg, especially for larger models.  The average increase in size of IVCs returned by \ucalg\ is approximately 8\% of the \ucbfalg\ algorithm.  Since the overhead of producing \ucalg\ is only approximately 31\% more expensive than the baseline analysis, this test may be efficient enough to run as a standard part of the model checking process.  %If guaranteed minimality is required, the \ucbfalg\ can be used, but
\fi




%\subsection{Optimizing Logic Synthesis}

\input{synthesis}


\subsection{Illustration of Traceability and Coverage}
\label{sec:disc}


\newcommand{\allp}{\texttt{all\_p}}
\newcommand{\onp}{\texttt{on\_p}}
\newcommand{\offp}{\texttt{off\_p}}
\newcommand{\hystp}{\texttt{hyst\_p}}
\newcommand{\aonebelow}{\texttt{a1\_below}}
\newcommand{\atwobelow}{\texttt{a2\_below}}
\newcommand{\aoneabove}{\texttt{a1\_above}}
\newcommand{\atwoabove}{\texttt{a2\_above}}
\newcommand{\doion}{\texttt{doi\_on}}
\newcommand{\done}{\texttt{d1}}
\newcommand{\dtwo}{\texttt{d2}}
\newcommand{\abovehyst}{\texttt{above\_hyst}}
\newcommand{\inhibit}{\texttt{inhibit}}

In this section we show how traceability and coverage analyses can be performed using IVCs, attempting to both illustrate the techniques and discuss the potential pitfalls of the analysis. To do so, we use the ASW example presented in Section \ref{sec:example}, and we illustrate the results with traceability matrices produced by the \agree\ tool suite~\cite{NFM2012:CoGaMiWhLaLu}.  We have extended this tool to add graphical support for displaying adequacy and traceability results.  Screenshots for our running example are shown in Figures~\ref{fig:propertyset1} and~\ref{fig:propertyset4}.

We first formulate two requirements that describe when the ASW should be {\em on} and when it should be {\em off} as follows:

%\begin{definition} {\emph{ASW Requirements Version 1} }
{\smaller
\begin{verbatim}
on_p = (a1_below and a2_below) and not inhibit =>
    doi_on = true;
off_p = (a1_above and a2_above) and inhibit =>
    doi_on = false;
all_p = on_p and off_p;
\end{verbatim}
}
%\end{definition}

\begin{figure}
  \centering
  \includegraphics[width=\columnwidth]{figs/spear_set1.png}
  \vspace{-0.1in}
  \caption{Elements covered by the initial property set}
  \vspace{-0.1in}
  \label{fig:propertyset1}
\end{figure}


\noindent When both altimeters are below the threshold and not inhibited, then the DOI should be on (\onp), and when both altimeters are below the threshold and the ASW is inhibited, then the DOI should be off (\offp).
For each of the \ivccov, \maycov, and \mustcov\ metrics, \allp\ only requires \texttt{\{below, d1, doi\_on\}}, as shown in Figure~\ref{fig:propertyset1}.   This small set of elements is due to a classic specification problem: using computed variables as the antecedents of implications.  If these values are computed incorrectly (say, we choose the wrong threshold for \aonebelow), it may cause the property to be valid for incorrect reasons.
%
%This is alarming, and somewhat puzzling, because one would think that at least the definitions of the `below' or `above' would be necessary.  However, because the specification used the model variables \aonebelow, \atwobelow, \aoneabove, and \atwoabove, the actual valuations of the thresholds do not matter.  This situation illustrates a classic specification problem: using computed variables in the antecedents of implications.
%\footnote{\noindent ~In this case, if the computation of the variables used in the antecedent is incorrect, then our property will not verify what it is expected to verify \mike{citation to one of our papers on specification here...}; note also that this does not mean the property is necessarily {\em vacuous}.}
%
We therefore modify our properties to use inputs as follows:

{\smaller
\begin{verbatim}
on_p = ((alt1 < THRESHOLD) and (alt2 < THRESHOLD))
   and not inhibit => doi_on = true;
off_p = ((alt1 >= T_HYST) and (alt2 >= T_HYST))
   and inhibit => doi_on = false;
\end{verbatim}
}
% all_p = on_p and off_p;
%\end{definition}

\noindent In this version, distinctions emerge between the metrics.  \allp\ has two \mivc s: \texttt{\{\{a1\_below, below, doi\_on, d1\}, \{a2\_below, below, doi\_on, d1\}\}}, because of the \onp\ property: in the implementation, the DOI is turned on when either of the altimeters is below the threshold, while our property states that they both must be below.
Domain experts determine that the requirement is correctly specified and that our implementation is a reasonable refinement, so there is no need to change the model or the property.  The \must\ elements are the same as version 1: \texttt{\{below, doi\_on, d1\}}, because neither \aonebelow\ or \atwobelow\ is required for all proofs.  %However, given the \must elements, we can no longer construct a proof, because at one of these definitions is necessary for either proof.
The \may\ elements contain both \aonebelow\ and \atwobelow.

The \abovehyst, \aoneabove, \atwoabove, and \dtwo\ equations are still missing, meaning that the ``above'' thresholds are irrelevant to our properties.  Examining \offp, we realize that we have a specification error; the DOI should be off if either \inhibit\ is true or both altimeters are above the threshold. The fix is:

%\begin{definition} {\emph{ASW Requirements Version 2} }
{\smaller
\begin{verbatim}
off_p = ((alt1 >= T_HYST) and (alt2 >= T_HYST))
   or inhibit => doi_on = false;
\end{verbatim}
}
%on_p = ((alt1 < THRESHOLD) and (alt2 < THRESHOLD))
%   and not inhibit => doi_on = true;
%all_p = on_p and off_p;
%\end{definition}

\noindent Now the \allp\ requirement proof yields a single \mivc ~that requires all variables except \{\texttt{d2}\}, so \mivc ~= \may ~= \must.  Interestingly, the \offp\ proof requires both the lower altimeter thresholds even though the \onp\ proof does not; the reason is that if either of these is false, then \doion\ will be true.  To cover \{\texttt{d2}\}, we realize no property covers the hysteresis case, so an additional property is added for this case:

%\begin{definition} {\emph{ASW Requirements Version 2} }
{\smaller
\begin{verbatim}
hyst_p = not inhibit and
         (alt1 > THRESHOLD and alt2 > THRESHOLD) and
         (alt1 < T_HYST or alt2 < T_HYST) =>
   (doi_on = false -> doi_on = pre(doi_on))
all_p = on_p and off_p and hyst_p;
\end{verbatim}
}
%on_p = ((alt1 < THRESHOLD) and (alt2 < THRESHOLD))
%   and not inhibit => doi_on = true;
%all_p = on_p and off_p;
%\end{definition}
\noindent The final property states that if the antecedent conditions hold, then in the initial state, the \doion\ variable is assigned false, and in subsequent steps, it retains the same value as it previously had.

\begin{figure}
  \centering
  \includegraphics[width=\columnwidth]{figs/spear_set4.png}
  \vspace{-0.1in}
  \caption{Elements covered by the final property set}
  \vspace{-0.1in}
  \label{fig:propertyset4}
\end{figure}

As shown in Figure~\ref{fig:propertyset4}, the measures again coincide and include all variables, and we appear to have a reasonably complete specification.  However, the measures are certainly not foolproof; it turns out that using {\em only} the hysteresis property \hystp\ will {\em also} yield a ``complete'' result for all of the metrics: to establish its validity, all of the equations that we have defined in the model are required.  This is because the partitioning of the transition system (i.e., the equations) is insufficiently {\em granular} to detect the incompleteness.
%However, this one property would not reasonably be considered a complete specification.
%We will examine this situation further in the discussion section~\mike{add this!}.
%
%\mike{How would we define a metric that would flag the model as incomplete?  Model transformation would do it: if we added separate variables for each assignment of doi\_on, then any of the metrics would flag the \hystp\ spec as incomplete.}

\iffalse As mentioned, IVCs are derived from inductive invariants; in other words, they are built upon the proof of the validity of a given property. One interesting fact about proofs
  is that a given property could be proved from different proof paths.
  The $\aivc$ captures this fact and gives a clear picture of various ways a property is satisfied. By getting all the MIVCs for the system properties and categorizing them, one can find if there are design artifacts that do not trace to any property: set $\bigcap \{IRR (P) | P \in \Delta \}$.  If this set is non-empty, it is a possible indication of ``gold plating'' or missing properties \cite{Murugesan16:renext}.
\fi

\subsection{Granularity}
\label{sec:granularity}

As we have described in Section \ref{sec:background}, a transition relation is considered
to be a conjunction of Boolean formulas. The granularity of these formulas substantially affects the analysis results.  In the presented example, it was possible to have a ``complete'' specification of the model involving only the hysteresis property \hystp.  The way that the model was structured, in order to determine the validity of the property, all of the equations in the model were required.  However, for this property certain subexpressions of the equations were irrelevant, notably the value assigned to the \texttt{doi\_on} variable in the \texttt{then} branches of equations (7) and (8).  If we decompose the equations into smaller pieces, e.g., creating separate equations for the \texttt{then} and \texttt{else} branches, this incompleteness becomes visible and the model is no longer completely covered.  It is often the case that splitting a model into more conjuncts, that is, making the model more {\em granular}, leads to lower coverage of the model.


%\mike{Oy...this section needs improvement}

%Splitting a model into more conjuncts will make coverage scores more accurate and usually lower, though it will not always lower coverage scores.
%
We have explored granularity within the context of the Lustre language.  Lustre provides a nice formalism for discussion because it is top-level conjunctive (as required by our IVC definition), equational, and {\em referentially transparent}~\cite{Halbwachs91:lustre}: the behavior of a Lustre program is defined by a system of equations, and any subexpression on the right side of an equation can be extracted and assigned to a fresh variable\footnote{A fresh variable is a variable with an identifier that has not been used within the program.} which is substituted into the original equation without changing the meaning of a program.  In this context, we can define a {\em granular refinement} as an extraction of a subexpression into a new equation assigning a fresh variable.

We call a Lustre model {\em totally decomposed} if (1) each computed (i.e., non-input) variable is used at most once in the right-hand side of an equation, and (2) each equation is either a single operator (which may be a ternary operator in the case of if-then-else) with leaves all variable expressions, or a constant expression, and (3) each model input is directly assigned to one or more fresh variables and is not used elsewhere in the model.  In this case, each instance of a subexpression and each use of an input in the original model is assigned its own variable, so it is maximally factored.  Intuitively, if the proofs require each of these elements, it means that there is a reasonably strong claim that the requirements are adequate.

If a set of requirements achieves 100\% coverage of a totally decomposed model, then no granular refinement will achieve less than 100\% coverage.  This is straightforward to show in a proof sketch.  Based on the right side of the equation, there are two cases: (1) the entire right-hand side of the equation is extracted (which may be a constant or a single operator expression). In this case, by assumption the variable assigned is necessary for proof, so its definition must be necessary for the proof; in this case, the fresh variable must also be necessary for the proof; (2) a leaf expression of the single right-hand side operator is extracted.  Since (by definition) the leaf is a variable expression that is used only once, and (by assumption) the variable is necessary for proof, then the fresh variable is also necessary for proof.  The resultant model is also a totally-decomposed model, so we can do any number of these extractions.

Unfortunately, such decompositions also appear to slow model checking performance.  In future work, we hope to explore efficient transformations leading to sufficiently granular models.

\iffalse
We have implemented a transformation that splits Lustre models into {\em totally decomposed} models.  In a small initial experiment involving 30 of the original models, we performed our transformation and re-ran the analysis.  By changing the granularity of the model, the analysis tools perform significantly slower for proofs, but the ratio of performance between the proof and the \ucalg\ and \mustalg\ algorithms is largely unchanged.  However, on some models, the \mustalg\ algorithm becomes unacceptably slow (analysis times of tens of hours) and occasionally causes the solver to run out of memory.

The issue of granularity of models is significant, but to the best of our knowledge, is not discussed in previous work.  In our approach, we allow the user to choose the level of granularity, but in certain cases, this may lead to misleading answers when checking the adequacy of requirements.  This aspect will be a focus of our future work, especially in situations in which the tool determines that a set of requirements is {\em complete}.  We believe that it is possible to substantially optimize the n{\"a}ive preprocessing algorithm that we have presented.
\fi
%\subsection{Use in Certification}
%Airborne software must undergo a rigorous software development process to ensure its airworthiness. This process is governed by DO-178C: Software Considerations in Airborne Systems and Equipment Certification \cite{DO178C} and when formal methods tools are used, DO-333: Formal Methods Supplement to DO-178C and DO-278A \cite{DO333}. DO-178C proposes a rigorous software development process that starts with an abstract requirements artifact that is iteratively refined into a software designs, source code, and finally, object code, and a set of {\em objectives} that should be met by critical avionics software.  Two of the key tenets of this process are traceability and adequacy; that is, each refinement of an artifact must be traceable to the artifact if was derived from. Further, each refinement must be shown not to introduce functionality not present in the artifact from which it was derived (adequacy). For example, DO-178C objectives A-3.6 (traceability of high-level requirements to system requirements) and A-4.6 (traceability of software design to high-level requirements) specifically require applicants to demonstrate bi-directional traceability.
%
%DO178C currently uses a variety of metrics to determine adequacy of requirements, but much of the effort involves code-level testing.  Test suites are derived from requirements and used to test the software and measured using different structural coverage test metrics.  If code-level test suites do not achieve full coverage, then an analysis is performed to determine whether there are missing requirements and test cases.  The kind of structural coverage required (e.g., statement, branch, MCDC) for adequate testing is driven by the criticality of the software in question.
%
%The utility of the proposed metrics are being evaluated by Rockwell Collins on a pilot project. The proposed metrics
%appear to be useful for both traceability and adequacy checking \cite{lucas17}.  The proposed metrics appear to be useful for both traceability and adequacy.  Previously, bi-directional traceability between artifacts involved rigorous manual peer review to determine that requirements were adequate and that additional functionality was not introduced in the implementation model.  In the pilot approach, both traceability and adequacy are assessed using metrics proposed in this paper.  The goal is to use this automation to satisfy the DO178C objectives related to traceability and adequacy.


%We plan to focus on efficient analysis of sufficiently granular models in future work.
%when the tool returns that the set of requirements are complete.






\section{Conclusions \& Future Work}
\label{sec:conclusion}

In this paper, we have defined a novel coverage notion for formal verification using
the IVC concept, a useful measure in relation to
a valid safety property for inductive model checking. We have shown that our method
 is computationally efficient while 
 being accurate about the covered parts of a given design. 
 We have referred to this accuracy as preserving provability, which means 
 that a set of elements considered covered by our algorithm is sufficient 
 to establish the validity proof for every requirement in the set of specifications.
 
 We have implemented
our algorithm as part of the open source model checker JKind. Using our approach, measuring coverage is quite possible as we have shown in our experiments.
 We also benchmarked our implementation and compared it with other techniques in the literature. 
 The experiments show that the computation imposes a small overhead to the verification process. We have described how the justifiable notion of coverage proposed in this paper can be used as a
means of quantifying requirements completeness.
 
 In addition, based on the idea of multiple support sets for a specification, we 
 have introduced and discussed some other complementary coverage notions in the context of formal verification. Finally, we are in the process of developing some efficient algorithms for exploring the space of IVCs, e.g., finding a
minimum, rather than minimal support set, or finding all support sets. Having such algorithms makes the utilization of other proposed coverage practical.
% % \input{RelatedWork}
% % \input{ModelDevelopment}
% % \input{Discussion}
%\chapter{Timeline and Conclusion}
Establishing a reference model for integrated process simulation---a process
situated within its expected project environment---is an important step toward
reducing process-adoption risk through \apriori process evaluation.
In this work, we propose creating and validating such a reference model to
capture the essential constructs and relationships underlying agile process
models situated within a project environment.
We plan to achieve this objective
by constructing a reference model from a number of agile practices and showing
it as the basis for credible simulation models by creating a simulation framework that can simulate concrete instances of the reference
model---validating the framework using metamorphic testing.  We will validate
the reference model's expressiveness by using it to encode a number of process
models.  Additionally, we will validate the simulation framework using
metamorphic testing, leveraging known relationships between input changes
and the corresponding change to the output currently captured in the literature.

%  by
% performing metamorphic testing on a simulation framework---leveraging known relationships between transformations on inputs and their expected, relative impact on the outputs to perform metamorphic testing on the simulation framework.
%
%
% We plan to achieve this objective
% by constructing a reference model from a number of agile practices, validating
% the model's expressiveness by using it to encode a number of process models, and
% showing it as the basis for credible simulation models by performing metamorphic
% testing on the simulation, leveraging known relationships between input changes
% and the corresponding change to the output.

The timeline for the proposed work is as follows:
\begin{description}
\item[Fall - Spring 2015:]  Develop simulation framework based on the reference
model.
\item[Fall 2015:]  Express agile process models in terms of reference models
\item[April - May 2015:]   Perform metamorphic testing and refine simulation
framework
\item[May - June 2015:]  Write dissertation
\item[July 2015:]  Complete requested edits
\item[Early August 2015:]  Defend dissertation
\item[August 2015:]  Complete final changes
\end{description}



Upon completion of the proposed research, we will have a verified reference
model for expressing simulatable, integrated process models and a
simulation framework for executing and evaluating models based on the reference
model.  Our simulation models expressed using our reference model will allow the
modeler to decoupling the product and process aspects of the overall
project.
%to The reference model will be able to capture agile processes and will support
% decoupling the product and process aspects of the overall project.
Further, the reference model will provide us with a means for expressing
integrated, agile process models---models of agile processes that take into account the
individuals on the team, the product under development, and the project
constraints.  This lays the groundwork for detailed, \apriori process evaluation
through simulation and---because of having to encode the process using the
reference model---provides some degree of manual analysis.  Such process
evaluation is important, not only for \apriori process selection and tailoring,
but also for developing new processes and improving our understanding of the
limitations of agile techniques.


% \printglossary[title={LIST OF TERMS}, toctitle={List of Terms}]

%%%%%%%%%%%%%%%%%%%%%%%%%%%%%%%%%%%%%%%%%%%%%%%%%%%%%%%%%%%%%%%%%%%%%%%%%%%%%%%%
% WORKS CITED
%%%%%%%%%%%%%%%%%%%%%%%%%%%%%%%%%%%%%%%%%%%%%%%%%%%%%%%%%%%%%%%%%%%%%%%%%%%%%%%%
% When you need a works cited section, uncomment this and update the name of the
%   file.
%%%
    \bibliographystyle{abbrv}
%     \bibliographystyle{abbrvnat}
\bibliography{noAbbreviations,citations}


%%%%%%%%%%%%%%%%%%%%%%%%%%%%%%%%%%%%%%%%%%%%%%%%%%%%%%%%%%%%%%%%%%%%%%%%%%%%%%%%
% APPENDICIES
%%%%%%%%%%%%%%%%%%%%%%%%%%%%%%%%%%%%%%%%%%%%%%%%%%%%%%%%%%%%%%%%%%%%%%%%%%%%%%%%
% If you need an appendix, uncomment the following lines and add sections.
%%%
%\appendix
%\input{sectionName}

\end{document}
