
\documentclass[10pt, conference]{IEEEtran}



\usepackage{cite}

\ifCLASSINFOpdf
\usepackage[pdftex]{graphicx}
  % declare the path(s) where your graphic files are
  % \graphicspath{{../pdf/}{../jpeg/}}
  % and their extensions so you won't have to specify these with
  % every instance of \includegraphics
  % \DeclareGraphicsExtensions{.pdf,.jpeg,.png}
\else
  % or other class option (dvipsone, dvipdf, if not using dvips). graphicx
  % will default to the driver specified in the system graphics.cfg if no
  % driver is specified.
\usepackage[dvips]{graphicx}
  % declare the path(s) where your graphic files are
  % \graphicspath{{../eps/}}
  % and their extensions so you won't have to specify these with
  % every instance of \includegraphics
  % \DeclareGraphicsExtensions{.eps}
\fi
% graphicx was written by David Carlisle and Sebastian Rahtz. It is
% required if you want graphics, photos, etc. graphicx.sty is already
% installed on most LaTeX systems. The latest version and documentation can
% be obtained at:
% http://www.ctan.org/tex-archive/macros/latex/required/graphics/
% Another good source of documentation is "Using Imported Graphics in
% LaTeX2e" by Keith Reckdahl which can be found as epslatex.ps or
% epslatex.pdf at: http://www.ctan.org/tex-archive/info/
%
% latex, and pdflatex in dvi mode, support graphics in encapsulated
% postscript (.eps) format. pdflatex in pdf mode supports graphics
% in .pdf, .jpeg, .png and .mps (metapost) formats. Users should ensure
% that all non-photo figures use a vector format (.eps, .pdf, .mps) and
% not a bitmapped formats (.jpeg, .png). IEEE frowns on bitmapped formats
% which can result in "jaggedy"/blurry rendering of lines and letters as
% well as large increases in file sizes.
%
% You can find documentation about the pdfTeX application at:
% http://www.tug.org/applications/pdftex





% *** MATH PACKAGES ***
%
\usepackage[cmex10]{amsmath}
\usepackage{amssymb}
\usepackage{stmaryrd}
\usepackage{amsthm}
\usepackage{algorithmic}
\usepackage{array}
\usepackage{mdwmath}
\usepackage{mdwtab}
\usepackage{eqparbox}
\usepackage[tight,normalsize]{subfigure}
\usepackage[font=normalsize]{caption}
\usepackage{tabularx,colortbl}
\usepackage[dvipsnames]{xcolor}
\usepackage{flushend}
\usepackage{cite}
\usepackage{amsmath}
%\usepackage[font=footnotesize]{subfig}
%\usepackage[caption=false,font=footnotesize]{subfig}
\usepackage{fixltx2e}
\usepackage[ruled, vlined, linesnumbered]{algorithm2e}
\usepackage{stfloats}
\usepackage{url}
\usepackage{xspace}

\hyphenation{op-tical net-works semi-conduc-tor}
\newcommand{\mkeyword}[1]{\mbox{\texttt{#1}}}
\DeclareMathOperator{\kuop}{uop}
\DeclareMathOperator{\kbop}{bop}
\DeclareMathOperator{\kite}{ite}
\DeclareMathOperator{\kpre}{pre}
\DeclareMathOperator{\dom}{dom}
\DeclareMathOperator{\ktrue}{true}
\DeclareMathOperator{\kfalse}{false}
\DeclareMathOperator{\kselect}{select}
\DeclareMathOperator{\ran}{range}
\newcommand{\lbb}{[\![}
\newcommand{\rbb}{]\!]}
\newcommand{\expr}{\phi}
\newcommand{\exprS}{\Phi}

\begin{document}

\definecolor{gold}{rgb}{0.90,.66,0}
\definecolor{dgreen}{rgb}{0,0.6,0}
\newcommand{\mike}[1]{\textcolor{red}{#1}}
\newcommand{\fixed}[1]{\textcolor{purple}{#1}}
\newcommand{\andrew}[1]{\textcolor{green}{#1}}
\newcommand{\ela}[1]{\textcolor{blue}{#1}}
\newcommand{\stateequiv}{\equiv_{s}}
\newcommand{\traceequiv}{\equiv_{\sigma}}
\newcommand{\ta}{\text{TA}}
\newcommand{\cta}{\text{TA$_{C}$}}
\newcommand{\tta}{\text{TA$_{T}$}}

\newcommand{\bfalg}{{IVC\_BF}\xspace}
\newcommand{\ucalg}{{IVC\_UC}\xspace}
\newcommand{\ucbfalg}{{IVC\_UCBF}\xspace}
\newcommand{\mustalg}{{IVC\_MUST}\xspace}

\newtheorem{definition}{Definition}
\newtheorem{lemma}{Lemma}
\newtheorem{theorem}{Theorem}
\newtheorem{coroll}{Corollary}
%\newdef{lemma}{Lemma}
%\newdef{definition}{Definition}
%\newdef{theorem}{Theorem}
%\newdef{note}{Note}
%
% paper title
% can use linebreaks \\ within to get better formatting as desired
\title{Inductive Validity Cores for Formal Verification}


% author names and affiliations
% use a multiple column layout for up to two different
% affiliations

\author{\IEEEauthorblockN{Elaheh Ghassabani and Michael W. Whalen}

\IEEEauthorblockA{Department of Computer Science and Engineering\\
University of Minnesota\\
Minneapolis, MN, USA\\
Email: ghassaba, whalen@cs.umn.edu}
\and
\IEEEauthorblockN{Andrew Gacek}
\IEEEauthorblockA{Rockwell Collins\\
Advanced Technology Center\\
Cedar Rapids, IA, USA\\
Email: andrew.gacek@rockwellcollins.com}
}

% conference papers do not typically use \thanks and this command
% is locked out in conference mode. If really needed, such as for
% the acknowledgment of grants, issue a \IEEEoverridecommandlockouts
% after \documentclass

% for over three affiliations, or if they all won't fit within the width
% of the page, use this alternative format:
%
%\author{\IEEEauthorblockN{Michael Shell\IEEEauthorrefmark{1},
%Homer Simpson\IEEEauthorrefmark{2},
%James Kirk\IEEEauthorrefmark{3},
%Montgomery Scott\IEEEauthorrefmark{3} and
%Eldon Tyrell\IEEEauthorrefmark{4}}
%\IEEEauthorblockA{\IEEEauthorrefmark{1}School of Electrical and Computer Engineering\\
%Georgia Institute of Technology,
%Atlanta, Georgia 30332--0250\\ Email: see http://www.michaelshell.org/contact.html}
%\IEEEauthorblockA{\IEEEauthorrefmark{2}Twentieth Century Fox, Springfield, USA\\
%Email: homer@thesimpsons.com}
%\IEEEauthorblockA{\IEEEauthorrefmark{3}Starfleet Academy, San Francisco, California 96678-2391\\
%Telephone: (800) 555--1212, Fax: (888) 555--1212}
%\IEEEauthorblockA{\IEEEauthorrefmark{4}Tyrell Inc., 123 Replicant Street, Los Angeles, California 90210--4321}}




% use for special paper notices
%\IEEEspecialpapernotice{(Invited Paper)}




% make the title area
\maketitle


\begin{abstract}
This paper introduces the concept of Inductive Validity Core (IVC), which provides an explanation of an inductive proof, such as those constructed by modern model checking algorithms using k-induction and Property-Directed Reachability (PDR). This notion has been implemented and evaluated in the JKind model checker. IVCs appear to be useful for several analyses related to traceability and completeness of properties, which are discussed in this paper.
\end{abstract}

\begin{IEEEkeywords}
  formal verification; SAT-based model checking; IVC; inductive proofs;
\end{IEEEkeywords}

\IEEEpeerreviewmaketitle

\section{Introduction}
\label{sec:intro}

In order to build software, one usually starts with {\em requirements}, a set of statements about what the software is intended to do, which is refined either prior to, or in tandem with, the software being developed.  Requirements are necessary to software both for shaping the development of the software and for determining its adequacy when performing verification activities.  Therefore, determining the {\em adequacy} of requirements is of substantial importance to the eventual quality of the software.

Zowghi and Gervasi~\cite{} define adequacy of requirements in terms of the ``three Cs'': Consistency, Completeness, and Correctness.  \mike{[MOST OF THE REST OF THIS PARAGRAPH IS BLATANTLY STOLEN FROM GERVASI - REWRITE]} Davis states that completeness is the most difficult of the specification attributes to
define and incompleteness of specification is the most difficult violation to detect~\cite{}.
According to Boehm~\cite{}, to be considered complete, the requirements document must exhibit three fundamental characteristics: (1) No information is left unstated or ``to be determined'', (2) The information does not contain any undefined objects or entities, (3) No information is missing from this document. The first two properties imply a closure of the existing information and are typically referred to as internal completeness.  The third property, however, concerns the external completeness of the document
\cite{}. External completeness ensures that all of the information required for problem definition is found within the specification.  However, {\em assessing} external completeness in a precise and formal way is difficult, if not impossible, because there is rarely an external reference that can be used to determine whether all relevant requirements have been defined.

What tends to happen instead is that we measure the {\em relative completeness} of requirements with respect to some other artifact.  Usually, the other artifact is some form of implementation of the requirements (it could an abstract ``model'' of the implementation, source code, or object code).  This idea underlies certification standards such as DO178B/C~\cite{}, which require that requirements-based tests are sufficient to achieve structural coverage of the code to a certain level of rigor.  More recent work by Zeller~\cite{} and Murugesan~\cite{} have attempted to adapt these measure towards automated test generation by examining coverage of {\em assertions} in the code.  

A drawback of the approach is that an implementation must exist prior to performing this analysis; if the implementation is only available late in the development process, then incompleteness in requirements is not exposed until very late in the development cycle, potentially leading to substantial rework.  Next, the approach usually requires thousands to hundreds of thousands of tests, which are expensive to construct and can be expensive to modify in the face of changing or incomplete requirements.  Finally, the test metrics that are used for measurement tend to substantially overapproximate which portions of a program are necessary to fulfill a requirement~\cite{} \mike{cite MCDC and OMCDC work here}.

In addition, what happens if we want to use formal methods to prove system requirements?  Arguably, proofs should lead to higher assurance than tests, leading to more confidence in system performance.  However, the problem of requirements completeness becomes, if anything, more critical.  Relatively recently, 
%
%\cite{} \mike{cite formal verification work here}, we have attempted to use 
%These problem is exacerbated if one wishes to use a formal verification to assess  
%
techniques have been devised for analyzing completeness of requirements against formal implementation models, specified as transition systems or Kripke structures~\cite{}\mike{Chockler, Kupferman, Vardi, Kroening, etc.}.  These models are agnostic to the ``abstraction level'' of the implementation: they can represent lower-level requirements, software architectures, or concrete implementations of system behavior.  The mechanism used is based on {\em mutation} and {\em proof}: is it possible to prove that the requirements still hold of the system after mutating the model in some way?  If so, then the requirements are incomplete with respect to that model element.  Unfortunately, these metrics can {\em underapproximate} which portions of a program are necessary to fulfill a requirement: the residual model returned by the analysis for the requirement is, in the general case, insufficient to prove the requirement.  In addition, these approaches tend to be very computationally expensive, having a runtime of (in the best case) approximately 5x the time required for model checking.

What we would like to have is an approach for checking the relative completeness of requirements against an implementation model that:
\begin{itemize}
    \item Can be applied early and throughout a development cycle on different implementation artifacts
    \item Is accurate: the portion of the implementation that is identified as necessary demonstrates the 
        fulfillment of the requirement but does not contain additional information.
    \item Is reasonably computationally efficient. 
\end{itemize}     

\noindent Towards this end, we propose a notion of requirements completeness that examines {\em minimal proofs of requirements}.  In this approach, we measure the completeness of a set of requirements by examining an (approximately) minimal set of model elements necessary to construct a proof of all the requirements.  Like earlier proof-based approaches, this idea is implementation agnostic, so can be applied early in the development cycle against abstract implementation models.  We then define an implementation of this idea using {\em Inductive Validity Cores} (IVCs)~\cite{} \mike{Cite our FSE paper} for transition systems.  We demonstrate that the IVC-based approach is considerably more computationally tractable than previous approaches based on mutation, averaging ~15\% overhead over model-checking alone, rather than (for our benchmark problems) ~900\% overhead required for mutation-based metrics.  In addition, by definition, it retains the portion of the model necessary to prove the requirements.

Thus, the contributions of this work are:
\begin{enumerate}
\item A notion of requirements completeness based on a proof involving a minimal number of model elements
\item A realization of this idea for symbolic transition systems using {\em inductive validity cores} that is a.) cheap to compute, given a model-checking proof, b.) more accurate than test-based methods, and c.) preserves the ``provability'' property from the residual model.
\item An implementation that computes this notion of completeness
\item An experiment that examines our notion of requirements completeness against a previous mutation-based notion of completeness.
\end{enumerate}

\noindent Our eventual goal is to provide a definition of completeness of requirements that can be established using formal verification-based approaches that is acceptable to certification authorities.  We believe that using minimal proofs provides a reasonable candidate metric for this discussion.

%\mike{something here about certification?}

In the rest of the paper is organized as follows.  In Section~\ref{sec:example}, we present a motivating example.  In Section~\ref{sec:background}, we provide the formal preliminaries for the approach.  In Section~\ref{sec:method} we present our approach to computing relative completeness and compare it with several other related approaches.  In Section~\ref{sec:experiments} we define an experiment to examine our algorithm with recent work by Chockler and Kroening~\cite{chockler2010coverage}.  In Section~\ref{sec:results} we describe our results with respect to algorithm performance and properties of the residual models, and discuss limitations of all ``relative completeness'' algorithms.  In Section~\ref{sec:related} we describe related work.  Finally, Section~\ref{sec:conclusion} describes conclusions and future work.

 
...\mike{fill in!}.




\iffalse
Different notions of coverage have been well defined in software testing, however, in formal verification, it is very complex to define and compute this notion.
Usually, coverage techniques in the property-based verification try to measure the quality of the specification with regards to the completeness of a set of properties.
In fact, the goal is to point out unspecified behaviors, hence the idea behind most of the existing work is to address the question of `have we specified enough properties (requirements)?'
Since the coverage notions are usually  and over-approximation, achieving a high coverage does not guarantee there will be no missing behavior. However, when the coverage is low, techniques will definitely reveal some unspecified cases \cite{claessen2007coverage}.
\fi 
\newcommand{\minproofcov}{\text{\sc MinProof-Cov}}


\section{Proof-Based Metrics}
\label{sec:method}

In this section, we propose a new approach for measuring property completeness based on proof rather than mutation.  We first define notation, then describe different possible metrics given a set of {\em minimal proofs}.  In this section, we do not describe how these proofs are discovered, but define an implementation for transition systems in Section~\ref{sec:impl}.
%\subsection{Coverage and Minimal Proofs}
%Alternatively, we can consider using the proofs themselves as a mechanism for determining adequacy of requirements.

\begin{definition} {\emph{IVC coverage (\ivccov):}} \\
\label{def:coverage-justi}
Given $S \in AIVC(P)$, $T_i$ is covered by $P$ via $S$ \emph{iff} $T_i \in \ivccov (P, S)$, where
$\ivccov (P, S) = \{ T_i \in T ~|~ T_i \in S \}$.
%Given $S \in AIVC(P)$, $T_i \in T$ is covered by $P$ \emph{iff} $T_i \in S$,
%denoted by $T_i \in \ivccov (P, S)$
\end{definition}
\vspace{2mm}

%For the sake of simplicity, we refer to the coverage function
%formalized in Definition \ref{def:coverage-justi} as \ivccov\.
%
We call Definition \ref{def:coverage-justi} a \emph{justifiable} metric because, with a set of the model elements marked as covered by \ivccov, $P$ is provable.  Other notions, as will be discussed in Section~\ref{subsec:method-disc}, may yield subsets of the model that are insufficient to reconstruct the proof of the property\footnote{\noindent ~Throughout the paper, when a coverage metric is justifiable, like \ivccov, it is said that it preserves provability of the property.}.
%Thus, the coverage score for \ivccov\ is often higher than the score for \nondetcov.
The coverage score for \ivccov\ can be calculated with: $$\frac{|S|}{|T|}$$
%Note that because minimal proofs are not unique, there are several possible coverage scores.
Because $P$ may have multiple IVCs, there can be a range of scores (with equal justification) for the \ivccov\ metric, which is shown by \minproofcov:
\[
   \minproofcov(T, P) = \{~\frac{ |S|}{|T|}~|~S \in AIVC(P)~\}
\]

\noindent Note that if an IVC contains all model elements (i.e., the model is {\em completely covered}), then there is only one possible IVC, so in this case there is no diversity of scores.

%the model is {\em completely covered}, on the other hand, then there is only one possible minimal set: the set of all elements.

Given {\em all} proofs of a particular property, it is possible to define additional, complementary coverage notions.  To do so, we first introduce categorizations of elements within proofs.
%
Considering $IVC$ and $AIVC$ relations for $P$, model elements can be categorized into one of the following groups:

\begin{itemize}
  \item \textbf{MUST} elements - target artifacts that are present in all the IVCs of a specification.
      %$$ MUST_x = \{\forall i (S_xi \in \Sigma_x) \mid \bigcap S_xi \}$$
      \[
      MUST (P) = \bigcap AIVC(P)
      \]

  \item \textbf{MAY} elements - target artifacts that are used in some, but not all, IVCs.
      \[
      MAY(P) = (\bigcup AIVC (P)) \setminus MUST (P))
      \]

  \item \textbf{IRRELEVANT} elements - target artifacts that are not in any of the IVCs.
  $$IRR(P) = T \setminus (\bigcup AIVC (P))$$
\end{itemize}

Given property $P$, functions MUST, MAY, and IRR partition the target artifacts (set $T$) into three disjoint sets \emph{must}, \emph{may}, and \emph{irrelevant}, respectively. This categorization helps to identify the role and relevance of each target artifact in satisfying a property. The \emph{must} set contains those target artifacts that are absolutely necessary for the property satisfaction.  Any change to these elements will affect provability of the property. On the other hand, any single element in the \emph{may} set may be modified without affecting provability of the property (though modifying multiple elements may require re-proof).   The \emph{irrelevant} artifacts never affect the satisfaction of the property \cite{Murugesan16:renext}.


Using the notions of $MAY$ and $MUST$, we can introduce additional coverage metrics.
%Since the primary goal of
% this paper has been to provide a complementary coverage notion in
%  formal verification, it is worth exploring other possible notions based on the idea of provability and $AIVC$, which is beneficial, as with testing, because if a coverage notion is an over-approximation, when the coverage
% is high, it does not necessarily mean the quality of
% the specification (or test suite) is high, or when it is an under-approximation, a low coverage score does not always mean the specification is of poor quality.

\begin{definition} {\emph{(\maycov):}}
  \label{def:comp-1}
 $T_i \in T$ is covered by $P$ \emph{iff} $T_i \in \maycov (P)$, where
   $\maycov (P) = \{T_i ~|~ \exists S \in AIVC(P)~.~T_i \in S \}$.
\end{definition}

\begin{definition} {\emph{(\mustcov):}}
  \label{def:mustcov}
 $T_i \in T$ is covered by $P$ \emph{iff} $T_i \in \mustcov (P)$, where
   $\mustcov (P) = \{T_i ~|~ \forall S \in AIVC(P)~.~T_i \in S \}$.
\end{definition}

The $\maycov$ notion aims to deal with the fact that a property $P$ may have
several distinct IVCs. In such cases, \ivccov\ only looks at an arbitrary IVC
that may contain a subset of $MAY(P)$, which means, depending on
which IVC it considers, every time it may report a different part of $MAY(P)$
as uncovered. However, \maycov\ resolves this issue reporting the entire set of $MAY(P)$ as covered, which also leads to higher coverage scores.  \mustcov\ takes the opposite view, considering model elements covered only if they appear in all proofs.


It is still possible to build more relaxed coverage metrics in which coverage
is captured by looking at individual properties, rather than their conjunction.
%for example, in the definition of \ivccov , it is wise to look at $P$ as
%the conjunction of all properties. However,
We can, for example, describe a metric in which any element used by an IVC for any property is considered covered.
%with this view,
%elements around IVCs that do not have common \emph{must}
%elements with others will be treated as uncovered while they are at least covered by one
% IVC of an individual property in the specification.
%
The next definition, \allcov, formalizes this notion.

\begin{definition} {\emph{(\allcov):}}
  \label{def:comp-2}
     Given a set of properties $\Delta$ over $T$, $T_i \in T$ is covered
   \emph{iff} $T_i \in \allcov (T)$, where
   $\allcov (T) = \{T_i ~|~ \exists P \in \Delta ,~ S \in AIVC(P).~T_i \in S \}$.
\end{definition}

%Considering $MAY$ and $MUST$ categorization, we can formalize another
%coverage metric that takes into account the \emph{must} set;
%however, such a metric is the same as \nondetcov\ as we discuss in the next sub-section.

\subsection{Discussion}
\label{subsec:method-disc}


Based on the categorization of elements, we will state some relationships about IVCs so to compare different proof-based metrics proposed earlier.

\begin{lemma}
  \label{lem:must-not-enough}
  If $MAY(P) \neq \varnothing$, then $P$ is not provable by $MUST(P)$.
\end{lemma}
\begin{proof}
  $MAY(P) \neq \varnothing \Rightarrow  \exists T_i \in MAY(P).$
$T_i \in \bigcup AIVC(P) \wedge T_i \notin MUST(P)$,
which implies $\exists S \in AIVC(P).~ T_i \in S$.
Considering the fact that $S$ is minimal and
$MUST(P) \subset S$ (since $T_i \in S \wedge T_i \notin MUST(P)$),
 $\nexists S' \subset S.~ (I,S') \vdash P$,  which means $(I, MUST(P)) \nvdash P$.
\end{proof}
\vspace{2mm}

%\begin{lemma}
%    \label{lem:must-mustcov}
%    $T_i \in MUST(P) \Leftrightarrow T_i \in \mustcov(P)$
%\end{lemma}
%\begin{proof}
%Immediate from the definition of $MUST$ and \mustcov.
%\end{proof}

Now we focus on the relationship between non-deterministic mutation-based coverage and proof-based metrics. In \cite{chockler2010coverage}, each mutant design changes the type of a single node to \inputnode.
Given a suitable encoding of the netlist,\footnote{The mapping is as follows: the netlist becomes a conjunction
of equations, where each vertex becomes a variable $v_i \in U$, and where each non-input vertex becomes an assignment equation $T_i \in T$.
For example, given an AND-vertex $v_i$ with three input edges from other vertexes $\{v_a, v_b, v_c\}$, we would define an equation $T_i \in T$ of the form $(v_i = (v_a \wedge v_b \wedge v_c))$. } assigning a ``fresh'' input is an isomorphic operation to simply removing a $T_i$ from $T$.
%
%As the variable is no longer constrained by a defining equation, it is effectively an %input.
So, we can reframe the non-deterministic coverage proposed in \cite{chockler2010coverage} as follows:

\begin{definition} {\emph{Nondeterministic coverage (alternate specification) (\nondetcovalt) ~\cite{chockler2010coverage}.} }
\label{def:non-det-2}
$T_i \in T$ is covered by property $P$ \emph{iff} $T_i \in \nondetcovalt (P)$, where
$\nondetcovalt (P) = \{T_i~|~ (I, T) \vdash P \wedge (I, T \setminus \{T_i\}) \nvdash P\}$.
\end{definition}


%\begin{definition} {\emph{Nondeterministic coverage alternate definition (\nondetcovalt) ~\cite{chockler2010coverage}.} }
%\label{def:non-det-2}
%$T_i \in T$ is covered by property $P$ \emph{iff} $T_i \in \nondetcovalt (P)$, where
%$\nondetcovalt (P) = \{T_i~|~ (I, T) \vdash P \wedge (I, T \setminus \{T_i\}) \nvdash P\}$.
%\end{definition}
%
%\begin{lemma}
%    \label{lem:nondet-nondetaltcov}
%    $\nondetcov(P) = \nondetcovalt(P)$
%\end{lemma}
%\begin{proof}
%\mike{obvious?} \ela{   not so sure if obvious}
%\end{proof}

\noindent Given this definition, it becomes straightforward to define some additional properties.

\begin{lemma}
  \label{lem:must-coverage}
$T_i \in \nondetcovalt (P) \Leftrightarrow T_i \in \mustcov(P)$.
\end{lemma}
\begin{proof}
$T_i \in \nondetcovalt (P)$ means that $(I, T \setminus \{ T_i \}) \nvdash P$ then
%$T_i$ is necessary to prove $P$,  which means
$\forall S \subset T .~ T_i \notin S \Rightarrow (I, S) \nvdash P$.
Therefore, since $(I, T) \vdash P$, $T_i \in \bigcap AIVC(P)$, which means  $T_i \in MUST(P)$.
On the other hand, let $T_i \in MUST(P)$; then $\forall S \in AIVC(P).~ T_i \in S$.
By definition, any proof of $P$ is a superset of some minimal IVC in $AIVC(P)$.
Thus, any subset $S$ of $T$ leading to proof contains $T_i$.
Therefore, $T \setminus \{ T_i \}$ does not lead to a proof.
%On the other hand, by definition, $MUST(P)$ is the intersection of all IVCs.
%From the definition of $MUST$, removing a $T_i \in MUST(P)$ from $T$
%results in $ \bigcap AIVC(P) \setminus \{ T_i \} $.
%And since all IVCs in $AIVC$ are \emph{minimal} removing an element from all possible IVCs makes
% $P$ unprovable by every single of them:
% $\forall S \in AIVC(P),~ T_i \in \bigcap AIVC(P).~ (I, S \setminus \{ T_i \}) \nvdash P$. And, we know $S \subseteq T$, so $S \setminus \{ T_i \} \subseteq T \setminus \{ T_i \}$, which means the reachable states of
% $(I, T \setminus \{ T_i \})$ are a subset of the reachable states from
%   $(I, S \setminus \{ T_i \})$. Therefore,
%   $ (I, S \setminus \{ T_i \}) \nvdash P \Rightarrow (I, T \setminus \{ T_i \}) \nvdash P$.
\end{proof}
\vspace{2mm}

In light of Lemma \ref{lem:must-coverage}, the \nondetcovalt\ coverage score of specification $P$ can be also calculated by
$$\frac{|MUST(P)|}{|T|}$$
%Therefore, for set of properties $\Delta$, the coverage score is computed by $$\frac{|MUST(\Pi)|}{|T|},\quad  \Pi= \bigwedge_{i} {P_i \in \Delta}$$
\vspace{0.2in}


%\mike{after all metrics presented, contrast them on the example.  Introduce the properties HERE and then discuss the coverage sets}
%
%\mike{Then, you can talk about justification, etc.}
\begin{coroll}
\label{cor:must-not-provable}
\nondetcovalt\ does not preserve provability.
\end{coroll}
\begin{proof}
Immediate from Lemma \ref{lem:must-not-enough} and Lemma \ref{lem:must-coverage}
\end{proof}
\vspace{2mm}
\begin{coroll}
\label{cor:ivc-provable}
\ivccov\ preserves provability.
\end{coroll}
\begin{proof}
Immediate from Definition~\ref{def:minimal-ivc} and Definition \ref{def:coverage-justi}
\end{proof}
\vspace{2mm}

%It should be pointed out that \ivccov\ is accurate meaning that it does not result in false positives. In other words, since IVCs are \emph{minimal}, \ivccov\ does not mark
%any \emph{actual} uncovered element as covered.

To conclude this section, we should mention that one can define many more proof-based coverage metrics based on the IVC/AIVC idea. Metrics that make use of $AIVC$ relation are computationally more expensive to compute than \ivccov\ although they might be easier to satisfy (i.e., result in higher coverage scores).

The proposed coverage metrics can be ranked in terms of their scores as follows:
$$\nondetcovalt\ \leq \ivccov\ \leq \maycov\ \leq \allcov$$
\ivccov\ and \nondetcovalt\ are equivalent when all elements within the model are covered: if all model elements are MUST elements, then there can only be one IVC, and this IVC uses all of the model elements.

%Based on our preliminary evaluation, we believe that metrics based on
%$AIVC$ relation (like \maycov\ and \allcov) are approximately as computationally expensive as \nondetcov.\footnote{\noindent ~The reason is that \nondetcov\ computes the must set which is also based on $AIVC$ relation. However, in terms of preserving provability, a set of design elements marked as covered by \allcov\ and \maycov\ are
%sufficient to reconstruct the proof of the properties.}
%In the following sections, we first illustrate how the different metrics measure coverage of our ASW example with some sample requirements, and then perform a larger experiment with the \nondetcov\ and \ivccov\ metrics.

%So, in order to examine the proof-based metrics, Section \ref{sec:impl} considers the implementation of two major notions: \nondetcov\ and
% \ivccov ; because \nondetcov\ is based on a recent work in the literature,
% and among all the other proposed notions, \ivccov\ is the
% one that does not take into account $AIVC$.
 %Besides, in terms of coverage score, \ivccov\ is not too easy (or hard) to satisfy.  

\section{Implementation}
\label{sec:impl}

We have implemented the inductive validity core algorithms in the
previous section in two tools: {\em JKind}, which performs the \ucalg
algorithm, and {\em JSupport}, which can compute either the \bfalg or
the \ucbfalg algorithm (using JKind as a subprocess). Moreover, our
implementation of \ucbfalg uses an additional feature of JKind to
store and re-use discovered invariants between separate runs. This
reduces some of the cost of attempting to re-prove a property multiple
times. These tools operate over the Lustre
language~\cite{Halbwachs91:lustre}, which we briefly illustrate below.

\subsection{Lustre and IVCs}

Lustre~\cite{Halbwachs91:lustre} is a synchronous dataflow language
used as an input language for various model checkers. The textual
models in Figures~\ref{fig:ex-before} and \ref{fig:ex-after} are
written in Lustre. We will use model in Figure~\ref{fig:ex-before} as
a running example in this section. For our purposes, a Lustre program
consists of 1) input variables, {\tt x} in the example, 2) output
variables, {\tt a}, {\tt b}, and {\tt y} in the example, and 3) an
equation for each output variable. A Lustre program runs over discrete
time steps. On each step, the input variables take on some values and
are used to compute values for the output variables on the same step.
In addition, equations may refer to the previous value of a variable
using the {\tt pre} operator. This operator is underspecified in the
initial step, so the arrow operator, {\tt ->}, is used to guard the
{\tt pre} operator. In the initial step the expression {\tt e1 -> e2}
evalutes to {\tt e1}, and it evaluates to {\tt e2} in all other steps.

We interpret a Lustre program as a model specification by considering
the behavior of the program under all possible input traces. Safety
properties over Lustre can then be expressed as Boolean expressions in
Lustre. A safety property holds if the corresponding expression is
always true for all input traces. For example, the property for
Figure~\ref{fig:ex-before} is {\tt y >= 0}, which is a valid property.

It is straightforward to translate this interpretation of Lustre into
the traditional initial and transition relations. We will show this by
continuing with the example in Figure~\ref{fig:ex-before}. First we
introduce a new Boolean variable $init$ into the state space to denote
when the system is in its initial step. Then we define,
\begin{align*}
  &I((x, a, b, y, \mathit{init})) = \mathit{init} \\
  &T((x, a, b, y, \mathit{init}), (x', a', b', y', \mathit{init'})) = \\
  &\hspace{1.5cm} (a' = f(x', \ite{init}{0}{y})) \land~ \\
  &\hspace{1.5cm} (b' = \ite{a' \geq 0}{a'}{-a'}) \land~ \\
  &\hspace{1.5cm} (y' = b' + (\ite{init}{0}{y})) \land ~\\
  &\hspace{1.5cm} \neg\mathit{init'}
\end{align*}
A safety property such as {\tt y >= 0} is translated into
$\mathit{init} \lor (y \geq 0)$. Nested uses of arrow and pre
operators are handled by introducing new output variables for nested
expressions, though such details are unimportant for our purposes.

Each equation in the Lustre program is translated into a single
top-level conjunct in the transition relation. This is very convenient
as the IVC of a Lustre property can be reported in terms of the output
variables whose equations are part of the IVC. Equivalently, the
interpretation of an IVC for a Lustre property is that any output
variable that is not part of the IVC can be turned into an input
variable, its equation thrown away, while preserving the validity of
the property.

\subsection{JKind}

JKind is an infinite-state model checker for safety properties. JKind
proves safety properties using multiple cooperative engines in
parallel including $k$-induction, property directed reachability, and
template-based lemma generation. JKind accepts Lustre programs written
over the theory of linear integer and real arithmetic. In the back-end,
JKind uses an SMT-solver such as Yices, Z3, CVC4, MathSAT, or
SMTInterpol.

JKind works on multiple properties simulatenously. When a property is
proven and IVC generation is enabled, an additional parallel engine
executes Algorithm~\ref{alg:ivc} to generate a near-minimal IVC.

JKind accepts an annotation on its input Lustre program indicating
which outputs variables to consider for IVC generation. Output
variables not mentioned in the annotation are implicitly included in
all IVCs. This allows the implementation focus on the variables
important to the user and ignore, for example, administrative
equations. This is even more important for tools which generate Lustre
as they often create many such administrative equations which simply
wire together more interesting expressions.


%\subsection{Experiment}
%\label{sec:experiment}

\newcommand{\takeaway}[1]{
\vspace{6pt}
\noindent\fbox{\parbox{0.975\columnwidth}{#1}}
\vspace{6pt}
}
 the overhead
in discovering all IVCs is a linear in the number of unique IVC
in the problem multiplied by the cost for finding a proof for
%\mike{What do we want to call our efficient algorithm: IVC?}

%We would like to investigate both the {\em efficiency} and {\em
%  minimality} of our three algorithms: the naive brute-force
%algorithm (\bfalg), the UNSAT core-based algorithm (\ucalg), and the
%combined UNSAT core followed by brute-force minimization algorithm
%(\ucbfalg). Efficiency is computed in terms of wall-clock time: how
%much overhead does the IVC algorithm introduce? Minimality is
%determined by the size of the IVC: cores with a smaller number of
%variables are preferred to cores with a larger number of variables.
%Finally, we are interested in the {\em diversity} of solutions: how
%often do different tools/algorithms generate different minimal IVCs?
%
%The use of \texttt{JKind} allows additional dimensions to our investigation: it supports two different inductive algorithms: $k$-induction and PDR, and a ``fastest'' mode, that runs both algorithms in parallel.  In addition, \texttt{JKind} supports multiple back-end SMT solvers including Z3~\cite{DeMoura08:z3}, Yices~\cite{Dutertre06:yices}, MathSAT~\cite{Cimatti2013:MathSAT}, and SMTInterpol~\cite{Christ2012:SMTInterpol}.  We would like to determine whether the choice of inductive algorithm affects the size of the IVC, whether different solvers are more or less efficient at producing IVCs, and whether running different solvers/algorithms leads to {\em diversity} of IVC solutions.
%
%Therefore, we investigate the following research questions:
%\begin{itemize}
%    \item \textbf{RQ1:} How expensive is it to compute inductive validity cores using the \bfalg, \ucalg, and \ucbfalg algorithms?
%    \item \textbf{RQ2:} How close to minimal are the IVC sets computed by \ucalg as opposed to the (guaranteed minimal) \ucbfalg?  How do the sizes of IVCs compare to static slices of the model?
%    \item \textbf{RQ3:} How much {\em diversity} exists in the solutions produced by different solver/induction algorithm configurations?
%\end{itemize}
%
%\subsection{Experimental Setup}
%In this study, we started from a suite of 700 Lustre models developed
%as a benchmark suite for~\cite{Hagen08:FMCAD}. We augmented this suite
%with 81 additional models from recent verification projects including
%avionics and medical devices~\cite{QFCS15:backes,hilt2013}. Most of
%the benchmark models from~\cite{Hagen08:FMCAD} are small (10kB or less,
%with 6-40 equations) and contain a range of hardware benchmarks and
%software problems involving counters. The additional models are much
%larger: around 80kB with over 300 equations. We added the new
%benchmarks to better check the scalability for the tools, especially
%with respect to the brute force algorithm.
%%
%%\mike{MORE HERE...stats on size, reasons for add'l models.}
%Each benchmark model has a single property to analyze.  For our purposes, we are only interested in models with a {\em valid} property (though it is perhaps worth noting that there is no additional computation---and thus no overhead---using the \texttt{JKind} IVC options for {\em invalid} properties).  In our benchmark set, 295 models yield counterexamples, and 10 additional models are neither provable nor yield counterexamples in our test configuration (see next paragraph for configuration information).  The benchmark suite therefore contains 476 models with valid properties, which we use as our test subjects.
%
%For each test model, we computed \ucalg in 12+1 configurations: the
%twelve configurations were the cross product of all solvers \{Z3,
%Yices, MathSAT, SMTInterpol\} and inductive algorithms
%\{$k$-induction, PDR, fastest\}, and the remaining (+1) configuration
%was an instance of \bfalg run on Yices, which is the default solver in
%\texttt{JKind}. In addition, for each of the 12 configurations, we ran an
%instance of \texttt{JKind} without IVC to examine overhead. The experiments
%were run on an Intel(R) i5-2430M, 2.40GHz, 4GB memory machine, with a
%1 hour timeout for each analysis on any model. The data gathered for
%each configuration of each model included the time required to check
%the model without IVC, with IVC, and also the set of elements in the
%computed IVC.\footnote{The benchmarks, all raw experimental results,
%  and computed data are available on \cite{expr}.}
%
%Note that not all analysis problems were solvable with all algorithms: for all solvers, $k$-induction (without IVC) was unable to solve 172 of the examples.  When comparing minimality of different solving algorithms, we only considered cases where both algorithms provided a solution (as will be discussed in more detail in Section~\ref{sec:minimality}).
%
%\iffalse
%\begin{itemize}
%    \item an algorithm to compute a truly minimal set of support, i.e. \texttt{JSupport}.
%    \item given a LUS model, a static crawler which automatically marks all equations of a node in the initial support set of a property.
%    \item some trackers that measure the verification time with/ without support computation.
%   % \item some minor changes in the XML writers.
%\end{itemize}
%
%\mike{My thoughts on this section: mostly, it needs more structure: more information on the properties of the models: size, provenance, etc., a broken out subsection on the description of the experimental setup, etc}
%
%\mike{I think we want to split out the results in another top-level section}
%
%Experiment:
%\begin{itemize}
%    \item (Overview) describe research questions and goals.
%    \item Experimental setup: tell me about the models: how many, how big are they?  Then, tell me about the experiment: the tool configurations, the machine used for test.
%    \item Data generation: Describe what you measured for each model analysis.
%\end{itemize}
%\fi
%
%
%%%  LocalWords:  minimality ive UNSAT IVC Minimality IVCs PDR Yices
%%%  LocalWords:  MathSAT SMTInterpol RQ JSupport


\section{Conclusions \& Future Work}
\label{sec:conc}
The idea of extracting a minimal IVC for a given property, and applications for doing so was recently introduced in \cite{Ghass16}.  However, a single IVC often does not provide a complete picture of the traceability from a property to a model.  In this paper,
we have addressed the problem of extracting {\em all minimal} IVCs. We have shown
the correctness and completeness of our method and algorithm.  In addition, we have a substantial evaluation that shows that the practicality and efficiency of our technique.

Our method is inspired by a recent work in the domain of satisfiability analysis \cite{marco2016fast}. One interesting future direction is to devise similar MIVC enumeration algorithms based on other studies on MUSes extraction such as \cite{nadel2014accelerated}.  We are also looking into improving our implementation by using more  efficient methods for the \isadeq ~and \getivc ~modules used by our algorithm. Another interesting direction is to parallelize the enumeration process: it is certainly possible to ask for multiple distinct maximal models to be solved in parallel.
%, though this may result in unnecessary work performed by some of the parallel solvers.

We also plan to investigate additional applications of the idea.  When performing {\em compositional verification}, the All-IVCs technique may be able to determine {\em minimal component sets} within an architecture that can satisfy a given set of requirements, which may be helpful for design-space exploration and synthesis. Finally, we are interested in adapting the notion of (all) validity cores for \emph{bounded} model checking for quantifying how much of models have been explored by bounded analysis. 

%\section*{Acknowledgment}
%
%
%The authors would like to thank...
%more thanks here


\bibliographystyle{IEEEtran}
\bibliography{biblio}
\end{document}


